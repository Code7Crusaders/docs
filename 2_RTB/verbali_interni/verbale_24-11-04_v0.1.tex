\documentclass{article}
\usepackage{graphicx}
\usepackage{fancyhdr}
\usepackage{geometry}
\usepackage{setspace}
\usepackage{tikz}
\usepackage[italian]{babel}
\usepackage{tabularx}
\usepackage[hidelinks]{hyperref}

% Margini della pagina
\geometry{a4paper, margin=1in}

% Intestazione personalizzata
\pagestyle{fancy}
\fancyhf{}
\fancyhead[L]{Code7Crusaders - Software Development Team}
\fancyhead[R]{\thepage}

% Spaziatura delle righe
\setstretch{1.2}

\begin{document}

% Sezione del titolo
\begin{titlepage}

    \AddToHookNext{shipout/background}{
    \begin{tikzpicture}[remember picture,overlay]
    \node at (current page.center) {
    \includegraphics{../../img/background.png}
    };
    \end{tikzpicture}
    }

    \centering
    \vspace*{2cm}
    
    \includegraphics[width=0.3\textwidth]{../../img/logo/7Crusaders_logo.png} % Aggiungi il logo qui
    \vspace{1cm}
    
    {\Huge \textbf{Code7Crusaders}}\\
    \vspace{0.5cm}
    {\Large Software Development Team}\\
    \vspace{2cm}
    
    {\large \textbf{Riunione Settimanale 04/11/2024}}\\
    \vspace{5cm}

    \textbf{Membri del Team:}\\
    Enrico Cotti Cottini, Gabriele Di Pietro, Tommaso Diviesti \\
    Francesco Lapenna, Matthew Pan, Eddy Pinarello, Filippo Rizzolo \\
    \vspace{0.5cm}
    
    \vspace{1cm}
\end{titlepage}

%Versioni
\newpage
\begin{table}[h!]
\centering
\textbf{Versioni} \\ % Titolo sopra la tabella
\vspace{2mm} % Spazio tra il titolo e la tabella
\begin{tabular}{|c|c|c|c|c|}
    \hline
    \textbf{Ver.} & \textbf{Data} & \textbf{Autore} & \textbf{Verificatore} & \textbf{Descrizione} \\
    \hline
    0.1 & 5/11/2025 & Gabriele Di Pietro & Eddy Pinarello & Prima stesura del documento \\ 
    \hline
\end{tabular}
\end{table}

% Indice
\newpage
\tableofcontents

% Registro Presenze
\newpage
\section{Registro Presenze}
\textbf{Piattaforma della riunione:} Piattaforma Discord \\
\textbf{Ora di Inizio:} 21:00\\
\textbf{Ora di Fine:} 22:00\\
\\
\begin{tabular}{|c|c|c|} %NON MI RICORDO CHI MANCASSE Mi pare Francesco
    \hline
    \textbf{Componente} & \textbf{Ruolo} & \textbf{Presenza}\\
    \hline
    Enrico Cotti Cottini &  & Presente \\ 
    \hline
    Gabriele Di Pietro & Redattore & Presente\\ 
    \hline
    Tommaso Diviesti &  & Presente \\ 
    \hline 
    Francesco Lapenna & & Assente \\ 
    \hline
    Matthew Pan &  & Presente\\ 
    \hline 
    Eddy Pinarello & Verificatore & Presente \\ 
    \hline 
    Filippo Rizzolo & Amministratore & Presente \\ 
    \hline 
\end{tabular}

% Sezione Verbale
\newpage
\section{Verbale}
\subsection{Approvazione della candidatura}
Prendiamo atto dell'approvazione della candidatura presentata per il progetto \textbf{LLM: Assistente virtuale} da parte del professor Vardanega.
Nella valutazione viene evidenziato come abbiamo sottostimato l'impegno dato all'analisi dei requisiti e per aver chiarito poco la rotazione dei ruoli di ogni membro del team, inoltre esaminiamo bene le critiche per migliorare la documentazione in futuro.
Discutiamo assieme su tali osservazioni e ci riserviamo di rivalutare tale pianificazione per il prossimo periodo.
Siamo soddisfatti nel complesso del risultato raggiunto e ci impegneremo a migliorare la documentazione futura.
Provvediamo nei prossimi giorni a comunicare via \textit{email} all'azienda \textbf{Ergon} l'approvazione della candidatura.
\subsection{Prossimi Step}
Le prossime revisioni saranno la \textbf{RTB} (\textit{Requirements and Technology Baseline}) e la \textbf{PB} (\textit{Product Baseline}).
Esaminiamo bene la documentazione necessaria per la prossima revisione \emph{RTB}:
\begin{itemize}
    \item \textit{analisi dei requisiti}
    \item \textit{piano di progetto}
    \item \textit{Piano di qualifica}
    \item \textit{norme di progetto}
    \item \textit{Glossario}
    \item \textit{Lettera di presentazione}
    \item \emph{PoC} \textbf{Proof of Concept}
\end{itemize}
Per quanto riguarda il Glossario, riteniamo utile trovare un modo per evidenziare ogni termine in tutti i documenti in cui compare, in modo che al lettore sia chiaro e non ambiguo.
Riteniamo inoltre opportunuo fissare una chiamata con l'azienda per discutere come affrontare le prossime fasi del progetto.
\end{document}
\documentclass{article}
\usepackage{graphicx}
\usepackage{fancyhdr}
\usepackage{geometry}
\usepackage{setspace}
\usepackage{hyperref}
\usepackage{tikz}
\usepackage[italian]{babel}

% Margini della pagina
\geometry{a4paper, margin=1in}

% Intestazione personalizzata
\pagestyle{fancy}
\fancyhf{}
\fancyhead[L]{Code7Crusaders - Software Development Team}
\fancyhead[R]{\thepage}

% Spaziatura delle righe
\setstretch{1.2}

\begin{document}

% Sezione del titolo
\begin{titlepage}

    \AddToHookNext{shipout/background}{
    \begin{tikzpicture}[remember picture,overlay]
    \node at (current page.center) {
    \includegraphics{../img/background.png}
    };
    \end{tikzpicture}
    }

    \centering
    \vspace*{2cm}
    
    \includegraphics[width=0.3\textwidth]{../img/logo/7Crusaders_logo.png} % Aggiungi il logo qui
    \vspace{1cm}
    
    {\Huge \textbf{Code7Crusaders}}\\
    \vspace{0.5cm}
    {\Large Software Development Team}\\
    \vspace{2cm}
    
    {\large \textbf{Valutazione Capitolati}}\\
    \vspace{5cm}

    \textbf{Membri del Team:}\\
    Enrico Cotti Cottini, Gabriele Di Pietro, Tommaso Diviesti \\
    Francesco Lapenna, Matthew Pan, Eddy Pinarello, Filippo Rizzolo \\
    \vspace{0.5cm}
    
    {\large \textbf{Data:}} \today\\
    
    \vspace{1cm}
\end{titlepage}

% Indice
\newpage
%Versioni
\begin{center}
    \textbf{Versioni}
    \\
    \\
    \begin{tabular}{|c|c|c|c|}
        \hline
        \textbf{Ver} & \textbf{Data} & \textbf{Autore} & \textbf{descrizione}\\
        \hline
        0.1 & 21/10/24 & Enrico Cotti Cottini & Prima stesura del documento \\
        \hline
    \end{tabular}
\end{center}
% Sezione Introduzione
\newpage

\section{Introduzione}

Questo documento raccoglie le considerazioni fatte dal team Code7Crusaders 
riguardo ai capitolati proposti per il progetto di Ingegneria del Software, 
consultabili a \url{https://www.math.unipd.it/~tullio/IS-1/2024/Progetto/Capitolati.html}. \newline

Segue una descrizione dei capitolati presi in considerazione dal team durante la fase di valutazione.
Sarà presente una sezione che elenca aspetti positivi e negativi dei vari capitolati.
In fine verrà presentata la scelta del capitolato scelto dal team.

\section{Descrizione dei Capitolati}

\subsection{Capitolato C2 VimarGENIALE - Vimar}

\subsubsection{Descrizione}

Il capitolato \textbf{Vimar GENIALE}, proposto da Vimar S.p.A., ha come obiettivo 
principale la realizzazione di un'applicazione per supportare gli installatori nella 
ricerca di informazioni tecniche sui prodotti Vimar. Il sistema sarà basato su modelli 
di intelligenza artificiale (LLM) che consentiranno di 
rispondere a richieste in linguaggio naturale riguardanti le specifiche tecniche e 
l'installazione dei prodotti.

\subsubsection{Dominio Applicativo}

Il dominio applicativo riguarda l'ambito della \textbf{domotica} e delle \textbf{smart home}. 
Vimar offre dispositivi come interruttori connessi, termostati e comandi per tapparelle, 
integrabili in impianti:
\begin{itemize}
    \item \textbf{Tradizionali}: dispositivi controllati manualmente senza automazioni.
    \item \textbf{Smart}: dispositivi connessi e controllabili da remoto tramite 
    tecnologie wireless come Bluetooth e ZigBee.
    \item \textbf{Domotici}: sistemi avanzati con automazioni e controllo remoto completo.
\end{itemize}

\subsubsection{Dominio Tecnologico}

Il dominio tecnologico prevede l'uso di tecnologie \textbf{Cloud-ready} 
come Docker e Terraform per la containerizzazione. 
L'intelligenza artificiale sarà integrata tramite modelli LLM open source (ad es. Llama, Mistral). 
L'applicazione sfrutterà tecniche di \textbf{Web Scraping} e \textbf{OCR} per raccogliere e indicizzare 
informazioni tecniche dai prodotti presenti sul sito di Vimar.

\subsection{Capitolato C5 3Dataviz - Sanmarco Informatica}

\subsubsection{Descrizione}
Il capitolato, proposto da Sanmarco Informatica S.p.A., 
ha come obiettivo la realizzazione di un'interfaccia web 
per la visualizzazione tridimensionale di dati.
Questo progetto mira a tradurre grandi volumi di informazioni in un formato visuale interattivo, 
facilitando l'interpretazione dei dati e supportando il processo decisionale.

\subsubsection{Dominio Applicativo}
Il dominio applicativo riguarda la visualizzazione dei dati. 
Il sistema consentirà di rappresentare informazioni quantitative in un ambiente 3D, 
utilizzando un istogramma tridimensionale per la presentazione e navigazione dei dati, 
permettendo all'utente di analizzare grandi dataset in maniera chiara e interattiva.

\subsubsection{Dominio Tecnologico}
Il progetto si inserisce nel dominio tecnologico dello sviluppo web, 
facendo uso di tecnologie moderne per la visualizzazione grafica e l'interazione utente. 
Le tecnologie principali includono \textit{Three.js}, una libreria JavaScript per 
la grafica 3D basata su WebGL, e \textit{D3.js}, utilizzata per produrre visualizzazioni 
dinamiche e interattive. Sono suggeriti framework frontend come \textit{React} e \textit{Angular} 
per la realizzazione dell'interfaccia utente.

\subsection{Capitolato C7 Assistente Virtuale - Ergon}

\subsubsection{Descrizione}
Il progetto prevede lo sviluppo di un assistente virtuale che permetta ai clienti 
di un'azienda di ricercare informazioni sui prodotti disponibili, 
rispondendo alle domande più frequenti. L'obiettivo è migliorare l'interazione uomo macchina, 
ottimizzando l'acquisizione delle informazioni sui prodotti e riducendo la necessità di intervento umano.

\subsubsection{Dominio Applicativo}
Il dominio applicativo si colloca nelle aziende che vendono prodotti attraverso 
cataloghi molto ampi e diversificati, in cui la conoscenza approfondita dei prodotti è 
solitamente affidata a specialisti. Il sistema mira a facilitare l'accesso alle informazioni sui prodotti, 
migliorando la gestione delle richieste da parte dei clienti.

\subsubsection{Dominio Tecnologico}
Il progetto sfrutta modelli linguistici di grandi dimensioni, 
come BLOOM o Falcon, che permettono di elaborare e generare risposte a domande complesse. 
Il sistema sarà composto da un database relazionale, un modello LLM e un’interfaccia 
utente mobile che consente una facile interazione con l'assistente virtuale, 
utilizzando API REST per la comunicazione tra i vari componenti.

\subsection{Capitolato C9 BuddyBot - AzzurroDigitale}

\subsubsection{Descrizione}
Il capitolato BuddyBot è proposto da Azzurrodigitale e prevede la realizzazione 
di un assistente virtuale basato su intelligenza artificiale. 
L'obiettivo principale è migliorare l'efficienza dei team di sviluppo aziendale, 
centralizzando e semplificando l'accesso alle informazioni provenienti da diverse piattaforme come GitHub, 
Confluence, Jira, e Slack tramite un'interfaccia chat.

\subsubsection{Dominio Applicativo}
Il dominio applicativo riguarda la gestione delle informazioni e 
delle conoscenze all'interno di team di sviluppo software. 
L'assistente virtuale mira a facilitare l'onboarding, ottimizzare i flussi di lavoro e ridurre 
il tempo speso a cercare risposte attraverso la centralizzazione delle fonti informative.

\subsubsection{Dominio Tecnologico}
Il dominio tecnologico include l'utilizzo di tecnologie basate su intelligenza artificiale, 
tra cui API di terze parti come OpenAI per la comprensione del linguaggio naturale. 
Inoltre, verranno utilizzate tecnologie web moderne come Angular per il front-end, 
Node/NestJS per il back-end, e database per la persistenza delle informazioni scambiate con l'assistente.

\section{Valutazione dei Capitolati}

\subsection{Pro e Contro Capitolato C2 VimarGENIALE - Vimar}

\subsubsection{Pro}
\begin{itemize}
    \item Utilizzo di tecnologie avanzate come LLM e OCR.
    \item Applicazione pratica nel settore della domotica.
    \item Supporto diretto agli installatori, migliorando l'efficienza lavorativa.
\end{itemize}

\subsubsection{Contro}
\begin{itemize}
    \item Complessità nell'integrazione di diverse tecnologie.
\end{itemize}

\subsection{Pro e Contro Capitolato C5 3Dataviz - Sanmarco Informatica}

\subsubsection{Pro}
\begin{itemize}
    \item Utilizzo di tecnologie moderne come Three.js e D3.js.
\end{itemize}

\subsubsection{Contro}
\begin{itemize}
    \item Necessità di competenze avanzate in grafica 3D e sviluppo web.
    \item Possibili problemi di performance con dataset molto grandi.
\end{itemize}

\subsection{Pro e Contro Capitolato C7 Assistente Virtuale - Ergon}

\subsubsection{Pro}
\begin{itemize}
    \item Utilizzo di modelli linguistici meno conosciuti come BLOOM (alternative italiane).
\end{itemize}

\subsubsection{Contro}
\begin{itemize}
    \item Complessità nella gestione e integrazione dei modelli LLM.
    \item Potenziali problemi di accuratezza nelle risposte generate dall'assistente.
\end{itemize}

\subsection{Pro e Contro Capitolato BuddyBot - AzzurroDigitale}

\subsubsection{Pro}
\begin{itemize}
    \item Centralizzazione delle informazioni provenienti da diverse piattaforme.
    \item Utilizzo di tecnologie moderne e API di terze parti.
\end{itemize}

\subsubsection{Contro}
\begin{itemize}
    \item Complessità nell'integrazione di diverse piattaforme.
    \item Potenziali problemi di sicurezza e privacy dei dati.
\end{itemize}

\section{Scelta del Capitolato}

Dopo la presentazione dei capitolati in aula e una discussione interna, abbiamo concordato che 
il capitolato più interessante per il nostro team fosse il \textbf{C2 VimarGENIALE} proposto da \textbf{Vimar}.
Abbiamo comunque deciso di mantenere un approccio aperto e valutare anche altri capitolati proposti che
ci sembravano interessanti, per avere una visione più completa delle opportunità offerte dai vari progetti.
Dopo aver contattato le aziende, abbiamo organizzato degli incontri per approfondire i dettagli dei capitolati. 
Ne è emerso che il capitolato \textbf{C7 Assistente Virtuale} - \textbf{Ergon} è quello che più ci ha convinto.
Purtroppo non siamo riusciti a organizzare un incontro con l'azienda Vimar, quindi, a parità di interesse
tra i due capitolati, abbiamo deciso di scegliere il capitolato \textbf{C7}, poiché nella valutazione abbiamo 
tenuto conto anche della disponibilità dell'azienda a collaborare con noi.
Grazie a tutti gli incontri organizzati con le aziende ci siamo fatti un idea più precisa di quali possano
essere gli strumenti e i requisiti necessari per lo sviluppo del progetto, sopratutto per quanto riguarda
il tema degli LLM trattati da 3 dei 4 capitolati che abbiamo valutato più approfonditamente.
In conclusione il capitolato scelto è il \textbf{C7 Assistente Virtuale} - \textbf{Ergon}, 
concordato tramite votazione interna al team.


\end{document}



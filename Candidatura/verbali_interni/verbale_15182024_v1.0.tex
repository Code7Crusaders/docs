\documentclass{article}
\usepackage{graphicx}
\usepackage{fancyhdr}
\usepackage{geometry}
\usepackage{setspace}
\usepackage{tikz}
\usepackage[italian]{babel}

% Margini della pagina
\geometry{a4paper, margin=1in}

% Intestazione personalizzata
\pagestyle{fancy}
\fancyhf{}
\fancyhead[L]{Code7Crusaders - Software Development Team}
\fancyhead[R]{\thepage}

% Spaziatura delle righe
\setstretch{1.2}

\begin{document}

% Sezione del titolo
\begin{titlepage}

    \AddToHookNext{shipout/background}{
    \begin{tikzpicture}[remember picture,overlay]
    \node at (current page.center) {
    \includegraphics{../../img/background.png} 
    };
    \end{tikzpicture}
    }

    \centering
    \vspace*{2cm}
    
    \includegraphics[width=0.3\textwidth]{../../img/logo/7Crusaders_logo.png} % MODIFICATO PER OVERLEAF
    \vspace{1cm}
    
    {\Huge \textbf{Code7Crusaders}}\\
    \vspace{0.5cm}
    {\Large Software Development Team}\\
    \vspace{2cm}
    
    {\large \textbf{Documentazione Progetto}}\\
    \vspace{5cm}

    \textbf{Membri del Team:}\\
    Enrico Cotti Cottini, Gabriele Di Pietro, Tommaso Diviesti \\
    Francesco Lapenna, Matthew Pan, Eddy Pinarello, Filippo Rizzolo \\
    \vspace{0.5cm}
    
    {\large \textbf{Data: }15 Ottobre 2024}     
    \vspace{1cm}
\end{titlepage}

\newpage
\begin{table}[h!]
\centering
\textbf{Versioni} \\ % Titolo sopra la tabella
\vspace{2mm} % Spazio tra il titolo e la tabella
\begin{tabular}{|c|c|c|l|}
    \hline
    \textbf{Ver.} & \textbf{Data} & \textbf{Autore} & \textbf{Descrizione} \\
    \hline
    1.0 & 17/10/2024 & Enrico Cotti Cottini & Approvazione documento \\ 
    \hline
    0.2 & 16/10/2024 & Filippo Rizzolo & Controllo errori grammaticali e di sintassi \\ 
    \hline
    0.1 & 15/10/2024 & Eddy Pinarello & Prima stesura del documento \\ 
    \hline
\end{tabular}
\end{table}



% Indice
\newpage
\tableofcontents
\newpage

% Sezione Introduzione
\section{Registro Presenze}
\textbf{Piattaforma della riunione:} Piattaforma Discord \\
\textbf{Ora di Inizio} 15:00\\
\textbf{Ora di Fine} 16:00
\vspace{10mm} 

\begin{tabular}{|c|c|c|}
    \hline
    \textbf{Componente} & \textbf{Ruolo} & \textbf{Presenza}\\
    \hline
    Enrico Cotti Cottini & Amministratore & Presente \\ 
    \hline
    Gabriele Di Pietro & Redattore & Presente \\ 
    \hline
    Tommaso Divesti & Redattore & Presente \\ 
    \hline 
    Francesco Lapenna & Verificatore & Presente \\ 
    \hline
    Matthew Pan & Verificatore & Presente \\ 
    \hline 
    Eddy Pinarello & Redattore & Presente \\ 
    \hline
    Filippo Rizzolo & Responsabile & Presente \\ 
    \hline
\end{tabular}



\newpage
\section{Verbale}

\label{sec:verbale}
{\Large
Il nostro gruppo, Code7Crusaders, ha tenuto la prima riunione con una presentazione iniziale di ogni membro. Uno dei primi passi è stato trovare un nome che rappresentasse al meglio il team. Dopo diverse proposte e un confronto tra tutti i membri, abbiamo optato per "Code7Crusaders", un nome che rispecchia il nostro gruppo formato da 7 persone e che ci vede come il settimo gruppo partecipante al progetto di Ingegneria del Software di quest’anno accademico.\newline
Conclusa la scelta del nome, ci siamo dedicati alla creazione della nostra identità visiva e anche creazione della nostra casella postale. Il logo è stato cercato online utilizzando strumenti di AI da Eddy Pinarello e Tommaso Diviesti, tenendo conto delle preferenze condivise dal gruppo. Gabriele Di Pietro ha inoltre contribuito sviluppando il template LaTeX per le consegne formali.\newline
Durante la discussione riguardante le tracce fornite dalle aziende coinvolte nel progetto, abbiamo deciso di contattare direttamente quelle che ci interessavano di più. La traccia 2 proposta da Vimar è stata quella che ha attirato maggiormente la nostra attenzione. Filippo Rizzolo ha redatto e inviato le prime email alle aziende selezionate con l'indirizzo aziendale creato poco prima, per la richiesta dei primi colloqui con esse.\newline
Per facilitare la gestione collaborativa del progetto, Enrico Cotti Cottini ha creato un repository su GitHub, che useremo per condividere il documentazione e codice per organizzare le varie fasi del lavoro in modo ordinato e trasparente. \newline
Infine, ci siamo organizzati per l'utilizzo di LaTeX durante la preparazione della documentazione del primo verbale, sfruttando al massimo il template creato da Gabriele.}
\end{document}
\documentclass{article}
\usepackage{graphicx}
\usepackage{fancyhdr}
\usepackage{geometry}
\usepackage{setspace}
\usepackage{tikz}
\usepackage[italian]{babel}
\usepackage{tabularx}

% Margini della pagina
\geometry{a4paper, margin=1in}

% Intestazione personalizzata
\pagestyle{fancy}
\fancyhf{}
\fancyhead[L]{Code7Crusaders - Software Development Team}
\fancyhead[R]{\thepage}

% Spaziatura delle righe
\setstretch{1.2}

\begin{document}

% Sezione del titolo
\begin{titlepage}

    \AddToHookNext{shipout/background}{
    \begin{tikzpicture}[remember picture,overlay]
    \node at (current page.center) {
    \includegraphics{../../img/background.png}
    };
    \end{tikzpicture}
    }

    \centering
    \vspace*{2cm}
    
    \includegraphics[width=0.3\textwidth]{../../img/logo/7Crusaders_logo.png} % Aggiungi il logo qui
    \vspace{1cm}
    
    {\Huge \textbf{Code7Crusaders}}\\
    \vspace{0.5cm}
    {\Large Software Development Team}\\
    \vspace{2cm}
    
    {\large \textbf{Riunione Settimanale 25/10/2024}}\\
    \vspace{5cm}

    \textbf{Membri del Team:}\\
    Enrico Cotti Cottini, Gabriele Di Pietro, Tommaso Diviesti \\
    Francesco Lapenna, Matthew Pan, Eddy Pinarello, Filippo Rizzolo \\
    \vspace{0.5cm}
    
    {\large \textbf{Data:}} \today\\
    
    \vspace{1cm}
\end{titlepage}

% Indice
\newpage
\tableofcontents

%Versioni
\newpage
\begin{center}
    \textbf{Versioni}
    \\
    \\
    \begin{tabular}{|c|c|c|c|}
        \hline
        \textbf{Ver} & \textbf{Data} & \textbf{Autore} & \textbf{descrizione}\\
        \hline
        0.1 & 25/10/24 & Lapenna Francesco & Prima stesura del documento \\ 
        \hline
    \end{tabular}
\end{center}

% Registro Presenze
\newpage
\section{Registro Presenze}
\textbf{Piattaforma della riunione:} Discord \\
\textbf{Ora di Inizio} 10:30\\
\textbf{Ora di Fine} 12:30\\
\\
\begin{tabular}{|c|c|c|}
    \hline
    \textbf{Componente} & \textbf{Ruolo} & \textbf{Presenza}\\
    \hline
    Enrico Cotti Cottini & Responsabile& Presente \\ 
    \hline
    Gabriele Di Pietro & Redattore & Presente\\ 
    \hline
    Tommaso Divesti & Redattore & Presente \\ 
    \hline 
    Francesco Lapenna & Redattore& Presente \\ 
    \hline
    Matthew Pan & Verificatore & Presente\\ 
    \hline 
    Eddy Pinarello & Redattore & Presente \\ 
    \hline 
    Filippo Rizzolo & Amministratore& Presente \\ 
    \hline 
\end{tabular}

% Sezione Verbale
\newpage
% \section{Verbale dell'incontro}
% \label{sec:introduzione}
\section*{Ordine del Giorno}
\begin{enumerate}
    \item Preventivo del progetto
    \item Impegni orari previsti a persona
    \item Impegni orari previsti per ruolo
    \item Stima dei costi totali relativi agli impegni
    \item Rischi attesi
    \item Data indicativa di fine progetto
\end{enumerate}

\section{Preventivo del Progetto}
\textbf{Sintesi:} Viene realizzata una prima stesura del preventivo dei costi finali basato sulle specifiche del progetto. Si è considerato anche il carico di lavoro previsto per ogni ruolo e membro del team. \\
\textbf{Decisioni:} Il team conferma il preventivo come base di partenza. Eventuali variazioni saranno riviste con committente e proponente.

\section{Impegni Orari Previsti a Persona}
\textbf{Sintesi:} Si è discusso il carico di lavoro individuale per i membri del team, basato sulle ore settimanali disponibili.

\section{Impegni Orari Previsti per Ruolo}
\textbf{Sintesi:} Viene stabilito il carico orario per ciascun ruolo specifico del progetto, adattato in base alla necessità delle fasi di sviluppo e della tipologia di progetto. \\
\textbf{Decisioni:} Visto l'orientamento del progetto verso l'integrazione con AI piuttosto che verso lo sviluppo SW, si è deciso di distribuire le ore più verso i ruoli concernenti la progettazione piuttosto che lo sviluppo.\\Per informazioni dettaliate su ruoli e rispettivi costi orari consultare la relativa il documento di analisi dei costi.

\section{Stima dei Costi Totali Relativi agli Impegni}
\textbf{Sintesi:} Si è effettuata una stima dei costi basata sulle ore previste e sui costi orari. \\
\textbf{Totale previsto:} €12,575. 

\section{Rischi Attesi}

\textbf{Sintesi:} Vengono identificati i principali rischi e le misure di mitigazione.\\
\textbf{Decisioni:} Monitoraggio costante dei rischi e adattamento in corso d’opera.

\begin{table}[h]
\centering
\begin{tabularx}{\textwidth}{|X|c|c|X|}
    \hline
    \textbf{Rischio} & \textbf{Gravità} & \textbf{Probabilità}& \textbf{Contromisure}\\
    \hline
    Difficoltà nell'uso di nuove tecnologie & 3& 3 & Studio approfondito degli LLM \\ 
    \hline
    Codice non portato a termine dal delegato & 2 & 2 & Chiedere supporto all'azienda o usare più risorse del team\\ 
    \hline
    Dimunuzione carico e ore di lavoro durante le festività & 1 & 2 & Cercare di mantenere i ritmi feriali impostandosi un tempo minimo di lavoro settimanale\\ 
    \hline 
    Scarsa collaborazione & 1& 1 & Chiarire ruoli e compiti \\ 
    \hline
    Impegni personali & 1 & 2 & Comunicazione anche grazie al calendario per schedulare bene i compiti\\ 
    \hline 
\end{tabularx}
\caption{Analisi dei rischi}
\end{table}

\section{Data Indicativa di Fine Progetto}
\textbf{Data stimata:} 21 Marzo 2025. \\
\textbf{Decisioni:} Si conferma la data come obiettivo iniziale, con l’impegno a rivederla se i tempi di sviluppo dovessero subire variazioni.

\section*{Conclusioni}
\begin{itemize}
    \item Prossima riunione pianificata per il 1° Novembre 2024.
\end{itemize}

\end{document}
\documentclass{article}
\usepackage{graphicx}
\usepackage{fancyhdr}
\usepackage{geometry}
\usepackage{setspace}
\usepackage{tikz}
\usepackage[italian]{babel}

% Margini della pagina
\geometry{a4paper, margin=1in}

% Intestazione personalizzata
\pagestyle{fancy}
\fancyhf{}
\fancyhead[L]{Code7Crusaders - Software Development Team}
\fancyhead[R]{\thepage}

% Spaziatura delle righe
\setstretch{1.2}

\begin{document}

% Sezione del titolo
\begin{titlepage}

    \AddToHookNext{shipout/background}{
    \begin{tikzpicture}[remember picture,overlay]
    \node at (current page.center) {
    \includegraphics{../../img/background.png}
    };
    \end{tikzpicture}
    }

    \centering
    \vspace*{2cm}
    
    \includegraphics[width=0.3\textwidth]{../../img/logo/7Crusaders_logo.png} % Aggiungi il logo qui
    \vspace{1cm}
    
    {\Huge \textbf{Code7Crusaders}}\\
    \vspace{0.5cm}
    {\Large Software Development Team}\\
    \vspace{2cm}
    
    {\large \textbf{Documentazione Progetto}}\\
    \vspace{5cm}

    \textbf{Membri del Team:}\\
    Enrico Cotti Cottini, Gabriele Di Pietro, Tommaso Diviesti \\
    Francesco Lapenna, Matthew Pan, Eddy Pinarello, Filippo Rizzolo \\
    \vspace{0.5cm}
    
    {\large \textbf{Data:}} \today\\
    
    \vspace{1cm}
\end{titlepage}

\newpage
\begin{table}[h!]
\centering
\textbf{Versioni} \\ % Titolo sopra la tabella
\vspace{2mm} % Spazio tra il titolo e la tabella
\begin{tabular}{|c|c|c|l|}
    \hline
    \textbf{Ver.} & \textbf{Data} & \textbf{Autore} & \textbf{Descrizione} \\
    \hline
    1.0 & 20/10/2024 & Enrico Cotti Cottini & Approvazione documento \\ 
    \hline
    0.2 & 19/10/2024 & Filippo Rizzolo & Controllo errori grammaticali e di sintassi \\ 
    \hline
    0.1 & 18/10/2024 & Eddy Pinarello & Prima stesura del documento \\ 
    \hline
\end{tabular}
\end{table}



% Indice
\newpage
\tableofcontents
\newpage

% Sezione Introduzione
\section{Registro Presenze}
\textbf{Piattaforma della riunione:} Piattaforma Discord \\
\textbf{Ora di Inizio} 15:00\\
\textbf{Ora di Fine} 16:00
\vspace{10mm} 

\begin{tabular}{|c|c|c|}
    \hline
    \textbf{Componente} & \textbf{Ruolo} & \textbf{Presenza}\\
    \hline
    Enrico Cotti Cottini & Amministratore & Presente \\ 
    \hline
    Gabriele Di Pietro & Redattore & Presente \\ 
    \hline
    Tommaso Divesti & Redattore & Presente \\ 
    \hline %linea di fine
    Francesco Lapenna & Verificatore & Presente \\ 
    \hline
    Matthew Pan & Verificatore & Presente \\ 
    \hline %linea di fine
    Eddy Pinarello & Redattore & Presente \\ 
    \hline %linea di fine
    Filippo Rizzolo & Responsabile & Presente \\ 
    \hline %linea di fine
\end{tabular}



\newpage
\section{Verbale}

\label{sec:verbale}
\subsection{Discussioni colloqui settimanali}
{\large
Dopo aver completato i tre colloqui con le aziende che avevamo selezionato, ci siamo ritrovati tutti insieme per discutere e confrontare le nostre impressioni. L'obiettivo di questi colloqui non era solo quello di fare una prima conoscenza con le aziende, ma soprattutto di comprendere meglio le loro aspettative e cosa realmente cercassero da noi come gruppo. \newline
Nel nostro incontro, abbiamo esaminato in dettaglio ciò che ogni azienda si aspetta dal progetto e come questo possa influire sul nostro lavoro. Abbiamo riflettuto anche sulle differenze tra le proposte, cercando di capire quale delle aziende ci offrisse l'opportunità migliore per metterci alla prova e per crescere come team.\newline
Tuttavia, nonostante i colloqui siano stati illuminanti e ci abbiano dato un quadro più chiaro delle esigenze di ciascuna azienda, non abbiamo ancora ricevuto feedback dalla nostra preferita. Questo ci ha impedito di prendere una decisione definitiva. Vogliamo essere certi di aver esplorato ogni opzione e di aver compreso appieno le richieste prima di fare il nostro passo finale.\newline
In sintesi, pur avendo acquisito molte informazioni utili, non siamo ancora pronti a scegliere, perché desideriamo continuare a valutare le proposte in modo approfondito, considerando al meglio tutte le variabili in gioco. }
\subsection{Creazione directory per documentazione GitHub}
{\large
Abbiamo creato una struttura chiara su GitHub per organizzare la documentazione del progetto, suddividendo i file in cartelle specifiche per verbali, documentazione tecnica, e altre risorse. Questo sistema ci permette di mantenere tutto ordinato e facilmente accessibile, facilitando la collaborazione tra i membri del gruppo. }
\subsection{Invio di una seconda email di sollecitazione per colloquio}
{\large
Dopo aver notato che non avevamo ricevuto risposta alla nostra prima email per il colloquio con la nostra azienda preferita, ci siamo riuniti per discutere insieme sulla situazione. Durante l'incontro, abbiamo valutato le possibili ragioni per cui non ci fosse stata alcuna risposta e abbiamo deciso di inviare una seconda email di sollecitazione. }
\end{document}

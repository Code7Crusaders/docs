
\documentclass{article}
\usepackage{graphicx}
\usepackage{fancyhdr}
\usepackage{geometry}
\usepackage{setspace}
\usepackage{tikz}
\usepackage[italian]{babel}
\usepackage[hidelinks]{hyperref}

% Margini della pagina
\geometry{a4paper, margin=1in}

% Intestazione personalizzata
\pagestyle{fancy}
\fancyhf{}
\fancyhead[L]{Code7Crusaders - Software Development Team}
\fancyhead[R]{\thepage}

% Spaziatura delle righe
\setstretch{1.2}

\begin{document}

% Sezione del titolo
\begin{titlepage}

    \AddToHookNext{shipout/background}{
    \begin{tikzpicture}[remember picture,overlay]
    \node at (current page.center) {
    \includegraphics{../img/background.png}
    };
    \end{tikzpicture}
    }

    \centering
    \vspace*{2cm}
    
    \includegraphics[width=0.3\textwidth]{../img/logo/7Crusaders_logo.png} % Aggiungi il logo qui
    \vspace{1cm}
    
    {\Huge \textbf{Code7Crusaders}}\\
    \vspace{0.5cm}
    {\Large Software Development Team}\\
    \vspace{2cm}
    
    \large \textbf{Analisi costi e assunzione impegni}
    \vspace{5cm}

    \textbf{Membri del Team:}\\
    Enrico Cotti Cottini, Gabriele Di Pietro, Tommaso Diviesti \\
    Francesco Lapenna, Matthew Pan, Eddy Pinllo, Filippo Rizzolo \\
    \vspace{0.5cm}
    
    \vspace{1cm}
\end{titlepage}

\newpage
%Versioni
\begin{center}
    \newline
    \textbf{Versioni}
    \\
    \\
    \begin{tabular}{|c|c|c|c|}
        \hline
        \textbf{Ver} & \textbf{Data} & \textbf{Autore} & \textbf{descrizione}\\
        \hline
        0.2 & 26/10/24 & Gabriele Di Pietro & Revisione del documento \\
        0.1 & 24/10/24 & Tommaso Diviesti & Prima stesura del documento \\
        \hline
    \end{tabular}
\end{center}

% Indice
\newpage
\tableofcontents
\newpage

% Sezione Introduzione
\section{Descrizione}
Il seguente documento rappresenta una dichiarazione degli impegni di lavoro che ogni componente del Team assume in modo da 
riuscire a completare al meglio il progetto entro la data di scadenza da noi stabilita. Viene riportato il quantitativo in 
ore di lavoro individuale e collettivo, in base al ruolo assunto, necessario per lo svolgimento del progetto. Viene in seguito 
descritto ciascun ruolo presente per il corretto svolgimento delle attività attraverso una serie di compiti delegati e obbiettivi. 
Viene, infine, preventivato il costo finale e, inoltre, stabilita la scadenza di consegna.
\newpage

\section{Impegni orari}
\label{sec:introduzione}
Tutti i componenti del Team Code7Crusaders si impegnano a dedicare un totale di 91 ore di lavoro effettivo partizionate 
settimanalmente in base al ruolo di riferimento. Inoltre, ciascun membro garantisce la conclusione del progetto entro la 
data prevista e preventivata nel paragrafo 5 di questo documento.
\\
\\
Di seguito, si riporta il costo orario in base al ruolo assunto:
\begin{table}[!h]
	\begin{center}
		\begin{tabular}{ |c|c|c|c| }
			\hline
			Ruolo          & Costo orario &  per ruolo & Ore per membro \\
			\hline
			Responsabile   & 30           &     55       &       8        \\
			Amministratore & 20           &     55       &       8        \\
			Analista       & 25           &     75       &       11       \\
			Progettista    & 25           &     120      &       17       \\
			Programmatore  & 15           &     163      &       23       \\
			Verificatore   & 15           &     167      &       24       \\
			\hline
			Totale         &    12575    &     635       &       91       \\
			\hline
		\end{tabular}
	\end{center}
\end{table}
\\
Ripartizione delle ore per membro del team:
\begin{table}[!h]
	\begin{center}
		\begin{tabular}{ |c|c|c|c|c|c|c|c| }
			\hline
			\textbf{Membro}    & \textbf{Re} & \textbf{Am} & \textbf{An} & \textbf{Pj} & \textbf{Pg} & \textbf{Ve} & \textbf{Totale} \\
			\hline
			Enrico Cotti Cottini     & 8           & 8           & 11          & 17          & 23          & 24          & 91              \\
			Gabriele Di Pietro       & 8           & 8           & 11          & 17          & 23          & 24          & 91              \\
			Tommaso Diviesti         & 8           & 8           & 11          & 17          & 23          & 24          & 91              \\
			Francesco Lapenna        & 8           & 8           & 11          & 17          & 23          & 24          & 91              \\
			Matthew Pan              & 8           & 8           & 11          & 17          & 23          & 24          & 91              \\
			Eddy Pinarello           & 8           & 8           & 11          & 17          & 23          & 24          & 91              \\
			Filippo Rizzolo          & 8           & 8           & 11          & 17          & 23          & 24          & 91              \\
			\hline
		\end{tabular}
	\end{center}
\end{table}
\\
\textsc{Leggenda:} \\
    \textbf{Re} = Responsabile \\
    \textbf{Am} = Amministratore \\
    \textbf{An} = Analista \\
    \textbf{Pj} = Progettista \\
    \textbf{Pg} = Programmatore \\
    \textbf{Ve} = Verificatore \\
    
\newpage
\section{Suddivisione dei ruoli}
I ruoli in seguito descritti sono equamente divisi tra i vari componenti del Team. Ogni ruolo possiede diversi incarichi e obbiettivi:
\subsection{Responsabile}

\subsection{Amministratore}

\subsection{Analista}

\subsection{Progettista}

\subsection{Programmatore}

\subsection{Verificatore}

\newpage
\section{Analisi dei rischi}
In questa sezione vengono elencati i rischi che potrebbero verificarsi durante lo svolgimento del progetto e le relative contromisure. Ad ogni rischio è associato un \textbf{indice di Gravità e Probabilità},
in modo da poter valutare la criticità di ciascuno di essi.
\subsection{Definizione degli indici}
I valori dell'\textbf{Indice di Gravità} e dell'\textbf{Indice di Probabilità} sono definiti come segue:
\begin{table}{|c|c|c|c|}
    \hline
    \textbf{Leggenda} & \textbf{Tipo} & \textbf{Gravità} & \textbf{Probabilità} \\
    \hline
    \textbf{1} & \textbf{Basso} & ha un impatto minimo o trascurabile sul progetto, come un lieve rallentamento che non incide sui tempi di consegna & Improbabile che si verifichi ma esistono fattori che potrebbero contribuire alla sua realizzazione \\
    \hline
    \textbf{2} & \textbf{Medio} & Il rischio, se si concretizza, richiede risorse aggiuntive o modifica parzialmente il piano di progetto, causando impatti gestibili ma che comportano sforzi supplementari & si potrebbe verificare & C'è una possibilità realistica che l’evento di rischio si verifichi \\
    \hline
    \textbf{3} & \textbf{Alto} & L’evento causa ritardi significativi, aumento dei costi o degrado della qualità che incide sull’esperienza utente, richiedendo interventi importanti per mantenere il progetto nei tempi e budget & Esistono molti fattori o segni che indicano che il rischio potrebbe accadere, e il team considera probabile la sua manifestazione \\
    \hline
    \caption{Definizione degli Indici}
\end{table}
\subsection{Rischi}
\begin{table}[h]
    \centering
    \begin{tabularx}{\textwidth}{|X|c|c|X|}
        \hline
        \textbf{ID} & \textbf{Rischio} & \textbf{Gravità} & \textbf{Probabilità}\\
        \hline
        1 & Difficoltà nell'uso di nuove tecnologie & 2& 3\\ 
        \hline
        2 & Codice non portato a termine dal delegato & 2 & 2\\ 
        \hline
        3 & Dimunuzione carico e ore di lavoro durante le festività & 1 & 2\\ 
        \hline 
        4 & Scarsa collaborazione da parte di uno o più membri & 3& 1\\ 
        \hline
        5 & Impegni personali e universitari & 1 & 2\\ 
        \hline 
        6 & Deviazione rispetto ai tempi e costi previsti & 3 & 1\\
        \hline
    \end{tabularx}
    \caption{Analisi dei rischi}
\end{table}
\subsubsection{Contromisura rischio 1}
Il gruppo si impegnerà a studiare in modo approfondite le tecnologie richieste dal capitolato in particolar modo lo studio dei \textbf{LLM}. E verranno organizzati incontri di formazione interna in modo tale da poter condividere le conoscenze acquisite per essere tutti sullo stesso livello.
\subsubsection{Contromisura rischio 2}
Il gruppo si impegnerà a chiedere supporto all'azienda e si cercherà di massimizzare le risorse nel team nella soluzione di un problema.
\subsubsection{Contromisura rischio 3}
Il gruppo cercherà di mantenere i ritmi feriali impostando un tempo minimo di lavoro settimanale.
\subsubsection{Contromisura rischio 4}
Il gruppo si impegna nella comprensione e nel chiarire quali siano i ruoli, inoltre una comunicazione costante e trasparente aiuterà sull'affrontare le diverse difficoltà e nel segnalare tempestivamente eventuali problemi
\subsubsection{Contromisura rischio 5}
Progettazione di un calendario condiviso dove ogni componente può segnalare i propri impegni personali con anticipo. Di conseguenza pianificare bene le varie attività evitando sovrapposizioni
\subsubsection{Contromisura rischio 6}
Monitorare il progresso delle attività e svolgere frequenti riunioni per valutare lo stato di avanzamento del progetto. 

\newpage
\section{Preventivo dei costi}
In base alle valutazioni effettuate nel paragrafo 2, il costo finale del progetto corrisponde a \textbf{12575}€.

\section{Scadenza di consegna}
Il Team ritiene di consegnare il materiale completo relativo al progetto a noi assegnato entro il giorno \textbf{21/03/2025}.


\end{document}


\documentclass{article}
\usepackage{graphicx}
\usepackage{fancyhdr}
\usepackage{geometry}
\usepackage{setspace}
\usepackage{tikz}
\usepackage{eurosym}
\usepackage[italian]{babel}
\usepackage[hidelinks]{hyperref}

% Margini della pagina
\geometry{a4paper, margin=1in}

% Intestazione personalizzata
\pagestyle{fancy}
\fancyhf{}
\fancyhead[L]{Code7Crusaders - Software Development Team}
\fancyhead[R]{\thepage}

% Spaziatura delle righe
\setstretch{1.2}

\begin{document}

% Sezione del titolo
\begin{titlepage}

    \AddToHookNext{shipout/background}{
    \begin{tikzpicture}[remember picture,overlay]
    \node at (current page.center) {
    \includegraphics{../img/background.png}
    };
    \end{tikzpicture}
    }

    \centering
    \vspace*{2cm}
    
    \includegraphics[width=0.3\textwidth]{../img/logo/7Crusaders_logo.png} % Aggiungi il logo qui
    \vspace{1cm}
    
    {\Huge \textbf{Code7Crusaders}}\\
    \vspace{0.5cm}
    {\Large Software Development Team}\\
    \vspace{2cm}
    
    \large \textbf{Analisi costi e assunzione impegni}
    \vspace{5cm}

    \textbf{Membri del Team:}\\
    Enrico Cotti Cottini, Gabriele Di Pietro, Tommaso Diviesti \\
    Francesco Lapenna, Matthew Pan, Eddy Pinllo, Filippo Rizzolo \\
    \vspace{0.5cm}
    
    \vspace{1cm}
\end{titlepage}

\newpage
%Versioni
\begin{center}
    \newline
    \textbf{Versioni}
    \\
    \\
    \begin{tabular}{|c|c|c|c|}
        \hline
        \textbf{Ver} & \textbf{Data} & \textbf{Autore} & \textbf{descrizione}\\
        0.3 & 29/10/24 & Enrico Cotti Cottini & Rivista analisi costo e considerazioni + Pianificazione Scadenze  \\
        \hline
        0.2 & 26/10/24 & Gabriele Di Pietro & Revisione del documento \\
        \hline
        0.1 & 24/10/24 & Tommaso Diviesti & Prima stesura del documento \\
        \hline
    \end{tabular}
\end{center}

% Indice
\newpage
\tableofcontents
\newpage

% Sezione Introduzione
\section{Descrizione}
Il seguente documento rappresenta una dichiarazione degli impegni di lavoro che ogni componente del Team assume in modo da 
riuscire a completare al meglio il progetto entro la data di scadenza da noi stabilita. Viene riportato il quantitativo in 
ore di lavoro individuale e collettivo, in base al ruolo assunto, necessario per lo svolgimento del progetto. Viene in seguito 
descritto ciascun ruolo presente per il corretto svolgimento delle attività attraverso una serie di compiti delegati e obbiettivi. 
Viene, infine, preventivato il costo finale e, inoltre, stabilita la scadenza di consegna.
\newpage

\section{Impegni orari}
\label{sec:introduzione}
Tutti i componenti del Team Code7Crusaders si impegnano a dedicare un totale di \textbf{95 ore} di lavoro effettivo partizionate 
settimanalmente in base al ruolo di riferimento, allo svolgimento del capitolato \textbf{C7} di \textbf{Ergon Informatica}. Inoltre,
ciascun membro garantisce la conclusione del progetto entro la data prevista e preventivata nel paragrafo 5 di questo documento.
\\
\\
Di seguito, si riporta il costo orario in base al ruolo assunto:
\begin{table}[!h]
	\begin{center}
		\begin{tabular}{ |c|c|c|c| }
			\hline
			Ruolo          & Costo orario &  per ruolo   & Ore per membro \\
			\hline          
			Responsabile   & 30           &     54       &       8        \\
			Amministratore & 20           &     64       &       9        \\
			Analista       & 25           &     65       &       9       \\
			Progettista    & 25           &     105      &       15       \\
			Programmatore  & 15           &     184      &       26       \\
			Verificatore   & 15           &     193      &       28       \\
			\hline
			Totale         &    12805    &     665       &       95       \\
			\hline
		\end{tabular}
        \caption{Costo orario e totale}
    \end{center}
\end{table}

\newpage




\\
Ripartizione delle ore per membro del team:
\begin{table}[!h]
	\begin{center}
		\begin{tabular}{ |c|c|c|c|c|c|c|c| }
			\hline
			\textbf{Membro}    & \textbf{Re} & \textbf{Am} & \textbf{An} & \textbf{Pj} & \textbf{Pg} & \textbf{Ve} & \textbf{Totale} \\
			\hline
			Enrico Cotti Cottini     & 8           & 9           & 9          & 15          & 26          & 28          & 95              \\
			Gabriele Di Pietro       & 8           & 9           & 9          & 15          & 26          & 28          & 95              \\
			Tommaso Diviesti         & 8           & 9           & 9          & 15          & 26          & 28          & 95              \\
			Francesco Lapenna        & 8           & 9           & 9          & 15          & 26          & 28          & 95              \\
			Matthew Pan              & 8           & 9           & 9          & 15          & 26          & 28          & 95              \\
			Eddy Pinarello           & 8           & 9           & 9          & 15          & 26          & 28          & 95              \\
			Filippo Rizzolo          & 8           & 9           & 9          & 15          & 26          & 28          & 95              \\
			\hline
		\end{tabular}
        \caption{Impegni orari a persona} 
    \end{center}
\end{table}
\\
\textsc{Leggenda:} \\
    \textbf{Re} = Responsabile \\
    \textbf{Am} = Amministratore \\
    \textbf{An} = Analista \\
    \textbf{Pj} = Progettista \\
    \textbf{Pg} = Programmatore \\
    \textbf{Ve} = Verificatore \\
    
\newpage

\section{Suddivisione dei ruoli e considerazioni su essi}
I ruoli in seguito descritti sono equamente divisi tra i vari componenti del Team. Ogni ruolo possiede diversi incarichi e obbiettivi:

\subsection{Responsabile}
Il \textbf{Responsabile} coordina il gruppo di lavoro, controlla le attività e gestisce le risorse. Si occupa di garantire che il progetto venga portato a termine nei tempi stabiliti e con le risorse disponibili.

\subsection{Amministratore}
L'\textbf{Amministratore} si occupa della gestione delle risorse e delle infrastrutture, incluso il setup degli strumenti di supporto alla produzione del software. Garantisce inoltre l’uso corretto delle procedure per assicurare efficienza e produttività.

\subsection{Analista}
L'\textbf{Analista} gioca un ruolo fondamentale nella fase iniziale del progetto. È responsabile della definizione dei requisiti e dell’analisi delle funzionalità del software, delineando i casi d'uso. Essendo necessario principalmente all'inizio del progetto, il numero di ore assegnato al ruolo è relativamente ridotto.

\subsection{Progettista}
Il \textbf{Progettista} definisce l'architettura del software, descrivendo le componenti e le loro interazioni sulla base dei requisiti stabiliti dall'Analista. Questo ruolo ha un numero di ore significativamente elevato perché è essenziale per garantire una struttura solida, soprattutto considerando l’implementazione di modelli LLM, che richiedono un'architettura ben progettata e adattata a tali tecnologie.

\subsection{Programmatore}
Il \textbf{Programmatore} si occupa di scrivere il codice del software seguendo le specifiche del progettista. Il numero di ore assegnato è alto, dato che rappresenta il cuore della fase di sviluppo. Tuttavia, il ruolo ha leggermente meno ore rispetto al Verificatore, poiché abbiamo scelto di adottare una metodologia incentrata sui test, che richiede un’accurata verifica del software.

\subsection{Verificatore}
Il \textbf{Verificatore} verifica che il software e la documentazione siano conformi alle norme e alle specifiche. Questo ruolo richiede un numero di ore superiore alla media, data la necessità di test approfonditi e continui, in particolare per un progetto basato su LLM, dove ogni componente deve essere rigorosamente validato per garantire la precisione e l’affidabilità del sistema.


\newpage
\section{Analisi dei rischi}
In questa sezione vengono elencati i rischi che potrebbero verificarsi durante lo svolgimento del progetto e le relative contromisure. Ad ogni rischio è associato un \textbf{indice di Gravità e Probabilità},
in modo da poter valutare la criticità di ciascuno di essi.
\subsection{Definizione degli indici}
I valori dell'\textbf{Indice di Gravità} e dell'\textbf{Indice di Probabilità} sono definiti come segue:
\begin{table}[h!]
    \centering
    \begin{tabular}{|c|c|p{6cm}|p{6cm}|}
        \hline
        \textbf{Indice} & \textbf{Tipo} & \textbf{Gravità} & \textbf{Probabilità} \\
        \hline
        \textbf{1} & Basso & Ha un impatto minimo o trascurabile sul progetto, come un lieve rallentamento che non incide sui tempi di consegna & Improbabile che si verifichi, ma esistono fattori che potrebbero contribuire alla sua realizzazione \\
        \hline
        \textbf{2} & Medio & Se si concretizza, richiede risorse aggiuntive o modifica parzialmente il piano di progetto, causando impatti gestibili ma che comportano sforzi supplementari & C'è una possibilità realistica che l'evento di rischio si verifichi \\
        \hline
        \textbf{3} & Alto & Causa ritardi significativi, aumento dei costi o degrado della qualità che incide sull’esperienza utente, richiedendo interventi importanti per mantenere il progetto nei tempi e nel budget & Esistono molti fattori o segni che indicano che il rischio potrebbe accadere, e il team considera probabile la sua manifestazione \\
        \hline
    \end{tabular}
    \caption{Definizione degli Indici di Gravità e Probabilità}
    \label{tab:definizione_indici}
\end{table}
    
\subsection{Rischi}
\begin{table}[h]
    \centering
    \begin{tabular}{|p{0.5cm}|p{7cm}|p{2cm}|p{2cm}|}
        \hline
        \textbf{ID} & \textbf{Rischio} & \textbf{Gravità} & \textbf{Probabilità} \\
        \hline
        1 & Difficoltà nell'uso di nuove tecnologie & 2 & 3 \\ 
        \hline
        2 & Codice non completato dal delegato & 2 & 2 \\ 
        \hline
        3 & Riduzione del carico e delle ore di lavoro durante le festività & 1 & 2 \\ 
        \hline 
        4 & Scarsa collaborazione da parte di uno o più membri del team & 3 & 1 \\ 
        \hline
        5 & Impegni personali e universitari & 1 & 2 \\ 
        \hline 
        6 & Deviazione dai tempi e costi previsti & 3 & 1 \\
        \hline
    \end{tabular}
    \caption{Analisi dei rischi}
    \label{tab:analisi_rischi}
\end{table}

\subsubsection{Contromisura rischio 1}
Il gruppo si impegnerà a studiare in modo approfondite le tecnologie richieste dal capitolato in particolar modo lo studio dei \textbf{LLM}. E verranno organizzati incontri di formazione interna in modo tale da poter condividere le conoscenze acquisite per essere tutti sullo stesso livello.
\subsubsection{Contromisura rischio 2}
Il gruppo si impegnerà a chiedere supporto all'azienda e si cercherà di massimizzare le risorse nel team nella soluzione di un problema.
\subsubsection{Contromisura rischio 3}
Il gruppo cercherà di mantenere i ritmi feriali impostando un tempo minimo di lavoro settimanale.
\subsubsection{Contromisura rischio 4}
Il gruppo si impegna nella comprensione e nel chiarire quali siano i ruoli, inoltre una comunicazione costante e trasparente aiuterà sull'affrontare le diverse difficoltà e nel segnalare tempestivamente eventuali problemi
\subsubsection{Contromisura rischio 5}
Progettazione di un calendario condiviso dove ogni componente può segnalare i propri impegni personali con anticipo. Di conseguenza pianificare bene le varie attività evitando sovrapposizioni
\subsubsection{Contromisura rischio 6}
Monitorare il progresso delle attività e svolgere frequenti riunioni per valutare lo stato di avanzamento del progetto. 

\newpage
\section{Preventivo dei costi}
In base alle valutazioni effettuate nel paragrafo 2, il costo finale del progetto corrisponde a \textbf{12805}\euro .

\section{Pianificazione Scadenze}
Il gruppo Code7Crusaders si impegna a consegnare il progetto entro il \textbf{14/03/2025}. La pianificazione prevede 19 settimane di lavoro, suddivise come segue:

\begin{itemize}
    \item \textbf{PoC (Proof of Concept): 6 settimane} \\
    Questa fase serve a testare rapidamente le tecnologie principali per verificare che siano adatte al progetto. Si sperimenteranno funzionalità essenziali per ridurre i rischi tecnici, specialmente per l'integrazione di modelli di Linguaggio di Modello (LLM).

    \item \textbf{MVP (Minimum Viable Product): 13 settimane} \\
    In questa fase, si svilupperà il prodotto base, con tutte le funzionalità principali pronte per l’uso. Verranno implementati anche i test necessari per garantire stabilità e usabilità, con un focus sui test per i modelli LLM.
\end{itemize}

Questa suddivisione permette di affrontare i rischi iniziali e di completare il progetto in modo graduale e sicuro.

\end{document}
 
\documentclass{article}
\usepackage{graphicx}
\usepackage{fancyhdr}
\usepackage{geometry}
\usepackage{setspace}
\usepackage{tikz}
\usepackage[italian]{babel}

% Margini della pagina
\geometry{a4paper, margin=1in}

% Intestazione personalizzata
\pagestyle{fancy}
\fancyhf{}
\fancyhead[L]{Code7Crusaders - Software Development Team}
\fancyhead[R]{\thepage}

% Spaziatura delle righe
\setstretch{1.2}

\begin{document}

% Sezione del titolo
\begin{titlepage}

    \AddToHookNext{shipout/background}{
    \begin{tikzpicture}[remember picture,overlay]
    \node at (current page.center) {
    \includegraphics{../../img/background.png}
    };
    \end{tikzpicture}
    }

    \centering
    \vspace*{2cm}
    
    \includegraphics[width=0.3\textwidth]{../../img/logo/7Crusaders_logo.png} % Aggiungi il logo qui
    \vspace{1cm}
    
    {\Huge \textbf{Code7Crusaders}}\\
    \vspace{0.5cm}
    {\Large Software Development Team}\\
    \vspace{2cm}
    
    {\large \textbf{Documentazione Progetto}}\\
    \vspace{5cm}

    \textbf{Membri del Team:}\\
    Enrico Cotti Cottini, Gabriele Di Pietro, Tommaso Diviesti \\
    Francesco Lapenna, Matthew Pan, Eddy Pinarello, Filippo Rizzolo \\
    \vspace{0.5cm}
    
    {\large \textbf{Data:}} \today\\
    
    \vspace{1cm}
\end{titlepage}

% Indice
\newpage
%Versioni
\begin{center}
    \textbf{Versioni}
    \\
    \\
    \begin{tabular}{|c|c|c|c|}
        \hline
        \textbf{Ver} & \textbf{Data} & \textbf{Autore} & \textbf{descrizione}\\
        \hline
        1.0 & 17/10/24 & XXXX & Prima stesura del documento \\
        \hline
    \end{tabular}
\end{center}
% Sezione Introduzione
\newpage
\section{Registro Presenze}
\textbf{Piattaforma della riunione:} XXXX \\
\textbf{Ora di Inizio} 00:00\\
\textbf{Ora di Fine} 00:00\\
\\
\begin{tabular}{|c|c|c|}
    \hline
    \textbf{Componente} & \textbf{Ruolo} & \textbf{Presenza}\\
    \hline
    XXXX & Redattore & Presente \\ %%continuare così
    \hline %linea di fine
\end{tabular}
\\
\begin{tabular}{|c|c|}
    \hline
    \textbf{Nome} & \textbf{Ruolo}\\
    \hline
    XXXX & XXXX\\
    \hline
\end{tabular}
\newpage
\section{Verbale dell'incontro}
\subsection{Complessità Progetto}
\textbf{Domanda:}Quali sono gli elementi di complessità maggiore, quindi le parti più difficili da implementare nel progetto?\\
il backend che gestisce lutilizzo di open ai, la ricezione di domanda e risosta e la ricezione dei dati dalle
varie piattaforme, anche la gestione di rug database vettoriale, postgesql usare come database
\subsection{Piattaforma da usare per la comunicazione}
\textbf{Domanda:}Quale strumento che usa l’azienda per eventuali colloqui e chiarimenti specifici, o se ci si può incontrare dal vivo?\\
usiamo meet per incontri, si informano a livello burocratico e si capisce, standard meet
\subsection{Framework utilizzabili}
\textbf{Domanda:}Quali framework possiamo usare per gestire il progetto?\\
angular, è obligatorio? per ottenere il risultato se siamo piu comodo con altro allora possiamo fare quello
\subsection{Linee guida per API}
\textbf{Domanda:}Ci sono linee guida per le applicazioni di API di terze parti?\\
api le varie piattaforme hanno api che hanno documentazione e farsi un idea per come interagire con esse?
se va male identificate quali sono gli api che danno info  get page conference, valutate di fare prima una
ricerca su se esistono servizi gia fatti e se hanno gia connettori cosi da non dovere interagire direttamente con 
gli strumenti, con gia lintegrazione.
\subsection{Modelli LLM utilizzabili}
\textbf{Domanda:}Possiamo usare modelli llm opensource?\\
si potrebbero usare ache modelli opensource se non riusciamo a fare qualcosa con chatgpt, 
trainando in locale ecc.veloce nelle risposte, non puo rispondere dopo dieci minuti, 
\subsection{alternative a LangChain e OpenAI}


\end{document}



Azzurro digitale

1 Complessita progetto? il backend che gestisce lutilizzo di open ai, la ricezione di domanda e risosta e la ricezione dei dati dalle
varie piattaforme, anche la gestione di rug database vettoriale, postgesql usare come database

2 Piattaforme per comunicare? usiamo meet per incontri, si informano a livello burocratico e si capisce, standard meet

3 Framework? angular, è obligatorio? per ottenere il risultato se siamo piu comodo con altro allora possiamo fare quello

4 api di terze parti, backend llm con piattaforme loro? 
api le varie piattaforme hanno api che hanno documentazione e farsi un idea per come interagire con esse?
se va male identificate quali sono gli api che danno info  get page conference, valutate di fare prima una
ricerca su se esistono servizi gia fatti e se hanno gia connettori cosi da non dovere interagire direttamente con 
gli strumenti, con gia lintegrazione.

4 Possiamo usare modelli llm opensource? si potrebbero usare ache modelli opensource se non riusciamo a fare qualcosa con chatgpt, 
trainando in locale ecc.veloce nelle risposte, non puo rispondere dopo dieci minuti, 

5 Ci sono alternative tra LangChain? LangChain framework da usare per situazioni come questa, limportante è
il risultato come lo si ottiene facciamo noi 6 difficota nelluso delle cose core

6 Consiglio su cosa concentrarsi nel progetto.
concentratevi sulla parte di studio sulla roba dei llm perche è la roba meno solida, concetratevi sulla cicca
il resto si fa easy

7 parte di testing e bug reporting? sezione di issues di github, parte di unit test è chiesta dal prof,
il codice che scriviamo deve essere coperto, possiamo usare o TDD oppure no pero bisogna arrivare al risultato.
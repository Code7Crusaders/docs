\documentclass{article}
\usepackage{graphicx}
\usepackage{fancyhdr}
\usepackage{geometry}
\usepackage{setspace}
\usepackage{tikz}
\usepackage[italian]{babel}

% Margini della pagina
\geometry{a4paper, margin=1in}

% Intestazione personalizzata
\pagestyle{fancy}
\fancyhf{}
\fancyhead[L]{Code7Crusaders - Software Development Team}
\fancyhead[R]{\thepage}

% Spaziatura delle righe
\setstretch{1.2}

\begin{document}

% Sezione del titolo
\begin{titlepage}

    \AddToHookNext{shipout/background}{
    \begin{tikzpicture}[remember picture,overlay]
    \node at (current page.center) {
    \includegraphics{../../img/background.png}
    };
    \end{tikzpicture}
    }

    \centering
    \vspace*{2cm}
    
    \includegraphics[width=0.3\textwidth]{../../img/logo/7Crusaders_logo.png} % Aggiungi il logo qui
    \vspace{1cm}
    
    {\Huge \textbf{Code7Crusaders}}\\
    \vspace{0.5cm}
    {\Large Software Development Team}\\
    \vspace{2cm}
    
    {\large \textbf{Documentazione Progetto}}\\
    \vspace{5cm}

    \textbf{Membri del Team:}\\
    Enrico Cotti Cottini, Gabriele Di Pietro, Tommaso Diviesti \\
    Francesco Lapenna, Matthew Pan, Eddy Pinarello, Filippo Rizzolo \\
    \vspace{0.5cm}
    
    {\large \textbf{Data:}} \today\\
    
    \vspace{1cm}
\end{titlepage}

% Indice
\newpage
%Versioni
\begin{center}
    \textbf{Versioni}
    \\
    \\
    \begin{tabular}{|c|c|c|c|}
        \hline
        \textbf{Ver} & \textbf{Data} & \textbf{Autore} & \textbf{descrizione}\\
        \hline
        0.1 & 18/10/24 & Gabriele Di Pietro & Prima stesura del documento \\
        \hline
    \end{tabular}
\end{center}
% Sezione Introduzione
\newpage
\section{Registro Presenze}
\textbf{Piattaforma della riunione:} Google Meet \\
\textbf{Ora di Inizio} 09:30\\
\textbf{Ora di Fine} 09:45\\
\\
\begin{tabular}{|c|c|c|}
    \hline
    \textbf{Componente} & \textbf{Ruolo} & \textbf{Presenza}\\
    \hline
    XXXX & Redattore & Presente \\ %%continuare così
    \hline %linea di fine
\end{tabular}
\newline
\newline
\begin{tabular}{|c|c|}
    \hline
    \textbf{Nome} & \textbf{Ruolo}\\
    \hline
    Alex Beggiato & XXXX\\
    \hline
\end{tabular}
\newpage

\section{Verbale dell'incontro}
\subsection{Dataset}
\textbf{Domanda:} Esiste qualche Dataset da utilizzare nel progetto?
\newline
Nel capitolato è stata inserita una lista di possibili fonti di cui una è stata utilizzata per l’esempio del clima, dove appunto venivano mostrati diversi dati impostando latitudine e logitudine. Questo è utile per il POC dove si lavora con un dataset già fornito. 
Quindi si ha un dataset grande in un file .csv, ed è possibile selezionare diversi paesi, nell’esempio mostrato in classe sono state mostrate 3 città. Quindi abbiamo un dataset popoloso senza fare troppa fatica.
La sfida del progetto è concentrarsi sulla parte grafica per rendere questi valori/tabelle chiare ed accessibili.
\subsection{Documentazione Bug e Unit Testing}
\textbf{Domanda:} Come svolgere la documentazione dei bug e unit testing, come funzionano e cosa si intende?
\newline
La fase di testing è diversa dalla documentazione dei bug, per unit test intendiamo test che verificano una singola funzionalità che girano senza dipendenze e questo può essere fatto sia in lato front-end ragionando ad esempio: “quali colonne mostrare, quindi ragionado su un data set fisso e aspettare che il risultato sia coerente”; Sul Lato backend molto simile ma si ragiona sull’api e configurazione
Per la lista dei bug risolti in fase di sviluppo, è una lista che a fine progetto ci permette di dire questi sono casi che sono emersi e che sono stati risolti e ipotizzare quelli che possono emergere alla fine e che non sono stati risolti e quindi sono emersi dei temi che non sono stati smarcati completamente e quindi possono creare dei problemi ed essere fonte di miglioramento oppure ci sono bug minori che non sono stati risolti. Quindi è una cosa da fare verso la fine
\subsection{Metodologia Test}
\textbf{Domanda:} Come e quando effettuiamo i test?
\newline
Ci sono diverse scuole di pensiero:
\begin{itemize}
    \item La prima garantisce lo sviluppo consapevole, scrivo prima il test pensando a cosa la funzione debba fare e dopodiché scrivo il codice avendo già il test garantendo che la funzione sia coerente con ciò che ho ipotizzato (\textit{test driven development})
    \item La strada più tradizionale invece è quella di sviluppare e poi fare il test garantendo così che modifiche successive non mi creino disturbi sul metodo scritto
\end{itemize}
Sta a noi quindi decidere come strutturare i test, ma è consigliato il test driven development
\subsection{Bug Report}
\textbf{Domanda:} Come report di bug, possiamo usare le issue di \textit{GitHub} oppure dobbiamo fare una lista a parte?
\newline
Se usiamo \textit{GitHub} come strumento di tracciatura va molto bene usare \textit{GitHub} ma andrebbe poi fatta una lista più consolidata e leggibile.
\subsection{Esperienze Passate}
\textbf{Domanda:} Secondo le Esperienze degli anni passati cosa si consiglia fare?
\newline
Nel progetto viene richiesto il \textbf{POC} quindi solitamente la gente si butta sul codice, ma se il \textbf{POC} viene buttato via non conviene ma invece conviene partire dai test perchè il \textbf{POC} serve per dimostrare la fattibilità.
Negli scorsi anni non ci sono stati problemi di consegna, forse qualche bug ma non problemi grossi tutti risolvibili
\subsection{Comunicazione}
\textbf{Domanda:} Nel caso di eventuali colloqui futuri che piattaforma usiamo per la comunicazione?
\newline
Le riunioni di avanzamento conviene farle via \textit{Google meet}, per le altre si possono fare in presenza a Grisignano oppure online come siamo più comodi, ed eventualmente con lo stage da fare evitando di sovrapporre il lavoro. Per dubbi veloci si può usare la mail tranne il lunedì e il martedì giorni nei quali il responsabile è via fuori città.
\end{document}




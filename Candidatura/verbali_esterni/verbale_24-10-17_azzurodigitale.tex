\documentclass{article}
\usepackage{graphicx}
\usepackage{fancyhdr}
\usepackage{geometry}
\usepackage{setspace}
\usepackage{tikz}
\usepackage[italian]{babel}
\usepackage[hidelinks]{hyperref}

% Margini della pagina
\geometry{a4paper, margin=1in}

% Intestazione personalizzata
\pagestyle{fancy}
\fancyhf{}
\fancyhead[L]{Code7Crusaders - Software Development Team}
\fancyhead[R]{\thepage}

% Spaziatura delle righe
\setstretch{1.2}

\begin{document}

% Sezione del titolo
\begin{titlepage}

    \AddToHookNext{shipout/background}{
    \begin{tikzpicture}[remember picture,overlay]
    \node at (current page.center) {
    \includegraphics{../../img/background.png}
    };
    \end{tikzpicture}
    }

    \centering
    \vspace*{2cm}

    \includegraphics[width=0.3\textwidth]{../../img/logo/7Crusaders_logo.png} % Aggiungi il logo qui
    \vspace{1cm}
    
    {\Huge \textbf{Code7Crusaders}}\\
    \vspace{0.5cm}
    {\Large Software Development Team}\\
    \vspace{2cm}
    
    {\large \textbf{Incontro del 17/10/2024 con AzzurroDigitale}}\\
    \vspace{5cm}

    \textbf{Membri del Team:}\\
    Enrico Cotti Cottini, Gabriele Di Pietro, Tommaso Diviesti \\
    Francesco Lapenna, Matthew Pan, Eddy Pinarello, Filippo Rizzolo \\
    \vspace{0.5cm}
    
    \vspace{1cm}
\end{titlepage}

% Versioni
\begin{center}
    \textbf{Versioni} % Title for Versions Table
    \\
    \begin{tabular}{|c|c|c|c|}
        \hline
        \textbf{Ver.} & \textbf{Data} & \textbf{Autore} & \textbf{Descrizione} \\
        \hline
        0.3 & 22/10/2024 & Filippo Rizzolo & Prima revisione del documento e controllo ortografia \\ 
        \hline
        0.2 & 21/10/2024 & Enrico Cotti Cottini & Aggiunte Presenze \\
        \hline
        0.1 & 17/10/2024 & Eddy Pinarello & Prima stesura del documento  \\ 
        \hline
    \end{tabular}
\end{center}

% Indice
\newpage
\tableofcontents
\newpage

% Sezione Introduzione
\section{Registro Presenze}
\textbf{Piattaforma della riunione:} Google Meet \\
\textbf{Ora di Inizio} 12:30\\
\textbf{Ora di Fine} 13:00\\
\\
\begin{tabular}{|c|c|c|}
    \hline
    \textbf{Componente} & \textbf{Ruolo} & \textbf{Presenza}\\
    \hline
    Enrico Cotti Cottini & Responsabile & Presente \\ 
    \hline
    Gabriele Di Pietro & Redattore & Assente \\ 
    \hline
    Tommaso Divesti & Redattore & Presente \\ 
    \hline 
    Francesco Lapenna & Verificatore & Presente \\ 
    \hline
    Matthew Pan & Verificatore & Assente \\ 
    \hline 
    Eddy Pinarello & Redattore & Presente \\ 
    \hline 
    Filippo Rizzolo & Amministratore & Presente \\ 
    \hline 
\end{tabular}
\\
\newline
\newline
\begin{tabular}{|c|c|}
    \hline
    \textbf{Nome} & \textbf{Ruolo}\\
    \hline
    Martina Daniele & Rappresentante Azienda \\
    \hline
    Giorgio Vallini & Rappresentante Azienda \\
    \hline
    Nicola Boscaro & Rappresentante Azienda \\
    \hline
    Mattia Gottardello & Rappresentante Azienda \\
    \hline
\end{tabular}
\newpage

\section{Verbale dell'incontro}

\subsection{Complessità Progetto} \textbf{Domanda:} Quali sono gli elementi di complessità maggiore nel progetto, ovvero le parti più difficili da implementare?
\newline
\textit{Risposta:} La gestione del backend che integra l'uso di OpenAI, la ricezione e invio di domande e risposte, l'acquisizione di dati da diverse piattaforme, e la gestione di un database vettoriale. Stiamo considerando PostgreSQL come database principale.

\subsection{Piattaforma per la comunicazione} \textbf{Domanda:} Quali strumenti utilizza l'azienda per comunicare e per i colloqui? È prevista la possibilità di incontri di persona?
\newline
\textit{Risposta:} Usiamo Google Meet per gli incontri e per la gestione burocratica. Questo è lo standard di comunicazione.

\subsection{Framework utilizzabili} \textbf{Domanda:} Quali framework possiamo utilizzare per lo sviluppo del progetto? Angular è obbligatorio?
\newline
\textit{Risposta:} Se Angular non è lo strumento con cui ci sentiamo più a nostro agio, possiamo scegliere un altro framework, purché si raggiunga il risultato.

\subsection{Linee guida per API di terze parti} \textbf{Domanda:} Ci sono linee guida specifiche per l'integrazione di API di terze parti?
\newline
\textit{Risposta:} Ogni piattaforma ha la propria documentazione per le API. È utile identificare prima i servizi già esistenti con connettori pronti, per evitare di dover creare integrazioni dirette da zero.

\subsection{Modelli LLM utilizzabili} \textbf{Domanda:} Possiamo utilizzare modelli LLM open-source se necessario?
\newline
\textit{Risposta:} Sì, possiamo utilizzare modelli open-source, soprattutto se ChatGPT non soddisfa completamente i requisiti. È essenziale che le risposte siano rapide e non abbiano latenze eccessive.

\subsection{Alternative a LangChain e OpenAI} \textbf{Domanda:} Ci sono alternative a LangChain o OpenAI per la gestione del progetto?
\newline
\textit{Risposta:} LangChain è uno dei framework consigliati, ma ciò che conta di più è il risultato. Se ci sono difficoltà con gli strumenti core, si possono considerare altre soluzioni.

\subsection{Aree di maggiore focus} \textbf{Domanda:} Su quali aspetti dovremmo concentrarci maggiormente nel progetto?
\newline
\textit{Risposta:} Concentratevi soprattutto sugli aspetti relativi ai modelli LLM, poiché rappresentano la parte meno solida del progetto. Il resto dovrebbe essere più agevole.

\subsection{Test e bug reporting} \textbf{Domanda:} Come gestire il testing e il reporting dei bug nel progetto?
\newline
\textit{Risposta:} Per il bug reporting, utilizziamo la sezione \textit{Issues} di GitHub. Il docente richiede anche unit test per coprire il codice. Possiamo seguire una metodologia TDD, ma l'importante è arrivare al risultato con una copertura sufficiente del codice.

\begin{table}[b]
    \begin{tabular}{@{}p{.5in}p{4in}@{}}
    Data:  & \hrulefill \\
           &     		\\
           &     		\\
    Firma: & \hrulefill \\
    \end{tabular}
\end{table}

\end{document}

\documentclass{article}
\usepackage{graphicx}
\usepackage{fancyhdr}
\usepackage{geometry}
\usepackage{setspace}
\usepackage{tikz}
\usepackage[italian]{babel}
\usepackage[hidelinks]{hyperref}

% Margini della pagina
\geometry{a4paper, margin=1in}

% Intestazione personalizzata
\pagestyle{fancy}
\fancyhf{}
\fancyhead[L]{Code7Crusaders - Software Development Team}
\fancyhead[R]{\thepage}

% Spaziatura delle righe
\setstretch{1.2}

\begin{document}

% Sezione del titolo
\begin{titlepage}

    \AddToHookNext{shipout/background}{
    \begin{tikzpicture}[remember picture,overlay]
    \node at (current page.center) {
    \includegraphics{../../img/background.png}
    };
    \end{tikzpicture}
    }

    \centering
    \vspace*{2cm}
    
    \includegraphics[width=0.3\textwidth]{../../img/logo/7Crusaders_logo.png} % Aggiungi il logo qui
    \vspace{1cm}
    
    {\Huge \textbf{Code7Crusaders}}\\
    \vspace{0.5cm}
    {\Large Software Development Team}\\
    \vspace{2cm}
    

    {\large \textbf{Incontro del 18/10/2024 con Sanmarco Informatica}}\\
    \vspace{5cm}

    \textbf{Membri del Team:}\\
    Enrico Cotti Cottini, Gabriele Di Pietro, Tommaso Diviesti \\
    Francesco Lapenna, Matthew Pan, Eddy Pinarello, Filippo Rizzolo \\
    \vspace{0.5cm}
    
    \vspace{1cm}
\end{titlepage}

%Versioni
\begin{center}
    \textbf{Versioni}
    \\
    \begin{tabular}{|c|c|c|c|}
        \hline
        \textbf{Ver} & \textbf{Data} & \textbf{Autore} & \textbf{Descrizione}\\
        \hline
        1.0 & 29/10/2024 & Filippo Rizzolo & Approvazione documento \\ 
        \hline
        0.3 & 22/10/2024 & Filippo Rizzolo & Prima revisione del documento e controllo ortografia \\ 
        \hline
	    0.2 & 21/10/2024 & Enrico Cotti Cottini & Aggiunte Presenze \\
	    \hline
        0.1 & 18/10/2024 & Gabriele Di Pietro & Prima stesura del documento \\
        \hline
    \end{tabular}
\end{center}

% Indice
\newpage
\tableofcontents
\newpage

% Sezione Introduzione
\section{Registro Presenze}
\textbf{Piattaforma della riunione:} Google Meet \\
\textbf{Ora di Inizio} 09:30\\
\textbf{Ora di Fine} 09:45\\
\\
\begin{tabular}{|c|c|c|}
    \hline
    \textbf{Componente} & \textbf{Ruolo} & \textbf{Presenza}\\
    \hline
    Enrico Cotti Cottini & Responsabile & Assente \\ 
    \hline
    Gabriele Di Pietro & Redattore & Presente \\ 
    \hline
    Tommaso Diviesti & Redattore & Assente \\ 
    \hline 
    Francesco Lapenna & Verificatore & Presente \\ 
    \hline
    Matthew Pan & Verificatore & Assente \\ 
    \hline 
    Eddy Pinarello & Redattore & Presente \\ 
    \hline 
    Filippo Rizzolo & Amministratore & Presente \\ 
    \hline 
\end{tabular}
\\
\newline
\newline
\begin{tabular}{|c|c|}
    \hline
    \textbf{Nome} & \textbf{Ruolo}\\
    \hline
    Alex Beggiato & Rappresentante Azienda \\
    \hline
\end{tabular}
\newpage

\section{Verbale dell'incontro}
\subsection{Dataset}
\textbf{Domanda:} Esiste qualche Dataset da utilizzare nel progetto?
\newline
\textit{Risposta:} Nel capitolato è stata inserita una lista di possibili fonti di cui una è stata utilizzata per l’esempio del clima, dove appunto venivano mostrati diversi dati impostando latitudine e longitudine. Questo è utile per il POC dove si lavora con un dataset già fornito. 
Quindi si ha un dataset grande in un file .csv, ed è possibile selezionare diversi paesi, nell’esempio mostrato in classe sono stati visualizzati i dati di tre città. Quindi abbiamo un dataset popoloso senza fare troppa fatica.
La sfida del progetto è concentrarsi sulla parte grafica per rendere questi valori/tabelle chiare ed accessibili.
\subsection{Documentazione Bug e Unit Testing}
\textbf{Domanda:} Come svolgere la documentazione dei bug e unit testing, come funzionano e cosa si intende?
\newline
\textit{Risposta:} La documentazione dei bug e l'unit testing sono due fasi importanti nello sviluppo software, con scopi differenti.
\newline
\underline{Unit Testing:}  
L'unit testing verifica singole unità di codice in isolamento, assicurandosi che ogni componente funzioni correttamente.  
\begin{itemize}
    \item \textit{Front-end}: Si testa, ad esempio, che una tabella mostri correttamente determinate colonne usando dati fissi.
    \item \textit{Back-end}: Si testano API o configurazioni per garantire che rispondano come previsto.
\end{itemize}
\underline{Documentazione dei Bug:}  
La documentazione dei bug raccoglie gli errori trovati durante lo sviluppo e i test, tenendo traccia dei problemi risolti e di quelli ancora presenti. Questa fase, utile soprattutto verso la fine del progetto, aiuta a identificare bug risolti, bug minori ancora presenti, e aree di possibile miglioramento.
\subsection{Metodologia Test}
\textbf{Domanda:} Come e quando effettuiamo i test?
\newline
\textit{Risposta:} Ci sono diverse scuole di pensiero:
\begin{itemize}
    \item La prima garantisce lo sviluppo consapevole, scrivo prima il test pensando a cosa la funzione debba fare e dopodiché scrivo il codice avendo già il test garantendo che la funzione sia coerente con ciò che ho ipotizzato (\textit{test driven development})
    \item La strada più tradizionale invece è quella di sviluppare e poi fare il test garantendo così che modifiche successive non mi creino disturbi sul metodo scritto
\end{itemize}
Sta a noi quindi decidere come strutturare i test, ma è consigliato il test driven development
\subsection{Bug Report}
\textbf{Domanda:} Come report di bug, possiamo usare le issue di \textit{GitHub} oppure dobbiamo fare una lista a parte?
\newline
\textit{Risposta:} Se usiamo \textit{GitHub} come strumento di tracciatura va molto bene usare \textit{GitHub} ma andrebbe poi fatta una lista più consolidata e leggibile.
\subsection{Esperienze Passate}
\textbf{Domanda:} Secondo le Esperienze degli anni passati cosa si consiglia fare?
\newline
\textit{Risposta:} Nel progetto viene richiesto il \textbf{POC} quindi solitamente la gente si butta sul codice, ma se il \textbf{POC} viene buttato via non conviene ma invece conviene partire dai test perchè il \textbf{POC} serve per dimostrare la fattibilità.
Negli scorsi anni non ci sono stati problemi di consegna, forse qualche bug ma non problemi grossi tutti risolvibili
\subsection{Comunicazione}
\textbf{Domanda:} Nel caso di eventuali colloqui futuri che piattaforma usiamo per la comunicazione?
\newline
\textit{Risposta:} Le riunioni di avanzamento conviene farle via \textit{Google meet}, per le altre si possono fare in presenza a Grisignano oppure online come siamo più comodi, ed eventualmente con lo stage da fare evitando di sovrapporre il lavoro. Per dubbi veloci si può usare la mail tranne il lunedì e il martedì giorni nei quali il responsabile è via fuori città.

\begin{table}[b]
	\begin{tabular}{@{}p{.5in}p{4in}@{}}
		Data:  & \hrulefill \\
			   &     		\\
			   &     		\\
		Firma: & \hrulefill \\
	\end{tabular}
	\end{table}

\end{document}
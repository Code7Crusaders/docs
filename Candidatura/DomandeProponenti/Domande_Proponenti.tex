\documentclass{article}
\usepackage{graphicx}
\usepackage{fancyhdr}
\usepackage{geometry}
\usepackage{setspace}
\usepackage{tikz}
\usepackage[italian]{babel}

% Margini della pagina
\geometry{a4paper, margin=1in}

% Intestazione personalizzata
\pagestyle{fancy}
\fancyhf{}
\fancyhead[L]{Code7Crusaders - Software Development Team}
\fancyhead[R]{\thepage}

% Spaziatura delle righe
\setstretch{1.2}

\begin{document}

% Sezione del titolo
\begin{titlepage}

    \AddToHookNext{shipout/background}{
    \begin{tikzpicture}[remember picture,overlay]
    \node at (current page.center) {
    \includegraphics{../../img/background.png}
    };
    \end{tikzpicture}
    }

    \centering
    \vspace*{2cm}
    
    \includegraphics[width=0.3\textwidth]{../../img/logo/7Crusaders_logo.png} % Aggiungi il logo qui
    \vspace{1cm}
    
    {\Huge \textbf{Code7Crusaders}}\\
    \vspace{0.5cm}
    {\Large Software Development Team}\\
    \vspace{2cm}
    
    {\large \textbf{Documentazione Progetto}}\\
    \vspace{5cm}

    \textbf{Membri del Team:}\\
    Enrico Cotti Cottini, Gabriele Di Pietro, Tommaso Diviesti \\
    Francesco Lapenna, Matthew Pan, Eddy Pinarello, Filippo Rizzolo \\
    \vspace{0.5cm}
    
    {\large \textbf{Data:}} \today\\
    
    \vspace{1cm}
\end{titlepage}

% Indice
\newpage
\tableofcontents
\newpage

% Sezione Introduzione
\section{Domande per le Aziende}\label{sec:domande-aziende}
\subsection{Domande Generali}\label{subsec:domande-generali}
\begin{itemize}
    \item Quali sono gli elementi di complessità maggiore, quindi le parti più difficili da implementare nel progetto?
    \item Quale strumento che usa l’azienda per eventuali colloqui e chiarimenti specifici, o se ci si può incontrare dal vivo?
    \item Quali framework possiamo usare per gestire il progetto? 
\end{itemize}

\subsection{Domande per Vimar C2}\label{subsec:domande-vimar}
\begin{itemize}
    \item Quale LLM sono più adatte allo scopo tra quelle proposte?
    \item Come funziona il cloud e se è possibile avere più chiarimenti o consulenze?
    \item Che frame usare per costruire il server?
    \item Che framework usare per il sito web, quello più consigliato o se è possibile cambiarlo.
    \item Se possiamo usare Python per lo sviluppo degli API, altrimenti quali sono le alternative che utilizzabili?
    \item Che strumenti o guide sono consigliate per i Testing?
    \item Cosa intendono nei requisiti obbligatori per “Infrastrutture as Code”?
\end{itemize}

\subsection{Domande per Ergon C7}\label{subsec:domande-ergon}
\begin{itemize}
    \item È possibile non utilizzare .NET MAUI?
\end{itemize}

\subsection{Domande per Azzurro Digitale C9}\label{subsec:domande-azzurro}
\begin{itemize}
    \item Ci sono linee guida per le applicazioni di API di terze parti?
    \item Ci sono alternative tra LangChain e OpenAI?
    \item Come possiamo raccogliere i dati da tutte le varie piattaforme che dobbiamo usare “Telegram, GitHub..”?
    \item Abbiamo alternative ad Angular? 
    \item Che datebase dobbiamo usare?
\end{itemize}

\subsection{Domande per SanMarco C5}\label{subsec:domande-sanmarco}
\begin{itemize}
    \item Esiste qualche dataset da utilizzare per il progetto?
\end{itemize}

\end{document}



Azzurro digitale

1 Complessita progetto? il backend che gestisce lutilizzo di open ai, la ricezione di domanda e risosta e la ricezione dei dati dalle
varie piattaforme, anche la gestione di rug database vettoriale, postgesql usare come database

2 Piattaforme per comunicare? usiamo meet per incontri, si informano a livello burocratico e si capisce, standard meet

3 Framework? angular, è obligatorio? per ottenere il risultato se siamo piu comodo con altro allora possiamo fare quello

4 api di terze parti, backend llm con piattaforme loro? 
api le varie piattaforme hanno api che hanno documentazione e farsi un idea per come interagire con esse?
se va male identificate quali sono gli api che danno info  get page conference, valutate di fare prima una
ricerca su se esistono servizi gia fatti e se hanno gia connettori cosi da non dovere interagire direttamente con 
gli strumenti, con gia lintegrazione.

4 Possiamo usare modelli llm opensource? si potrebbero usare ache modelli opensource se non riusciamo a fare qualcosa con chatgpt, 
trainando in locale ecc.veloce nelle risposte, non puo rispondere dopo dieci minuti, 

5 Ci sono alternative tra LangChain? LangChain framework da usare per situazioni come questa, limportante è
il risultato come lo si ottiene facciamo noi 6 difficota nelluso delle cose core

6 Consiglio su cosa concentrarsi nel progetto.
concentratevi sulla parte di studio sulla roba dei llm perche è la roba meno solida, concetratevi sulla cicca
il resto si fa easy

7 parte di testing e bug reporting? sezione di issues di github, parte di unit test è chiesta dal prof,
il codice che scriviamo deve essere coperto, possiamo usare o TDD oppure no pero bisogna arrivare al risultato.
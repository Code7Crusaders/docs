\documentclass{article}
\usepackage{graphicx}
\usepackage{fancyhdr}
\usepackage{geometry}
\usepackage{setspace}
\usepackage{tikz}
\usepackage[italian]{babel}
\usepackage[hidelinks]{hyperref}
\usepackage{float}

% Margini della pagina
\geometry{a4paper, margin=1in}

% Intestazione personalizzata
\pagestyle{fancy}
\fancyhf{}
\fancyhead[L]{Code7Crusaders - Software Development Team}
\fancyhead[R]{\thepage}

% Spaziatura delle righe
\setstretch{1.2}

\begin{document}

% Sezione del titolo
\begin{titlepage}

    \AddToHookNext{shipout/background}{
    \begin{tikzpicture}[remember picture,overlay]
    \node at (current page.center) {
    \includegraphics[width=1.05\paperwidth]{../../img/background.png}
    };
    \end{tikzpicture}
    }

    \centering
    \vspace*{2cm}
    
    \includegraphics[width=0.3\textwidth]{../../img/logo/7Crusaders_logo.png} % Aggiungi il logo qui
    \vspace{1cm}
    
    {\Huge \textbf{Code7Crusaders}}\\
    \vspace{0.5cm}
    {\Large Software Development Team}\\
    \vspace{2cm}
    
    {\large \textbf{Analisi e confronto tra MAUI, React e Angular}}\\
    \vspace{5cm}

    \textbf{Membri del Team:}\\
    Enrico Cotti Cottini, Gabriele Di Pietro, Tommaso Diviesti \\
    Francesco Lapenna, Matthew Pan, Eddy Pinarello, Filippo Rizzolo \\
    \vspace{0.5cm}
    
    {\large \textbf{Data:}} \today\\
    
    \vspace{1cm}
\end{titlepage}

% Indice
\newpage
\tableofcontents
\newpage

% Sezione Introduzione
\section{Obiettivo}
Questo documento si pone l'obiettivo di confrontare tre framework di sviluppo per applicazioni front-end: MAUI (Multi-platform App UI), React e Angular. Il confronto considera aspetti tecnici, caratteristiche, vantaggi, svantaggi e casi d'uso. La scelta finale ricade su React, motivata da una valutazione completa delle necessit\`a del nostro progetto e delle risorse disponibili.

\section{Caratteristiche dei Framework}

\subsection{MAUI (Multi-platform App UI)}
MAUI \`e un framework sviluppato da Microsoft che consente la creazione di applicazioni multipiattaforma utilizzando un unico codice base scritto in C\#.

\paragraph*{Caratteristiche principali}
\begin{itemize}
    \item \textbf{Linguaggio}: Utilizza C\# e .NET, strumenti consolidati per lo sviluppo.
    \item \textbf{Multipiattaforma}: Permette di creare applicazioni per Android, iOS, macOS e Windows.
    \item \textbf{Integrazione con Visual Studio}: Sviluppo semplificato grazie agli strumenti nativi di Microsoft.
\end{itemize}

\paragraph*{Vantaggi}
\begin{itemize}
    \item \textbf{Codice Condiviso}: Riduzione dei tempi di sviluppo grazie all'utilizzo di un unico codice base.
    \item \textbf{Ecosistema Microsoft}: Forte integrazione con i servizi Microsoft.
    \item \textbf{Prestazioni native}: Applicazioni con performance elevate su tutte le piattaforme.
\end{itemize}

\paragraph*{Svantaggi}
\begin{itemize}
    \item \textbf{Curva di apprendimento}: Richiede una conoscenza approfondita di .NET e C\#.
    \item \textbf{Comunit\`a Limitata}: Rispetto a React e Angular, MAUI ha una comunit\`a di sviluppatori pi\`u piccola.
    \item \textbf{Supporto non uniforme}: Le funzionalit\`a possono variare tra piattaforme.
\end{itemize}

\subsection{React}
React \`e una libreria JavaScript sviluppata da Facebook per creare interfacce utente dinamiche e reattive.

\paragraph*{Caratteristiche principali}
\begin{itemize}
    \item \textbf{Componenti}: Architettura basata su componenti riutilizzabili.
    \item \textbf{Virtual DOM}: Aggiornamenti efficienti del DOM grazie al sistema virtuale.
    \item \textbf{Ecosistema Vastissimo}: Ampia disponibilit\`a di librerie e strumenti.
\end{itemize}

\paragraph*{Vantaggi}
\begin{itemize}
    \item \textbf{Flessibilit\`a}: Pu\`o essere utilizzato con diversi stack tecnologici.
    \item \textbf{Comunit\`a Attiva}: Documentazione eccellente e supporto continuo da parte della comunit\`a.
    \item \textbf{Performance}: Aggiornamenti rapidi grazie al Virtual DOM.
\end{itemize}

\paragraph*{Svantaggi}
\begin{itemize}
    \item \textbf{Curva di apprendimento iniziale}: Richiede una comprensione profonda di JSX e del ciclo di vita dei componenti.
    \item \textbf{Gestione dello stato}: Necessita di librerie aggiuntive come Redux o Context API per progetti complessi.
\end{itemize}

\subsection{Angular}
Angular \`e un framework full-stack sviluppato da Google, ideale per applicazioni complesse e scalabili.

\paragraph*{Caratteristiche principali}
\begin{itemize}
    \item \textbf{Full-Stack}: Offre un set completo di strumenti per lo sviluppo front-end.
    \item \textbf{Two-Way Data Binding}: Sincronizzazione automatica tra modello e vista.
    \item \textbf{TypeScript}: Sviluppo strutturato e mantenibile grazie a TypeScript.
\end{itemize}

\paragraph*{Vantaggi}
\begin{itemize}
    \item \textbf{Robustezza}: Ideale per applicazioni di grandi dimensioni.
    \item \textbf{Struttura}: Offre un'architettura ben definita e coerente.
    \item \textbf{Integrazione}: Supporto nativo per funzionalit\`a avanzate come routing e gestione dello stato.
\end{itemize}

\paragraph*{Svantaggi}
\begin{itemize}
    \item \textbf{Complessit\`a}: Curva di apprendimento ripida.
    \item \textbf{Pesantezza}: Applicazioni iniziali pi\`u lente rispetto a quelle sviluppate con React.
    \item \textbf{Minor Flessibilit\`a}: Meno adatto a piccoli progetti o applicazioni non standard.
\end{itemize}

\section{Confronto}
Il seguente confronto \`e basato su diversi fattori chiave:

\begin{table}[H]
    \centering
    \begin{tabular}{|l|c|c|c|}
        \hline
        \textbf{Caratteristica} & \textbf{MAUI} & \textbf{React} & \textbf{Angular} \\
        \hline
        Facilità di apprendimento & Media & Alta & Bassa \\
        \hline
        Performance & Alta (Nativo) & Alta (Virtual DOM) & Media \\
        \hline
        Ecosistema & Limitato & Vastissimo & Robusto ma meno flessibile \\
        \hline
        Scalabilità & Media & Alta & Alta \\
        \hline
        Comunità & Limitata & Ampia & Media \\
        \hline
        Flessibilità & Bassa & Alta & Media \\
        \hline
    \end{tabular}
    \caption{Confronto tra MAUI, React e Angular}
    \label{tab:confronto_framework}
\end{table}

\section{Conclusioni}
Dopo un'attenta analisi, la scelta finale ricade su \textbf{React}, grazie ai seguenti motivi:

\begin{itemize}
    \item \textbf{Facilit\`a di apprendimento e flessibilit\`a}: Ideale per un team con competenze variegate e per progetti di diverse dimensioni.
    \item \textbf{Ecosistema vasto}: Disponibilit\`a di numerosi strumenti e librerie per semplificare

\end{document}
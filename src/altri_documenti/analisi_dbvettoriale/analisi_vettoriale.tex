\documentclass{article}
\usepackage{graphicx}
\usepackage{fancyhdr}
\usepackage{geometry}
\usepackage{setspace}
\usepackage{tikz}
\usepackage[italian]{babel}
\usepackage[hidelinks]{hyperref}
\usepackage{float}

% Margini della pagina
\geometry{a4paper, margin=1in}

% Intestazione personalizzata
\pagestyle{fancy}
\fancyhf{}
\fancyhead[L]{Code7Crusaders - Software Development Team}
\fancyhead[R]{\thepage}

% Spaziatura delle righe
\setstretch{1.2}

\begin{document}

% Sezione del titolo
\begin{titlepage}

    \AddToHookNext{shipout/background}{
    \begin{tikzpicture}[remember picture,overlay]
    \node at (current page.center) {
    \includegraphics[width=1.05\paperwidth]{../../img/background.png}
    };
    \end{tikzpicture}
    }

    \centering
    \vspace*{2cm}
    
    \includegraphics[width=0.3\textwidth]{../../img/logo/7Crusaders_logo.png} % Aggiungi il logo qui
    \vspace{1cm}
    
    {\Huge \textbf{Code7Crusaders}}\\
    \vspace{0.5cm}
    {\Large Software Development Team}\\
    \vspace{2cm}
    
    {\large \textbf{Analisi per la scelta del DB Vettoriale}}\\
    \vspace{5cm}

    \textbf{Membri del Team:}\\
    Enrico Cotti Cottini, Gabriele Di Pietro, Tommaso Diviesti \\
    Francesco Lapenna, Matthew Pan, Eddy Pinarello, Filippo Rizzolo \\
    \vspace{0.5cm}
    
    {\large \textbf{Data:}} 15 Febbraio 2025\\
    
    \vspace{1cm}
\end{titlepage}
\clearpage

% Indice
\newpage
\tableofcontents
\newpage

% Sezione del documento

\section{Introduzione}
Il seguente documento ha lo scopo di analizzare i database vettoriali e di fornire una valutazione per la scelta del database vettoriale più adatto per il progetto in corso.\\
Il documento è strutturato in tre sezioni principali:

\begin{itemize}
    \item \textbf{Introduzione ai database vettoriali}: in cui vengono introdotti i database vettoriali e le loro caratteristiche principali.
    \item \textbf{Confronto}: In maniera tabellare vengono confrontati i database vettoriali analizzati.
    \item \textbf{Conclusione e Motivazione della Scelta}: in cui viene presentata la scelta del database vettoriale più adatto per il progetto in corso.
\end{itemize}

\section{Introduzione ai database vettoriali}

\subsection{FAISS (Facebook AI Similarity Search)}
FAISS è una libreria sviluppata da Facebook AI per la ricerca di similarità tra vettori ad alta dimensione. È ottimizzata per operazioni di nearest neighbor search su dataset di grandi dimensioni ed è ampiamente utilizzata in applicazioni di machine learning e intelligenza artificiale.

\textbf{Pro:}
\begin{itemize}
    \item Alta velocità: ottimizzato per operazioni di nearest neighbor search su dataset di grandi dimensioni.
    \item Supporto GPU: usa CUDA per accelerare la ricerca su hardware Nvidia.
    \item Ottimizzazioni avanzate: supporta tecniche come IVF (Inverted File Index), PQ (Product Quantization), e HNSW (Hierarchical Navigable Small World).
\end{itemize}
\textbf{Contro:}
\begin{itemize}
    \item Nessuna persistenza nativa: gli indici devono essere salvati e caricati manualmente.
    \item Uso elevato di memoria: può richiedere molta RAM, specialmente senza compressione.
    \item Configurazione complessa: necessita di tuning per ottimizzare velocità, accuratezza e memoria.
\end{itemize}
\textbf{Risorse:} \href{https://github.com/facebookresearch/faiss}{GitHub} \textbar{} \href{https://faiss.ai/}{Documentazione}



\end{document}
\documentclass{article}
\usepackage{graphicx}
\usepackage{fancyhdr}
\usepackage{geometry}
\usepackage{setspace}
\usepackage{tikz}
\usepackage[italian]{babel}
\usepackage[hidelinks]{hyperref}
\usepackage{float}

% Margini della pagina
\geometry{a4paper, margin=1in}

% Intestazione personalizzata
\pagestyle{fancy}
\fancyhf{}
\fancyhead[L]{Code7Crusaders - Software Development Team}
\fancyhead[R]{\thepage}

% Spaziatura delle righe
\setstretch{1.2}

\begin{document}

% Sezione del titolo
\begin{titlepage}

    \AddToHookNext{shipout/background}{
    \begin{tikzpicture}[remember picture,overlay]
    \node at (current page.center) {
    \includegraphics[width=1.05\paperwidth]{../../img/background.png}
    };
    \end{tikzpicture}
    }

    \centering
    \vspace*{2cm}
    
    \includegraphics[width=0.3\textwidth]{../../img/logo/7Crusaders_logo.png} % Aggiungi il logo qui
    \vspace{1cm}
    
    {\Huge \textbf{Code7Crusaders}}\\
    \vspace{0.5cm}
    {\Large Software Development Team}\\
    \vspace{2cm}
    
    {\large \textbf{Analisi per la scelta delle tecnologie backend}}\\
    \vspace{5cm}

    \textbf{Membri del Team:}\\
    Enrico Cotti Cottini, Gabriele Di Pietro, Tommaso Diviesti \\
    Francesco Lapenna, Matthew Pan, Eddy Pinarello, Filippo Rizzolo \\
    \vspace{0.5cm}
    
    {\large \textbf{Data:}} 12 Febbraio 2025\\
    
    \vspace{1cm}
\end{titlepage}
\clearpage

% Indice
\newpage
\tableofcontents
\newpage

% Sezione Introduzione
\section{Obiettivo}
Questo documento si pone l'obiettivo di confrontare le diverse 
tecnologie backend in modo da avere un'idea precisa e prendere 
una decisione sicura per la scelta di esse e per la loro integrazione 
all'interno del nostro progetto. Il confronto considera aspetti 
tecnici, caratteristiche, vantaggi e svantaggi dei vari framework 
discussi. Le nostre scelte finali sono l'utilizzo di Flask e di 
Langchain per diverse motivazioni chiarite nei paragrafi sucessivi.

\section{Analisi tecnologie backend scelte}

\subsection{Flask}
Flask è un \textit{microframework} web per Python, il che significa 
che fornisce gli strumenti di base necessari per sviluppare 
un'applicazione web, lasciando allo sviluppatore la possibilità di 
aggiungere altre funzionalità secondo le necessità. È una tecnologia 
open source e gratuita che fornisce un modo semplice per creare e 
distribuire applicazioni web dinamiche, offrendo molta libertà e 
controllo sullo sviluppo dell’applicazione.
\paragraph*{Vantaggi}
\begin{itemize}
    \item \textbf{Semplicità e flessibilità}: si può iniziare con 
    un’applicazione piccola e espanderla facilmente man mano che cresce
    \item \textbf{Personalizzazione}: permette di costruire 
    un’applicazione che si adatti perfettamente alle proprie esigenze 
    specifiche
    \item \textbf{Documentazione completa e community attiva}: è molto 
    facile trovare aiuto quando si ha qualche tipo di problema
    \item \textbf{Facile integrazione con altre tecnologie}: è 
    compatibile con una vasta gamma di tecnologie, incluse basi di dati, 
    sistemi di autenticazione e altri ancora
    \item \textbf{Sviluppo rapido ed efficiente}: Garantisce di portare 
    a termine l’applicazione velocemente e in modo efficace
    \item \textbf{Scalabilità}: è possibile adattare la propria 
    applicazione a un numero maggiore di utenti e carichi di lavoro 
    senza dover riscrivere il codice da zero
    \item \textbf{Tempo di caricamento ridotto}: è leggero ed efficiente 
    in termini di risorse, il che significa che le pagine web si 
    caricano rapidamente e senza problemi
\end{itemize}
\paragraph*{Svantaggi}
\begin{itemize}
    \item \textbf{Maggiore quantità di codice}: Flask richiede più 
    codice personalizzato rispetto a framework più completi
    \item \textbf{Manutenzione e aggiornamenti}:  molte funzionalità 
    devono essere aggiunte tramite estensioni o pacchetti di terze parti
    \item \textbf{Dipendenza da terze parti}:molte funzionalità 
    avanzate per costruire applicazioni web complete devono essere 
    aggiunte tramite librerie esterne
\end{itemize}

\subsection{Langchain}




\section{Possibili alternative da valutare}

\subsection{FastAPI}
FastAPI è un framework web moderno, ad alte prestazioni, per la 
costruzione di API RESTful con Python. È progettato per essere facile 
da usare, veloce da sviluppare, e altamente efficiente, sfruttando 
le potenzialità delle moderne caratteristiche di Python. Inoltre, 
possiede la capacità di produrre API estremamente veloci e sicure, 
con un focus sull'efficienza sia in termini di prestazioni che di 
sviluppo rapido.
\paragraph*{Vantaggi}
\begin{itemize}
    \item \textbf{Prestazioni elevate}: FastAPI è uno dei framework 
    più veloci in Python grazie al supporto nativo per la programmazione 
    asincrona
    \item \textbf{Supporto per la programmazione asincrona}: permette 
    di gestire richieste asincrone in modo molto efficiente, riducendo 
    il tempo di attesa e migliorando la gestione delle risorse durante 
    le operazioni I/O intensive
    \item \textbf{Facilità di testing}: facilita il testing delle API 
    grazie alla sua struttura di testing integrata e alla capacità di 
    generare automaticamente mock di richieste e risposte
    \item \textbf{Validazione automatica dei dati}: i dati inviati 
    tramite le richieste (come JSON) sono automaticamente validati 
    contro i modelli definiti, riducendo il codice necessario per la 
    gestione degli errori e migliorando la sicurezza
\end{itemize}
\paragraph*{Svantaggi}
\begin{itemize}
    \item \textbf{Curva di apprendimento}: FastAPI, pur essendo molto 
    potente, può avere una curva di apprendimento ripida per chi non è 
    familiare con Python e con la programmazione asincrona
    \item \textbf{Documentazione e community}: ha una community più 
    piccola rispetto ad altri framework più consolidati come Django o 
    Flask. Questo significa che potrebbe esserci meno documentazione, 
    risorse online o librerie di terze parti da utilizzare
    \item \textbf{Meno funzionalità predefinite}: fornisce solo le 
    funzionalità essenziali per la creazione di API. Funzionalità come 
    autenticazione, gestione dei permessi, e admin panel devono essere 
    implementate tramite estensioni o personalizzazioni
\end{itemize}

\subsection{Django REST Framework}
Django è un framework di alto livello, open source, costruito su Python 
che incoraggia uno sviluppo rapido e un design pulito e pragmatico. 
Si basa sul paradigma MTV, ossia \textit{“Model-Template-View”}. In 
questo esplicita la sua natura full-stack, in quanto gestiamo in modo 
olistico le interazioni tra la parte back-end (i modelli) e la parte 
front-end (i template) tramite viste (view).
\paragraph*{Vantaggi}
\begin{itemize}
    \item \textbf{Serializzazione dei dati}: fornisce un sistema di 
    serializzazione che consente di convertire facilmente i dati tra 
    formati (come JSON o XML) e oggetti Python, semplificando la 
    gestione dei dati tra il client e il server
    \item \textbf{Autenticazione e permessi avanzati}: include supporto 
    nativo per vari metodi di autenticazione (come token-based o 
    sessione) e per la gestione dei permessi
    \item \textbf{Paginazione}: offre un sistema di paginazione per 
    gestire grandi quantità di dati nelle risposte API, migliorando 
    le prestazioni quando ci sono molte risorse
    \item \textbf{Struttura sicura}:si basa su diversi meccanismi 
    integrati che aiutano a proteggere le applicazioni API da 
    vulnerabilità comuni e garantire una gestione sicura dei dati e 
    degli accessi
    \item \textbf{Comunità e supporto}: ha una grande comunità di 
    sviluppatori, documentazione completa e numerosi esempi
\end{itemize}
\paragraph*{Svantaggi}
\begin{itemize}
    \item \textbf{Overhead}: in alcuni casi, per progetti molto semplici 
    o con API leggere, questa completezza potrebbe introdurre overhead 
    non necessario, aumentando la complessità e i tempi di sviluppo
    \item \textbf{Performance}: Poiché  gestisce molte operazioni, come 
    la serializzazione, la validazione e la gestione dei permessi, in 
    alcuni casi può essere meno performante rispetto a soluzioni più 
    leggere per API molto grandi o ad alte prestazioni
    \item \textbf{Apprendimento complesso}: apprendimento che coinvolge 
    concetti avanzati, abilità e conoscenze che richiedono tempo, sforzo 
    e un certo livello di comprensione approfondita
\end{itemize}

\subsection{LlamaIndex}

\subsection{Haystack}



\section{Confronto}

\begin{table}[H]
    \renewcommand{\arraystretch}{1.7}
    \centering
    \begin{tabular}{|p{2.55cm}|p{4cm}|p{4cm}|p{4cm}|}
        \hline
        \textbf{Aspetto} & \textbf{\large Flask} & \textbf{\large Django} & \textbf{\large FastAPI} \\
        \hline
        \textbf{Ambito di applicazione} & Microframework per applicazioni leggere & Framework completo per applicazioni web & Ottimo per API ad alte prestazioni \\
        \hline
        \textbf{Sicurezza} & Funzionalità di base, personalizzabile & Sicurezza integrata (autenticazione, CSRF) & Sicurezza per API, richiede librerie esterne \\
        \hline
        \textbf{Flessibilità} & Altamente flessibile, poche convenzioni & Struttura rigida con molte convenzioni & Flessibile, ma con focus su API asincrone \\
        \hline
        \textbf{Prestazioni} & Molto buone & Prestazioni medio-alte, non ottimizzato per async & Eccellenti, soprattutto per operazioni asincrone \\
        \hline
        \textbf{Velocità di apprendimento} & Facile, ma richiede più lavoro per funzionalità avanzate & Curva di apprendimento più ripida & Facile per chi conosce async/await e type hints \\
        \hline
        \textbf{Community} & Molto ampia e matura & Grande e attiva & In rapida crescita, ma più piccola \\
        \hline
    \end{tabular}
    \caption{Tabella di confronto tecnico tra Flask, Django e FastAPI}
\end{table}

\section{Conclusioni}
\textbf{Flask} è stato scelto per la sua leggerezza e semplicità nel creare 
API RESTful. Essendo un micro-framework, permette di sviluppare 
rapidamente un backend senza imporre dei vincoli rigidi. 
Inoltre, la sua ampia documentazione e la sua flessibilità lo rendono 
ideale per prototipi e progetti in evoluzione.
Per quanto riguarda \textbf{Langchain}, invece, si tratta di una libreria
progettata per facilitare l'integrazione dei \textit{Large Language Models (LLMs)}
nei sistemi Software. Esso permette di gestire conversazioni, memoria
contestuale e connettori a database vettoriali, rendendo l'interazione 
con i modelli più strutturata e personalizzabile.

\end{document}
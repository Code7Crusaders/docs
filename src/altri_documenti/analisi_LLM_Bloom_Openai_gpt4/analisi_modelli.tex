\documentclass{article}
\usepackage{graphicx}
\usepackage{fancyhdr}
\usepackage{geometry}
\usepackage{setspace}
\usepackage{tikz}
\usepackage[italian]{babel}
\usepackage{hyperref}

% Margini della pagina
\geometry{a4paper, margin=1in}

% Intestazione personalizzata
\pagestyle{fancy}
\fancyhf{}
\fancyhead[L]{Code7Crusaders - Software Development Team}
\fancyhead[R]{\thepage}

% Spaziatura delle righe
\setstretch{1.2}

\begin{document}

% Sezione del titolo
\begin{titlepage}

    \AddToHookNext{shipout/background}{
    \begin{tikzpicture}[remember picture,overlay]
    \node at (current page.center) {
    \includegraphics[width=1.05\paperwidth]{../../img/background.png}
    };
    \end{tikzpicture}
    }

    \centering
    \vspace*{2cm}
    
    \includegraphics[width=0.3\textwidth]{../../img/logo/7Crusaders_logo.png} % Aggiungi il logo qui
    \vspace{1cm}
    
    {\Huge \textbf{Code7Crusaders}}\\
    \vspace{0.5cm}
    {\Large Software Development Team}\\
    \vspace{2cm}
    
    {\large \textbf{Documentazione Progetto}}\\
    \vspace{5cm}

    \textbf{Membri del Team:}\\
    Enrico Cotti Cottini, Gabriele Di Pietro, Tommaso Diviesti \\
    Francesco Lapenna, Matthew Pan, Eddy Pinarello, Filippo Rizzolo \\
    \vspace{0.5cm}
    
    {\large \textbf{Data:}} \today\\
    
    \vspace{1cm}
\end{titlepage}

% Indice
\newpage
\tableofcontents
\newpage

% Sezione Introduzione
\section{Introduzione}
\label{sec:introduzione}
Questo documento si pone l’obiettivo di confrontare i Modelli di Huggingface in particolare i modelli di BigScience Workshop (\href{https://huggingface.co/bigscience}{https://huggingface.co/bigscience}) 
e i modelli di OpenAi (\href{https://openai.com/}{https://openai.com/}) utilizzati tramite interfaccia LangChain (\href{https://python.langchain.com/docs/introduction/}{https://python.langchain.com/docs/introduction/}). In
particolare, verranno analizzate le caratteristiche, i vantaggi e gli svantaggi dei mezzi.

\end{document}

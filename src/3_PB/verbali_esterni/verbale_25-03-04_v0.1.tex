
%%% INTRO %%%%%%%%%%%%%%%%%%%%%%%%%%%%%%%%%%%%%%%%%%%%%%%%%%%%%%%%%%%%%%%%%%%%%%%%%%%%
% Template sia per verbali interni che esterni
% Segui i commenti "TODO" per ricordarti cosa modificare
% In caso di verbali esterni ricordati di settare isEsterno ad 1
%%%%%%%%%%%%%%%%%%%%%%%%%%%%%%%%%%%%%%%%%%%%%%%%%%%%%%%%%%%%%%%%%%%%%%%%%%%%%%%%%%%%%



%%% Settings %%%%%%%%%%%%%%%%%%%%%%%%%%%%%%%%%%%%%%%%%%%%%%%%%%%%%%%%%%%%%%%%%%%%%%%%%
\documentclass{article}

\usepackage{graphicx}  % serve per inserire immagini
\usepackage{fancyhdr}  % creazione header-footer
\usepackage{tabularx}  % serve per creare tabelle con colonne a larghezza variabile
\usepackage{ifthen}  % serve per mostrare cose diverse in base a condizioni
\usepackage{geometry}
\usepackage{setspace}
\usepackage{tikz}
\usepackage[italian]{babel}
\usepackage[hidelinks]{hyperref}
\usepackage{longtable}

% TODO: setta a 1 se il verbale è esterno, 0 se è interno
\newcommand{\isEsterno}{1}

% Margini della pagina
\geometry{a4paper, margin=1in}

% Intestazione personalizzata
\pagestyle{fancy}
\fancyhf{}
\fancyhead[L]{Code7Crusaders - Software Development Team}
\fancyhead[R]{\thepage}

% Spaziatura delle righe
\setstretch{1.2}

\begin{document}
%%%%%%%%%%%%%%%%%%%%%%%%%%%%%%%%%%%%%%%%%%%%%%%%%%%%%%%%%%%%%%%%%%%%%%%%%%%%%%%%%%%%%%



%%% Sezione del titolo %%%%%%%%%%%%%%%%%%%%%%%%%%%%%%%%%%%%%%%%%%%%%%%%%%%%%%%%%%%%%%%
\begin{titlepage}

    \AddToHookNext{shipout/background}{
        \begin{tikzpicture}[remember picture,overlay]
        \node at (current page.center) {
            \includegraphics{../../img/background.png}
        };
        \end{tikzpicture}
    }

    \centering
    \vspace*{2cm}
    
    \includegraphics[width=0.3\textwidth]{../../img/logo/7Crusaders_logo.png} % logo
    \vspace{1cm}
    
    {\Huge \textbf{Code7Crusaders}}\\
    \vspace{0.5cm}
    {\Large Software Development Team}\\
    \vspace{2cm}
    
    {\large \textbf{Incontro del 04/03/2025 con Ergon}}\\ % TODO: inserire titolo del verbale
    \vspace{5cm}                           % esempio: Riunione Settimanale 04/11/2024
    
    
    \textbf{Membri del Team:}\\
    Enrico Cotti Cottini, Gabriele Di Pietro, Tommaso Diviesti \\
    Francesco Lapenna, Matthew Pan, Eddy Pinarello, Filippo Rizzolo \\
    \vspace{0.5cm}
    
    \vspace{1cm}
\end{titlepage}
%%%%%%%%%%%%%%%%%%%%%%%%%%%%%%%%%%%%%%%%%%%%%%%%%%%%%%%%%%%%%%%%%%%%%%%%%%%%%%%%%%%%%%



% Versioni %%%%%%%%%%%%%%%%%%%%%%%%%%%%%%%%%%%%%%%%%%%%%%%%%%%%%%%%%%%%%%%%%%%%%%%%%%%
\newpage
\begin{table}[h!]
\centering
\textbf{Versioni} \\ % Titolo sopra la tabella
\vspace{2mm} % Spazio tra il titolo e la tabella
\begin{tabular}{|c|c|c|c|c|}
    \hline
    \textbf{Ver.} & \textbf{Data} & \textbf{Autore} & \textbf{Verificatore} & \textbf{Descrizione} \\
    \hline
    0.1 & 05/03/2025 & Francesco Lapenna & Nome Verificatore & Prima stesura del documento \\ 
    \hline                                  % TODO: inserire data, nomi e descrizione
\end{tabular}
\end{table}
%%%%%%%%%%%%%%%%%%%%%%%%%%%%%%%%%%%%%%%%%%%%%%%%%%%%%%%%%%%%%%%%%%%%%%%%%%%%%%%%%%%%%%



% Indice %%%%%%%%%%%%%%%%%%%%%%%%%%%%%%%%%%%%%%%%%%%%%%%%%%%%%%%%%%%%%%%%%%%%%%%%%%%%%
\newpage
\tableofcontents
%%%%%%%%%%%%%%%%%%%%%%%%%%%%%%%%%%%%%%%%%%%%%%%%%%%%%%%%%%%%%%%%%%%%%%%%%%%%%%%%%%%%%%



% Registro Presenze %%%%%%%%%%%%%%%%%%%%%%%%%%%%%%%%%%%%%%%%%%%%%%%%%%%%%%%%%%%%%%%%%%
\newpage
\section{Registro Presenze}
\textbf{Piattaforma della riunione:} Piattaforma Zoom \\
\textbf{Ora di Inizio} 09:30\\
\textbf{Ora di Fine} 10:00\\  % TODO: inserire orari ed eventualmente piattaforma
\\
\begin{tabular}{|c|c|c|}  % TODO: inserire ruoli e presenze
    \hline
    \textbf{Componente} & \textbf{Ruolo} & \textbf{Presenza}\\
    \hline
    Enrico Cotti Cottini & Responsabile & Presente \\ 
    \hline
    Gabriele Di Pietro & Redattore & Presente\\ 
    \hline
    Tommaso Diviesti & Redattore & Presente \\ 
    \hline 
    Francesco Lapenna & Redattore& Presente \\ 
    \hline
    Matthew Pan & Verificatore & Presente\\ 
    \hline 
    Eddy Pinarello & Redattore & Presente \\ 
    \hline 
    Filippo Rizzolo & Amministratore& Presente \\ 
    \hline 
\end{tabular}
% Presenze Rappresentanti Azienda %%%%%%%%%%%%%%%%%%%%%%%%%%%%%%%%%%%%%%%%%%%%%%%%%%%%
% non toccare, modifica invece la variabile isEsterno
\ifthenelse{\equal{\isEsterno}{1}}{
    \\
    \newline
    \newline
    \begin{tabular}{|c|c|}  % TODO: eventualmente modificare nomi rappresentanti
        \hline
        \textbf{Nome} & \textbf{Ruolo}\\
        \hline
        Gianluca Carlesso & Rappresentante Azienda \\
        \hline
    \end{tabular}
}{}
%%%%%%%%%%%%%%%%%%%%%%%%%%%%%%%%%%%%%%%%%%%%%%%%%%%%%%%%%%%%%%%%%%%%%%%%%%%%%%%%%%%%%%



% Sezione Verbale %%%%%%%%%%%%%%%%%%%%%%%%%%%%%%%%%%%%%%%%%%%%%%%%%%%%%%%%%%%%%%%%%%%%
\newpage
\section{Verbale}
    % TODO: per ogni punto discusso / attività svolta
    % inserire una sottosezione, sintesi ed eventuali decisioni

    \subsection{Negoziazione Requisiti}
    \textbf{Sintesi:} A causa di un ritardo nel progetto, come indicato nella correzione 
    del colloquio RTB, sono stati negoziati alcuni requisiti con l'azienda. \\
    In particolare i requisiti riguardanti il monitoraggio delle metriche da parte 
    dell'amministratore, i quali sono stati ritenuti non essenziali.
    \textbf{Decisioni:} I seguenti requisiti funzionali diventano facoltativi:

    %%%%%%%%%%%%%%%%%%%%%%%%%%%%%%%%%%%%%%%%%%%%%%%%%%%%%%%%%%%%%%%%%%%%%%%%%%%%%%%%%%
    \begin{longtable}{|>{\centering\arraybackslash}m{0.10\textwidth}|>{\centering\arraybackslash}m{0.20\textwidth}|>{\arraybackslash}m{0.6\textwidth}|}
    	\hline
    	\textbf{Codice} & \textbf{Fonte} & \textbf{Descrizione}\\\hline
    	\endfirsthead
    	\hline
    	\textbf{Codice} & \textbf{Fonte} & \textbf{Descrizione}\\\hline
    	\endhead
    	\hline
    	\textbf{RFO28} & Interno 			& L'amministratore deve poter accedere alla dashboard di monitoraggio delle metriche. \\
    	\hline
    	\textbf{RFO29} & Interno 			& L’accesso alla dashboard delle metriche delle run è consentito solo agli utenti con ruolo di amministratore. \\
    	\hline
    	\textbf{RFO30} & Interno 			& Dopo l’accesso da parte dell'amministratore, la pagina di gestione mostra la dashboard delle metriche delle run. \\
    	\hline
    	\textbf{RFD31} & Interno 			& L’amministratore deve poter selezionare criteri di filtro per visualizzare solo le run di interesse. \\
    	\hline
    	\textbf{RFD32} & Interno 			& Il sistema deve permettere la selezione di filtri come ID, nome, input, data di inizio e fine, errore, output, tag, numero di token, costo. \\
    	\hline
    	\textbf{RFF33} & Interno 			& Una volta selezionati i filtri, il sistema deve aggiornare la visualizzazione senza ricaricare l'intera pagina. \\
    	\hline
    	\textbf{RFO34} & Interno 			& Se nessun filtro è selezionato, il sistema mostra le prime dieci run per impostazione predefinita. \\
    	\hline
    	\textbf{RFD35} & Interno 			& Dopo aver applicato i filtri, l’amministratore deve poter visualizzare le metriche principali delle run selezionate. \\
    	\hline
    	\textbf{RFD36} & Interno 			& Il sistema deve mostrare le metriche principali delle run filtrate (ID, nome, input, data di inizio e fine, errore, output, tag, token totali, costo totale). \\
    	\hline
    	\textbf{RFF37} & Interno 			& La visualizzazione deve essere chiara e strutturata, con possibilità di ordinare le colonne. \\
    	\hline
    	\textbf{RFD47} & Interno 			& Le metriche delle run del chatbot devono essere esportabili in JSON. \\
    	\hline
    	\textbf{RFD48} & Interno 			& Le metriche della run devono includere ID univoco della run, nome assegnato alla sessione, dati di input elaborati dal modello, timestamp di avvio e completamento dell'esecuzione, eventuali errori incontrati, risultato generato dal modello, numero totale di token utilizzati e stima dei costi basata sul consumo di token. \\
    	\hline
    	\caption{Requisiti funzionali}
    \end{longtable}
    %%%%%%%%%%%%%%%%%%%%%%%%%%%%%%%%%%%%%%%%%%%%%%%%%%%%%%%%%%%%%%%%%%%%%%%%%%%%%%%%%%
    

    % \subsection{Attività 2}
    % \textbf{Sintesi:} Sintesi attività 2. \\
    % \textbf{Decisioni:} Decisioni attività 2.

    % ...

    % \subsection*{Conclusioni}  % TODO: inserire conclusioni della riunione
    % Prossima riunione pianificata per il \textbf{D° Month YYYY}.
%%%%%%%%%%%%%%%%%%%%%%%%%%%%%%%%%%%%%%%%%%%%%%%%%%%%%%%%%%%%%%%%%%%%%%%%%%%%%%%%%%%%%%



% Sezione Firme %%%%%%%%%%%%%%%%%%%%%%%%%%%%%%%%%%%%%%%%%%%%%%%%%%%%%%%%%%%%%%%%%%%%%%
% non toccare, modifica invece la variabile isEsterno
\ifthenelse{\equal{\isEsterno}{1}}{
    \begin{table}[b]
        \begin{tabular}{@{}p{.5in}p{4in}@{}}
            Data:  & \hrulefill \\
                   &     		\\
                   &     		\\
            Firma: & \hrulefill \\
        \end{tabular}
        \end{table}
}{}
%%%%%%%%%%%%%%%%%%%%%%%%%%%%%%%%%%%%%%%%%%%%%%%%%%%%%%%%%%%%%%%%%%%%%%%%%%%%%%%%%%%%%%


\end{document} 

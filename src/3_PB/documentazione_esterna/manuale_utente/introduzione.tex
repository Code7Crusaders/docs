\section{Introduzione}

\subsection{Scopo del manuale}
Il seguente manuale ha lo scopo di fornire le informazioni necessarie al corretto utilizzo dell'assistente virtuale. E' rivolto a tutti gli utenti interessati al suo utilizzo con l'obiettivo di fornire una guida dettagliata e ben strutturata sul suo funzionamento. Questo garantisce che gli utenti possano sfruttare al meglio le potenzialità.

\subsection{Scopo del progetto}
L'obiettivo principale del progetto è fornire un assistente virtuale in grado di supportare i clienti seguiti da \textit{Ergon Informatica}. Questa soluzione mira a garantire un'assistenza costante agli utenti in difficoltà, riducendo il carico di lavoro degli operatori e migliorando l'efficienza del servizio. L'assistente virtuale è progettato per essere intuitivo e facile da utilizzare, assicurando un'interazione fluida e naturale con gli utenti. Il prodotto finale sarà un prototipo funzionante del sistema, che potrà essere successivamente perfezionato e ampliato in base ai feedback degli utenti e alle esigenze del mercato.

\subsection{Glossario}
Per evitare ambiguità e problemi di comprensione all'interno del documento, verrà presentato anche un glossario. Al suo interno, ogni termine presente sarà seguito da una descrizione che ne chiarirà il significato e sarà indicato applicando uno stile specifico:
\begin{itemize}
    \item \textbf{Termine}: aggiungengo una "G" all'apice della parola;
    \item \textbf{Termine}: fornendo un link al glossario online;
\end{itemize}

\subsection{Riferimenti}
    \subsubsection{Normativi}
        \begin{itemize}
            \item \textbf{Capitolato d'appalto C7:} LLM: Assistente virtuale \\ \url{https://www.math.unipd.it/~tullio/IS-1/2024/Progetto/C7.pdf}
            \item \textbf{Norme di Progetto v.2.0} \\ \url{https://code7crusaders.github.io/docs/PB/documentazione_interna/norme_di_progetto.html}
            \item \textbf{Regole del progetto didattico} \\ \url{https://www.math.unipd.it/~tullio/IS-1/2024/Dispense/PD1.pdf}
        \end{itemize}
    \subsubsection{Informativi}
        \begin{itemize}
            \item \textbf{Analisi dei Requisiti v3.0} \\ \url{https://code7crusaders.github.io/docs/PB/documentazione_esterna/analisi_dei_requisiti.html}
            \item \textbf{Specifica Tecnice v.1.0} \\ \url{https://code7crusaders.github.io/docs/PB/documentazione_esterna/specifica_tecnica.html}
            \item \textbf{OpeAI} [Ultima consultazione: 2025-04-01] \\ \url{https://platform.openai.com/docs/overview} 
            \item \textbf{LangChain} [Ultima consultazione: 2025-04-01] \\ \url{https://python.langchain.com/docs/introduction/} 
            \item \textbf{Docker} [Ultima consultazione: 2025-04-01] \\ \url{https://docs.docker.com/} 
        \end{itemize}
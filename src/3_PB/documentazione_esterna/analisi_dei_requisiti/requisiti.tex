\section{Requisiti}
In questa sezione vengono presentati i requisiti emersi durante l'attività di analisi, 
condotta a partire dai casi d'uso, dall'esame del capitolato d'appalto e dagli incontri, 
sia interni che con il proponente. 

\subsection{Classificazione dei requisiti}
I requisiti sono classificati in tre categorie principali:  
\begin{itemize}
	\item \textbf{Funzionali}: riguardano l'usabilità del prodotto finale;  
	\item \textbf{Di qualità}: includono gli strumenti e la documentazione da fornire;  
	\item \textbf{Di vincolo}: fanno riferimento alle tecnologie da utilizzare.
\end{itemize}
Ciascun requisito è indicato da:
\begin{itemize}
	\item \textbf{Codice Identificativo}: codice univoco che identifica il requisito;
	\item \textbf{Descrizione}: breve spiegazione del requisito;
	\item \textbf{Fonte}: origine del requisito (es. capitolato, interno, ecc.);
	\item \textbf{Priorità}: importanza del requisito rispetto agli altri;
\end{itemize} 

\subsection{Fonti dei requisiti}
I requisiti sono stati identificati a partire dalle seguenti fonti:
\begin{itemize}
	\item \textbf{Capitolato}: requisiti individuati tramite analisi del capitolato;
	\item \textbf{Interno}: requisiti individuati durante riunioni interne al gruppo di lavoro;
	\item \textbf{Esterno}: requisiti individuati durante incontri con il proponente;
	\item \href{https://code7crusaders.github.io/docs/RTB/documentazione_interna/glossario.html#piano-di-qualifica}{\textbf{Piano di Qualifica}\textsuperscript{G}}: requisiti necessari per rispettare standard di qualità definiti nel documento \href{https://code7crusaders.github.io/docs/RTB/documentazione_interna/glossario.html#piano-di-qualifica}{Piano di Qualifica\textsuperscript{G}};
	\item \href{https://code7crusaders.github.io/docs/RTB/documentazione_interna/glossario.html#norme-di-progetto}{\textbf{Norme di Progetto}\textsuperscript{G}}: requisiti necessari per rispettare le norme di progetto definite nel documento \href{https://code7crusaders.github.io/docs/RTB/documentazione_interna/glossario.html#norme-di-progetto}{Norme di Progetto\textsuperscript{G}}\href{https://code7crusaders.github.io/docs/RTB/documentazione_interna/glossario.html#norme-di-progetto}{norme di progetto\textsuperscript{G}} definite nel documento Norme di Progetto;
\end{itemize}
\newpage
\subsection{Codifica dei requisiti}
I requisiti sono codificati come segue: \textbf{R[Tipo][Importanza][Numero]}
\newline
Dove \textbf{Tipo} può essere:
\begin{itemize}
	\item \textbf{F (funzionale)}
	\item \textbf{Q (di qualità)}
	\item \textbf{V (di vincolo)}
\end{itemize}
\textbf{Importanza} può essere:
\begin{itemize}
	\item \textbf{O (obbligatorio)}
	\item \textbf{D (desiderabile)}
	\item \textbf{F (facoltativo)}
\end{itemize}
\textbf{Numero} è un numero identificativo univoco del requisito.

\textbf{Esempio}:
\begin{itemize}
	\item \textbf{RFO1}: requisito funzionale obbligatorio numero 1
	\item \textbf{RQD2}: requisito di qualità desiderabile numero 2
	\item \textbf{RVF3}: requisito di vincolo facoltativo numero 3
\end{itemize}

\pagebreak
\subsection{Requisiti funzionali}
\begin{longtable}{|>{\centering\arraybackslash}m{0.10\textwidth}|>{\centering\arraybackslash}m{0.20\textwidth}|>{\arraybackslash}m{0.6\textwidth}|}
	\hline
	\textbf{Codice} & \textbf{Fonte} & \textbf{Descrizione}\\\hline
	\endfirsthead
	\hline
	\textbf{Codice} & \textbf{Fonte} & \textbf{Descrizione}\\\hline
	\endhead
	\hline
	\textbf{RFO1} & Capitolato, Interno & L'amministratore inserisce dalla pagina di gestione i dati semantici aziendali da cui apprendere la conoscenza da file in formato .pdf. \\
	\hline
	\textbf{RFO2} & Capitolato, Interno & L'amministratore inserisce dalla pagina di gestione i dati semantici aziendali da cui apprendere la conoscenza da file in formato .txt. \\
	\hline
	\textbf{RFO3} & Capitolato 			& I testi recuperati dai documenti verranno suddivisi in blocchi, ovvero pezzi più piccoli di dati che rappresentano una piccola porzione del contesto. \\
	\hline
	\textbf{RFO4} & Capitolato 			& I vettori generati verranno memorizzati all’interno di un database vettoriale e opportunamente indicizzati. \\
	\hline
	\textbf{RFO5} & Capitolato, Esterno & Da un’interfaccia utente della web app, viene catturata una domanda da parte dell’utente. \\
	\hline
	\textbf{RFO6} & Capitolato 			& La domanda viene inoltrata al sistema attraverso delle API REST risiedenti in un Web Server. \\
	\hline
	\textbf{RFO7} & Capitolato 			& La rappresentazione vettoriale viene utilizzata per effettuare una ricerca all’interno del database vettoriale da dove vengono reperiti i vettori più simili. \\
	\hline
	\textbf{RFO8} & Capitolato 			& La domanda viene inviata al sistema LLM tramite API. \\
	\hline
	\textbf{RFO9} & Capitolato 			& Viene attesa la risposta dall'LLM tramite API. \\
	\hline
	\textbf{RFO10} & Capitolato 		& Attraverso API REST, il sistema inoltra la risposta all'account dell’utente. \\
	\hline
	\textbf{RFO11} & Interno 			& L'utente deve essere in grado di ottenere informazioni riguardo un prodotto attraverso la conversazione con il bot. \\
	\hline
	\textbf{RFO12} & Interno 			& L'utente deve essere in grado di ottenere informazioni riguardo una serie di prodotti attraverso la conversazione con il bot. \\
	\hline
	\textbf{RFO13} & Interno 			& La conversazione tra utente e bot deve essere salvata. \\
	\hline
	\textbf{RFO14} & Interno 			& L'utente deve essere in grado di visualizzare una delle conversazioni precedentemente salvate. \\
	\hline
	\textbf{RFO15} & Interno 			& L'utente deve essere in grado di riprendere una delle conversazioni precedentemente salvata. \\
	\hline
	\textbf{RFO16} & Interno 			& L'utente o l'amministratore devono poter accedere al sistema inserendo Username e Password. \\
	\hline
	\textbf{RFO17} & Interno 			& L'utente si registra inserendo Username e Password. \\
	\hline
	\textbf{RFO18} & Interno 			& Gli input del form di registrazione devono essere sanificati per prevenire attacchi SQL Injection. \\
	\hline
	\textbf{RFO19} & Interno 			& Gli input del form di accesso devono essere sanificati per prevenire attacchi SQL Injection. \\
	\hline
	\textbf{RFO20} & Interno, Esterno 	& L'utente deve essere in grado di dare un feedback (thumbsup/thumbsdown) sulla qualità della conversazione dopo averla provata. \\
	\hline
	\textbf{RFO21} & Esterno 			& L’accesso alla dashboard dei "template di domanda e risposta" è consentito solo agli utenti con ruolo di amministratore. \\
	\hline
	\textbf{RFO22} & Esterno 			& Dopo l’accesso da parte dell'amministratore, la pagina di gestione mostra la dashboard dei "template di domanda e risposta". \\
	\hline
	\textbf{RFO23} & Esterno 			& Un "template di domanda e risposta" è formato da una domanda (possibilmente una domanda posta frequentemente che l'amministratore decide di inserire per risparmiare una chiamata al modello) associata ad una corrispondente risposta. \\
	\hline
	\textbf{RFO24} & Esterno 			& L'amministratore deve essere in grado di creare un template, che è formato da una domanda associata ad una corrispondente risposta. \\
	\hline
	\textbf{RFO25} & Esterno 			& L'amministratore deve essere in grado di modificare uno dei template esistenti. \\
	\hline
	\textbf{RFO26} & Interno 			& L'amministratore deve essere in grado di eliminare un template esistente. \\
	\hline
	\textbf{RFO27} & Interno, Esterno 	& Il sistema deve poter fermare la creazione di un template invalido, ovvero quando il template non rispetta il formato Json. \\
	\hline
	\textbf{RFF28} & Interno 			& L'amministratore deve poter accedere alla dashboard di monitoraggio delle metriche. \\
	\hline
	\textbf{RFF29} & Interno 			& L’accesso alla dashboard delle metriche delle run è consentito solo agli utenti con ruolo di amministratore. \\
	\hline
	\textbf{RFF30} & Interno 			& Dopo l’accesso da parte dell'amministratore, la pagina di gestione mostra la dashboard delle metriche delle run. \\
	\hline
	\textbf{RFF31} & Interno 			& L’amministratore deve poter selezionare criteri di filtro per visualizzare solo le run di interesse. \\
	\hline
	\textbf{RFF32} & Interno 			& Il sistema deve permettere la selezione di filtri come ID, nome, input, data di inizio e fine, errore, output, tag, numero di token, costo. \\
	\hline
	\textbf{RFF33} & Interno 			& Una volta selezionati i filtri, il sistema deve aggiornare la visualizzazione senza ricaricare l'intera pagina. \\
	\hline
	\textbf{RFF34} & Interno 			& Se nessun filtro è selezionato, il sistema mostra le prime dieci run per impostazione predefinita. \\
	\hline
	\textbf{RFF35} & Interno 			& Dopo aver applicato i filtri, l’amministratore deve poter visualizzare le metriche principali delle run selezionate. \\
	\hline
	\textbf{RFF36} & Interno 			& Il sistema deve mostrare le metriche principali delle run filtrate (ID, nome, input, data di inizio e fine, errore, output, tag, token totali, costo totale). \\
	\hline
	\textbf{RFF37} & Interno 			& La visualizzazione deve essere chiara e strutturata, con possibilità di ordinare le colonne. \\
	\hline
	\textbf{RFO38} & Interno 			& L'amministratore deve poter visualizzare i feedback dati dagli utenti. \\
	\hline
	\textbf{RFO39} & Esterno 			& Il sistema deve poter rifiutare l'importazione dati di file non compatibili, ovvero file non nel formato pdf o txt. \\
	\hline
	\textbf{RFO40} & Interno 			& L'utente deve poter eliminare una conversazione precedentemente effettuata. \\
	\hline
	\textbf{RFO41} & Esterno 			& L'utente deve poter mandare richieste di assistenza per poter parlare con un operatore umano. \\
	\hline
	\textbf{RFO42} & Interno 			& L’accesso alla dashboard delle richieste di assistenza è consentito solo agli utenti con ruolo di amministratore. \\
	\hline
	\textbf{RFO43} & Interno 			& Dopo l’accesso da parte dell'amministratore, la pagina di gestione mostra la dashboard delle richieste di assistenza. \\
	\hline
	\textbf{RFO44} & Esterno 			& L'amministratore deve poter visualizzare le richieste di assistenza ricevute da parte dell'utente. \\
	\hline
	\textbf{RFO45} & Interno 			& L'amministratore deve poter segnalare ad altri amministratori che una richiesta è stata presa in carico. \\
	\hline
	\textbf{RFD46} & Esterno, Interno 	& L'amministratore deve essere in grado di poter rispondere all'utente tramite contatto via e-mail. \\
	\hline
	\textbf{RFF47} & Interno 			& Le metriche delle run del chatbot devono essere esportabili in JSON. \\
	\hline
	\textbf{RFF48} & Interno 			& Le metriche della run devono includere ID univoco della run, nome assegnato alla sessione, dati di input elaborati dal modello, timestamp di avvio e completamento dell'esecuzione, eventuali errori incontrati, risultato generato dal modello, numero totale di token utilizzati e stima dei costi basata sul consumo di token. \\
	\hline
	\textbf{RFO49} & Interno 			& Il bot per rispondere a una domanda deve ricordarsi i messaggi precedenti nella singola conversazione. \\
	\hline
	\textbf{RFD50} & Interno 			& Il sistema deve notificare l'utente quando la memoria per le chat salvate è piena e non è possibile salvare ulteriori conversazioni. \\
	\hline
	\textbf{RFO51} & Interno 			& L'utente seleziona una delle domande tra quelle predefinite. \\
	\hline
	\textbf{RFO52} & Interno 			& L'utente deve essere in grado di visualizzare una lista delle conversazioni precedentemente salvate. \\
	\hline
	\textbf{RFO53} & Interno 			& La lunghezza massima dell'username è di 256 caratteri. \\
	\hline
	\textbf{RFO54} & Interno 			& La lunghezza massima della password è di 256 caratteri. \\
	\hline
	\textbf{RFO55} & Interno 			& Il Sistema rifiuta la registrazione di un nuovo account con username già presente. \\
	\hline
	\caption{Requisiti funzionali}
\end{longtable}

\pagebreak
\subsection{Requisiti qualitativi}
\begin{longtable}{|>{\centering\arraybackslash}m{0.10\textwidth}|>{\centering\arraybackslash}m{0.20\textwidth}|>{\centering\arraybackslash}m{0.6\textwidth}|}
	\hline
	\textbf{Codice} & \textbf{Fonte} & \textbf{Descrizione}\\\hline
	\endfirsthead
	\hline
	\textbf{Codice} & \textbf{Fonte} & \textbf{Descrizione}\\\hline
	\endhead
	\hline
	\textbf{RQO1} & Capitolato, \href{https://code7crusaders.github.io/docs/RTB/documentazione_interna/glossario.html#piano-di-qualifica}{Piano di Qualifica\textsuperscript{G}} & Schema di progettazione della base di dati \\
	\hline
	\textbf{RQO2} & Capitolato, \href{https://code7crusaders.github.io/docs/RTB/documentazione_interna/glossario.html#piano-di-qualifica}{Piano di Qualifica\textsuperscript{G}} & Codice prodotto in formato sorgente reso disponibile tramite repository pubblici \\
	\hline
	\textbf{RQO3} & \href{https://code7crusaders.github.io/docs/RTB/documentazione_interna/glossario.html#piano-di-qualifica}{Piano di Qualifica\textsuperscript{G}} & Documentazione riassuntiva delle metriche e dei risultati\\
	\hline
	\textbf{RQO4} & \href{https://code7crusaders.github.io/docs/RTB/documentazione_interna/glossario.html#piano-di-qualifica}{Piano di Qualifica\textsuperscript{G}} & Il software deve essere testato con una copertura di codice minima dell'80\% e una copertura dei rami dell'80\%, con un obiettivo ottimale del 100\% \\
	\hline
	\textbf{RQO5} & \href{https://code7crusaders.github.io/docs/RTB/documentazione_interna/glossario.html#piano-di-qualifica}{Piano di Qualifica\textsuperscript{G}} & Il 90\% dei test deve essere superato come requisito minimo, mentre l'obiettivo ottimale è il 100\% \\
	\hline
	\textbf{RQO6} & \href{https://code7crusaders.github.io/docs/RTB/documentazione_interna/glossario.html#piano-di-qualifica}{Piano di Qualifica\textsuperscript{G}} & La metodologia di sviluppo deve seguire il paradigma del Test Driven Development (TDD), garantendo che il codice venga scritto partendo dai test \\
	\hline
	\caption{Requisiti qualitativi}
\end{longtable}


\pagebreak
\subsection{Requisiti di vincolo}
\begin{longtable}{|>{\centering\arraybackslash}m{0.10\textwidth}|>{\centering\arraybackslash}m{0.20\textwidth}|>{\centering\arraybackslash}m{0.60\textwidth}|}
	\hline
	\textbf{Codice} & \textbf{Fonte} & \textbf{Descrizione}\\\hline
	\endfirsthead
	\hline
	\textbf{Codice} & \textbf{Fonte} & \textbf{Descrizione}\\\hline
	\endhead
	\hline
	\textbf{RVO1} & Capitolato & Il chatbot deve rispondere con il contesto dato dai file di allenamento (pdf o file di testo inseriti)\\
	\hline
	\textbf{RVO2} & Capitolato & \href{https://code7crusaders.github.io/docs/RTB/documentazione_interna/glossario.html#llm-large-language-model}{LLM\textsuperscript{G}} deve essere integrato tramite API\\
	\hline
	\textbf{RVO3} & Interno (Analisi dei modelli) & LLM utilizzato deve essere quello di OpenAI\\
	\hline
	\textbf{RVO4} & Capitolato & Deve essere usato un database relazionale\\
	\hline
	\textbf{RVO5} & Interno (Analisi del Database) & Deve essere gestito il salvataggio delle chat precedenti con tutti i messaggi in esse tramite un database relazionale con PostgreSQL\\
	\hline
	\textbf{RVO6} & Capitolato & Deve essere implementato un database vettoriale\\
	\hline
	\textbf{RVO7} & Interno (Analisi dei modelli) & Deve essere implementato un database vettoriale FAISS per poter rendere possibile la ricerca con contesto dall'LLM\\
	\hline
	\textbf{RVO8} & Capitolato & Deve essere implementato un \href{https://code7crusaders.github.io/docs/RTB/documentazione_interna/glossario.html#embedding}{embedding\textsuperscript{G}} model\\
	\hline
	\textbf{RVO9} & Interno (Analisi dei modelli) & L'embedding model deve essere quello di OpenAI\\
	\hline
	\textbf{RVO10} & Capitolato & Deve essere implementata una WebApp che permetta di comunicare con il chatbot\\
	\hline
	\textbf{RVO11} & Interno (Analisi \href{https://code7crusaders.github.io/docs/RTB/documentazione_interna/glossario.html#frontend}{Frontend\textsuperscript{G}}) & L’interfaccia deve essere costruita utilizzando componenti funzionali React.\\
	\hline
	\textbf{RVO12} & Interno & Si deve creare un backend che gestisca le chiamate HTTP, il database vettoriale e il database relazionale con Flask.\\
	\hline
	\textbf{RVO13} & Interno & La gestione dello stato locale deve essere implementata tramite useState.\\
	\hline
	\textbf{RVO14} & Interno & La WebApp deve utilizzare React Router per gestire la navigazione tra le pagine.\\
	\hline
	\textbf{RVO15} & Interno & Gli stili devono essere gestiti tramite CSS inline o con className per garantire modularità.\\
	\hline
	\textbf{RVO16} & Interno & La comunicazione tra componenti deve essere gestita inviando funzioni come \href{https://code7crusaders.github.io/docs/RTB/documentazione_interna/glossario.html#props}{props\textsuperscript{G}}.\\
	\hline
	\textbf{RVO17} & Interno & La WebApp deve essere responsiva e adattarsi dinamicamente alle dimensioni della finestra.\\
	\hline
	\textbf{RVO18} & Interno (Analisi Vettoriale) & La gestione dei blocchi di testo vettorializzati deve essere gestita tramite Faiss\\
	\hline
	\textbf{RVD19} & Interno (Analisi Backend) & Le metriche delle run del chatbot devono essere recuperate tramite Langsmith\\
	\hline
	\textbf{RVO20} & Interno (Analisi Backend) & Bisogna usare la libreria LangChain per la interazione con i modelli LLM e Embedding\\
	\hline
	\caption{Requisiti di vincolo}
\end{longtable}


\pagebreak
\subsection{Tracciamento}
\subsubsection{Requisito - Fonte}
\begin{longtable}{|>{\centering\arraybackslash}m{0.40\textwidth}|>{\centering\arraybackslash}m{0.4\textwidth}|}
	\hline
	\textbf{Requisito} & \textbf{Fonte} 
	\endfirsthead
	\hline
	\textbf{Requisito} & \textbf{Fonte} 
	\endhead
	\hline
	\textbf{RFO1} & Capitolato, Interno \\\hline
	\textbf{RFO2} & Capitolato, Interno \\\hline
	\textbf{RFO3} & Capitolato \\\hline
	\textbf{RFO4} & Capitolato \\\hline
	\textbf{RFO5} & Capitolato, Esterno \\\hline
	\textbf{RFO6} & Capitolato \\\hline
	\textbf{RFO7} & Capitolato \\\hline
	\textbf{RFO8} & Capitolato \\\hline
	\textbf{RFO9} & Capitolato \\\hline
	\textbf{RFO10} & Capitolato \\\hline
	\textbf{RFO11} & Interno \\\hline
	\textbf{RFO12} & Interno \\\hline
	\textbf{RFO13} & Interno \\\hline
	\textbf{RFO14} & Interno \\\hline
	\textbf{RFO15} & Interno \\\hline
	\textbf{RFO16} & Interno \\\hline
	\textbf{RFO17} & Interno \\\hline
	\textbf{RFO18} & Interno \\\hline
	\textbf{RFO19} & Interno \\\hline
	\textbf{RFO20} & Interno, Esterno \\\hline
	\textbf{RFO21} & Esterno \\\hline
	\textbf{RFO22} & Esterno \\\hline
	\textbf{RFO23} & Esterno \\\hline
	\textbf{RFO24} & Esterno \\\hline
	\textbf{RFO25} & Esterno \\\hline
	\textbf{RFO26} & Interno \\\hline
	\textbf{RFO27} & Interno, Esterno \\\hline
	\textbf{RFF28} & Interno \\\hline
	\textbf{RFF29} & Interno \\\hline
	\textbf{RFF30} & Interno \\\hline
	\textbf{RFF31} & Interno \\\hline
	\textbf{RFF32} & Interno \\\hline
	\textbf{RFF33} & Interno \\\hline
	\textbf{RFF34} & Interno \\\hline
	\textbf{RFF35} & Interno \\\hline
	\textbf{RFF36} & Interno \\\hline
	\textbf{RFF37} & Interno \\\hline
	\textbf{RFO38} & Interno \\\hline
	\textbf{RFO39} & Esterno \\\hline
	\textbf{RFO40} & Interno \\\hline
	\textbf{RFO41} & Esterno \\\hline
	\textbf{RFO42} & Interno \\\hline
	\textbf{RFO43} & Interno \\\hline
	\textbf{RFO44} & Esterno \\\hline
	\textbf{RFO45} & Interno \\\hline
	\textbf{RFD46} & Esterno, Interno \\\hline
	\textbf{RFF47} & Interno \\\hline
	\textbf{RFF48} & Interno \\\hline
	\textbf{RFO49} & Interno \\\hline
	\textbf{RFD50} & Interno \\\hline
	\textbf{RFO51} & Interno \\\hline
	\textbf{RFO52} & Interno \\\hline
	\textbf{RFO53} & Interno \\\hline
	\textbf{RFO54} & Interno \\\hline
	\textbf{RFO55} & Interno \\\hline
    
	\textbf{RQO1}            & Capitolato, \href{https://code7crusaders.github.io/docs/RTB/documentazione_interna/glossario.html#piano-di-qualifica}{Piano di Qualifica\textsuperscript{G}}\\\hline
	\textbf{RQO2}            & Capitolato, \href{https://code7crusaders.github.io/docs/RTB/documentazione_interna/glossario.html#piano-di-qualifica}{Piano di Qualifica\textsuperscript{G}}\\\hline
	\textbf{RQO3}            & \href{https://code7crusaders.github.io/docs/RTB/documentazione_interna/glossario.html#piano-di-qualifica}{Piano di Qualifica\textsuperscript{G}}\\\hline
    \textbf{RQO4}            & \href{https://code7crusaders.github.io/docs/RTB/documentazione_interna/glossario.html#piano-di-qualifica}{Piano di Qualifica\textsuperscript{G}}\\\hline
	\textbf{RQO5}            & \href{https://code7crusaders.github.io/docs/RTB/documentazione_interna/glossario.html#piano-di-qualifica}{Piano di Qualifica\textsuperscript{G}}\\\hline
	\textbf{RQO6}            & \href{https://code7crusaders.github.io/docs/RTB/documentazione_interna/glossario.html#piano-di-qualifica}{Piano di Qualifica\textsuperscript{G}}\\\hline

    \textbf{RVO1}			 & Capitolato \\\hline
	\textbf{RVO2}			 & Capitolato \\\hline
	\textbf{RVO3}			 & Interno (Analisi dei modelli)\\\hline
	\textbf{RVO4}			 & Capitolato \\\hline
	\textbf{RVO5}			 & Interno (Analisi Database) \\\hline
	\textbf{RVO6}			 & Capitolato \\\hline
	\textbf{RVO7}			 & Interno (Analisi dei modelli) \\\hline
	\textbf{RVO8}			 & Capitolato \\\hline
	\textbf{RVF9}			 & Interno (Analisi dei modelli) \\\hline
	\textbf{RVF10}			 & Capitolato \\\hline
	\textbf{RVD11}			 & Interno (Analisi \href{https://code7crusaders.github.io/docs/RTB/documentazione_interna/glossario.html#frontend}{Frontend\textsuperscript{G}})\\\hline
	\textbf{RVO12}			 & Interno\\\hline
	\textbf{RVO13}			 & Interno\\\hline
	\textbf{RVO14}			 & Interno\\\hline
	\textbf{RVO16}           & Interno  \\\hline
	\textbf{RVO15}           & Interno  \\\hline
	\textbf{RVO17}           & Interno  \\\hline
	\textbf{RVO18}           & Interno (Analisi Vettoriale)\\\hline
	\textbf{RVD19}           & Interno (Analisi Backend)\\\hline
	\textbf{RVO20}           & Interno (Analisi Backend)\\\hline
	\caption{Requisito - Fonte}
\end{longtable}
\pagebreak
\subsection{Caso d'uso - Requisito}
\begin{longtable}{|>{\centering\arraybackslash}m{0.40\textwidth}|>{\centering\arraybackslash}m{0.4\textwidth}|}
	\hline
	\textbf{Caso d'uso} & \textbf{Requisito}\\
	\endfirsthead
	\hline
	\textbf{U.C.1} & \textbf{RFO5} \\\hline
	\textbf{U.C.2} & \textbf{RFO3, RFO6, RFO7, RFO8, RFO9, RFO10, RFO11, RFO12, RFO49} \\\hline
	\textbf{U.C.2.1} & \textbf{RFO7, RFO8, RFO9} \\\hline
	\textbf{U.C.3} & \textbf{RFO5, RFO51} \\\hline
	\textbf{U.C.4} & \textbf{RFO52} \\\hline
	\textbf{U.C.5} & \textbf{RFO14} \\\hline
	\textbf{U.C.6} & \textbf{RFO16} \\\hline
	\textbf{U.C.6.1} & \textbf{RFO16} \\\hline
	\textbf{U.C.6.2} & \textbf{RFO16} \\\hline
	\textbf{U.C.6.3} & \textbf{RFO16} \\\hline
	\textbf{U.C.6.4} & \textbf{RFO16} \\\hline
	\textbf{U.C.6.5} & \textbf{RFO16, RFO19} \\\hline
	\textbf{U.C.7} & \textbf{RFO17} \\\hline
	\textbf{U.C.7.1} & \textbf{RFO17} \\\hline
	\textbf{U.C.7.2} & \textbf{RFO17} \\\hline
	\textbf{U.C.7.3} & \textbf{RFO17, RFO18} \\\hline
	\textbf{U.C.7.4} & \textbf{RFO17, RFO53} \\\hline
	\textbf{U.C.7.5} & \textbf{RFO17, RFO54} \\\hline
	\textbf{U.C.7.6} & \textbf{RFO17, RFO55} \\\hline
	\textbf{U.C.8} & \textbf{RFO13} \\\hline
	\textbf{U.C.8.1} & \textbf{RFD50} \\\hline
	\textbf{U.C.9} & \textbf{RFO20} \\\hline
	\textbf{U.C.10} & \textbf{RFO24} \\\hline
	\textbf{U.C.11} & \textbf{RFO25} \\\hline
	\textbf{U.C.12} & \textbf{RFO26} \\\hline
	\textbf{U.C.13} & \textbf{RFF27} \\\hline
	\textbf{U.C.14} & \textbf{RFF28, RFF29, RFF30, RFF31, RFF32, RFF33, RFF34, RFF35, RFF36, RFF37, RFF48} \\\hline
	\textbf{U.C.15} & \textbf{RFO38} \\\hline
	\textbf{U.C.16} & \textbf{RFO1, RFO2, RFO4} \\\hline
	\textbf{U.C.16.1} & \textbf{RFO1, RFO2, RFO39} \\\hline
	\textbf{U.C.17} & \textbf{RFF47, RFF48} \\\hline
	\textbf{U.C.18} & \textbf{RFO40} \\\hline
	\textbf{U.C.19} & \textbf{RFO15} \\\hline
	\textbf{U.C.20} & \textbf{RFO42, RFO43, RFO44, RFD46} \\\hline
	\textbf{U.C.21} & \textbf{RFO45} \\\hline
	\textbf{U.C.22} & \textbf{RFO41} \\\hline
	\textbf{U.C.23} & \textbf{RFO21, RFO22, RFO23} \\\hline
	\caption{Caso d'uso - Requisito}
\end{longtable}
%%% Settings %%%%%%%%%%%%%%%%%%%%%%%%%%%%%%%%%%%%%%%%%%%%%%%%%%%%%%%%%%%%%%%%%%%%%%%%%
\documentclass{article}

\usepackage{graphicx}  % serve per inserire immagini
\usepackage{fancyhdr}  % creazione header-footer
\usepackage{tabularx}  % serve per creare tabelle con colonne a larghezza variabile
\usepackage{ifthen}  % serve per mostrare cose diverse in base a condizioni
\usepackage{geometry}
\usepackage{setspace}
\usepackage{tikz}
\usepackage[italian]{babel}
\usepackage[hidelinks]{hyperref}
\usepackage{pgfgantt}  % per i diagrammi di Gantt
\usepackage{eurosym}
\usepackage{float}
\usepackage{longtable}


% setta a 1 se il verbale è esterno, 0 se è interno
\newcommand{\isEsterno}{1}

% Margini della pagina
\geometry{a4paper, margin=1in}

% Intestazione personalizzata
\pagestyle{fancy}
\fancyhf{}
\fancyhead[L]{Code7Crusaders - Software Development Team}
\fancyhead[R]{\thepage}

% Spaziatura delle righe
\setstretch{1.2}

\begin{document}
\setcounter{secnumdepth}{5} % Permette la numerazione fino a \subparagraph
%%%%%%%%%%%%%%%%%%%%%%%%%%%%%%%%%%%%%%%%%%%%%%%%%%%%%%%%%%%%%%%%%%%%%%%%%%%%%%%%%%%%%%



%%% Sezione del titolo %%%%%%%%%%%%%%%%%%%%%%%%%%%%%%%%%%%%%%%%%%%%%%%%%%%%%%%%%%%%%%%
\begin{titlepage}

    \AddToHookNext{shipout/background}{
        \begin{tikzpicture}[remember picture,overlay]
        \node at (current page.center) {
            \includegraphics{../../../img/background.png}
        };
        \end{tikzpicture}
    }

    \centering
    \vspace*{2cm}
    
    \includegraphics[width=0.3\textwidth]{../../../img/logo/7Crusaders_logo.png} % logo
    \vspace{1cm}
    
    {\Huge \textbf{Code7Crusaders}}\\
    \vspace{0.5cm}
    {\Large Software Development Team}\\
    \vspace{2cm}
    
    {\large \href{https://code7crusaders.github.io/docs/RTB/documentazione_interna/glossario.html#piano-di-progetto}{\textbf{Piano di Qualifica}\textsuperscript{G}}}\\
    \vspace{5cm}
    
    
    \textbf{Membri del Team:}\\
    Enrico Cotti Cottini, Gabriele Di Pietro, Tommaso Diviesti \\
    Francesco Lapenna, Matthew Pan, Eddy Pinarello, Filippo Rizzolo \\
    \vspace{0.5cm}
    
    \vspace{1cm}
\end{titlepage}
%%%%%%%%%%%%%%%%%%%%%%%%%%%%%%%%%%%%%%%%%%%%%%%%%%%%%%%%%%%%%%%%%%%%%%%%%%%%%%%%%%%%%%



% Versioni %%%%%%%%%%%%%%%%%%%%%%%%%%%%%%%%%%%%%%%%%%%%%%%%%%%%%%%%%%%%%%%%%%%%%%%%%%%
% \newpage
\begin{table}[h!]
\centering
\textbf{Versioni} \\ % Titolo sopra la tabella
\vspace{2mm} % Spazio tra il titolo e la tabella
\begin{tabular}{|c|c|c|c|c|}
    \hline
    \textbf{Ver.} & \textbf{Data} & \textbf{Autore} & \textbf{Verificatore} & \textbf{Descrizione} \\
    \hline
    1.1 & 10/02/2025 & Francesco Lapenna & Gabriele Di Pietro & Correzione errori RTB, riscritta sezione 6 \\
    1.0 & 10/02/2025 & Gabriele Di Pietro & Filippo Rizzolo & Approvazione documento \\
    0.5 & 06/02/2025 & Gabriele Di Pietro & Matthew Pan & Stesura sezione 3.2 \\
    0.4 & 20/01/2025 & Matthew Pan & Filippo Rizzolo & Stesura sezione 3.1 - Test Sistema \\
    0.3 & 16/12/2024 & Gabriele Di Pietro & Matthew Pan & Stesura sezione 5 \\
    0.2 & 10/12/2024 & Gabriele Di Pietro & Francesco Lapenna & Aggiunte tabelle \\
    0.1 & 05/12/2024 & Gabriele Di Pietro & Enrico Cotti Cottini & Prima stesura del documento \\  
    \hline
\end{tabular}
%\caption{Versioni del documento}
\label{tab:versioni}
\end{table}
%%%%%%%%%%%%%%%%%%%%%%%%%%%%%%%%%%%%%%%%%%%%%%%%%%%%%%%%%%%%%%%%%%%%%%%%%%%%%%%%%%%%%%
\newpage


% Indice %%%%%%%%%%%%%%%%%%%%%%%%%%%%%%%%%%%%%%%%%%%%%%%%%%%%%%%%%%%%%%%%%%%%%%%%%%%%%
% \newpage
\tableofcontents
\listoftables
\listoffigures
%%%%%%%%%%%%%%%%%%%%%%%%%%%%%%%%%%%%%%%%%%%%%%%%%%%%%%%%%%%%%%%%%%%%%%%%%%%%%%%%%%%%%%



% Sezione Introduzione %%%%%%%%%%%%%%%%%%%%%%%%%%%%%%%%%%%%%%%%%%%%%%%%%%%%%%%%%%%%%%%
\newpage
\section{Introduzione}
\subsection{Obiettivo del Documento}
Il documento ha lo scopo di definire le strategie di verifica e validazione per assicurare il corretto funzionamento e uno standard di qualità
dello strumento sviluppato e delle attività che lo accompagnano. Sarà sottoposto a revisioni continue, così da poter seguire l'evoluzione del progetto.

\subsection{Glossario}
Il \href{https://code7crusaders.github.io/docs/RTB/documentazione_interna/glossario.html#glossario}{Glossario\textsuperscript{G}} è uno strumento utilizzato per risolvere eventuali dubbi su termini specifici utilizzati nella redazione del documento. Esso conterrà la definizione dei 
termini evidenziati e sarà consultabile al seguente \href{https://code7crusaders.github.io/docs/RTB/documentazione_interna/glossario.html}{link}. I termini presenti in tale documento
saranno evidenziati da una 'G' al pedice.

\subsection{Riferimenti}
\subsubsection{Normativi}
\begin{itemize}
    \item \textbf{Regolamento del progetto} \\ \texttt{\url{https://www.math.unipd.it/~tullio/IS-1/2024/Dispense/PD1.pdf}}
    \item \textbf{Norme del Progetto} \\ \texttt{\url{https://code7crusaders.github.io/docs/RTB/documentazione_interna/norme_di_progetto.html}}
\end{itemize}
\subsubsection{Informativi}
\begin{itemize}
    \item \textbf{Standard ISO/IEC 25010} \\ \texttt{\url{https://iso25000.com/index.php/en/iso-25000-standards/iso-25010}}
    \item \textbf{Standard ISO/IEC 12207:1995} \\ \texttt{\url{https://www.math.unipd.it/~tullio/IS-1/2009/Approfondimenti/ISO_12207-1995.pdf}}
    \item \textbf{Qualità di prodotto} \\ \texttt{\url{https://www.math.unipd.it/~tullio/IS-1/2024/Dispense/T07.pdf}}
    \item \textbf{Qualità di processo} \\ \texttt{\url{https://www.math.unipd.it/~tullio/IS-1/2024/Dispense/T08.pdf}}
    \item \textbf{Verifica e validazione}
    \begin{itemize}
        \item Introduzione \\ \texttt{\url{https://www.math.unipd.it/~tullio/IS-1/2024/Dispense/T09.pdf}}
        \item Analisi Statica \\ \texttt{\url{https://www.math.unipd.it/~tullio/IS-1/2024/Dispense/T10.pdf}}
        \item Analisi Dinamica \\ \texttt{\url{https://www.math.unipd.it/~tullio/IS-1/2024/Dispense/T11.pdf}}
    \end{itemize}
    \item \textbf{Capitolato d'appalto C7} \\ \texttt{\url{https://www.math.unipd.it/~tullio/IS-1/2024/Progetto/C7.pdf}}
    \item \textbf{Verbali esterni ed interni} \\ \texttt{\url{https://code7crusaders.github.io/docs/RTB/index.html}} \\ \texttt{\url{https://code7crusaders.github.io/docs/PB/index.html}}
    \item \textbf{Analisi dei requisiti V2.0} \\ \texttt{\url{https://code7crusaders.github.io/docs/PB/documentazione_esterna/analisi_dei_requisiti.html}}
    \item \href{https://code7crusaders.github.io/docs/PB/documentazione_interna/glossario.html#glossario}{\textbf{Glossario V2.0}\textsuperscript{G}} \\ \texttt{\url{https://code7crusaders.github.io/docs/RTB/documentazione_interna/glossario.html}}
\end{itemize}
\newpage

%%%%%%%%%%%%%%%%%%%%%%%%%%%%%%%%%%%%%%%%%%%%%%%%%%%%%%%%%%%%%%%%%%%%%%%%%%%%%%%%%%%%%%%
\section{Qualità di processo}
La qualità di processo è un criterio fondamentale ed è alla base di ogni prodotto che
rispecchi lo stato dell’arte. Per raggiungere tale obiettivo è necessario sfruttare delle
pratiche rigorose che consentano lo svolgimento di ogni attività in maniera ottimale.
Al fine di valutare nel miglior modo possibile la qualità del prodotto e l’efficacia dei
processi, sono state definite delle metriche, meglio specificate nel documento Norme
di ProgettoG e qui di seguito riepilogate. Esse sono state suddivise utilizzando lo \textbf{Standard \texttt{ISO/IEC12207:1995}}, il quale separa i processi di ciclo di vita del software in processi di
base e/o primari, processi di supporto e processi organizzativi.
\subsection{Processi di base e/o primari}
\subsubsection{Fornitura} %Tabella
\begin{table}[H]
    \centering
    \renewcommand{\arraystretch}{1.5} % Aumenta lo spazio tra le righe
    \begin{tabular}{|c|l|c|c|}
        \hline
        \textbf{Codice} & \textbf{Nome} & \textbf{Ammissibile} & \textbf{Ottimale} \\
        \hline
        1PBM-PV & \href{https://code7crusaders.github.io/docs/RTB/documentazione_interna/glossario.html#planned-value}{Planned Value\textsuperscript{G}} & $PV \geq 0$ & $PV \leq BAC$ \\
        2PBM-ETC & Estimated to Complete & $ETC \geq 0$ & $ETC \leq EAC$ \\
        3PBM-EAC & Estimated at Completion & $EAC \leq BAC + 10\%$ & $EAC \leq BAC$ \\
        4PBM-EV & \href{https://code7crusaders.github.io/docs/RTB/documentazione_interna/glossario.html#earned-value}{Earned Value\textsuperscript{G}} & $EV \geq 0$ & $EV \leq EAC$ \\
        5PBM-AC & \href{https://code7crusaders.github.io/docs/RTB/documentazione_interna/glossario.html#actual-cost}{Actual Cost\textsuperscript{G}} & $AC \geq 0$ & $AC \leq EAC$ \\
        6PBM-SV & \href{https://code7crusaders.github.io/docs/RTB/documentazione_interna/glossario.html#scheduled-variance}{Scheduled Variance\textsuperscript{G}} & $SV \geq -10\%$ & $SV \geq 0\%$ \\
        7PBM-CV & \href{https://code7crusaders.github.io/docs/RTB/documentazione_interna/glossario.html#cost-variance}{Cost Variance\textsuperscript{G}} & $CV \geq -10\%$ & $CV \geq 0\%$ \\
        8PBM-CPI & Cost Performance Index & $CPI \geq 0.8$ & $CPI \geq 1$ \\
        9PBM-SPI & Scheduled Performance Index & $SPI \geq 0.8$ & $SPI \geq 1$ \\
        10PBM-OTDR & On-Time Delivery Rate & $OTDR \geq 90\%$ & $OTDR \geq 95\%$ \\
        \hline
    \end{tabular}
    \label{tab:fornitura}
    \caption{Metriche di qualità per il processo di Fornitura}
\end{table}
% \newpage

\subsubsection{Sviluppo} %Non inserire nulla
% \paragraph{Analisi dei requisiti}%Tabella
% \textbf{(11PBM - 13PBM)}

% % TODO: togliere (è stata messa in qualità di prodotto > Funzionalità)
% \begin{table}[H]
% \centering
% \renewcommand{\arraystretch}{1.5} % Aumenta lo spazio tra le righe
% \begin{tabular}{|c|l|c|c|}
%     \hline
%     \textbf{Codice} & \textbf{Nome} & \textbf{Ammissibile} & \textbf{Ottimale} \\
%     \hline
%     11PBM-PRO & Percentuale Requisiti Obbligatori & $PRO = 100\%$ & $PRO = 100\%$ \\
%     12PBM-PRD & Percentuale Requisiti Desiderabili & $PRD \geq 30\%$ & $PRD = 100\%$ \\
%     13PBM-PRF & Percentuale Requisiti Facoltativi & $PRF \geq 0\%$ & $PRF = 100\%$ \\
%     \hline
% \end{tabular}
% \label{tab:analisi_requisiti}
% \caption{Metriche di qualità per il processo di Analisi dei requisiti}
% \end{table}
% \newpage
\paragraph{Progettazione}%Tabella
\textbf{(14PBM)}
\begin{table}[H]
    \centering
    \renewcommand{\arraystretch}{1.5} % Aumenta lo spazio tra le righe
    \begin{tabular}{|c|l|c|c|}
        \hline
        \textbf{Codice} & \textbf{Nome} & \textbf{Ammissibile} & \textbf{Ottimale} \\
        \hline
        14PBM-PG & Profondità delle Gerarchie & $PG \leq 7$ & $PG \leq 5$ \\
        \hline
    \end{tabular}
    \label{tab:progettazione}
    \caption{Metriche di qualità per il processo di Progettazione}
\end{table}
\newpage
\paragraph{Implementazione}%Tabella
\textbf{(15PBM - 18PBM)}
% TODO: togliere 18PBM (è stata messa in Qualità di prodotto > Manutenibilità)
\begin{table}[H]
    \centering
    \renewcommand{\arraystretch}{1.5} % Aumenta lo spazio tra le righe
    \begin{tabular}{|c|l|c|c|}
        \hline
        \textbf{Codice} & \textbf{Nome} & \textbf{Ammissibile} & \textbf{Ottimale} \\
        \hline
        15PBM-PPM & Parametri per Metodo & $PPM \leq 7$ & $PPM \leq 5$\\
        16PBM-CPC & Campi per Classe & $CPC \leq 8$ & $CPC \leq 5$ \\
        17PBM-LCPM & Linee di Commento per Metodo & $LCPM \geq 50$ & $LCPM \geq 20$ \\
        % 18PBM-CCM & Complessità Ciclomatica Metrica & $CCM \leq 6$ & $CCM \leq 3$ \\
        \hline
    \end{tabular}
    \label{tab:codifica}
    \caption{Metriche di qualità per il processo di Codifica}
\end{table}

% TODO: togliere 9, 10 e 11PSM (sono stati messi in Qualità di prodotto > Affidabilità)
\paragraph{Verifica e Validazione}%Tabella
\textbf{(8PSM-CC - 12PSM-PTCP)}
\begin{table}[H]
    \centering
    \renewcommand{\arraystretch}{1.5} % Aumenta lo spazio tra le righe
    \begin{tabular}{|c|l|c|c|}
        \hline
        \textbf{Codice} & \textbf{Nome} & \textbf{Ammissibile} & \textbf{Ottimale} \\
        \hline
        8PSM-CC & Code Coverage & $CC \geq 80\%$ & $CC = 100\%$ \\
        % 9PSM-BC & Branch Coverage & $BC \geq 80\%$ & $BC = 100\%$ \\
        % 10PSM-SC & Statement Coverage & $SC \geq 80\%$ & $SC = 100\%$ \\
        % 11PSM-FD & Failure Density & $FD \leq 15\%$ & $FD = 0\%$ \\
        12PSM-PTCP & Passed Test Case Percentage & $PTCP \geq 90\%$ & $PTCP \geq 100\%$ \\
        \hline
    \end{tabular}
    \label{tab:verifica}
    \caption{Metriche di qualità per il processo di Verifica}
\end{table}

\subsection{Processi di Supporto}
\subsubsection{Documentazione}%Tabella
\textbf{(1PSM-IG - 2PSM-CO)}

\begin{table}[H]
    \centering
    \renewcommand{\arraystretch}{1.5} % Aumenta lo spazio tra le righe
    \begin{tabular}{|c|l|c|c|}
        \hline
        \textbf{Codice} & \textbf{Nome} & \textbf{Ammissibile} & \textbf{Ottimale} \\
        \hline
        1PSM-IG & Indice di Gulpease & $IG \geq 50$ & $IG \geq 75$ \\
        2PSM-CO & Correttezza Ortografica & $CO = 0$ errori & $CO = 0$ errori \\ 
        \hline
    \end{tabular}
    \label{tab:documentazione}
    \caption{Metriche di qualità per il processo di Documentazione}
\end{table}
\newpage
\subsubsection{Gestione della qualità}%Tabella
\textbf{(3PSM-FU - 7PSM-QMS)}
% TODO: togliere 3, 4, 5 e 6PSM (sono stati messi in Qualità di prodotto > Usabilità)
\begin{table}[H]
    \centering
    \renewcommand{\arraystretch}{1.5} % Aumenta lo spazio tra le righe
    \begin{tabular}{|c|l|c|c|}
        \hline
        \textbf{Codice} & \textbf{Nome} & \textbf{Ammissibile} & \textbf{Ottimale} \\
        \hline
        % 3PSM-FU & Facilità di Utilizzo & $FU \geq 3$ errori & $FU \geq 0$ errori \\
        % 4PSM-TA & Tempo di Apprendimento & $TA \leq 12$ minuti & $TA \leq 8$ minuti \\
        % 5PSM-TR & Tempo di Risposta & $TR \leq 8$ secondi & $TR \leq 4$ secondi \\
        % 6PSM-TE & Tempo di Elaborazione & $TE \leq 10$ secondi & $TE \leq 5$ secondi \\
        7PSM-\href{https://code7crusaders.github.io/docs/RTB/documentazione_interna/glossario.html#qms}{QMS\textsuperscript{G}} & Metriche di Qualità Soddisfatte & $\href{https://code7crusaders.github.io/docs/RTB/documentazione_interna/glossario.html#qms}{QMS\textsuperscript{G}} \geq 90\%$ & $\href{https://code7crusaders.github.io/docs/RTB/documentazione_interna/glossario.html#qms}{QMS\textsuperscript{G}} \geq 90\%$\\
        \hline
    \end{tabular}
    \label{tab:gestione_qualità}
    \caption{Metriche di qualità per il processo di Gestione della Qualità}
\end{table}

\subsubsection{Risoluzione dei Problemi}%Tabella
\textbf{(13PSM-RMR - 14PSM-NCR)}
\begin{table}[H]
    \centering
    \renewcommand{\arraystretch}{1.5} % Aumenta lo spazio tra le righe
    \begin{tabular}{|c|l|c|c|}
        \hline
        \textbf{Codice} & \textbf{Nome} & \textbf{Ammissibile} & \textbf{Ottimale} \\
        \hline
        13PSM-RMR & Risk Mitigation Rate & $RMR \geq 80\%$ & $RMR = 100\%$ \\
        14PSM-NCR & Richi non Calcolati & $NCR \leq 3$ & $NCR = 0$ \\
        \hline
    \end{tabular}
    \label{tab:risoluzione_problemi}
    \caption{Metriche di qualità per il processo di Risoluzione dei Problemi}
\end{table}
% \newpage
\subsection{Processi organizzativi}
\subsubsection{Pianificazione} %Tabella
\textbf{(1POM-RSI)}
\begin{table}[H]
    \centering
    \renewcommand{\arraystretch}{1.5} % Aumenta lo spazio tra le righe
    \begin{tabular}{|c|l|c|c|}
        \hline
        \textbf{Codice} & \textbf{Nome} & \textbf{Ammissibile} & \textbf{Ottimale} \\
        \hline
        1POM-RSI & \href{https://code7crusaders.github.io/docs/RTB/documentazione_interna/glossario.html#requirements-stability-index}{Requirements Stability Index\textsuperscript{G}} & $RSI \geq 75\%$ & $RSI = 100\%$ \\
        \hline
    \end{tabular}
    \label{tab:pianificazione}
    \caption{Metriche di qualità per il processo di Pianificazione}
\end{table}
\newpage

%%%%%%%%%%%%%%%%%%%%%%%%%%%%%%%%%%%%%%%%%%%%%%%%%%%%%%%%%%%%%%%%%%%%%%%%%%%%%%%%%%%%%%%
\section{Qualità di prodotto}
La sezione Qualità di Prodotto del Piano di Qualifica definisce i criteri e le metriche adottate per garantire che il software sviluppato soddisfi i requisiti di qualità previsti. Questa sezione descrive gli attributi fondamentali del prodotto, come affidabilità, manutenibilità, usabilità e prestazioni, e le strategie adottate per monitorarne e migliorarne la qualità durante il ciclo di sviluppo. L'obiettivo è assicurare che il software sia conforme agli standard richiesti e risponda efficacemente alle esigenze degli utenti finali.
    
    \subsection{Funzionalità}
    \begin{table}[H]
        \centering
        \renewcommand{\arraystretch}{1.5} % Aumenta lo spazio tra le righe
        \begin{tabular}{|c|l|c|c|}
            \hline
            \textbf{Codice} & \textbf{Nome} & \textbf{Ammissibile} & \textbf{Ottimale} \\
            \hline
            11PBM-PRO & Percentuale Requisiti Obbligatori & $PRO = 100\%$ & $PRO = 100\%$ \\
            12PBM-PRD & Percentuale Requisiti Desiderabili & $PRD \geq 30\%$ & $PRD = 100\%$ \\
            13PBM-PRF & Percentuale Requisiti Facoltativi & $PRF \geq 0\%$ & $PRF = 100\%$ \\
            \hline
        \end{tabular}
        \label{tab:analisi_requisiti}
        \caption{Metriche di qualità di funzionalità del prodotto}
    \end{table}

    \subsection{Affidabilità}
    \begin{table}[H]
        \centering
        \renewcommand{\arraystretch}{1.5} % Aumenta lo spazio tra le righe
        \begin{tabular}{|c|l|c|c|}
            \hline
            \textbf{Codice} & \textbf{Nome} & \textbf{Ammissibile} & \textbf{Ottimale} \\
            \hline
            9PSM-BC & Branch Coverage & $BC \geq 80\%$ & $BC = 100\%$ \\
            10PSM-SC & Statement Coverage & $SC \geq 80\%$ & $SC = 100\%$ \\
            11PSM-FD & Failure Density & $FD \leq 15\%$ & $FD = 0\%$ \\
            \hline
        \end{tabular}
        \label{tab:verifica}
        \caption{Metriche di qualità di affidabilità del prodotto}
    \end{table}

    \subsection{Usabilità}
    \begin{table}[H]
        \centering
        \renewcommand{\arraystretch}{1.5} % Aumenta lo spazio tra le righe
        \begin{tabular}{|c|l|c|c|}
            \hline
            \textbf{Codice} & \textbf{Nome} & \textbf{Ammissibile} & \textbf{Ottimale} \\
            \hline
            3PSM-FU & Facilità di Utilizzo & $FU \geq 3$ errori & $FU \geq 0$ errori \\
            4PSM-TA & Tempo di Apprendimento & $TA \leq 12$ minuti & $TA \leq 8$ minuti \\
            6PSM-TE & Tempo di Elaborazione & $TE \leq 10$ secondi & $TE \leq 5$ secondi \\
            \hline
        \end{tabular}
        \label{tab:gestione_qualità}
        \caption{Metriche di qualità di usabilità del prodotto}
    \end{table}

    \subsection{Efficienza}
    \begin{table}[H]
        \centering
        \renewcommand{\arraystretch}{1.5} % Aumenta lo spazio tra le righe
        \begin{tabular}{|c|l|c|c|}
            \hline
            \textbf{Codice} & \textbf{Nome} & \textbf{Ammissibile} & \textbf{Ottimale} \\
            \hline
            5PSM-TR & Tempo di Risposta & $TR \leq 8$ secondi & $TR \leq 4$ secondi \\
            \hline
        \end{tabular}
        \label{tab:gestione_qualità}
        \caption{Metriche di qualità di efficienza del prodotto}
    \end{table}

    \subsection{Manutenibilità}
    \begin{table}[H]
        \centering
        \renewcommand{\arraystretch}{1.5} % Aumenta lo spazio tra le righe
        \begin{tabular}{|c|l|c|c|}
            \hline
            \textbf{Codice} & \textbf{Nome} & \textbf{Ammissibile} & \textbf{Ottimale} \\
            \hline
            18PBM-CCM & Complessità Ciclomatica Metrica & $CCM \leq 6$ & $CCM \leq 3$ \\
            \hline
        \end{tabular}
        \label{tab:codifica}
        \caption{Metriche di qualità di manutenibilità del prodotto}
    \end{table}
\newpage
%%%%%%%%%%%%%%%%%%%%%%%%%%%%%%%%%%%%%%%%%%%%%%%%%%%%%%%%%%%%%%%%%%%%%%%%%%%%%%%%%%%%%%%

\section{Metodologie e Testing}
In questa sezione si illustrano le metodologie di \textit{Testing} adottate per garantire il rispetto dei vincoli individuati
nella sezione \textit{Requisiti} del documento Analisi dei Requisiti. I test sono suddivisi in cinque categorie:
\begin{enumerate}
    \item Test di unità
    \item Test di integrazione
    \item Test di Sistema
    \item Test di Regressione
    \item Test di Accettazione
\end{enumerate}
Verranno elencate le varie tipologie di test eseguite, indicando il codice del test, una breve descrizione di ciò che viene verificato e lo stato di avanzamento del test, espresso come segue.

\begin{table}[H]
    \centering
    \renewcommand{\arraystretch}{1.5}
\begin{tabular}{|c|c|}
    \hline
    \textbf{S} & Test Superato \\
    \hline
    \textbf{NS} & Test NON Superato \\
    \hline
    \textbf{NI} & Test NON Implementato \\
    \hline
\end{tabular}
\caption{Legenda per il Test}
\end{table}


\subsection{Test di Sistema} %tabellare
I test di sistema sono finalizzati alla verifica del soddisfacimento dei requisiti richiesti ed evidenziati nel documento
\href{https://code7crusaders.github.io/docs/RTB/documentazione_interna/glossario.html#analisi-dei-requisiti}{Analisi dei Requisiti\textsuperscript{G}}. Questi test vengono effettuati sul sistema nel suo complesso, per verificare che il software funzioni correttamente
e che sia in grado di eseguire le operazioni richieste.

\renewcommand{\arraystretch}{1.5}  % Aumenta lo spazio tra le righe

\begin{longtable}{|>{\centering\arraybackslash}m{0.10\textwidth}|>{\raggedright\arraybackslash}m{0.70\textwidth}|c|}
    \hline
    \textbf{Codice} & \textbf{Descrizione} & \textbf{Stato} \\
    \hline
    \endfirsthead
    \hline
    \textbf{Codice} & \textbf{Descrizione} & \textbf{Stato} \\
    \hline
    \endhead
    \hline
    \endfoot
    \hline
    \textbf{1T-S} & Verificare che il caricamento dei dati semantici aziendali avvenga correttamente nei formati accettati. & NI \\
    \hline
    \textbf{2T-S} & Verificare che il sistema gestisca correttamente documenti in formati non compatibili. & NI \\
    \hline
    \textbf{3T-S} & Verificare che i testi vengano suddivisi correttamente in blocchi. & NI \\
    \hline
    \textbf{4T-S} & Verificare che i blocchi di testo vengano trasformati in vettori tramite l’\href{https://code7crusaders.github.io/docs/RTB/documentazione_interna/glossario.html#embedding}{Embedding\textsuperscript{G}} Model. & NI \\
    \hline
    \textbf{5T-S} & Verificare che i vettori siano memorizzati e indicizzati correttamente nel database vettoriale. & NI \\
    \hline
    \textbf{6T-S} & Verificare che l’utente possa inviare una domanda attraverso l’interfaccia utente. & NI \\
    \hline
    \textbf{7T-S} & Verificare che la query venga gestita correttamente tramite \href{https://code7crusaders.github.io/docs/RTB/documentazione_interna/glossario.html#api-rest-representational-state-transfer}{API REST\textsuperscript{G}} e inoltrata al sistema. & NI \\
    \hline
    \textbf{8T-S} & Verificare che l’\href{https://code7crusaders.github.io/docs/RTB/documentazione_interna/glossario.html#embedding}{Embedding\textsuperscript{G}} Model trasformi la domanda in una rappresentazione vettoriale. & NI \\
    \hline
    \textbf{9T-S} & Verificare che la ricerca nel database vettoriale restituisca i vettori più simili. & NI \\
    \hline
    \textbf{10T-S} & Verificare che il sistema \href{https://code7crusaders.github.io/docs/RTB/documentazione_interna/glossario.html#llm-large-language-model}{LLM\textsuperscript{G}} costruisca la risposta utilizzando il contesto fornito. & NI \\
    \hline
    \textbf{11T-S} & Verificare che la risposta venga inviata correttamente al dispositivo dell’utente tramite \href{https://code7crusaders.github.io/docs/RTB/documentazione_interna/glossario.html#api-rest-representational-state-transfer}{API REST\textsuperscript{G}}. & NI \\
    \hline
    \textbf{12T-S} & Verificare che l’utente registrato possa avviare e gestire una conversazione con il bot. & NI \\
    \hline
    \textbf{13T-S} & Verificare che l’utente possa richiedere e ricevere informazioni sui prodotti durante una conversazione. & NI \\
    \hline
    \textbf{14T-S} & Verificare che l’utente possa salvare una conversazione avviata. & NI \\
    \hline
    \textbf{15T-S} & Verificare che l’utente possa visualizzare le conversazioni precedentemente salvate. & NI \\
    \hline
    \textbf{16T-S} & Verificare che l’utente possa recuperare e riprendere una conversazione salvata. & NI \\
    \hline
    \textbf{17T-S} & Verificare che l’utente possa eliminare una conversazione salvata. & NI \\
    \hline
    \textbf{18T-S} & Verificare che l’accesso al sistema sia consentito solo con credenziali valide. & NI \\
    \hline
    \textbf{19T-S} & Verificare che il sistema blocchi gli utenti non registrati. & NI \\
    \hline
    \textbf{20T-S} & Verificare che il sistema prevenga attacchi come SQL Injection. & NI \\
    \hline
    \textbf{21T-S} & Verificare che l’utente possa inviare feedback positivo o negativo sulla qualità della conversazione. & NI \\
    \hline
    \textbf{22T-S} & Verificare che l’amministratore possa creare template di domande e risposte. & NI \\
    \hline
    \textbf{23T-S} & Verificare che l’amministratore possa modificare template di domande e risposte. & NI \\
    \hline
    \textbf{24T-S} & Verificare che l’amministratore possa eliminare un template esistente. & NI \\
    \hline
    \textbf{25T-S} & Verificare che il sistema blocchi la creazione di template in formato non valido. & NI \\
    \hline
    \textbf{26T-S} & Verificare che l’amministratore possa monitorare le prestazioni del sistema dalla dashboard. & NI \\
    \hline
    \textbf{27T-S} & Verificare che l’amministratore possa visualizzare i feedback forniti dagli utenti. & NI \\
    \hline
    \textbf{28T-S} & Verificare che l’amministratore possa importare dati da documenti esterni. & NI \\
    \hline
    \textbf{29T-S} & Verificare che il sistema blocchi l’importazione di file non compatibili. & NI \\
    \hline
    \textbf{30T-S} & Verificare che l’amministratore possa visualizzare le richieste di assistenza degli utenti. & NI \\
    \hline
    \textbf{31T-S} & Verificare che l’amministratore possa segnalare una richiesta di assistenza presa in carico. & NI \\
    \hline
    \textbf{32T-S} & Verificare che l’amministratore possa rispondere agli utenti via e-mail. & NI \\
    \hline
    \textbf{33T-S} & Verificare che l’amministratore possa visualizzare l’utilizzo generale del servizio. & NI \\
    \hline
    \textbf{34T-S} & Verificare che l’amministratore possa visualizzare i costi del sistema. & NI \\
    \hline
    \textbf{35T-S} & Verificare che lo schema di progettazione della base di dati sia conforme ai requisiti. & NI \\
    \hline
    \textbf{36T-S} & Verificare che il codice prodotto sia disponibile in formato sorgente tramite repository pubblici. & NI \\
    \hline
    \textbf{37T-S} & Verificare che la documentazione descrittiva del sistema di raccomandazione sia completa e accessibile. & NI \\
    \hline
    \textbf{38T-S} & Verificare che la documentazione riassuntiva delle metriche e dei risultati sia conforme ai requisiti. & NI \\
    \hline
    \textbf{39T-S} & Verificare che l’\href{https://code7crusaders.github.io/docs/RTB/documentazione_interna/glossario.html#llm-large-language-model}{LLM\textsuperscript{G}} sia integrato correttamente tramite API. & NI \\
    \hline
    \textbf{40T-S} & Verificare che sia stato implementato almeno un database relazionale e che funzioni correttamente. & NI \\
    \hline
    \textbf{41T-S} & Verificare che sia stato implementato almeno un database vettoriale e che funzioni correttamente. & NI \\
    \hline
    \textbf{42T-S} & Verificare che sia stato implementato un embedding model, locale o tramite API. & NI \\
    \hline
    \textbf{43T-S} & Verificare che la WebApp consenta di comunicare correttamente con il chatbot. & NI \\
    \hline
\caption{Test di Sistema}
\end{longtable}



\newpage
\subsection{Test di Accettazione} %tabellare
I test di Accettazione vengono effettuati per verificare che il Software soddisfi i requisiti richiesti e consentono di ultimare il processo di validazione finale.

\begin{longtable}{|>{\centering\arraybackslash}m{0.10\textwidth}|>{\raggedright\arraybackslash}m{0.70\textwidth}|c|}
    \hline
    \textbf{Codice} & \textbf{Descrizione} & \textbf{Stato} \\
    \hline
    \textbf{TA01} & Verificare che il sistema accetti documenti nei formati \texttt{.pdf} e \texttt{.txt} in input & NI \\
    \hline
    \textbf{TA02} & Verificare che i documenti vengano suddivisi in blocchi di testo. & NI \\
    \hline
    \textbf{TA03} & Verificare che il modello di embedding generi rappresentazioni vettoriali dei blocchi di testo. & NI\\
    \hline
    \textbf{TA04} & Verificare che i vettori generati siano memorizzati nel database vettoriale. & NI\\
    \hline
    \textbf{TA05} & Verificare che l’utente possa inviare domande tramite l’interfaccia della web app.& NI\\
    \hline
    \textbf{TA06} & Verificare che la domanda venga inoltrata al sistema tramite \href{https://code7crusaders.github.io/docs/RTB/documentazione_interna/glossario.html#api-rest-representational-state-transfer}{API REST\textsuperscript{G}}. & NI\\
    \hline
    \textbf{TA07} & Verificare che la domanda venga trasformata in una rappresentazione vettoriale. & NI\\
    \hline
    \textbf{TA08} & Verificare che il sistema recuperi i vettori più simili dal database vettoriale. & NI\\
    \hline
    \textbf{TA09} & Verificare che il sistema \href{https://code7crusaders.github.io/docs/RTB/documentazione_interna/glossario.html#llm-large-language-model}{LLM\textsuperscript{G}} costruisca una risposta basata sulla domanda e sul contesto. & NI\\
    \hline
    \textbf{TA10} & Verificare che la risposta venga inviata all’utente tramite \href{https://code7crusaders.github.io/docs/RTB/documentazione_interna/glossario.html#api-rest-representational-state-transfer}{API REST\textsuperscript{G}}.& NI\\
    \hline
    \textbf{TA11} & Verificare che l’utente registrato possa avviare una conversazione con il bot.& NI\\
    \hline
    \textbf{TA12} & Verificare che l’utente possa salvare una conversazione.& NI\\
    \hline
    \textbf{TA13} & Verificare che il login con username e password funzioni.& NI\\
    \hline
    \textbf{TA14} & Verificare la protezione contro SQL Injection e altri attacchi. & NI\\
    \hline
    \textbf{TA15} & Verificare che l’utente possa fornire un feedback sulla conversazione. & NI\\
    \hline
    \textbf{TA16} & Verificare che l’amministratore possa monitorare le prestazioni del sistema tramite dashboard. & NI\\
    \hline
    \textbf{TA17} & Verificare che l’utente possa eliminare una conversazione salvata.& NI\\
    \hline
    \textbf{TA18} & Verificare che l’utente possa inviare richieste di assistenza per contattare un operatore umano.& NI\\
    \hline
    \caption{Test di Accettazione}
\end{longtable}

%%%%%%%%%%%%%%%%%%%%%%%%%%%%%%%%%%%%%%%%%%%%%%%%%%%%%%%%%%%%%%%%%%%%%%%%%%%%%%%%%%%

\section{Cruscotto valutazione della qualità}


    \subsection{Qualità processo di Fornitura}
        \subsubsection{1PBM-PV - Planned Value e 4PBM-EV - Earned Value}
        \begin{figure}[H]
            \centering
            \includegraphics[width=0.8\textwidth]{../../../img/pdq_charts/chart1-proiezionePVEV.png}
            \caption{Proiezione di PV ed EV}
        \end{figure}
        Il grafico mostra l'andamento di Planned Value (PV) ed Earned Value (EV) nel tempo, evidenziando una forte sovrapposizione tra le due curve. Questo implica che il team di progetto ha seguito fedelmente il piano iniziale, eseguendo le attività previste nei tempi stabiliti.
        La convergenza quasi lineare di PV ed EV verso il Estimate at Completion (EAC) indica una distribuzione omogenea del lavoro lungo tutto l'orizzonte temporale del progetto. Ciò suggerisce che il progetto non ha subito ritardi significativi né ha registrato accelerazioni improvvise, ma ha mantenuto un ritmo costante di avanzamento.
        L'aderenza tra PV ed EV è un segnale positivo in termini di gestione del progetto, poiché significa che le attività sono state completate secondo le stime iniziali, senza deviazioni rilevanti. Questo può derivare da una pianificazione accurata, un'efficace allocazione delle risorse e una buona esecuzione da parte del team.

        \subsubsection{5PBM-AC - Actual Cost e 2PBM-ETC - Estimate to Complete}
        \begin{figure}[H]
            \centering
            \includegraphics[width=0.8\textwidth]{../../../img/pdq_charts/chart2-proiezioneACETC.png}
            \caption{Proiezione di AC e ETC}
        \end{figure}
        Il grafico mostra l’andamento di tre metriche fondamentali nella gestione dei costi di progetto:
        \begin{itemize}
            \item Actual Cost (AC): il costo effettivamente sostenuto fino a un determinato momento. Questo valore cresce progressivamente nel tempo, indicando il consumo di risorse economiche man mano che il progetto avanza.  
            \item Estimate to Complete (ETC): la stima dei costi necessari per completare il progetto. Si osserva un andamento decrescente, segno che, con l’avanzare delle attività, il budget residuo necessario si riduce.  
            \item Estimate at Completion (EAC): la stima del costo totale previsto al completamento del progetto. La stabilità di questa metrica nel tempo suggerisce che non si stanno verificando scostamenti significativi rispetto al budget iniziale.  
        \end{itemize}
        La stabilità dell’Estimate at Completion è un segnale positivo, poiché indica che le previsioni di spesa fatte in fase di pianificazione si stanno rivelando accurate e che il progetto non sta subendo variazioni di costo significative. In altre parole, i costi effettivi e quelli stimati rimangono allineati, suggerendo una gestione finanziaria efficace e senza imprevisti di rilievo.

        \subsubsection{6PBM-SV - Schedule Variance e 7PBM-CV - Cost Variance}
        \begin{figure}[H]
            \centering
            \includegraphics[width=0.8\textwidth]{../../../img/pdq_charts/chart3-proiezioneSVCV.png}
            \caption{Proiezione di SV e CV}
        \end{figure}
        Il grafico mette in evidenza l’andamento di Scheduled Variance (SV) e Cost Variance (CV) nel tempo, mostrando come questi due indicatori siano quasi sempre sovrapposti. Questo significa che il progetto è gestito con un buon livello di efficienza, sia in termini di pianificazione che di controllo dei costi.
        L’andamento simile di SV e CV indica che le attività vengono completate nei tempi previsti senza significative deviazioni dal budget. Questo riflette un bilanciamento efficace tra l’esecuzione dei lavori e il controllo delle spese.
        Nel complesso, il grafico evidenzia un buon equilibrio tra tempistiche e risorse, dimostrando un efficace processo di gestione del progetto.

        \subsubsection{3PBM-EAC - Estimated at Completion}
        \begin{figure}[H]
            \centering
            \includegraphics[width=0.8\textwidth]{../../../img/pdq_charts/chart4-proiezioneEAC.png}
            \caption{Proiezione di EAC}
        \end{figure}
        Il grafico rappresenta l'andamento dell'Estimate at Completion (EAC) rispetto al costo preventivato durante il corso del progetto. Si osserva che, nella maggior parte del tempo, l'EAC rimane allineato con il budget iniziale, indicando un controllo efficace dei costi e una gestione finanziaria coerente con le previsioni.\\
        Tuttavia, emergono alcuni momenti critici in cui l'EAC si avvicina alla soglia massima ammissibile, segnalando possibili situazioni di rischio, come ritardi, variazioni nei costi delle risorse o inefficienze operative. Nonostante queste fluttuazioni, il valore dell'EAC rientra rapidamente nei parametri ottimali, suggerendo l'adozione di misure correttive tempestive che hanno permesso di riportare il progetto in linea con il budget.\\
        Questo comportamento evidenzia una gestione attenta della valutazione dei costi a finire, con interventi mirati a contenere gli scostamenti e garantire che il progetto si mantenga all'interno delle soglie di sostenibilità economica.


    \subsection{Qualità processo di Documentazione}
        \subsubsection{1PSM-IG - Indice Gulpease}
        \begin{figure}[H]
            \centering
            \includegraphics[width=0.8\textwidth]{../../../img/pdq_charts/chart5-indiceGulpease.png}
            \caption{Indice di Gulpease per documento}
        \end{figure}
        L'indice di Gulpease è una metrica utilizzata per valutare la leggibilità di un testo in lingua italiana.
        L’indice varia tra 0 e 100, dove valori più alti indicano una maggiore leggibilità.
        Dal grafico emerge che l’indice di Gulpease della documentazione analizzata si mantiene costantemente in un intervallo compreso tra il valore ammissibile e quello ottimale.
        Ciò implica che la documentazione è chiara e ben strutturata e permette una fruizione veloce e immediata, riducendo il tempo di comprensione del contenuto.


    \subsection{Qualità del processo di gestione della qualità}
        \subsubsection{7PSM-QMS - Metriche di Qualità Soddisfatte}
        \begin{figure}[H]
            \centering
            \includegraphics[width=0.8\textwidth]{../../../img/pdq_charts/chart6-metricheSoddisfatte.png}
            \caption{Metriche di qualità soddisfatte}
        \end{figure}
        Come evidenziato dal grafico, la quantità di metriche soddisfatte si colloca appena al di sotto del range ammissibile. Questo indica che, sebbene il progetto sia vicino a raggiungere gli standard richiesti, non ha ancora soddisfatto pienamente i criteri stabiliti.
        Tuttavia, è importante considerare che il calcolo delle Metriche di Qualità Soddisfatte include anche metriche relative alla qualità del codice e ai test effettuati su di esso. Poiché il codice non è ancora stato sviluppato in questa fase del progetto, è naturale che il valore complessivo del QMS risulti inferiore.
        Di conseguenza, il risultato attuale non rappresenta necessariamente una criticità, ma piuttosto una condizione prevista, destinata a migliorare man mano che lo sviluppo del codice procede e vengono introdotti i relativi test.


    \subsection{Qualità del processo di pianificazione}
        \subsubsection{1POM-RSI - Requirements Stability Index}
        \begin{figure}[H]
            \centering
            \includegraphics[width=0.8\textwidth]{../../../img/pdq_charts/chart7-proiezioneRSI.png}
            \caption{Metriche di qualità soddisfatte}
        \end{figure}
        Il Requirements Stability Index (RSI) è un indicatore che misura la stabilità dei requisiti di un progetto nel tempo. Un RSI elevato indica che i requisiti sono rimasti sostanzialmente invariati, mentre un valore basso suggerisce modifiche frequenti e potenzialmente destabilizzanti per lo sviluppo.
        L’andamento del grafico mostra che, a partire dalla fase di analisi, i requisiti hanno subito variazioni limitate e coerenti con le previsioni iniziali. Questo suggerisce un processo di gestione dei requisiti ben controllato, con revisioni e aggiornamenti minimi che non hanno compromesso la stabilità complessiva del progetto.
        L’oscillazione contenuta dell’indice evidenzia un buon livello di maturità nella definizione dei requisiti, riducendo il rischio di impatti negativi su tempi, costi e qualità del prodotto finale.

%%%%%%%%%%%%%%%%%%%%%%%%%%%%%%%%%%%%%%%%%%%%%%%%%%%%%%%%%%%%%%%%%%%%%%%%%%%%%%%%%%%

\newpage
\section{Iniziative di automiglioramento per la qualità}
\subsection{Introduzione}
In questa sezione vengono descritte le azioni intraprese per migliorare la qualità del progetto \textit{Software}. 
Ogni iniziativa è stata identificata attraverso l’esperienza accumulata durante lo sviluppo del progetto, mano a mano che emergono problematiche specifiche. 
Essendo questa la nostra prima esperienza con un progetto di tale complessità, è stato necessario affrontare numerosi tentativi per capire come organizzarci e gestire le diverse attività in modo efficace. 
Durante il percorso, siamo riusciti a riconoscere i punti di forza e le aree di miglioramento nel nostro lavoro, individuando così gli aspetti su cui focalizzarci per ottimizzare il processo. 
\subsection{Valutazione sull'organizzazione}
\begin{center}
\begin{longtable}{|>{\centering\arraybackslash}p{0.45\textwidth}|>{\centering\arraybackslash}p{0.45\textwidth}|}
    \hline
    \textbf{Descrizione problema} & \textbf{Contromisura} \\
    \hline
    Mancanza di tracciabilità delle attività rende difficile l'avanzamento produttivo e la pianificazione del lavoro & Implementazione di una Project Board su GitHub per migliorare la gestione e il monitoraggio delle attività \\
    \hline
    Attività che non venivano concluse per il tempo previsto rallentando il lavoro & Introduzione di una data di scadenza per ogni attività e creazione di tag specifici per categorizzare i problemi \\ 
    \hline
    Difficoltà nel comunicare la rendicondazione delle ore durante ogni sprint & Utilizzata la roadmap di \textit{GitHub} per la costruzione dei diagrammi di Gantt, in modo tale da iniziare a prevedere il tempo per ogni singola attività e avere uno schema di riferimento \\
    \hline
    Difficoltà nel comunicare e organizzare le riunioni o eventi & Utilizzo di un calendario condiviso tramite l'uso di \textit{Google Calendar} per pianificare le riunioni e le attività da svolgere \\
    \hline
    Diversi software di scrittura per stendere i documenti & Standardizzazione dell'uso di \LaTeX{} per la stesura dei documenti, e creazione di template per specifici documenti come i Verbali. Tutto al fine di migliorare la coerenza, la manutenibilità e la leggibilità della documentazione all'interno del progetto. \\
    \hline
    \caption{Contromisure adottate per migliorare l'organizzazione}
\end{longtable}
\end{center}
\newpage
\subsection{Valutazione sui Ruoli}
\begin{center}
    \begin{longtable}{|>{\centering\arraybackslash}p{0.30\textwidth}|>{\centering\arraybackslash}p{0.30\textwidth}|>{\centering\arraybackslash}p{0.30\textwidth}|}
        \hline
        \textbf{Ruolo} & \textbf{Descrizione problema} & \textbf{Contromisura} \\
        \hline
        Verificatore & Modificare o aggiornare documenti senza prima averli sottoposti a verifica non è conforme ai buoni standard progettuali & Il verificatore è tenuto a compilare ad ogni modifica la tabella relativa per ciascun documento\\
        \hline %%%AGGIUNGERE ALTRE RIGHE SE NECESSARIO
        \caption{Contromisure adottate per miglirare la gestione dei ruoli}
    \end{longtable}
\end{center}
\newpage
\subsection{Valutazione sugli Strumenti}
\begin{center}
    \begin{longtable}{|>{\centering\arraybackslash}p{0.30\textwidth}|>{\centering\arraybackslash}p{0.30\textwidth}|>{\centering\arraybackslash}p{0.30\textwidth}|}
        \hline
        \textbf{Strumento} & \textbf{Descrizione problema} & \textbf{Contromisura} \\
        \hline
        Glossario & Difficoltà nel riconoscere se una parola in un documento è stata definita nel glossario & Creazione di uno script Python che verifica la presenza di ogni parola definita nel glossario nei documenti.\\
        \hline
        Sito Web & Visitare il repository per leggere la documentazione risulta scomodo e macchinoso & Creazione di un sito web di facile accesso dove è possibile scaricare i pdf per la documentazione, fornendo la possibilità di visulizzare i documenti più piccoli direttamente sulle pagine. \\
        \hline
        LaTeX & Difficoltà nell'uso di questo linguaggio, soprattutto per i membri che non lo conoscevano & Ogni membro ha intrapreso un percorso di apprendimento per colmare le proprie lacune \\
        \hline %%%AGGIUNGERE ALTRE RIGHE SE NECESSARIO
        \caption{Contromisure adottate per miglirare il rapporto con gli strumenti}
    \end{longtable}
\end{center}

%\subsection{Problemi Rilevati ed iniziative adottate}
%\subsubsection{Presentazioni del diario di Bordo}
%\begin{itemize}
%    \item \textbf{Fase del Progetto:} Iniziale;
%    \item \textbf{Descrizione:} Ogni settimana è richiesta una presentazione che illustri le attività svolte durante la settimana. È necessario preparare delle slide ed esporle di persona, ma per alcuni membri questo risulta difficoltoso a causa di impegni lavorativi o distanza. Nonostante una tabella che riportasse i ruoli settimanali di ciascun membro.
%    \item \textbf{Contromisura:} Abbiamo deciso che i membri responsabili durante il periodo delle \textit{vacanze natalizie}, in cui non sono previste attività di diario di bordo, sostituiranno i colleghi che non possono presentare. In caso di diari di bordo online, saranno loro a presentare per primi.
%\end{itemize}
%\subsubsection{Organizzazione delle riunioni}
%\begin{itemize}
%    \item \textbf{Fase del Progetto:} Intermedia;
%    \item \textbf{Descrizione:} In realtà siamo rimasti stupiti che questo problema non si sia presentato fin da subito, ma durante il mese di dicembre ci sono stati problemi sulle sprint interne con molti membri assenti, questo ha portato ad un rallentamento del lavoro anche per la discrepanza di conoscenze che ogni membro ha. A volte, alcuni membri non sapevano dove recuperare determinate informazioni o se i documenti fossero pronti. La situazione è peggiorata quando solo pochi membri si autoassegnavano le \textit{issue}.
%    \item \textbf{Contromisura:} Abbiamo deciso di rendere la stesura dei verbali un'attività prioritaria da svolgere durante la riunione, cercando di completarla il più rapidamente possibile. Questo aiuta anche chi deve presentare il diario di bordo. Inoltre, per ogni \textit{issue} creata, abbiamo notificato tutti i membri tramite le piattaforme di comunicazione, in modo che tutti fossero consapevoli del lavoro da svolgere durante la sprint. Inoltre per evitare riunioni con pochi membri abbiamo preso in considerazione di essere più flessibili sull'orario delle riunioni settimanali.
%\end{itemize}
\subsection{Considerazioni Finali}
Il processo di automiglioramento è un’attività continua e fondamentale per garantire la qualità del progetto. Le contromisure adottate hanno permesso di risolvere o ridurre i problemi individuati, migliorando l’efficienza e l’efficacia del lavoro svolto. Il gruppo si impegna a mantenere un approccio proattivo e collaborativo per individuare e risolvere tempestivamente eventuali criticità. Continueremo a monitorare e valutare le nostre pratiche, implementando ulteriori miglioramenti ove necessario, per assicurare il successo del progetto e la soddisfazione degli stakeholder.

\end{document}
\section{Tecnologie}
Questa sezione ha lo scopo di offrire una panoramica delle tecnologie adottate per la realizzazione del sistema software. Vengono analizzati in dettaglio le piattaforme, gli strumenti, i linguaggi di programmazione, i framework e altre risorse tecnologiche utilizzate nel corso dello sviluppo.
\subsection{Docker}
È una piattaforma di containerizzazione leggera che facilita lo sviluppo, il testing e il rilascio delle applicazioni, fornendo un ambiente isolato e riproducibile. 
Viene utilizzato per creare ambienti di sviluppo uniformi, migliorare la scalabilità delle applicazioni e semplificare la gestione delle risorse.
\subsection{Linguaggi di programmazione e formato dati}
\begin{longtable}{|>{\centering\arraybackslash}m{0.10\textwidth}|>{\centering\arraybackslash}m{0.10\textwidth}|>{\centering\arraybackslash}m{0.40\textwidth}|>{\centering\arraybackslash}m{0.30\textwidth}|}
	\hline
	\textbf{Nome} & \textbf{Versione} & \textbf{Descrizione} & \textbf{Impiego} \\\hline
	\endfirsthead
    Python & 3.0 & Linguaggio di programmazione ad alto livello, dinamico e interpretato & Sviluppo backend, gestione API ed embedding model \\\hline
    JavaScript & ES6 & Linguaggio di programmazione interpretato, principalmente utilizzato per lo sviluppo frontend & Sviluppo frontend, interattività delle pagine web, utilizzo di React \\\hline
    SQL & - & Linguaggio di programmazione per la gestione e manipolazione di database relazionali & Gestione database, query, manipolazione dati \\\hline
    YAML & 1.2 & Formato di serializzazione dati leggibile dall'uomo & Configurazione, script GitHub Actions \\\hline
    JSON & - & Formato di interscambio dati leggero e leggibile dall'uomo & Gestione database, scambio dati tra client e server \\\hline
    \caption{Linguaggi e formati utilizzati} 

\end{longtable}

\subsection{Librerie}   % per python rivedere requirments.txt e per JS rivedere package.json
\begin{longtable}{|>{\centering\arraybackslash}m{0.35\textwidth}|>{\centering\arraybackslash}m{0.10\textwidth}|>{\centering\arraybackslash}m{0.45\textwidth}|}
    \hline
    \multicolumn{3}{|c|}{\textbf{Python}} \\ \hline
    \textbf{Nome} & \textbf{Versione} & \textbf{Impiego} \\ \hline
    \endfirsthead
    \hline
    \textbf{Nome} & \textbf{Versione} & \textbf{Impiego} \\ \hline
    \endhead
    \texttt{Flask} & 3.1.0 & Framework per applicazioni web in Python. \\ \hline
    \texttt{Flask-Cors} & 5.0.0 & Estensione per Flask per gestire le richieste CORS. \\ \hline
    \texttt{langchain-core} & 0.3.31 & Modulo per la gestione dei documenti in LangChain. \\ \hline
    \texttt{langchain-openai} & 0.3.1 & Integrazione di OpenAI con LangChain. \\ \hline
    \texttt{requests} & 2.32.3 & Libreria per effettuare richieste HTTP. \\ \hline
    \texttt{python-dotenv} & 1.0.1 & Gestione delle variabili d’ambiente da file \texttt{.env}. \\ \hline
    \texttt{faiss-cpu} & 1.9.0.post1 & Gestione dei database vettoriali FAISS in LangChain. \\ \hline
    \texttt{numpy} & 2.2.2 & Libreria per il calcolo scientifico e la manipolazione di array. \\ \hline
    \texttt{openai} & 1.60.0 & Libreria per interfacciarsi con l’API di OpenAI. \\ \hline
    \texttt{SQLAlchemy} & 2.0.37 & Toolkit SQL per Python. \\ \hline

    \multicolumn{3}{|c|}{\textbf{JavaScript}} \\ \hline
    \texttt{-} & - & - \\ \hline
    \caption{Librerie utilizzate} 
\end{longtable}


\subsection{Servizi}
\subsubsection{OpenAI API}
L'API di OpenAI fornisce accesso a modelli di intelligenza artificiale avanzati, tra cui modelli di embedding. Un embedding model è un tipo di modello di machine learning che trasforma dati di input, come parole o frasi, in vettori di numeri in uno spazio continuo a bassa dimensione. Questi vettori catturano le caratteristiche semantiche dei dati di input, permettendo di misurare la similarità tra diversi input in modo efficiente.

\paragraph{Vantaggi:}
L'utilizzo di embedding models offre numerosi vantaggi, tra cui:
\begin{itemize}
    \item \textbf{Efficienza}: I vettori di embedding permettono di rappresentare dati complessi in modo compatto e computazionalmente efficiente.
    \item \textbf{Versatilità}: Possono essere utilizzati in una vasta gamma di applicazioni, tra cui il processamento del linguaggio naturale (NLP), la raccomandazione di contenuti e la classificazione dei dati.
\end{itemize}

\paragraph{Casi d'uso:}
Gli embedding models sono utilizzati in vari casi d'uso, tra cui:
\begin{itemize}
    \item \textbf{Ricerca di documenti}: Migliorano la ricerca di documenti trovando risultati più rilevanti basati sulla similarità semantica.
    \item \textbf{Raccomandazione di contenuti}: Personalizzano le raccomandazioni di contenuti in base alle preferenze dell'utente.
    \item \textbf{Classificazione del testo}: Aiutano nella classificazione automatica di testi in categorie predefinite.
\end{itemize}

\paragraph{Impiego del progetto:}
Nel progetto, l'API di OpenAI viene utilizzata per convertire il testo in token e generando embedding che rappresentano le caratteristiche semantiche del testo in uno spazio vettoriale.


\section{Introduzione}

\subsection{Scopo specifica tecnica}
Questo documento è rivolto a tutti gli stakeholder coinvolti nel progetto Code7Crusaders, un chatbot B2B pensato per semplificare la ricerca di prodotti all'interno dei cataloghi dei distributori.
Il documento fornisce una visione dettagliata dell’architettura del sistema, dei design pattern utilizzati, delle tecnologie adottate e delle scelte progettuali effettuate. Inoltre, include diagrammi UML delle classi e delle attività per descrivere il funzionamento del sistema in modo chiaro e strutturato.
  
\subsection{Scopo del prodotto}
Lo scopo del prodotto è realizzare un \textbf{Assistente Virtuale basato su \href{https://code7crusaders.github.io/docs/\href{https://code7crusaders.github.io/docs/PB/documentazione_interna/glossario.html#pb-product-baseline}{PB\textsuperscript{G}}/documentazione_interna/glossario.html#llm-large-language-model}{LLM\textsuperscript{G}} }, 
per supportare aziende produttrici di bevande nel fornire 
informazioni dettagliate e personalizzate sui loro prodotti. 
Il sistema si rivolge principalmente ai proprietari di locali, 
consentendo loro di ottenere risposte rapide e precise su caratteristiche, 
disponibilità e dettagli delle bevande, come se interagissero con uno specialista umano.

\subsection{\href{https://code7crusaders.github.io/docs/\href{https://code7crusaders.github.io/docs/PB/documentazione_interna/glossario.html#pb-product-baseline}{PB\textsuperscript{G}}/documentazione_interna/glossario.html#glossario}{Glossario}\textsuperscript{G}}
Per garantire una chiara comprensione della terminologia utilizzata nel documento,
è stato predisposto un \href{https://code7crusaders.github.io/docs/PB/documentazione_interna/glossario.html#glossario}{\emph{Glossario}\textsuperscript{G}} in un file dedicato. Questo strumento 
serve a evitare ambiguità nella definizione dei termini impiegati nell’attività progettuale, 
offrendo descrizioni precise e condivise. 


\subsection{Riferimenti}
\subsubsection{Riferimenti normativi}
\begin{itemize}
    \item \textbf{Capitolato C7 \href{https://code7crusaders.github.io/docs/PB/documentazione_interna/glossario.html#llm-large-language-model}{LLM}\textsuperscript{G}}: ASSISTENTE VIRTUALE \\ \url{https://www.math.unipd.it/~tullio/IS-1/2024/Progetto/C7.pdf}
    \item \textbf{Regolamento del progetto didattico} \\ \url{https://www.math.unipd.it/~tullio/IS-1/2024/Dispense/PD1.pdf}
    \item \href{https://code7crusaders.github.io/docs/PB/documentazione_interna/glossario.html#norme-di-progetto}{\textbf{Norme di Progetto}\textsuperscript{G}} v.2.0 \\ \url{https://code7crusaders.github.io/docs/\href{https://code7crusaders.github.io/docs/PB/documentazione_interna/glossario.html#pb-product-baseline}{PB\textsuperscript{G}}/documentazione_interna/norme_di_progetto.html}
\end{itemize}

\subsubsection{Riferimenti informativi}
\begin{itemize}
    \item \textbf{Slide Corso Ingegneria del software: \href{https://code7crusaders.github.io/docs/PB/documentazione_interna/glossario.html#analisi-dei-requisiti}{Analisi dei Requisiti}\textsuperscript{G}} \\ \url{https://www.math.unipd.it/~tullio/IS-1/2024/Dispense/T05.pdf}
    \item \textbf{Slide Corso Ingegneria del software: Diagrammi delle classi} \\ \url{https://www.math.unipd.it/~rcardin/swea/2023/Diagrammi%20delle%20Classi.pdf}
    \item \textbf{Slide Corso Ingegneria del software: Diagrammi dei casi d'uso}\\ \url{https://www.math.unipd.it/~rcardin/swea/2022/Diagrammi%20Use%20Case.pdf}
    \item \href{https://code7crusaders.github.io/docs/PB/documentazione_interna/glossario.html#glossario}{\textbf{Glossario}\textsuperscript{G}} v.2.0 \\ \url{https://code7crusaders.github.io/docs/PB/documentazione_interna/glossario.html}
    \item \textbf{Analisi \href{https://code7crusaders.github.io/docs/PB/documentazione_interna/glossario.html#llm-large-language-model}{LLM}\textsuperscript{G}} \\ \url{https://code7crusaders.github.io/docs/altri_documenti/analisi_modelli_firmato.html}
    \item \textbf{Analisi framework frontend} \\ \url{https://code7crusaders.github.io/docs/altri_documenti/analisi_frontend_firmato.html}
    \item \textbf{Analisi framework backend} \\ \url{https://code7crusaders.github.io/docs/altri_documenti/analisi_framework_backend.html}
    \item \textbf{Analisi database Vettoriale} \\ \url{https://code7crusaders.github.io/docs/altri_documenti/analisi_dbvettoriale.html}
    \item \href{https://code7crusaders.github.io/docs/PB/documentazione_interna/glossario.html#langchain}{\textbf{LangChain}\textsuperscript{G}} \\ \url{https://python.langchain.com/docs/introduction/}
    \item \textbf{OpenAI} \\ \url{https://openai.com/}
\end{itemize}

\section{Tracciamento dei requisiti}
In questa sezione vengono descritti i requisiti del sistema e il loro tracciamento. Ogni requisito è identificato da un codice univoco che ne facilita la gestione e il monitoraggio. I requisiti sono suddivisi in categorie in base alla loro natura (funzionali, di qualità, di vincolo) e alla loro importanza
(obbligatori, desiderabili, facoltativi). Di seguito viene presentata una tabella che traccia i requisiti funzionali del sistema, indicando per ciascuno di essi il codice identificativo, la descrizione e lo stato di soddisfacimento.
\newline I requisiti sono codificati come segue: \textbf{R[Tipo][Importanza][Numero]}
\newline
Dove \textbf{Tipo} può essere:
\begin{itemize}
	\item \textbf{F (funzionale)}
	\item \textbf{Q (di qualità)}
	\item \textbf{V (di vincolo)}
\end{itemize}
\textbf{Importanza} può essere:
\begin{itemize}
	\item \textbf{O (obbligatorio)}
	\item \textbf{D (desiderabile)}
	\item \textbf{F (facoltativo)}
\end{itemize}
\textbf{Numero} è un numero identificativo univoco del requisito.

\subsection{Tracciamento requisiti funzionali}

\begin{longtable}{|>{\centering\arraybackslash}m{0.08\textwidth}|>{\centering\arraybackslash}m{0.64\textwidth}|>{\centering\arraybackslash}m{0.18\textwidth}|}
	\hline
	\textbf{Codice} & \textbf{Descrizione} & \textbf{Stato}\\\hline
	\endfirsthead
	\hline
	\textbf{Codice} & \textbf{Descrizione} & \textbf{Stato}\\\hline
	\endhead
	\hline
    \textbf{RFO1} & L'amministratore inserisce dalla pagina di gestione i dati semantici aziendali da cui apprendere la conoscenza da file in formato .pdf. & - \\
    \hline
    \textbf{RFO2} & L'amministratore inserisce dalla pagina di gestione i dati semantici aziendali da cui apprendere la conoscenza da file in formato .txt. & - \\
    \hline
    \textbf{RFO3} & I testi recuperati dai documenti verranno suddivisi in blocchi, ovvero pezzi più piccoli di dati che rappresentano una piccola porzione del contesto. & - \\
    \hline
    \textbf{RFO4} & I vettori generati verranno memorizzati all’interno di un database vettoriale e opportunamente indicizzati. & - \\
    \hline
    \textbf{RFO5} & Da un’interfaccia utente della web app, viene catturata una domanda da parte dell’utente. & - \\
    \hline
    \textbf{RFO6} & La domanda viene inoltrata al sistema attraverso delle API REST risiedenti in un Web Server. & - \\
    \hline
    \textbf{RFO7} & La rappresentazione vettoriale viene utilizzata per effettuare una ricerca all’interno del database vettoriale da dove vengono reperiti i vettori più simili. & - \\
    \hline
    \textbf{RFO8} & La domanda viene inviata al sistema LLM tramite API. & - \\
    \hline
    \textbf{RFO9} & Viene attesa la risposta dall'LLM tramite API. & - \\
    \hline
    \textbf{RFO10} & Attraverso API REST, il sistema inoltra la risposta all'account dell’utente. & - \\
    \hline
    \textbf{RFO11} & L'utente deve essere in grado di ottenere informazioni riguardo un prodotto attraverso la conversazione con il bot. & - \\
    \hline
    \textbf{RFO12} & L'utente deve essere in grado di ottenere informazioni riguardo una serie di prodotti attraverso la conversazione con il bot. & - \\
    \hline
    \textbf{RFO13} & La conversazione tra utente e bot deve essere salvata. & - \\
    \hline
    \textbf{RFO14} & L'utente deve essere in grado di visualizzare una delle conversazioni precedentemente salvate. & - \\
    \hline
    \textbf{RFO15} & L'utente deve essere in grado di riprendere una delle conversazioni precedentemente salvata. & - \\
    \hline
    \textbf{RFO16} & L'utente o l'amministratore devono poter accedere al sistema inserendo Username e Password. & - \\
    \hline
    \textbf{RFO17} & L'utente si registra inserendo Username e Password. & - \\
    \hline
    \textbf{RFO18} & Gli input del form di registrazione devono essere sanificati per prevenire attacchi SQL Injection. & - \\
    \hline
    \textbf{RFO19} & Gli input del form di accesso devono essere sanificati per prevenire attacchi SQL Injection. & - \\
    \hline
    \textbf{RFO20} & L'utente deve essere in grado di dare un feedback (thumbsup/thumbsdown) sulla qualità della conversazione dopo averla provata. & - \\
    \hline
    \textbf{RFO21} & L’accesso alla dashboard dei "template di domanda e risposta" è consentito solo agli utenti con ruolo di amministratore. & - \\
    \hline
    \textbf{RFO22} & Dopo l’accesso da parte dell'amministratore, la pagina di gestione mostra la dashboard dei "template di domanda e risposta". & - \\
    \hline
    \textbf{RFO23} & Un "template di domanda e risposta" è formato da una domanda (possibilmente una domanda posta frequentemente che l'amministratore decide di inserire per risparmiare una chiamata al modello) associata ad una corrispondente risposta. & - \\
    \hline
    \textbf{RFO24} & L'amministratore deve essere in grado di creare un template, che è formato da una domanda associata ad una corrispondente risposta. & - \\
    \hline
    \textbf{RFO25} & L'amministratore deve essere in grado di modificare uno dei template esistenti. & - \\
    \hline
    \textbf{RFO26} & L'amministratore deve essere in grado di eliminare un template esistente. & - \\
    \hline
    \textbf{RFO27} & Il sistema deve poter fermare la creazione di un template invalido, ovvero quando il template non rispetta il formato Json. & - \\
    \hline
    \textbf{RFF28} & L'amministratore deve poter accedere alla dashboard di monitoraggio delle metriche. & - \\
    \hline
    \textbf{RFF29} & L’accesso alla dashboard delle metriche delle run è consentito solo agli utenti con ruolo di amministratore. & - \\
    \hline
    \textbf{RFF30} & Dopo l’accesso da parte dell'amministratore, la pagina di gestione mostra la dashboard delle metriche delle run. & - \\
    \hline
    \textbf{RFF31} & L’amministratore deve poter selezionare criteri di filtro per visualizzare solo le run di interesse. & - \\
    \hline
    \textbf{RFF32} & Il sistema deve permettere la selezione di filtri come ID, nome, input, data di inizio e fine, errore, output, tag, numero di token, costo. & - \\
    \hline
    \textbf{RFF33} & Una volta selezionati i filtri, il sistema deve aggiornare la visualizzazione senza ricaricare l'intera pagina. & - \\
    \hline
    \textbf{RFF34} & Se nessun filtro è selezionato, il sistema mostra le prime dieci run per impostazione predefinita. & - \\
    \hline
    \textbf{RFF35} & Dopo aver applicato i filtri, l’amministratore deve poter visualizzare le metriche principali delle run selezionate. & - \\
    \hline
    \textbf{RFF36} & Il sistema deve mostrare le metriche principali delle run filtrate (ID, nome, input, data di inizio e fine, errore, output, tag, token totali, costo totale). & - \\
    \hline
    \textbf{RFF37} & La visualizzazione deve essere chiara e strutturata, con possibilità di ordinare le colonne. & - \\
    \hline
    \textbf{RFO38} & L'amministratore deve poter visualizzare i feedback dati dagli utenti. & - \\
    \hline
    \textbf{RFO39} & Il sistema deve poter rifiutare l'importazione dati di file non compatibili, ovvero file non nel formato pdf o txt. & - \\
    \hline
    \textbf{RFO40} & L'utente deve poter eliminare una conversazione precedentemente effettuata. & - \\
    \hline
    \textbf{RFO41} & L'utente deve poter mandare richieste di assistenza per poter parlare con un operatore umano. & - \\
    \hline
    \textbf{RFO42} & L’accesso alla dashboard delle richieste di assistenza è consentito solo agli utenti con ruolo di amministratore. & - \\
    \hline
    \textbf{RFO43} & Dopo l’accesso da parte dell'amministratore, la pagina di gestione mostra la dashboard delle richieste di assistenza. & - \\
    \hline
    \textbf{RFO44} & L'amministratore deve poter visualizzare le richieste di assistenza ricevute da parte dell'utente. & - \\
    \hline
    \textbf{RFO45} & L'amministratore deve poter segnalare ad altri amministratori che una richiesta è stata presa in carico. & - \\
    \hline
    \textbf{RFD46} & L'amministratore deve essere in grado di poter rispondere all'utente tramite contatto via e-mail. & - \\
    \hline
    \textbf{RFF47} & Le metriche delle run del chatbot devono essere esportabili in JSON. & - \\
    \hline
    \textbf{RFF48} & Le metriche della run devono includere ID univoco della run, nome assegnato alla sessione, dati di input elaborati dal modello, timestamp di avvio e completamento dell'esecuzione, eventuali errori incontrati, risultato generato dal modello, numero totale di token utilizzati e stima dei costi basata sul consumo di token. & - \\
    \hline
    \textbf{RFO49} & Il bot per rispondere a una domanda deve ricordarsi i messaggi precedenti nella singola conversazione. & - \\
    \hline
    \textbf{RFD50} & Il sistema deve notificare l'utente quando la memoria per le chat salvate è piena e non è possibile salvare ulteriori conversazioni. & - \\
    \hline
    \textbf{RFO51} & L'utente seleziona una delle domande tra quelle predefinite. & - \\
    \hline
    \textbf{RFO52} & L'utente deve essere in grado di visualizzare una lista delle conversazioni precedentemente salvate. & - \\
    \hline
    \textbf{RFO53} & La lunghezza massima dell'username è di 256 caratteri. & - \\
    \hline
    \textbf{RFO54} & La lunghezza massima della password è di 256 caratteri. & - \\
    \hline
    \textbf{RFO55} & Il Sistema rifiuta la registrazione di un nuovo account con username già presente. & - \\
    \hline

\caption{Tabella Requisiti funzionali soddisfatti}
\end{longtable}

\subsection{Tracciamento requisiti di vincolo}

\begin{longtable}{|>{\centering\arraybackslash}m{0.08\textwidth}|>{\centering\arraybackslash}m{0.64\textwidth}|>{\centering\arraybackslash}m{0.18\textwidth}|}
	\hline
	\textbf{Codice} & \textbf{Descrizione} & \textbf{Stato}\\\hline
	\endfirsthead
	\hline
	\textbf{Codice} & \textbf{Descrizione} & \textbf{Stato}\\\hline
	\endhead
    \hline
    \textbf{RVO1} & Il chatbot deve rispondere con il contesto dato dai file di allenamento (pdf o file di testo inseriti) & - \\
    \hline
    \textbf{RVO2} & LLM deve essere integrato tramite API & - \\
    \hline
    \textbf{RVO3} & LLM utilizzato deve essere quello di OpenAI & - \\
    \hline
    \textbf{RVO4} & Deve essere usato un database relazionale & - \\
    \hline
    \textbf{RVO5} & Deve essere gestito il salvataggio delle chat precedenti con tutti i messaggi in esse tramite un database relazionale con PostgreSQL & - \\
    \hline
    \textbf{RVO6} & Deve essere implementato un database vettoriale & - \\
    \hline
    \textbf{RVO7} & Deve essere implementato un database vettoriale FAISS per poter rendere possibile la ricerca con contesto dall'LLM & - \\
    \hline
    \textbf{RVO8} & Deve essere implementato un embedding model & - \\
    \hline
    \textbf{RVO9} & L'embedding model deve essere quello di OpenAI & - \\
    \hline
    \textbf{RVO10} & Deve essere implementata una WebApp che permetta di comunicare con il chatbot & - \\
    \hline
    \textbf{RVO11} & L’interfaccia deve essere costruita utilizzando componenti funzionali React & - \\
    \hline
    \textbf{RVO12} & Si deve creare un backend che gestisca le chiamate HTTP, il database vettoriale e il database relazionale con Flask & - \\
    \hline
    \textbf{RVO13} & La gestione dello stato locale deve essere implementata tramite useState & - \\
    \hline
    \textbf{RVO14} & La WebApp deve utilizzare React Router per gestire la navigazione tra le pagine & - \\
    \hline
    \textbf{RVO15} & Gli stili devono essere gestiti tramite CSS inline o con className per garantire modularità & - \\
    \hline
    \textbf{RVO16} & La comunicazione tra componenti deve essere gestita inviando funzioni come props & - \\
    \hline
    \textbf{RVO17} & La WebApp deve essere responsiva e adattarsi dinamicamente alle dimensioni della finestra & - \\
    \hline
    \textbf{RVO18} & La gestione dei blocchi di testo vettorializzati deve essere gestita tramite Faiss & - \\
    \hline
    \textbf{RVD19} & Le metriche delle run del chatbot devono essere recuperate tramite Langsmith & - \\
    \hline
    \textbf{RVO20} & Bisogna usare la libreria LangChain per la interazione con i modelli LLM e Embedding & - \\
    \hline
\caption{Tabella Requisiti di vincolo soddisfatti}
\end{longtable}
\newpage
\subsection{Tracciamento requisiti di qualità}

\begin{longtable}{|>{\centering\arraybackslash}m{0.08\textwidth}|>{\centering\arraybackslash}m{0.64\textwidth}|>{\centering\arraybackslash}m{0.18\textwidth}|}
	\hline
	\textbf{Codice} & \textbf{Descrizione} & \textbf{Stato}\\\hline
	\endfirsthead
	\hline
	\textbf{Codice} & \textbf{Descrizione} & \textbf{Stato}\\\hline
	\endhead
	\hline
    \textbf{RQO1} & Schema di progettazione della base di dati & - \\
    \hline
    \textbf{RQO2} & Codice prodotto in formato sorgente reso disponibile tramite repository pubblici & - \\
    \hline
    \textbf{RQO3} & Documentazione riassuntiva delle metriche e dei risultati & - \\
    \hline
    \textbf{RQO4} & Il software deve essere testato con una copertura di codice minima dell'80\% e una copertura dei rami dell'80\%, con un obiettivo ottimale del 100\% & - \\
    \hline
    \textbf{RQO5} & Il 90\% dei test deve essere superato come requisito minimo, mentre l'obiettivo ottimale è il 100\% & - \\
    \hline
    \textbf{RQO6} & La metodologia di sviluppo deve seguire il paradigma del Test Driven Development (TDD), garantendo che il codice venga scritto partendo dai test & - \\
    \hline

\caption{Tabella Requisiti di qualità soddisfatti}
\end{longtable}


\subsection{Soddisfazione totale dei requisiti}
Il gruppo Code7Crusaders ha soddisfatto - su -, arrivando ad una copertura del -\%.
\begin{center}
\begin{tabular}{|c|c|}
\hline
\textbf{Soddisfatti} & \textbf{Non soddisfatti} \\
\hline
- & - \\
\hline
\end{tabular}
\end{center}
\begin{center}
\textbf{Grafico 1: Requisiti soddisfatti rispetto al totale.}
\end{center}

Per quanto riguarda la copertura dei requisiti obbligatori, la copertura rilevata è di - su - requisiti, arrivando quindi ad un -\% sul totale.
\begin{center}
\begin{tabular}{|c|c|}
\hline
\textbf{Soddisfatti} & \textbf{Non soddisfatti} \\
\hline
- & - \\
\hline
\end{tabular}
\end{center}
\begin{center}
\textbf{Grafico 2: Requisiti obbligatori soddisfatti rispetto al totale.}
\end{center}

In termini di soddisfacimento dei requisiti desiderabili, è stata raggiunta una copertura del -\%, con - su -.
\begin{center}
\begin{tabular}{|c|c|}
\hline
\textbf{Soddisfatti} & \textbf{Non soddisfatti} \\
\hline
- & - \\
\hline
\end{tabular}
\end{center}
\begin{center}
\textbf{Grafico 3: Requisiti desiderabili soddisfatti rispetto al totale.}
\end{center}
Per quanto concerne l’adempimento dei requisiti opzionali, abbiamo conseguito una percentuale del -\% sul totale, con - su - requisiti considerati.
\begin{center}
    \begin{tabular}{|c|c|}
    \hline
    \textbf{Soddisfatti} & \textbf{Non soddisfatti} \\
    \hline
    - & - \\
    \hline
    \end{tabular}
    \end{center}
    \begin{center}
    \textbf{Grafico 4: Requisiti opzionali soddisfatti rispetto al totale.}
    \end{center}
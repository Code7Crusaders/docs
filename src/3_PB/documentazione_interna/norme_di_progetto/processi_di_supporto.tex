\section{Processi di Supporto}
% Suddivoso in parti:
% Documentazione
% Gestione della configurazione
% Gestione della qualità 
% Verifica 
% Validazione




%------------DOCUMENTAZIONE---------------------------
\subsection{Documentazione}
\subsubsection{Introduzione}
La documentazione software si riferisce al testo che accompagna un programma, 
descrivendo il prodotto sia per gli sviluppatori che per gli utilizzatori. 
Essa ha l'obiettivo di supportare i membri del team durante lo sviluppo, monitorando i 
processi e documentando tutte le attività, per facilitare anche la manutenzione del software 
e migliorare la qualità del prodotto finale.

In base a quanto sopra, la documentazione svolge un ruolo cruciale nel ciclo di vita del software. 
Le aspettative nei suoi confronti includono:
\begin{itemize}
    \item Definizione di regole chiare e concise per la redazione dei documenti.
    \item Adozione di una struttura uniforme e standard per tutti i documenti nel ciclo di vita del software, per garantire omogeneità.
\end{itemize}

\subsubsection{Ciclo di Vita del Documento}

Il ciclo di vita di un documento software si articola in tre fasi principali:
\begin{itemize}
    \item \textbf{Redazione}: la fase di creazione del documento, che viene suddivisa tra i membri del gruppo e supportata dall'uso di un sistema di versionamento.
    \item \textbf{Verifica}: una volta completata la stesura, il documento passa alla fase di verifica, 
    che può essere effettuata su parti del documento o su tutto il contenuto. 
    Ogni sezione deve essere verificata da una persona distinta dal redattore della sezione stessa.
    \item \textbf{Approvazione}: il documento, una volta completato e verificato, viene approvato dal Responsabile di Progetto.
\end{itemize}

\subsubsection{Template}
Il gruppo ha deciso di utilizzare un template semplice, creato con Latex. Questo modello è stato 
standardizzato e viene utilizzato per la redazione di tutti i documenti ufficiali.

\subsubsection{Documenti Prodotti}

I documenti generati durante il ciclo di vita del software sono suddivisi in due categorie principali:

\subsubsection*{Formali}
I documenti formali sono quelli con un nome univoco e utilizzati per regolare le attività interne al gruppo 
durante tutto il ciclo di vita del software. 
Sono versionati e approvati dal Responsabile di Progetto. 
Questi documenti si suddividono in:
\begin{itemize}
    \item \textbf{Ad uso interno}: destinati esclusivamente ai membri del gruppo, come ad esempio:
    \begin{itemize}
        \item Norme di progetto
        \item Verbali interni
    \end{itemize}
    \item \textbf{Ad uso esterno}: destinati a enti esterni come il committente o il proponente, e consegnati 
    nell'ultima versione approvata. Tra questi:
    \begin{itemize}
        \item \href{https://code7crusaders.github.io/docs/RTB/documentazione_interna/glossario.html#analisi-dei-requisiti}{Analisi dei Requisiti\textsuperscript{G}}
        \item \href{https://code7crusaders.github.io/docs/RTB/documentazione_interna/glossario.html#piano-di-progetto}{Piano di Progetto\textsuperscript{G}}
        \item \href{https://code7crusaders.github.io/docs/RTB/documentazione_interna/glossario.html#piano-di-qualifica}{Piano di Qualifica\textsuperscript{G}}
        \item \href{https://code7crusaders.github.io/docs/RTB/documentazione_interna/glossario.html#glossario}{Glossario\textsuperscript{G}}
        \item Verbali esterni
    \end{itemize}
\end{itemize}

\subsubsection*{Informali}
I documenti informali comprendono:
\begin{itemize}
    \item Documenti non ancora approvati dal Responsabile di Progetto.
    \item Bozze e appunti brevi.
    \item Documenti che non necessitano di essere versionati.
\end{itemize}
Questi documenti sono gestiti in una sezione separata, dove il gruppo ha creato un Google Drive condiviso per 
facilitarne la gestione.

\subsubsection{Struttura del Documento}

Tutti i documenti ufficiali seguono una struttura rigida che deve essere rispettata. La struttura include:

\subsubsection*{Prima Pagina}
La prima pagina include:
\begin{itemize}
    \item Il titolo del gruppo.
    \item Il nome del documento.
    \item Il logo del gruppo.
    \item I membri del team.
\end{itemize}

\subsubsection*{Registro dei Cambiamenti - Changelog}
Il registro dei cambiamenti tiene traccia della storia del documento. In questa sezione sono inclusi:
\begin{itemize}
    \item La versione del documento.
    \item La data di ogni modifica.
    \item L'autore che ha effettuato la modifica.
    \item Il verificatore delle modifiche.
    \item Una breve descrizione delle modifiche.
\end{itemize}

\subsubsection*{Indice}
Ogni documento include un indice subito dopo il registro dei cambiamenti. 
Questo indice aiuta a navigare nel documento, rendendo più facile la ricerca di sezioni specifiche.

\subsubsection{Verbali}
I verbali sono documenti speciali con una struttura diversa rispetto agli altri. Non includono il registro dei cambiamenti né l'indice.
La struttura dopo la prima pagina prevede:
\begin{itemize}
    \item \textbf{Partecipanti}: orario di inizio e fine dell'incontro, seguito da una tabella con i nomi e le durate di presenza dei partecipanti.
    \item \textbf{Sintesi ed elaborazione incontro}: un riassunto degli argomenti trattati e una sezione per eventuali dubbi o indicazioni per i prossimi incontri.
\end{itemize}
I verbali sono suddivisi in interni (tra i membri del gruppo) ed esterni (con l'azienda o il committente).


\subsubsection{Nome del File}
I file devono avere nomi coerenti, con la lettera iniziale minuscola, 
Per quanto riguarda i verbali, sia interni che esterni, il nome del file deve essere "verbale\_YY-MM-DD\_vXX", dove:
\begin{itemize}
    \item \textbf{YY-MM-DD}: data dell'incontro.
    \item \textbf{XX}: numero progressivo del verbale.
\end{itemize}

\subsubsection{Stile del Testo}
Lo stile del testo nei documenti ufficiali include:
\begin{itemize}
    \item \textbf{Grassetto}: per titoli e parole di rilevanza.
    \item \underline{Sottolineato}: solo per i link.
\end{itemize}

\subsubsection{Glossario}
Il glossario è un documento contenente termini e definizioni utili per comprendere meglio il linguaggio tecnico, evitando ambiguità. 
I termini sono registrati in ordine alfabetico.


\subsubsection{Tabelle}
Le tabelle nei documenti ufficiali devono avere un titolo che descriva il contenuto e devono essere centrate orizzontalmente nella pagina.

\subsubsection{Immagini}
Le immagini devono essere centrate orizzontalmente. Anche i diagrammi UML sono trattati come immagini.

\subsubsection{Metriche}
\begin{table}[h!]
    \centering
    \renewcommand{\arraystretch}{1.5}
    \begin{tabular}{|l|l|}
        \hline
        \textbf{Metrica} & \textbf{Nome} \\
        \hline
        1PSM-IG & Indice di Gulpease \\
        \hline
        2PSM-CO & Correttezza Ortografica \\
        \hline
    \end{tabular}
    \caption{Metriche relative al processo di Documentazione.}
\end{table}









%------------CONFIGURAZIONE--------------------------------
\subsection{Gestione della Configurazione}
La gestione della configurazione è un processo fondamentale per mantenere il software in uno stato coerente, 
garantendo che il sistema continui a funzionare correttamente nonostante le modifiche apportate nel tempo.
Problemi di configurazione possono causare incoerenze o non conformità, con un impatto negativo sulle operazioni del sistema.
Il gruppo, attraverso la gestione della configurazione, mira a:
\begin{itemize}
    \item Individuare e risolvere i problemi prima che diventino critici;
    \item Facilitare il tracciamento delle modifiche e l’identificazione degli errori.
\end{itemize}

\subsubsection{Versionamento}
Il versionamento è un processo che consente di tracciare le modifiche apportate a un documento. 
Inoltre, permette di ripristinare il documento a uno stadio precedente e di visualizzare i cambiamenti effettuati nel tempo, 
associandoli al relativo autore.
Il nostro gruppo ha adottato il seguente formato per identificare la versione di un documento:  
\[
[x].[y]
\]
Dove:
\begin{itemize}
    \item \textbf{x}: numero intero che parte da 0 e viene incrementato dal Responsabile di Progetto (\textbf{RdP}) dopo l'approvazione del documento (versione di produzione);
    \item \textbf{y}: numero intero incrementato dal Verificatore (\textbf{Ve}) a ogni verifica del documento;
\end{itemize}

\subsubsection{Repository}
Il nostro gruppo ha deciso di utilizzare per la gestione della configurazione il servizio GitHub, 
basato sul sistema di controllo di versione distribuito Git.








%------------QUALITA--------------------------------
\subsection{Gestione della Qualità}
\subsubsection{Descrizione}
La gestione della qualità di progetto comprende i processi e le attività svolte all’interno di un progetto per garantire 
che la qualità dei deliverable e delle performance siano in linea con gli obiettivi e i requisiti definiti.
I membri del nostro gruppo si pongono i seguenti obiettivi:
\begin{itemize}
    \item Comprendere, valutare e gestire le aspettative del committente, assicurandosi che i requisiti siano rispettati;
    \item Definire chiaramente i requisiti di qualità e documentare tutte le procedure necessarie per completare 
            il progetto in conformità con le aspettative richieste;
    \item Consegnare il progetto in linea con il piano di qualità, garantendo che il prodotto finale sia consegnato 
            nei tempi previsti, rispettando il budget e soddisfacendo i requisiti e le aspettative del committente.
\end{itemize}

\subsubsection{Piano di Qualifica}
Per garantire il rispetto di tutti gli aspetti del processo di gestione della qualità, utilizziamo il 
\href{https://code7crusaders.github.io/docs/RTB/documentazione_interna/glossario.html#piano-di-qualifica}{\textbf{Piano di Qualifica}\textsuperscript{G}}, un documento che include un elenco strutturato dei dati necessari per assicurare un piano di alta qualità. In particolare, il \href{https://code7crusaders.github.io/docs/RTB/documentazione_interna/glossario.html#piano-di-qualifica}{Piano di Qualifica\textsuperscript{G}} prevede:  
\begin{itemize}
    \item La definizione dei requisiti richiesti dal committente;
    \item L’identificazione di metriche e parametri per l’analisi dei dati;
    \item L’implementazione di un sistema per il controllo della qualità durante l’intero ciclo di vita del progetto;
    \item La pianificazione di un sistema di miglioramento che descriva le azioni necessarie per analizzare le prestazioni di qualità e individuare le attività utili a incrementare il valore del progetto.
\end{itemize}


\subsubsection{Metriche}
\begin{table}[h!]
    \centering
    \renewcommand{\arraystretch}{1.5}
    \begin{tabular}{|l|l|}
        \hline
        \textbf{Metrica} & \textbf{Nome} \\
        \hline
        3PSM-FU & Facilità di Utilizzo \\
        \hline
        4PSM-TA & Tempo di Apprendimento \\
        \hline
        5PSM-TR & Tempo di Risposta \\
        \hline
        6PSM-TE & Tempo di Elaborazione\\
        \hline
        7PSM-QMS & Metriche di Qualità Soddisfatte \\
        \hline
    \end{tabular}
    \caption{Metriche relative al processo di Gestione della Qualità.}
\end{table}











%------------VERIFICA--------------------------------
\subsection{Verifica}
\subsubsection{Introduzione}
La verifica del software è il processo di valutazione del prodotto per garantire che la fase di sviluppo 
sia eseguita correttamente, al fine di costruire il prodotto desiderato. 
Questo processo si svolge durante lo sviluppo del software, consentendo di rilevare difetti e guasti 
nelle fasi iniziali del ciclo di vita e di verificare che il prodotto soddisfi i requisiti del cliente.

Le aspettative per questo processo includono:
\begin{itemize}
    \item Incrementare la fiducia del gruppo nel proseguire lo sviluppo del progetto in modo corretto;
    \item Garantire il raggiungimento del prodotto finale atteso;
    \item Identificare precocemente gli errori, riducendo così i costi e il tempo necessari per le correzioni.
\end{itemize}

\subsubsection{Tipi di Verifica}
Il processo di verifica si compone di due tipi principali, ciascuno focalizzato su diversi aspetti del software. 
Insieme, questi due tipi garantiscono che il software sia conforme ai requisiti specificati. 
Inoltre, viene considerato un terzo tipo per la verifica della documentazione.

\subsubsection*{Analisi Statica}
L’analisi statica consiste nell’ispezione del codice prima della sua esecuzione, assicurando che il software soddisfi 
i requisiti e le specifiche definiti. Poiché non richiede l’esecuzione dell’oggetto in verifica, questo tipo di analisi 
può essere applicato non solo al codice, ma anche alla documentazione.
Questo approccio analizza gli aspetti statici del sistema software, come le convenzioni del codice o il calcolo di metriche. 
Include sia tecniche di test manuali che automatizzate, come quelle orientate alla coerenza.  
L’analisi statica si divide in due metodi principali:
\begin{itemize}
    \item \textbf{Walkthroughs}: una lettura di ampio spettro che consente di esaminare e discutere 
    eventuali errori o difetti trovati. Questo metodo è utile quando non si ha certezza sulla posizione dei problemi.
    \item \textbf{Inspection}: un metodo mirato per identificare e rimuovere errori e difetti. 
    Si utilizza un approccio più focalizzato, avendo già un'idea delle possibili problematiche.
\end{itemize}

\subsubsection*{Analisi Dinamica}
L’analisi dinamica viene eseguita durante l’esecuzione del software e consiste nella fase di test. A differenza della verifica statica, comporta l’esecuzione del sistema e dei suoi componenti.  

Il gruppo adotterà un insieme di test ripetibili e automatizzati. L’automatizzazione sarà possibile attraverso strumenti dedicati, che verranno definiti successivamente.

\subsubsection*{Verifica della Documentazione}
La verifica della documentazione si compone delle seguenti attività:
\begin{itemize}
    \item Controllo di ortografia e sintassi;
    \item Controllo dell’utilizzo corretto delle norme tipografiche o di altre norme di stile e formattazione concordate;
    \item Verifica della pertinenza dei contenuti scritti.
\end{itemize}


\subsubsection{Metriche}
\begin{table}[h!]
    \centering
    \renewcommand{\arraystretch}{1.5}
    \begin{tabular}{|l|l|}
        \hline
        \textbf{Metrica} & \textbf{Nome} \\
        \hline
        8PSM-CC & Code Coverage \\
        \hline
        9PSM-BC & Branch Coverage \\
        \hline
        10PSM-SC & Statement Coverage \\
        \hline
        11PSM-FD & Failure Density\\
        \hline
        12PSM-PTCP & Passed Test Case Percentage \\
        \hline
    \end{tabular}
    \caption{Metriche relative al processo di Verifica.}
\end{table}







%------------VALIDAZIONE--------------------------------
\subsection{Validazione}
\subsubsection{Introduzione}
La validazione del software è un processo di valutazione del prodotto, finalizzato a garantire che il 
software soddisfi i requisiti predefiniti e specificati dal richiedente, nonché le richieste e le aspettative 
degli utenti finali/proponente. Un processo di validazione ha successo quando è stata effettuata una buona 
verifica durante tutta la fase di sviluppo.
L’esito finale positivo della validazione assicura che il prodotto finale sia allineato con le aspettative.
\begin{itemize}
    \item Rilevare possibili errori ignorati o trascurati durante la fase di verifica;
    \item Soddisfare i requisiti specificati nell’\href{https://code7crusaders.github.io/docs/RTB/documentazione_interna/glossario.html#analisi-dei-requisiti}{Analisi dei Requisiti\textsuperscript{G}} per il prodotto finale;
    \item Contribuire a migliorare la qualità e il valore del prodotto software finale.
\end{itemize}



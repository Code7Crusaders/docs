%%% INTRO %%%%%%%%%%%%%%%%%%%%%%%%%%%%%%%%%%%%%%%%%%%%%%%%%%%%%%%%%%%%%%%%%%%%%%%%%%%%
% Template sia per verbali interni che esterni
% Segui i commenti "TODO" per ricordarti cosa modificare
% In caso di verbali esterni ricordati di settare isEsterno ad 1
%%%%%%%%%%%%%%%%%%%%%%%%%%%%%%%%%%%%%%%%%%%%%%%%%%%%%%%%%%%%%%%%%%%%%%%%%%%%%%%%%%%%%



%%% Settings %%%%%%%%%%%%%%%%%%%%%%%%%%%%%%%%%%%%%%%%%%%%%%%%%%%%%%%%%%%%%%%%%%%%%%%%%
\documentclass{article}

\usepackage{graphicx}  % serve per inserire immagini
\usepackage{fancyhdr}  % creazione header-footer
\usepackage{tabularx}  % serve per creare tabelle con colonne a larghezza variabile
\usepackage{ifthen}  % serve per mostrare cose diverse in base a condizioni
\usepackage{geometry}
\usepackage{setspace}
\usepackage{tikz}
\usepackage[italian]{babel}
\usepackage[hidelinks]{hyperref}

% TODO: setta a 1 se il verbale è esterno, 0 se è interno
\newcommand{\isEsterno}{0}

% Margini della pagina
\geometry{a4paper, margin=1in}

% Intestazione personalizzata
\pagestyle{fancy}
\fancyhf{}
\fancyhead[L]{Code7Crusaders - Software Development Team}
\fancyhead[R]{\thepage}

% Spaziatura delle righe
\setstretch{1.2}

\begin{document}
%%%%%%%%%%%%%%%%%%%%%%%%%%%%%%%%%%%%%%%%%%%%%%%%%%%%%%%%%%%%%%%%%%%%%%%%%%%%%%%%%%%%%%



%%% Sezione del titolo %%%%%%%%%%%%%%%%%%%%%%%%%%%%%%%%%%%%%%%%%%%%%%%%%%%%%%%%%%%%%%%
\begin{titlepage}

    \AddToHookNext{shipout/background}{
        \begin{tikzpicture}[remember picture,overlay]
        \node at (current page.center) {
            \includegraphics{../../img/background.png}
        };
        \end{tikzpicture}
    }

    \centering
    \vspace*{2cm}
    
    \includegraphics[width=0.3\textwidth]{../../img/logo/7Crusaders_logo.png} % logo
    \vspace{1cm}
    
    {\Huge \textbf{Code7Crusaders}}\\
    \vspace{0.5cm}
    {\Large Software Development Team}\\
    \vspace{2cm}
        
        {\large \textbf{Riunione Settimanale 28/02/2025}}\\
    \vspace{5cm}                           % esempio: Riunione Settimanale 04/11/2024
    
    
    \textbf{Membri del Team:}\\
    Enrico Cotti Cottini, Gabriele Di Pietro, Tommaso Diviesti \\
    Francesco Lapenna, Matthew Pan, Eddy Pinarello, Filippo Rizzolo \\
    \vspace{0.5cm}
    
    \vspace{1cm}
\end{titlepage}
%%%%%%%%%%%%%%%%%%%%%%%%%%%%%%%%%%%%%%%%%%%%%%%%%%%%%%%%%%%%%%%%%%%%%%%%%%%%%%%%%%%%%%



% Versioni %%%%%%%%%%%%%%%%%%%%%%%%%%%%%%%%%%%%%%%%%%%%%%%%%%%%%%%%%%%%%%%%%%%%%%%%%%%
\newpage
\begin{table}[h!]
\centering
\textbf{Versioni} \\ % Titolo sopra la tabella
\vspace{2mm} % Spazio tra il titolo e la tabella
\begin{tabular}{|c|c|c|c|c|}
    \hline
    \textbf{Ver.} & \textbf{Data} & \textbf{Autore} & \textbf{Verificatore} & \textbf{Descrizione} \\
    \hline
    1.0 & 01/03/2025 & Filippo Rizzolo & Eddy Pinarello & Stesura del Documento \\ 
    \hline                                  % TODO: inserire data, nomi e descrizione
\end{tabular}
\end{table}
\vspace{3cm}
\tableofcontents
%%%%%%%%%%%%%%%%%%%%%%%%%%%%%%%%%%%%%%%%%%%%%%%%%%%%%%%%%%%%%%%%%%%%%%%%%%%%%%%%%%%%%%
%%%%%%%%%%%%%%%%%%%%%%%%%%%%%%%%%%%%%%%%%%%%%%%%%%%%%%%%%%%%%%%%%%%%%%%%%%%%%%%%%%%%%%



% Registro Presenze %%%%%%%%%%%%%%%%%%%%%%%%%%%%%%%%%%%%%%%%%%%%%%%%%%%%%%%%%%%%%%%%%%
\newpage
\section{Registro Presenze}
\textbf{Piattaforma della riunione:} Piattaforma Discord \\
\textbf{Ora di Inizio} 14:30\\
\textbf{Ora di Fine} 15:30\\  % TODO: inserire orari ed eventualmente piattaforma
\\
\begin{tabular}{|c|c|c|}  % TODO: inserire ruoli e presenze
    \hline
    \textbf{Componente} & \textbf{Ruolo} & \textbf{Presenza}\\
    \hline
    Enrico Cotti Cottini & Analista & Presente \\ 
    \hline
    Gabriele Di Pietro & Progettista & Presente \\ 
    \hline
    Tommaso Diviesti & Verificatore & Presente \\ 
    \hline 
    Francesco Lapenna & Amministratore & Presente \\ 
    \hline
    Matthew Pan & Programmatore & Assente \\ 
    \hline 
    Eddy Pinarello & Responsabile & Presente \\ 
    \hline 
    Filippo Rizzolo & Verificatore & Presente \\ 
    \hline 
\end{tabular}

\section{Ordine del giorno}
\begin{itemize}
	\item retrospettiva sullo sprint appena concluso;
	\item suddivisione del lavoro e prossimi step;
	\item analisi delle criticità segnalate;
	\item decisioni prese e conclusioni.
\end{itemize}

% Presenze Rappresentanti Azienda %%%%%%%%%%%%%%%%%%%%%%%%%%%%%%%%%%%%%%%%%%%%%%%%%%%%
% non toccare, modifica invece la variabile isEsterno
\ifthenelse{\equal{\isEsterno}{1}}{
    \\
    \newline
    \newline
    \begin{tabular}{|c|c|}  % TODO: eventualmente modificare nomi rappresentanti
        \hline
        \textbf{Nome} & \textbf{Ruolo}\\
        \hline
        Gianluca Carlesso & Rappresentante Azienda \\
        \hline
        Anna Tieppo & Rappresentante Azienda \\
        \hline
    \end{tabular}
}{}
%%%%%%%%%%%%%%%%%%%%%%%%%%%%%%%%%%%%%%%%%%%%%%%%%%%%%%%%%%%%%%%%%%%%%%%%%%%%%%%%%%%%%%
\newpage


% Sezione Verbale %%%%%%%%%%%%%%%%%%%%%%%%%%%%%%%%%%%%%%%%%%%%%%%%%%%%%%%%%%%%%%%%%%%%
\newpage
\section{Verbale Retrospettiva}
Nel corso di questa riunione, ci siamo allineati sull'andamento generale del progetto, discutendo in particolare dell'esposizione RTB e della riunione avvenuta con il professor Cardin, durante la quale sono stati evidenziati alcuni errori nei requisiti. Abbiamo poi analizzato le criticità emerse e valutato come procedere per migliorare il lavoro svolto fino a questo momento. A seguito di queste riflessioni, abbiamo concordato di organizzare un incontro con l'azienda proponente per la giornata di martedì \textbf{04/03/2025}, con l'obiettivo di chiarire eventuali dubbi e ottenere un feedback diretto sui prossimi passi da seguire.
\subsection{Suddivisione lavoro e prossimi step}
Per ottimizzare la gestione delle attività, abbiamo deciso di iniziare con una fase di studio autonomo, che ci permetterà di comprendere meglio i passaggi successivi da intraprendere. Durante questa fase, ogni membro del team si dedicherà all'analisi dei documenti già esistenti, individuando le modifiche necessarie e identificando eventuali nuovi documenti da produrre. Per garantire un'organizzazione più efficace, abbiamo stabilito di suddividerci in sottogruppi, assegnando a ciascuno la responsabilità di specifiche sezioni del lavoro, così da distribuire equamente il carico e procedere in modo più strutturato.
\subsection{Analisi delle criticità segnalate dal professor Cardin}
Nel corso del colloquio con il professor Cardin, è emersa una problematica significativa legata alla poca specificità della maggior parte dei requisiti attualmente definiti nel progetto. Il professore ci ha fortemente consigliato di rivedere e correggere alcuni di essi, affinché risultino più chiari, dettagliati e conformi agli standard richiesti per le prossime revisioni. Di conseguenza, ci siamo impegnati a lavorare in modo approfondito su questi aspetti, garantendo che ogni requisito venga formulato in maniera più precisa e inequivocabile.
\subsection{Conclusioni}
Per monitorare l'avanzamento delle modifiche e verificare che il lavoro proceda nella giusta direzione, abbiamo deciso di convocare una riunione straordinaria nella stessa giornata del \textbf{04/03/2025}, subito dopo l'incontro con l'azienda. Questo ci consentirà di confrontarci sui feedback ricevuti, valutare i progressi compiuti e definire in maniera più dettagliata le azioni da intraprendere nei giorni successivi, assicurandoci di mantenere una gestione efficace del progetto.


% Sezione Firme %%%%%%%%%%%%%%%%%%%%%%%%%%%%%%%%%%%%%%%%%%%%%%%%%%%%%%%%%%%%%%%%%%%%%%
% non toccare, modifica invece la variabile isEsterno
\ifthenelse{\equal{\isEsterno}{1}}{
    \begin{table}[b]
        \begin{tabular}{@{}p{.5in}p{4in}@{}}
            Data:  & \hrulefill \\
                   &     		\\
                   &     		\\
            Firma: & \hrulefill \\
        \end{tabular}
        \end{table}
}{}
%%%%%%%%%%%%%%%%%%%%%%%%%%%%%%%%%%%%%%%%%%%%%%%%%%%%%%%%%%%%%%%%%%%%%%%%%%%%%%%%%%%%%%


\end{document} 

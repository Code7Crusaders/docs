%%% INTRO %%%%%%%%%%%%%%%%%%%%%%%%%%%%%%%%%%%%%%%%%%%%%%%%%%%%%%%%%%%%%%%%%%%%%%%%%%%%
% Template sia per verbali interni che esterni
% Segui i commenti "TODO" per ricordarti cosa modificare
% In caso di verbali esterni ricordati di settare isEsterno ad 1
%%%%%%%%%%%%%%%%%%%%%%%%%%%%%%%%%%%%%%%%%%%%%%%%%%%%%%%%%%%%%%%%%%%%%%%%%%%%%%%%%%%%%



%%% Settings %%%%%%%%%%%%%%%%%%%%%%%%%%%%%%%%%%%%%%%%%%%%%%%%%%%%%%%%%%%%%%%%%%%%%%%%%
\documentclass{article}

\usepackage{graphicx}  % serve per inserire immagini
\usepackage{fancyhdr}  % creazione header-footer
\usepackage{tabularx}  % serve per creare tabelle con colonne a larghezza variabile
\usepackage{ifthen}  % serve per mostrare cose diverse in base a condizioni
\usepackage{geometry}
\usepackage{setspace}
\usepackage{tikz}
\usepackage[italian]{babel}
\usepackage[hidelinks]{hyperref}

% TODO: setta a 1 se il verbale è esterno, 0 se è interno
\newcommand{\isEsterno}{0}

% Margini della pagina
\geometry{a4paper, margin=1in}

% Intestazione personalizzata
\pagestyle{fancy}
\fancyhf{}
\fancyhead[L]{Code7Crusaders - Software Development Team}
\fancyhead[R]{\thepage}

% Spaziatura delle righe
\setstretch{1.2}

\begin{document}
%%%%%%%%%%%%%%%%%%%%%%%%%%%%%%%%%%%%%%%%%%%%%%%%%%%%%%%%%%%%%%%%%%%%%%%%%%%%%%%%%%%%%%



%%% Sezione del titolo %%%%%%%%%%%%%%%%%%%%%%%%%%%%%%%%%%%%%%%%%%%%%%%%%%%%%%%%%%%%%%%
\begin{titlepage}

    \AddToHookNext{shipout/background}{
        \begin{tikzpicture}[remember picture,overlay]
        \node at (current page.center) {
            \includegraphics{../../img/background.png}
        };
        \end{tikzpicture}
    }

    \centering
    \vspace*{2cm}
    
    \includegraphics[width=0.3\textwidth]{../../img/logo/7Crusaders_logo.png} % logo
    \vspace{1cm}
    
    {\Huge \textbf{Code7Crusaders}}\\
    \vspace{0.5cm}
    {\Large Software Development Team}\\
    \vspace{2cm}
        
        {\large \textbf{Riunione Settimanale 07/02/2025}}\\
    \vspace{5cm}                           % esempio: Riunione Settimanale 04/11/2024
    
    
    \textbf{Membri del Team:}\\
    Enrico Cotti Cottini, Gabriele Di Pietro, Tommaso Diviesti \\
    Francesco Lapenna, Matthew Pan, Eddy Pinarello, Filippo Rizzolo \\
    \vspace{0.5cm}
    
    \vspace{1cm}
\end{titlepage}
%%%%%%%%%%%%%%%%%%%%%%%%%%%%%%%%%%%%%%%%%%%%%%%%%%%%%%%%%%%%%%%%%%%%%%%%%%%%%%%%%%%%%%



% Versioni %%%%%%%%%%%%%%%%%%%%%%%%%%%%%%%%%%%%%%%%%%%%%%%%%%%%%%%%%%%%%%%%%%%%%%%%%%%
\newpage
\begin{table}[h!]
\centering
\textbf{Versioni} \\ % Titolo sopra la tabella
\vspace{2mm} % Spazio tra il titolo e la tabella
\begin{tabular}{|c|c|c|c|c|}
    \hline
    \textbf{Ver.} & \textbf{Data} & \textbf{Autore} & \textbf{Verificatore} & \textbf{Descrizione} \\
    \hline
    1.0 & 07/02/2025 & Gabriele Di Pietro & Tommaso Diviesti & Stesura del Documento \\ 
    \hline                                  % TODO: inserire data, nomi e descrizione
\end{tabular}
\end{table}
%%%%%%%%%%%%%%%%%%%%%%%%%%%%%%%%%%%%%%%%%%%%%%%%%%%%%%%%%%%%%%%%%%%%%%%%%%%%%%%%%%%%%%
%%%%%%%%%%%%%%%%%%%%%%%%%%%%%%%%%%%%%%%%%%%%%%%%%%%%%%%%%%%%%%%%%%%%%%%%%%%%%%%%%%%%%%



% Registro Presenze %%%%%%%%%%%%%%%%%%%%%%%%%%%%%%%%%%%%%%%%%%%%%%%%%%%%%%%%%%%%%%%%%%
\newpage
\section{Registro Presenze}
\textbf{Piattaforma della riunione:} Piattaforma Discord \\
\textbf{Ora di Inizio} 14:30\\
\textbf{Ora di Fine} 15:30\\  % TODO: inserire orari ed eventualmente piattaforma
\\
\begin{tabular}{|c|c|c|}  % TODO: inserire ruoli e presenze
    \hline
    \textbf{Componente} & \textbf{Ruolo} & \textbf{Presenza}\\
    \hline
    Enrico Cotti Cottini & Verificatore & Presente \\ 
    \hline
    Gabriele Di Pietro & Programmatore & Presente \\ 
    \hline
    Tommaso Diviesti & Responsabile & Presente \\ 
    \hline 
    Francesco Lapenna & Progettista & Presente \\ 
    \hline
    Matthew Pan & Analista & Presente \\ 
    \hline 
    Eddy Pinarello & Amministratore & Presente \\ 
    \hline 
    Filippo Rizzolo & Verificatore & Presente \\ 
    \hline 
\end{tabular}
% Presenze Rappresentanti Azienda %%%%%%%%%%%%%%%%%%%%%%%%%%%%%%%%%%%%%%%%%%%%%%%%%%%%
% non toccare, modifica invece la variabile isEsterno
\ifthenelse{\equal{\isEsterno}{1}}{
    \\
    \newline
    \newline
    \begin{tabular}{|c|c|}  % TODO: eventualmente modificare nomi rappresentanti
        \hline
        \textbf{Nome} & \textbf{Ruolo}\\
        \hline
        Gianluca Carlesso & Rappresentante Azienda \\
        \hline
        Anna Tieppo & Rappresentante Azienda \\
        \hline
    \end{tabular}
}{}
%%%%%%%%%%%%%%%%%%%%%%%%%%%%%%%%%%%%%%%%%%%%%%%%%%%%%%%%%%%%%%%%%%%%%%%%%%%%%%%%%%%%%%
\vspace{3cm}
\tableofcontents


% Sezione Verbale %%%%%%%%%%%%%%%%%%%%%%%%%%%%%%%%%%%%%%%%%%%%%%%%%%%%%%%%%%%%%%%%%%%%
\newpage
\section{Verbale Retrospettiva}
La riunione è inizia in ritardo in attesa che tutti i membri si collegassero. In data odierna c'è stata la prima revisione da parte del professor Cardin dove abbiamo ottenuto un semaforo rosso, di conseguenza è necessario rivedere il progetto e apportare le modifiche richieste necessarie per prosiguire.
\subsection{Lavoro Arretrato}
Visto che ora il Team dovrebbe essere più libero dal periodo di esami, e viste anche le numerose task che avevamo durante la precedente riunione, abbiamo discusso di cosa è stato fatto e di cosa è stato risolto. Tuttavia, è stato fatto notare come nella project board ci sia stata per parecchio tempo una issue assegnata a 4 persone e di come nessuno ci abbia lavorato attivamente.
\subsection{Analisi delle criticità segnalate dal professor Cardin}
Il professor Cardin ha segnalato alcune criticità presenti nel progetto, in particolare:
\begin{itemize}
    \item La mancata giustificazione delle tecnologie di backend.
    \item La mancata giustificazione delle tecnologie del database \textit{(in particolar modo quello vettoriale)}.
    \item La poca argomentazione delle tecnologie del frontend.
    \item La mancata discussione sul linguaggio di programmazione da utilizzare.
\end{itemize}
Per risolvere ciò, e chiedere il più velocemente possibile la prossima revisione senza perdere troppo tempo, abbiamo deciso di dividerci in gruppi da 1 o 2 persone e far si che ognuno decida e discuta di ogni tecnologia usata; in modo tale da avere un Team consapevole di ogni scelta fatta ed essere in grado di giustificarla.
\subsection{Assegnazione dei compiti ai vari membri del team}
Dopo esserci divisi in sottogruppi abbiamo preso i compiti non ancora svolti e abbiamo impostato una scadenza abbastanza vicina in modo tale che ogni membro del gruppo sia intenzionato a lavorare in modo da recuperare il tempo perso. Per le task più grandi invece abbiamo deciso di mantenere come scadenza il venerdì sucessivo.

\subsection{Conclusioni}
Si è deciso di organizzare una riunione straordinaria fissata per il giorno \textbf{11/02/2025} alle 9:00 per discutere sull'andamento del lavoro e vedere se alcune delle task più vecchie saranno chiuse. Mentre si proseguirà con la consueta riunione settimanale venerdì \textbf{14/02/2025} alle 14:30.
L'obiettivo di avere 2 riunione e quindi sprint più brevi è quello di avere un controllo maggiore sul lavoro svolto e sulle scadenze da rispettare, ma anche di velocizzare il lavoro visto il ritardo che ci stiamo portando dietro.


% Sezione Firme %%%%%%%%%%%%%%%%%%%%%%%%%%%%%%%%%%%%%%%%%%%%%%%%%%%%%%%%%%%%%%%%%%%%%%
% non toccare, modifica invece la variabile isEsterno
\ifthenelse{\equal{\isEsterno}{1}}{
    \begin{table}[b]
        \begin{tabular}{@{}p{.5in}p{4in}@{}}
            Data:  & \hrulefill \\
                   &     		\\
                   &     		\\
            Firma: & \hrulefill \\
        \end{tabular}
        \end{table}
}{}
%%%%%%%%%%%%%%%%%%%%%%%%%%%%%%%%%%%%%%%%%%%%%%%%%%%%%%%%%%%%%%%%%%%%%%%%%%%%%%%%%%%%%%


\end{document} 

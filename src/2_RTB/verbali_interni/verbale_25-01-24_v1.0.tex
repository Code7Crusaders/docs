%%% INTRO %%%%%%%%%%%%%%%%%%%%%%%%%%%%%%%%%%%%%%%%%%%%%%%%%%%%%%%%%%%%%%%%%%%%%%%%%%%%
% Template sia per verbali interni che esterni
% Segui i commenti "TODO" per ricordarti cosa modificare
% In caso di verbali esterni ricordati di settare isEsterno ad 1
%%%%%%%%%%%%%%%%%%%%%%%%%%%%%%%%%%%%%%%%%%%%%%%%%%%%%%%%%%%%%%%%%%%%%%%%%%%%%%%%%%%%%



%%% Settings %%%%%%%%%%%%%%%%%%%%%%%%%%%%%%%%%%%%%%%%%%%%%%%%%%%%%%%%%%%%%%%%%%%%%%%%%
\documentclass{article}

\usepackage{graphicx}  % serve per inserire immagini
\usepackage{fancyhdr}  % creazione header-footer
\usepackage{tabularx}  % serve per creare tabelle con colonne a larghezza variabile
\usepackage{ifthen}  % serve per mostrare cose diverse in base a condizioni
\usepackage{geometry}
\usepackage{setspace}
\usepackage{tikz}
\usepackage[italian]{babel}
\usepackage[hidelinks]{hyperref}

% TODO: setta a 1 se il verbale è esterno, 0 se è interno
\newcommand{\isEsterno}{0}

% Margini della pagina
\geometry{a4paper, margin=1in}

% Intestazione personalizzata
\pagestyle{fancy}
\fancyhf{}
\fancyhead[L]{Code7Crusaders - Software Development Team}
\fancyhead[R]{\thepage}

% Spaziatura delle righe
\setstretch{1.2}

\begin{document}
%%%%%%%%%%%%%%%%%%%%%%%%%%%%%%%%%%%%%%%%%%%%%%%%%%%%%%%%%%%%%%%%%%%%%%%%%%%%%%%%%%%%%%



%%% Sezione del titolo %%%%%%%%%%%%%%%%%%%%%%%%%%%%%%%%%%%%%%%%%%%%%%%%%%%%%%%%%%%%%%%
\begin{titlepage}

    \AddToHookNext{shipout/background}{
        \begin{tikzpicture}[remember picture,overlay]
        \node at (current page.center) {
            \includegraphics{../../img/background.png}
        };
        \end{tikzpicture}
    }

    \centering
    \vspace*{2cm}
    
    \includegraphics[width=0.3\textwidth]{../../img/logo/7Crusaders_logo.png} % logo
    \vspace{1cm}
    
    {\Huge \textbf{Code7Crusaders}}\\
    \vspace{0.5cm}
    {\Large Software Development Team}\\
    \vspace{2cm}
        
        {\large \textbf{Riunione Settimanale 24/01/2025}}\\
    \vspace{5cm}                           % esempio: Riunione Settimanale 04/11/2024
    
    
    \textbf{Membri del Team:}\\
    Enrico Cotti Cottini, Gabriele Di Pietro, Tommaso Diviesti \\
    Francesco Lapenna, Matthew Pan, Eddy Pinarello, Filippo Rizzolo \\
    \vspace{0.5cm}
    
    \vspace{1cm}
\end{titlepage}
%%%%%%%%%%%%%%%%%%%%%%%%%%%%%%%%%%%%%%%%%%%%%%%%%%%%%%%%%%%%%%%%%%%%%%%%%%%%%%%%%%%%%%



% Versioni %%%%%%%%%%%%%%%%%%%%%%%%%%%%%%%%%%%%%%%%%%%%%%%%%%%%%%%%%%%%%%%%%%%%%%%%%%%
\newpage
\begin{table}[h!]
\centering
\textbf{Versioni} \\ % Titolo sopra la tabella
\vspace{2mm} % Spazio tra il titolo e la tabella
\begin{tabular}{|c|c|c|c|c|}
    \hline
    \textbf{Ver.} & \textbf{Data} & \textbf{Autore} & \textbf{Verificatore} & \textbf{Descrizione} \\
    \hline
    1.0 & 26/01/2025 & Gabriele Di Pietro &  & Stesura del Documento \\ 
    \hline                                  % TODO: inserire data, nomi e descrizione
\end{tabular}
\end{table}
%%%%%%%%%%%%%%%%%%%%%%%%%%%%%%%%%%%%%%%%%%%%%%%%%%%%%%%%%%%%%%%%%%%%%%%%%%%%%%%%%%%%%%
%%%%%%%%%%%%%%%%%%%%%%%%%%%%%%%%%%%%%%%%%%%%%%%%%%%%%%%%%%%%%%%%%%%%%%%%%%%%%%%%%%%%%%



% Registro Presenze %%%%%%%%%%%%%%%%%%%%%%%%%%%%%%%%%%%%%%%%%%%%%%%%%%%%%%%%%%%%%%%%%%
\newpage
\section{Registro Presenze}
\textbf{Piattaforma della riunione:} Piattaforma Discord \\
\textbf{Ora di Inizio} 15:40\\
\textbf{Ora di Fine} 15:45\\  % TODO: inserire orari ed eventualmente piattaforma
\\
\begin{tabular}{|c|c|c|}  % TODO: inserire ruoli e presenze
    \hline
    \textbf{Componente} & \textbf{Ruolo} & \textbf{Presenza}\\
    \hline
    Enrico Cotti Cottini & Amministratore & Presente \\ 
    \hline
    Gabriele Di Pietro & Verificatore & Presente \\ 
    \hline
    Tommaso Diviesti & Programmatore & Presente \\ 
    \hline 
    Francesco Lapenna & Verificatore & Presente \\ 
    \hline
    Matthew Pan & Responsabile & Presente \\ 
    \hline 
    Eddy Pinarello & Analista & Presente \\ 
    \hline 
    Filippo Rizzolo & Progettista & Presente \\ 
    \hline 
\end{tabular}
% Presenze Rappresentanti Azienda %%%%%%%%%%%%%%%%%%%%%%%%%%%%%%%%%%%%%%%%%%%%%%%%%%%%
% non toccare, modifica invece la variabile isEsterno
\ifthenelse{\equal{\isEsterno}{1}}{
    \\
    \newline
    \newline
    \begin{tabular}{|c|c|}  % TODO: eventualmente modificare nomi rappresentanti
        \hline
        \textbf{Nome} & \textbf{Ruolo}\\
        \hline
        Gianluca Carlesso & Rappresentante Azienda \\
        \hline
        Anna Tieppo & Rappresentante Azienda \\
        \hline
    \end{tabular}
}{}
%%%%%%%%%%%%%%%%%%%%%%%%%%%%%%%%%%%%%%%%%%%%%%%%%%%%%%%%%%%%%%%%%%%%%%%%%%%%%%%%%%%%%%



% Sezione Verbale %%%%%%%%%%%%%%%%%%%%%%%%%%%%%%%%%%%%%%%%%%%%%%%%%%%%%%%%%%%%%%%%%%%%
\newpage
\section{Verbale Retrospettiva}
L'incontro consueto del venerdì si svolge dopo un incontro mattutino avvenuto con il prof.Cardin
dove abbiamo risolto alcuni dei dubbi che avevamo sui casi d'uso. Subito dopo il colloquio ci siamo messi al lavoro
per correggere gli errori individuati su di essi (\textit{Casi d'uso}) e ci siamo messi a lavorare ad uno schema entità-relazione da presentare alla prossima riunione aziendale
prevista per il 27/01/25.
\subsection{Frontend}
L'analisi del frontend è completa e ora bisognerà mostrarla all'azienda durante il prossimo incontro. Nel frattempo abbiamo sviluppato un PoC da presentare anche esso all'azienda e al futuro colloquio con il professor Cardin.
\subsection{Lavoro arretrato}
Si fa notare ad alcuni componenti del gruppo che il repository non viene toccato da parecchio tempo e che nella project board ci sono molte issue ancora nel to-do. Quindi si decide che per le prossime volte all'assegnazione delle issue sarà legata anche una scadenza
in modo tale da incentivare il lavoro. Siamo ben consapevoli del problema di avere esami durante lo svolgimento del progetto quindi prevediamo un impennata di commit nel repository dopo la fine di alcuni esami.

\subsection{Pianificazione per la Prossima Settimana}
\begin{itemize}
    \item Completare le attività di routine e verifica dei documenti.
    \item Incontro con l'azienda per mostrare l'analisi del frontend, lo schema del database relazionale e mostrare il PoC.
    \item Aggiornare le definizioni del glossario e eseguire lo script python su alcuni documenti in modo da collegare termini ambigui al glossario.
    \item Completare le attività in vista della revisione RTB e la consegna del POC.
    \item Richiedere un colloquio con il professor Cardin per la revisione della RTB.
\end{itemize}

\subsection{Conclusioni}
Concludiamo la riunione assegnandoci nuove issue dando una scadenza più larga alla conclusione della sprint vista anche la presenza di numerosi esami durante la settimana. 
La prossima riunione è programmata per il \textbf{4/02/2025} subito dopo l'esame di ingegneria del Software con una futura riunione il \textbf{07/02/2025}, con l'obiettivo di concludere questa fase di progetto e organizzare il lavoro al massimo delle nostre capacità.

%%%%%%%%%%%%%%%%%%%%%%%%%%%%%%%%%%%%%%%%%%%%%%%%%%%%%%%%%%%%%%%%%%%%%%%%%%%%%%



% Sezione Firme %%%%%%%%%%%%%%%%%%%%%%%%%%%%%%%%%%%%%%%%%%%%%%%%%%%%%%%%%%%%%%%%%%%%%%
% non toccare, modifica invece la variabile isEsterno
\ifthenelse{\equal{\isEsterno}{1}}{
    \begin{table}[b]
        \begin{tabular}{@{}p{.5in}p{4in}@{}}
            Data:  & \hrulefill \\
                   &     		\\
                   &     		\\
            Firma: & \hrulefill \\
        \end{tabular}
        \end{table}
}{}
%%%%%%%%%%%%%%%%%%%%%%%%%%%%%%%%%%%%%%%%%%%%%%%%%%%%%%%%%%%%%%%%%%%%%%%%%%%%%%%%%%%%%%


\end{document} 

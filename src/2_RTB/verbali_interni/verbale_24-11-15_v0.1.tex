\documentclass{article}
\usepackage{graphicx}
\usepackage{fancyhdr}
\usepackage{geometry}
\usepackage{setspace}
\usepackage{tikz}
\usepackage[italian]{babel}
\usepackage{tabularx}
\usepackage[hidelinks]{hyperref}

% Margini della pagina
\geometry{a4paper, margin=1in}

% Intestazione personalizzata
\pagestyle{fancy}
\fancyhf{}
\fancyhead[L]{Code7Crusaders - Software Development Team}
\fancyhead[R]{\thepage}

% Spaziatura delle righe
\setstretch{1.2}

\begin{document}

% Sezione del titolo
\begin{titlepage}

    \AddToHookNext{shipout/background}{
    \begin{tikzpicture}[remember picture,overlay]
    \node at (current page.center) {
    \includegraphics{../../img/background.png}
    };
    \end{tikzpicture}
    }

    \centering
    \vspace*{2cm}
    
    \includegraphics[width=0.3\textwidth]{../../img/logo/7Crusaders_logo.png} % Aggiungi il logo qui
    \vspace{1cm}
    
    {\Huge \textbf{Code7Crusaders}}\\
    \vspace{0.5cm}
    {\Large Software Development Team}\\
    \vspace{2cm}
    
    {\large \textbf{Riunione Settimanale 15/11/2024}}\\
    \vspace{5cm}

    \textbf{Membri del Team:}\\
    Enrico Cotti Cottini, Gabriele Di Pietro, Tommaso Diviesti \\
    Francesco Lapenna, Matthew Pan, Eddy Pinarello, Filippo Rizzolo \\
    \vspace{0.5cm}
    
    \vspace{1cm}
\end{titlepage}

%Versioni
\newpage
\begin{table}[h!]
\centering
\textbf{Versioni} \\ % Titolo sopra la tabella
\vspace{2mm} % Spazio tra il titolo e la tabella
\begin{tabular}{|c|c|c|c|c|}
    \hline
    \textbf{Ver.} & \textbf{Data} & \textbf{Autore} & \textbf{Verificatore} & \textbf{Descrizione} \\
    \hline
    0.1 & 20/11/2024 & Filippo Rizzolo & Matthew Pan & Prima stesura del documento \\ 
    \hline
\end{tabular}
\end{table}

% Indice
\newpage
\tableofcontents

% Registro Presenze
\newpage
\section{Registro Presenze}
\textbf{Piattaforma della riunione:} Piattaforma Discord \\
\textbf{Ora di Inizio:} 14:00\\
\textbf{Ora di Fine:} 15:00\\
\\
\begin{tabular}{|c|c|c|}
    \hline
    \textbf{Componente} & \textbf{Ruolo} & \textbf{Presenza}\\
    \hline
    Enrico Cotti Cottini & Verificatore & Assente \\ 
    \hline
    Gabriele Di Pietro & Responsabile & Presente\\ 
    \hline
    Tommaso Diviesti & Redattore & Presente \\ 
    \hline 
    Francesco Lapenna & Redattore & Presente \\ 
    \hline
    Matthew Pan & Verificatore & Presente\\ 
    \hline 
    Eddy Pinarello & Redattore & Presente \\ 
    \hline 
    Filippo Rizzolo & Amministratore & Presente \\ 
    \hline 
\end{tabular}
\vspace{3cm}
\section{Ordine del giorno}
\begin{itemize} 
    \item Discussione sull'incontro con Ergon avvenuto il 14/11/2024 
    \item Stato di avanzamento del file Norme di Progetto 
    \item Revisione del Glossario 
    \item Rendicontazione delle ore e ripartizione dei ruoli 
    \item Prossimi step 
\end{itemize}

% Sezione Verbale
\newpage
\section{Verbale}
\subsection{Discussione sull'incontro con Ergon del 14/11/2024} 
Durante l'incontro con l'azienda, abbiamo concordato sulla necessità di sviluppare due interfacce per la piattaforma: una per l'\textit{admin} e una per l'\textit{utente generico}.
\begin{itemize}
    \item La sezione \textit{admin} richiederà la possibilità di aggiornare i dati caricando file PDF.
    \item La sezione \textit{utente} permetterà di visualizzare le informazioni attraverso un chatbot.
\end{itemize}
Sono state inoltre esaminate le tecnologie disponibili, affrontando alcuni dubbi tecnici emersi all'interno del team.

\subsection{Stato di avanzamento del file Norme di Progetto} 
Abbiamo chiarito alcuni dubbi riguardanti il documento Norme di Progetto, riconoscendone il ruolo centrale per il \textit{way of working} del team.

Si è deciso che il documento sarà completato nel lungo termine, mentre altre sezioni della documentazione, come l'analisi dei requisiti, i diagrammi dei casi d'uso, i design pattern e la selezione dei framework dovranno essere finalizzate il prima possibile.

\subsection{Revisione del Glossario} 
Per il Glossario, si è deciso di evidenziare nei documenti tecnici i termini chiave, rendendoli riconoscibili e uniformi in tutta la documentazione.

\subsection{Rendicontazione delle ore e ripartizione dei ruoli} 
Per migliorare l'organizzazione, abbiamo deciso di:
\begin{itemize}
    \item Fissare un incontro breve ogni lunedì per l'assegnazione dei ruoli e la gestione delle attività.
    \item Assegnare i ruoli considerando le disponibilità individuali, evitando di sovraccaricare i membri del team.
\end{itemize}

\subsection{Prossimi step} 
Le prossime revisioni previste sono:
\begin{itemize}
    \item RTB (\textit{Requirements and Technology Baseline})
    \item PB (\textit{Product Baseline})
\end{itemize}
Per preparare l'RTB sarà necessario completare i seguenti documenti:
\begin{itemize} 
    \item Analisi dei requisiti 
    \item Piano di progetto 
    \item Piano di qualifica 
    \item Norme di progetto 
    \item Glossario 
    \item Lettera di presentazione 
    \item PoC (\textbf{Proof of Concept}) 
\end{itemize}
Sarà utile pianificare ulteriori incontri con l'azienda per discutere i passaggi successivi e raccogliere feedback.

\section{Conclusioni}
Si conferma l'appuntamento fisso del venerdì per discutere le difficoltà emerse durante la settimana.
L'incontro del lunedì rimane dedicato all'assegnazione dei ruoli e dei compiti settimanali.

\end{document}

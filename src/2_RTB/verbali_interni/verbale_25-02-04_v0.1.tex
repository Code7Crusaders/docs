%%% INTRO %%%%%%%%%%%%%%%%%%%%%%%%%%%%%%%%%%%%%%%%%%%%%%%%%%%%%%%%%%%%%%%%%%%%%%%%%%%%
% Template sia per verbali interni che esterni
% Segui i commenti "TODO" per ricordarti cosa modificare
% In caso di verbali esterni ricordati di settare isEsterno ad 1
%%%%%%%%%%%%%%%%%%%%%%%%%%%%%%%%%%%%%%%%%%%%%%%%%%%%%%%%%%%%%%%%%%%%%%%%%%%%%%%%%%%%%



%%% Settings %%%%%%%%%%%%%%%%%%%%%%%%%%%%%%%%%%%%%%%%%%%%%%%%%%%%%%%%%%%%%%%%%%%%%%%%%
\documentclass{article}

\usepackage{graphicx}  % serve per inserire immagini
\usepackage{fancyhdr}  % creazione header-footer
\usepackage{tabularx}  % serve per creare tabelle con colonne a larghezza variabile
\usepackage{ifthen}  % serve per mostrare cose diverse in base a condizioni
\usepackage{geometry}
\usepackage{setspace}
\usepackage{tikz}
\usepackage[italian]{babel}
\usepackage[hidelinks]{hyperref}

% TODO: setta a 1 se il verbale è esterno, 0 se è interno
\newcommand{\isEsterno}{0}

% Margini della pagina
\geometry{a4paper, margin=1in}

% Intestazione personalizzata
\pagestyle{fancy}
\fancyhf{}
\fancyhead[L]{Code7Crusaders - Software Development Team}
\fancyhead[R]{\thepage}

% Spaziatura delle righe
\setstretch{1.2}

\begin{document}
%%%%%%%%%%%%%%%%%%%%%%%%%%%%%%%%%%%%%%%%%%%%%%%%%%%%%%%%%%%%%%%%%%%%%%%%%%%%%%%%%%%%%%



%%% Sezione del titolo %%%%%%%%%%%%%%%%%%%%%%%%%%%%%%%%%%%%%%%%%%%%%%%%%%%%%%%%%%%%%%%
\begin{titlepage}

    \AddToHookNext{shipout/background}{
        \begin{tikzpicture}[remember picture,overlay]
        \node at (current page.center) {
            \includegraphics{../../img/background.png}
        };
        \end{tikzpicture}
    }

    \centering
    \vspace*{2cm}
    
    \includegraphics[width=0.3\textwidth]{../../img/logo/7Crusaders_logo.png} % logo
    \vspace{1cm}
    
    {\Huge \textbf{Code7Crusaders}}\\
    \vspace{0.5cm}
    {\Large Software Development Team}\\
    \vspace{2cm}
        
        {\large \textbf{Riunione Settimanale 24/01/2025}}\\
    \vspace{5cm}                           % esempio: Riunione Settimanale 04/11/2024
    
    
    \textbf{Membri del Team:}\\
    Enrico Cotti Cottini, Gabriele Di Pietro, Tommaso Diviesti \\
    Francesco Lapenna, Matthew Pan, Eddy Pinarello, Filippo Rizzolo \\
    \vspace{0.5cm}
    
    \vspace{1cm}
\end{titlepage}
%%%%%%%%%%%%%%%%%%%%%%%%%%%%%%%%%%%%%%%%%%%%%%%%%%%%%%%%%%%%%%%%%%%%%%%%%%%%%%%%%%%%%%



% Versioni %%%%%%%%%%%%%%%%%%%%%%%%%%%%%%%%%%%%%%%%%%%%%%%%%%%%%%%%%%%%%%%%%%%%%%%%%%%
\newpage
\begin{table}[h!]
\centering
\textbf{Versioni} \\ % Titolo sopra la tabella
\vspace{2mm} % Spazio tra il titolo e la tabella
\begin{tabular}{|c|c|c|c|c|}
    \hline
    \textbf{Ver.} & \textbf{Data} & \textbf{Autore} & \textbf{Verificatore} & \textbf{Descrizione} \\
    \hline
    0.1 & 04/02/2025 & Filippo Rizzolo &  & Stesura del Documento \\ 
    \hline                                  % TODO: inserire data, nomi e descrizione
\end{tabular}
\end{table}
%%%%%%%%%%%%%%%%%%%%%%%%%%%%%%%%%%%%%%%%%%%%%%%%%%%%%%%%%%%%%%%%%%%%%%%%%%%%%%%%%%%%%%
%%%%%%%%%%%%%%%%%%%%%%%%%%%%%%%%%%%%%%%%%%%%%%%%%%%%%%%%%%%%%%%%%%%%%%%%%%%%%%%%%%%%%%



% Registro Presenze %%%%%%%%%%%%%%%%%%%%%%%%%%%%%%%%%%%%%%%%%%%%%%%%%%%%%%%%%%%%%%%%%%
\newpage
\section{Registro Presenze}
\textbf{Piattaforma della riunione:} Piattaforma Discord \\
\textbf{Ora di Inizio} 15:30\\
\textbf{Ora di Fine} 16:45\\  % TODO: inserire orari ed eventualmente piattaforma
\\
\begin{tabular}{|c|c|c|}  % TODO: inserire ruoli e presenze
    \hline
    \textbf{Componente} & \textbf{Ruolo} & \textbf{Presenza}\\
    \hline
    Enrico Cotti Cottini & Verificatore & Presente \\ 
    \hline
    Gabriele Di Pietro & Programmatore & Presente \\ 
    \hline
    Tommaso Diviesti & Responsabile & Assente \\ 
    \hline 
    Francesco Lapenna & Progettista & Presente \\ 
    \hline
    Matthew Pan & Analista & Presente \\ 
    \hline 
    Eddy Pinarello & Amministratore & Presente \\ 
    \hline 
    Filippo Rizzolo & Verificatore & Presente \\ 
    \hline 
\end{tabular}
% Presenze Rappresentanti Azienda %%%%%%%%%%%%%%%%%%%%%%%%%%%%%%%%%%%%%%%%%%%%%%%%%%%%
% non toccare, modifica invece la variabile isEsterno
\ifthenelse{\equal{\isEsterno}{1}}{
    \\
    \newline
    \newline
    \begin{tabular}{|c|c|}  % TODO: eventualmente modificare nomi rappresentanti
        \hline
        \textbf{Nome} & \textbf{Ruolo}\\
        \hline
        Gianluca Carlesso & Rappresentante Azienda \\
        \hline
        Anna Tieppo & Rappresentante Azienda \\
        \hline
    \end{tabular}
}{}
%%%%%%%%%%%%%%%%%%%%%%%%%%%%%%%%%%%%%%%%%%%%%%%%%%%%%%%%%%%%%%%%%%%%%%%%%%%%%%%%%%%%%%



% Sezione Verbale %%%%%%%%%%%%%%%%%%%%%%%%%%%%%%%%%%%%%%%%%%%%%%%%%%%%%%%%%%%%%%%%%%%%
\newpage
\section{Verbale Retrospettiva}
Questo incontro è stato aggiunto rispetto alle settimane precedenti per effettuare un controllo approfondito sull'avanzamento delle varie task di ciascun membro del team. In seguito, abbiamo discusso alcuni casi d'uso che, durante il colloquio con il professor Cardin, erano risultati fuorvianti o errati. Al termine della riunione, abbiamo deciso di inviare un'email al professore per richiedere una data disponibile per l'esposizione dell'RTB.
\subsection{Lavoro arretrato}
Dopo la giornata odierna, in cui si è svolto l'esame scritto di Ingegneria del Software, abbiamo ritenuto necessario rivedere la gestione delle issue all'interno della project board. In particolare, abbiamo deciso di stabilire scadenze più ravvicinate sia per le nuove issue appena create sia per quelle già presenti da tempo, al fine di ottimizzare il flusso di lavoro e ridurre l'accumulo di attività in sospeso. Questo approccio ci permetterà di mantenere un migliore controllo sull'avanzamento del progetto e di garantire una maggiore efficienza nella risoluzione delle problematiche.

\subsection{Pianificazione per il prossimo incontro di venerdì 7/02/2025}
\begin{itemize}
    \item Completare le attività di routine e verificare la correttezza e la coerenza dei documenti.
    \item Portare a termine le issue ancora aperte in preparazione alla revisione dell'RTB.
    \item Svolgere l'incontro dedicato alla revisione dell'RTB, analizzando eventuali criticità emerse.
    \item Aggiornare e affinare le definizioni presenti nel glossario, eseguendo lo script Python su alcuni documenti per identificare e collegare i termini ambigui alle relative definizioni.
\end{itemize}

\subsection{Conclusioni}
Concludiamo la riunione assegnando nuove issue e definendo scadenze più ravvicinate per la chiusura della sprint, tenendo conto del previsto termine della sessione d'esami per alcuni membri del team.
La prossima riunione è programmata per il \textbf{07/02/2025}, subito dopo l'incontro di revisione dell'RTB. L'obiettivo sarà organizzare al meglio il lavoro in vista della fase successiva del progetto, assicurando una distribuzione efficace delle attività e una gestione ottimale delle priorità.

%%%%%%%%%%%%%%%%%%%%%%%%%%%%%%%%%%%%%%%%%%%%%%%%%%%%%%%%%%%%%%%%%%%%%%%%%%%%%%



% Sezione Firme %%%%%%%%%%%%%%%%%%%%%%%%%%%%%%%%%%%%%%%%%%%%%%%%%%%%%%%%%%%%%%%%%%%%%%
% non toccare, modifica invece la variabile isEsterno
\ifthenelse{\equal{\isEsterno}{1}}{
    \begin{table}[b]
        \begin{tabular}{@{}p{.5in}p{4in}@{}}
            Data:  & \hrulefill \\
                   &     		\\
                   &     		\\
            Firma: & \hrulefill \\
        \end{tabular}
        \end{table}
}{}
%%%%%%%%%%%%%%%%%%%%%%%%%%%%%%%%%%%%%%%%%%%%%%%%%%%%%%%%%%%%%%%%%%%%%%%%%%%%%%%%%%%%%%


\end{document} 

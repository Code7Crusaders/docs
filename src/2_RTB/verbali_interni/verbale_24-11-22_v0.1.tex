 %%% INTRO %%%%%%%%%%%%%%%%%%%%%%%%%%%%%%%%%%%%%%%%%%%%%%%%%%%%%%%%%%%%%%%%%%%%%%%%%%%%
% Template sia per verbali interni che esterni
% Segui i commenti "TODO" per ricordarti cosa modificare
% In caso di verbali esterni ricordati di settare isEsterno ad 1
%%%%%%%%%%%%%%%%%%%%%%%%%%%%%%%%%%%%%%%%%%%%%%%%%%%%%%%%%%%%%%%%%%%%%%%%%%%%%%%%%%%%%



%%% Settings %%%%%%%%%%%%%%%%%%%%%%%%%%%%%%%%%%%%%%%%%%%%%%%%%%%%%%%%%%%%%%%%%%%%%%%%%
\documentclass{article}

\usepackage{graphicx}  % serve per inserire immagini
\usepackage{fancyhdr}  % creazione header-footer
\usepackage{tabularx}  % serve per creare tabelle con colonne a larghezza variabile
\usepackage{ifthen}  % serve per mostrare cose diverse in base a condizioni
\usepackage{geometry}
\usepackage{setspace}
\usepackage{tikz}
\usepackage[italian]{babel}
\usepackage[hidelinks]{hyperref}

% TODO: setta a 1 se il verbale è esterno, 0 se è interno
\newcommand{\isEsterno}{0}

% Margini della pagina
\geometry{a4paper, margin=1in}

% Intestazione personalizzata
\pagestyle{fancy}
\fancyhf{}
\fancyhead[L]{Code7Crusaders - Software Development Team}
\fancyhead[R]{\thepage}

% Spaziatura delle righe
\setstretch{1.2}

\begin{document}
%%%%%%%%%%%%%%%%%%%%%%%%%%%%%%%%%%%%%%%%%%%%%%%%%%%%%%%%%%%%%%%%%%%%%%%%%%%%%%%%%%%%%%



%%% Sezione del titolo %%%%%%%%%%%%%%%%%%%%%%%%%%%%%%%%%%%%%%%%%%%%%%%%%%%%%%%%%%%%%%%
\begin{titlepage}

    \AddToHookNext{shipout/background}{
        \begin{tikzpicture}[remember picture,overlay]
        \node at (current page.center) {
            \includegraphics{../../img/background.png}
        };
        \end{tikzpicture}
    }

    \centering
    \vspace*{2cm}
    
    \includegraphics[width=0.3\textwidth]{../../img/logo/7Crusaders_logo.png} % logo
    \vspace{1cm}
    
    {\Huge \textbf{Code7Crusaders}}\\
    \vspace{0.5cm}
    {\Large Software Development Team}\\
    \vspace{2cm}
    
    {\large \textbf{Riunione Settimanale 22/11/2024}}\\ 
    \vspace{5cm}                           
    
    
    \textbf{Membri del Team:}\\
    Enrico Cotti Cottini, Gabriele Di Pietro, Tommaso Diviesti \\
    Francesco Lapenna, Matthew Pan, Eddy Pinarello, Filippo Rizzolo \\
    \vspace{0.5cm}
    
    \vspace{1cm}
\end{titlepage}
%%%%%%%%%%%%%%%%%%%%%%%%%%%%%%%%%%%%%%%%%%%%%%%%%%%%%%%%%%%%%%%%%%%%%%%%%%%%%%%%%%%%%%



% Versioni %%%%%%%%%%%%%%%%%%%%%%%%%%%%%%%%%%%%%%%%%%%%%%%%%%%%%%%%%%%%%%%%%%%%%%%%%%%
\newpage
\begin{table}[h!]
\centering
\textbf{Versioni} \\ % Titolo sopra la tabella
\vspace{2mm} % Spazio tra il titolo e la tabella
\begin{tabular}{|c|c|c|c|>{\raggedright\arraybackslash}p{0.3\textwidth}|}
    \hline
    \textbf{Ver.} & \textbf{Data} & \textbf{Autore} & \textbf{Verificatore} & \textbf{Descrizione} \\
    \hline
    1.0 & 22/11/2024 & Filippo Rizzolo & Gabriele Di Pietro & Prima stesura del verbale interno \\ 
    \hline                                  
\end{tabular}
\end{table}
%%%%%%%%%%%%%%%%%%%%%%%%%%%%%%%%%%%%%%%%%%%%%%%%%%%%%%%%%%%%%%%%%%%%%%%%%%%%%%%%%%%%%%



% Indice %%%%%%%%%%%%%%%%%%%%%%%%%%%%%%%%%%%%%%%%%%%%%%%%%%%%%%%%%%%%%%%%%%%%%%%%%%%%%
\newpage
\tableofcontents
%%%%%%%%%%%%%%%%%%%%%%%%%%%%%%%%%%%%%%%%%%%%%%%%%%%%%%%%%%%%%%%%%%%%%%%%%%%%%%%%%%%%%%



% Registro Presenze %%%%%%%%%%%%%%%%%%%%%%%%%%%%%%%%%%%%%%%%%%%%%%%%%%%%%%%%%%%%%%%%%%
\newpage
\section{Registro Presenze}
\textbf{Piattaforma della riunione:} Piattaforma Discord \\
\textbf {Ora di Inizio:} 14:00\\
\textbf {Ora di Fine:} 15:00\\  % TODO: inserire orari ed eventualmente piattaforma   						
\\
\begin{tabular}{|c|c|c|}  % TODO: inserire ruoli e presenze
    \hline
    \textbf{Componente} & \textbf{Ruolo} & \textbf{Presenza}\\
    \hline
    Enrico Cotti Cottini & XXXXX & Assente \\ 
    \hline
    Gabriele Di Pietro & XXXXX & Presente\\ 
    \hline
    Tommaso Diviesti & XXXXX & Presente \\ 
    \hline 
    Francesco Lapenna & XXXXX & Presente \\ 
    \hline
    Matthew Pan & XXXXX & Presente\\ 
    \hline 
    Eddy Pinarello & XXXXX & Presente \\ 
    \hline 
    Filippo Rizzolo & XXXXX & Presente \\ 
    \hline 
\end{tabular}
% Presenze Rappresentanti Azienda %%%%%%%%%%%%%%%%%%%%%%%%%%%%%%%%%%%%%%%%%%%%%%%%%%%%
% non toccare, modifica invece la variabile isEsterno
\ifthenelse{\equal{\isEsterno}{1}}{
    \\
    \newline
    \newline
    \begin{tabular}{|c|c|}  % TODO: eventualmente modificare nomi rappresentanti
        \hline
        \textbf{Nome} & \textbf{Ruolo}\\
        \hline
        Gianluca Carlesso & Rappresentante Azienda \\
        \hline
        Anna Tieppo & Rappresentante Azienda \\
        \hline
    \end{tabular}
}{}
%%%%%%%%%%%%%%%%%%%%%%%%%%%%%%%%%%%%%%%%%%%%%%%%%%%%%%%%%%%%%%%%%%%%%%%%%%%%%%%%%%%%%%




% Sezione Verbale %%%%%%%%%%%%%%%%%%%%%%%%%%%%%%%%%%%%%%%%%%%%%%%%%%%%%%%%%%%%%%%%%%%%
\newpage
\section{Verbale Retrospettiva}
    % TODO: per ogni punto discusso / attività svolta
    % inserire una sottosezione, sintesi ed eventuali decisioni

    \subsection{Aggiornamento del sito dei documenti}
    \textbf{Sintesi:} Il sito per la gestione dei documenti è stato aggiornato con una struttura più organizzata, che facilita la consultazione e la collaborazione tra i membri del team. \\
    \textbf{Decisioni:} L'aggiornamento sarà monitorato per eventuali miglioramenti, ma il lavoro può considerarsi concluso. 

    \subsection{Norme di progetto e analisi dei requisiti}
    \textbf{Sintesi:} Sono stati compiuti notevoli progressi nella definizione delle norme di progetto e nell’analisi dei requisiti. Questo ci consente di operare su basi più solide e strutturate. \\

    \subsection{Condivisione delle risorse per l'apprendimento delle tecnologie}
    \textbf{Sintesi:} Abbiamo distribuito risorse utili per approfondire le tecnologie del progetto, migliorando la conoscenza collettiva del team. \\
    \textbf{Decisioni:} È stato deciso di pianificare ulteriori momenti di condivisione e formazione per consolidare le competenze.

    \subsection{Capire come segnalare la presenza di una parola nel glossario}
    \textbf{Sintesi:} Abbiamo discusso la necessità di identificare un sistema per marcare le parole del glossario nei documenti. Le opzioni esplorate includono un'automazione tramite script o una gestione manuale. \\
    \textbf{Difficoltà:} L'automazione richiede uno sviluppo tecnico ulteriore, mentre il metodo manuale potrebbe risultare poco scalabile. \\
    \textbf{Decisioni:} Abbiamo deciso di testare entrambe le opzioni per valutare quale sia la più adatta al nostro caso.

    \subsection{Individuazione dei casi d'uso}
    \textbf{Sintesi:} Definire i casi d’uso principali si è rivelato complesso, data la necessità di coprire scenari specifici e generici. \\
    \textbf{Difficoltà:} Mancano informazioni dettagliate su alcune funzionalità richieste. \\
    \textbf{Decisioni:} Si procederà con il contatto diretto con l'azienda per chiarire meglio le aspettative.
    
    % ...



Conclusioni


Pianificazione per la prossima settimana











    \subsection*{Conclusioni e Pianificazione} 
    \textbf{Conclusioni:} Abbiamo raggiunto traguardi significativi, come l’aggiornamento del sito dei documenti, il progresso sulle norme di progetto e l’analisi dei requisiti. Sono state condivise risorse utili per l’apprendimento e implementata una turnazione efficace dei ruoli. Tuttavia, alcune sfide come l’automazione del glossario e l’individuazione dei casi d’uso richiedono ulteriori analisi.

    \textbf{Pianificazione per la prossima settimana:} 
    \begin{itemize}
        \item Proseguire con la stesura e la validazione dei casi d’uso.
        \item Contattare l’azienda per discutere hardware e chiarire i requisiti.
        \item Sviluppare ulteriormente le norme di progetto e l’analisi dei requisiti.
        \item Esplorare la fattibilità di una bozza/prototipo del prodotto.
    \end{itemize}

    La prossima riunione è pianificata per il \textbf{29 11 2024}, con l'obiettivo di analizzare i progressi ottenuti e pianificare i prossimi passi.

%%%%%%%%%%%%%%%%%%%%%%%%%%%%%%%%%%%%%%%%%%%%%%%%%%%%%%%%%%%%%%%%%%%%%%%%%%%%%%%%%%%%%%



% Sezione Firme %%%%%%%%%%%%%%%%%%%%%%%%%%%%%%%%%%%%%%%%%%%%%%%%%%%%%%%%%%%%%%%%%%%%%%
% non toccare, modifica invece la variabile isEsterno
\ifthenelse{\equal{\isEsterno}{1}}{
    \begin{table}[b]
        \begin{tabular}{@{}p{.5in}p{4in}@{}}
            Data:  & \hrulefill \\
                   &     		\\
                   &     		\\
            Firma: & \hrulefill \\
        \end{tabular}
        \end{table}
}{}
%%%%%%%%%%%%%%%%%%%%%%%%%%%%%%%%%%%%%%%%%%%%%%%%%%%%%%%%%%%%%%%%%%%%%%%%%%%%%%%%%%%%%%


\end{document} 

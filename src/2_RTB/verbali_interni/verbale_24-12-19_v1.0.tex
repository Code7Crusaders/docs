%%% INTRO %%%%%%%%%%%%%%%%%%%%%%%%%%%%%%%%%%%%%%%%%%%%%%%%%%%%%%%%%%%%%%%%%%%%%%%%%%%%
% Template sia per verbali interni che esterni
% Segui i commenti "TODO" per ricordarti cosa modificare
% In caso di verbali esterni ricordati di settare isEsterno ad 1
%%%%%%%%%%%%%%%%%%%%%%%%%%%%%%%%%%%%%%%%%%%%%%%%%%%%%%%%%%%%%%%%%%%%%%%%%%%%%%%%%%%%%



%%% Settings %%%%%%%%%%%%%%%%%%%%%%%%%%%%%%%%%%%%%%%%%%%%%%%%%%%%%%%%%%%%%%%%%%%%%%%%%
\documentclass{article}

\usepackage{graphicx}  % serve per inserire immagini
\usepackage{fancyhdr}  % creazione header-footer
\usepackage{tabularx}  % serve per creare tabelle con colonne a larghezza variabile
\usepackage{ifthen}  % serve per mostrare cose diverse in base a condizioni
\usepackage{geometry}
\usepackage{setspace}
\usepackage{tikz}
\usepackage[italian]{babel}
\usepackage[hidelinks]{hyperref}

% TODO: setta a 1 se il verbale è esterno, 0 se è interno
\newcommand{\isEsterno}{0}

% Margini della pagina
\geometry{a4paper, margin=1in}

% Intestazione personalizzata
\pagestyle{fancy}
\fancyhf{}
\fancyhead[L]{Code7Crusaders - Software Development Team}
\fancyhead[R]{\thepage}

% Spaziatura delle righe
\setstretch{1.2}

\begin{document}
%%%%%%%%%%%%%%%%%%%%%%%%%%%%%%%%%%%%%%%%%%%%%%%%%%%%%%%%%%%%%%%%%%%%%%%%%%%%%%%%%%%%%%



%%% Sezione del titolo %%%%%%%%%%%%%%%%%%%%%%%%%%%%%%%%%%%%%%%%%%%%%%%%%%%%%%%%%%%%%%%
\begin{titlepage}

    \AddToHookNext{shipout/background}{
        \begin{tikzpicture}[remember picture,overlay]
        \node at (current page.center) {
            \includegraphics{../../img/background.png}
        };
        \end{tikzpicture}
    }

    \centering
    \vspace*{2cm}
    
    \includegraphics[width=0.3\textwidth]{../../img/logo/7Crusaders_logo.png} % logo
    \vspace{1cm}
    
    {\Huge \textbf{Code7Crusaders}}\\
    \vspace{0.5cm}
    {\Large Software Development Team}\\
    \vspace{2cm}
        
        {\large \textbf{Riunione Settimanale 19/12/2024}}\\
    \vspace{5cm}                           % esempio: Riunione Settimanale 04/11/2024
    
    
    \textbf{Membri del Team:}\\
    Enrico Cotti Cottini, Gabriele Di Pietro, Tommaso Diviesti \\
    Francesco Lapenna, Matthew Pan, Eddy Pinarello, Filippo Rizzolo \\
    \vspace{0.5cm}
    
    \vspace{1cm}
\end{titlepage}
%%%%%%%%%%%%%%%%%%%%%%%%%%%%%%%%%%%%%%%%%%%%%%%%%%%%%%%%%%%%%%%%%%%%%%%%%%%%%%%%%%%%%%



% Versioni %%%%%%%%%%%%%%%%%%%%%%%%%%%%%%%%%%%%%%%%%%%%%%%%%%%%%%%%%%%%%%%%%%%%%%%%%%%
\newpage
\begin{table}[h!]
\centering
\textbf{Versioni} \\ % Titolo sopra la tabella
\vspace{2mm} % Spazio tra il titolo e la tabella
\begin{tabular}{|c|c|c|c|c|}
    \hline
    \textbf{Ver.} & \textbf{Data} & \textbf{Autore} & \textbf{Verificatore} & \textbf{Descrizione} \\
    \hline
    1.0 & 22/12/2024 & Enrico Cotti Cottini & Gabriele Di Pietro & Prima stesura del documento \\ 
    \hline                                  % TODO: inserire data, nomi e descrizione
\end{tabular}
\end{table}
%%%%%%%%%%%%%%%%%%%%%%%%%%%%%%%%%%%%%%%%%%%%%%%%%%%%%%%%%%%%%%%%%%%%%%%%%%%%%%%%%%%%%%



% Indice %%%%%%%%%%%%%%%%%%%%%%%%%%%%%%%%%%%%%%%%%%%%%%%%%%%%%%%%%%%%%%%%%%%%%%%%%%%%%
\newpage
\tableofcontents
%%%%%%%%%%%%%%%%%%%%%%%%%%%%%%%%%%%%%%%%%%%%%%%%%%%%%%%%%%%%%%%%%%%%%%%%%%%%%%%%%%%%%%



% Registro Presenze %%%%%%%%%%%%%%%%%%%%%%%%%%%%%%%%%%%%%%%%%%%%%%%%%%%%%%%%%%%%%%%%%%
\newpage
\section{Registro Presenze}
\textbf{Piattaforma della riunione:} Piattaforma Discord \\
\textbf{Ora di Inizio} 10:30\\
\textbf{Ora di Fine} 11:30\\  % TODO: inserire orari ed eventualmente piattaforma
\\
\begin{tabular}{|c|c|c|}  % TODO: inserire ruoli e presenze
    \hline
    \textbf{Componente} & \textbf{Ruolo} & \textbf{Presenza}\\
    \hline
    Enrico Cotti Cottini & Verificatore & Presente \\ 
    \hline
    Gabriele Di Pietro & Programmatore & Presente \\ 
    \hline
    Tommaso Diviesti & Responsabile & Presente \\ 
    \hline 
    Francesco Lapenna & Progettista& Presente \\ 
    \hline
    Matthew Pan & Analista & Assente \\ 
    \hline 
    Eddy Pinarello & Amministratore & Presente \\ 
    \hline 
    Filippo Rizzolo & Verificatore& Presente \\ 
    \hline 
\end{tabular}
% Presenze Rappresentanti Azienda %%%%%%%%%%%%%%%%%%%%%%%%%%%%%%%%%%%%%%%%%%%%%%%%%%%%
% non toccare, modifica invece la variabile isEsterno
\ifthenelse{\equal{\isEsterno}{1}}{
    \\
    \newline
    \newline
    \begin{tabular}{|c|c|}  % TODO: eventualmente modificare nomi rappresentanti
        \hline
        \textbf{Nome} & \textbf{Ruolo}\\
        \hline
        Gianluca Carlesso & Rappresentante Azienda \\
        \hline
        Anna Tieppo & Rappresentante Azienda \\
        \hline
    \end{tabular}
}{}
%%%%%%%%%%%%%%%%%%%%%%%%%%%%%%%%%%%%%%%%%%%%%%%%%%%%%%%%%%%%%%%%%%%%%%%%%%%%%%%%%%%%%%



% Sezione Verbale %%%%%%%%%%%%%%%%%%%%%%%%%%%%%%%%%%%%%%%%%%%%%%%%%%%%%%%%%%%%%%%%%%%%
\newpage
\section{Verbale Retrospettiva}

\subsection{Progresso Settimanale}
\subsubsection{Front-end}
Questa settimana abbiamo proseguito la ricerca dello strumento più adatto per la realizzazione del front-end, considerando React come possibile scelta. La finalizzazione del front-end è prevista per la prossima settimana.  
Abbiamo deciso di redigere un documento di analisi del front-end, simile a quello realizzato per il back-end, per presentarlo all'azienda in quanto richiesto dal capitolato.

\subsubsection{Requisiti e Casi d'Uso}
Abbiamo raccolto i requisiti basandoci sui casi d'uso e contattato l'azienda per comunicare i requisiti identificati. Abbiamo richiesto una loro valutazione per confermare se fossero corretti o se fosse necessario integrare nuovi requisiti.  
Abbiamo organizzato una chiamata interna per discutere dei casi d'uso in vista del ricevimento con il Professor Cardin. Durante la discussione, sono emerse domande specifiche da sottoporre al professore. Siamo pronti a correggere i casi d'uso sulla base delle sue indicazioni.

\subsubsection{Back-end}
L'architettura del back-end è stata aggiornata in seguito alla conferma e al feedback positivo dell'azienda. La stessa ci ha fornito le chiavi per utilizzare il modello sulla piattaforma OpenAI con un saldo iniziale di \$25. Questo ci permette di iniziare a lavorare al \textit{Proof of Concept (POC)}.

\subsubsection{Verifica e Correzione Documenti}
Abbiamo continuato la verifica dei documenti, correggendo errori riscontrati, come alcune formule errate nel Piano di Qualifica. Inoltre, sono proseguiti i lavori sui documenti di Piano di Progetto, Piano di Qualifica e Norme di Progetto.

\subsubsection{Problemi di Comunicazione}
È emerso un problema di comunicazione nella redazione delle Norme di Progetto con un membro specifico del team. Questo problema, che si ripete, è caratterizzato da una scarsa partecipazione a chiamate, attività di pianificazione e incontri interni, sia sul gruppo WhatsApp che in chiamata, e anche con interlocutori esterni.  
Abbiamo deciso di affrontare la situazione promuovendo un cambiamento di approccio e incentivando un maggiore impegno nella comunicazione e collaborazione. Se il problema si ripresenterà, dovremo prendere decisioni più radicali.

\subsection{Pianificazione per la Prossima Settimana}
\begin{itemize}
    \item Completare le attività di routine e verifica dei documenti.
    \item Iniziare l'analisi sul database da utilizzare.
    \item Aggiornare le definizioni del glossario con i termini raccolti fino a questo momento.
    \item Completare le attività in vista della revisione RTB e la consegna del POC.
\end{itemize}

\subsection{Conclusioni}
Nonostante le festività natalizie, ci aspettiamo di mantenere un livello di produttività costante, completando le attività pianificate. La prossima riunione è programmata per il \textbf{27/12/2024}, con l'obiettivo di analizzare i progressi e pianificare i prossimi passi.


%%%%%%%%%%%%%%%%%%%%%%%%%%%%%%%%%%%%%%%%%%%%%%%%%%%%%%%%%%%%%%%%%%%%%%%%%%%%%%%%%%%%%%



% Sezione Firme %%%%%%%%%%%%%%%%%%%%%%%%%%%%%%%%%%%%%%%%%%%%%%%%%%%%%%%%%%%%%%%%%%%%%%
% non toccare, modifica invece la variabile isEsterno
\ifthenelse{\equal{\isEsterno}{1}}{
    \begin{table}[b]
        \begin{tabular}{@{}p{.5in}p{4in}@{}}
            Data:  & \hrulefill \\
                   &     		\\
                   &     		\\
            Firma: & \hrulefill \\
        \end{tabular}
        \end{table}
}{}
%%%%%%%%%%%%%%%%%%%%%%%%%%%%%%%%%%%%%%%%%%%%%%%%%%%%%%%%%%%%%%%%%%%%%%%%%%%%%%%%%%%%%%


\end{document} 

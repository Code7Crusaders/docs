%%% INTRO %%%%%%%%%%%%%%%%%%%%%%%%%%%%%%%%%%%%%%%%%%%%%%%%%%%%%%%%%%%%%%%%%%%%%%%%%%%%
% Template sia per verbali interni che esterni
% Segui i commenti "TODO" per ricordarti cosa modificare
% In caso di verbali esterni ricordati di settare isEsterno ad 1
%%%%%%%%%%%%%%%%%%%%%%%%%%%%%%%%%%%%%%%%%%%%%%%%%%%%%%%%%%%%%%%%%%%%%%%%%%%%%%%%%%%%%



%%% Settings %%%%%%%%%%%%%%%%%%%%%%%%%%%%%%%%%%%%%%%%%%%%%%%%%%%%%%%%%%%%%%%%%%%%%%%%%
\documentclass{article}

\usepackage{graphicx}  % serve per inserire immagini
\usepackage{fancyhdr}  % creazione header-footer
\usepackage{tabularx}  % serve per creare tabelle con colonne a larghezza variabile
\usepackage{ifthen}  % serve per mostrare cose diverse in base a condizioni
\usepackage{geometry}
\usepackage{setspace}
\usepackage{tikz}
\usepackage[italian]{babel}
\usepackage[hidelinks]{hyperref}

% TODO: setta a 1 se il verbale è esterno, 0 se è interno
\newcommand{\isEsterno}{1}

% Margini della pagina
\geometry{a4paper, margin=1in}

% Intestazione personalizzata
\pagestyle{fancy}
\fancyhf{}
\fancyhead[L]{Code7Crusaders - Software Development Team}
\fancyhead[R]{\thepage}

% Spaziatura delle righe
\setstretch{1.2}

\begin{document}
%%%%%%%%%%%%%%%%%%%%%%%%%%%%%%%%%%%%%%%%%%%%%%%%%%%%%%%%%%%%%%%%%%%%%%%%%%%%%%%%%%%%%%



%%% Sezione del titolo %%%%%%%%%%%%%%%%%%%%%%%%%%%%%%%%%%%%%%%%%%%%%%%%%%%%%%%%%%%%%%%
\begin{titlepage}

    \AddToHookNext{shipout/background}{
        \begin{tikzpicture}[remember picture,overlay]
        \node at (current page.center) {
            \includegraphics{../../img/background.png}
        };
        \end{tikzpicture}
    }

    \centering
    \vspace*{2cm}
    
    \includegraphics[width=0.3\textwidth]{../../img/logo/7Crusaders_logo.png} % logo
    \vspace{1cm}
    
    {\Huge \textbf{Code7Crusaders}}\\
    \vspace{0.5cm}
    {\Large Software Development Team}\\
    \vspace{2cm}
    
    {\large \textbf{Incontro del 12/12/2024 con Ergon}}\\ % TODO: inserire titolo del verbale
    \vspace{5cm}                           % esempio: Riunione Settimanale 04/11/2024
    
    
    \textbf{Membri del Team:}\\
    Enrico Cotti Cottini, Gabriele Di Pietro, Tommaso Diviesti \\
    Francesco Lapenna, Matthew Pan, Eddy Pinarello, Filippo Rizzolo \\
    \vspace{0.5cm}
    
    \vspace{1cm}
\end{titlepage}
%%%%%%%%%%%%%%%%%%%%%%%%%%%%%%%%%%%%%%%%%%%%%%%%%%%%%%%%%%%%%%%%%%%%%%%%%%%%%%%%%%%%%%



% Versioni %%%%%%%%%%%%%%%%%%%%%%%%%%%%%%%%%%%%%%%%%%%%%%%%%%%%%%%%%%%%%%%%%%%%%%%%%%%
\newpage
\begin{table}[h!]
\centering
\textbf{Versioni} \\ % Titolo sopra la tabella
\vspace{2mm} % Spazio tra il titolo e la tabella
\begin{tabular}{|c|c|c|c|c|}
    \hline
    \textbf{Ver.} & \textbf{Data} & \textbf{Autore} & \textbf{Verificatore} & \textbf{Descrizione} \\
    \hline
    1.0 & 15/12/2024 & Gabriele Di Pietro & Tommaso Diviesti & Stesura del documento \\ 
    \hline                                  % TODO: inserire data, nomi e descrizione
\end{tabular}
\end{table}
%%%%%%%%%%%%%%%%%%%%%%%%%%%%%%%%%%%%%%%%%%%%%%%%%%%%%%%%%%%%%%%%%%%%%%%%%%%%%%%%%%%%%%



% Indice %%%%%%%%%%%%%%%%%%%%%%%%%%%%%%%%%%%%%%%%%%%%%%%%%%%%%%%%%%%%%%%%%%%%%%%%%%%%%
\newpage
\tableofcontents
%%%%%%%%%%%%%%%%%%%%%%%%%%%%%%%%%%%%%%%%%%%%%%%%%%%%%%%%%%%%%%%%%%%%%%%%%%%%%%%%%%%%%%



% Registro Presenze %%%%%%%%%%%%%%%%%%%%%%%%%%%%%%%%%%%%%%%%%%%%%%%%%%%%%%%%%%%%%%%%%%
\newpage
\section{Registro Presenze}
\textbf{Piattaforma della riunione:} Piattaforma Zoom \\
\textbf{Ora di Inizio} 12:30\\
\textbf{Ora di Fine} 13:00\\  % TODO: inserire orari ed eventualmente piattaforma
\\
\begin{tabular}{|c|c|c|}  % TODO: inserire ruoli e presenze
    \hline
    \textbf{Componente} & \textbf{Ruolo} & \textbf{Presenza}\\
    \hline
    Enrico Cotti Cottini & Programmatore & Presente \\ 
    \hline
    Gabriele Di Pietro & Amministratore & Presente\\ 
    \hline
    Tommaso Diviesti & Verificatore & Presente \\ 
    \hline 
    Francesco Lapenna & Responsabile & Presente \\ 
    \hline
    Matthew Pan & Progettista & Assente\\ 
    \hline 
    Eddy Pinarello & Verificatore & Presente \\ 
    \hline 
    Filippo Rizzolo & Analista & Presente \\ 
    \hline 
\end{tabular}
% Presenze Rappresentanti Azienda %%%%%%%%%%%%%%%%%%%%%%%%%%%%%%%%%%%%%%%%%%%%%%%%%%%%
% non toccare, modifica invece la variabile isEsterno
\ifthenelse{\equal{\isEsterno}{1}}{
    \\
    \newline
    \newline
    \begin{tabular}{|c|c|}  % TODO: eventualmente modificare nomi rappresentanti
        \hline
        \textbf{Nome} & \textbf{Ruolo}\\
        \hline
        Gianluca Carlesso & Rappresentante Azienda \\
        \hline
    \end{tabular}
}{}
%%%%%%%%%%%%%%%%%%%%%%%%%%%%%%%%%%%%%%%%%%%%%%%%%%%%%%%%%%%%%%%%%%%%%%%%%%%%%%%%%%%%%%



% Sezione Verbale %%%%%%%%%%%%%%%%%%%%%%%%%%%%%%%%%%%%%%%%%%%%%%%%%%%%%%%%%%%%%%%%%%%%
\newpage
\section{Verbale}
    % TODO: per ogni punto discusso / attività svolta
    % inserire una sottosezione, sintesi ed eventuali decisioni
    \subsection{Introduzione}
    L'Incontro è avvenuto con una presentazione da parte di Enrico dei vari modelli LLM da utilizzare.
    In particolare abbiamo confrontato i modelli di BigScience Workshop \textit{Hugging Face} e i modelli di OpenAI e abbiamo chiesto
    quale potessimo utilizzare, parlando dei vari vantaggi dei vari svantaggi dei costi. In particolare abbiamo proposto la nostra visione 
    nell'utilizzo dei modelli di OpenAI, più precisamente del modello \textit{gpt-o4-mini}, chiedendo, inoltre, se fosse una soluzione adatta all'implementazione pratica.

    \subsection{Scelta del modello OpenAI}
    \textbf{Domanda:} Possiamo Usare il modello di OpenAI? Se si, questo viene pagato dall'azienda? \\
    \textbf{Discussione:} Abbiamo presentato i vari vantaggi e svantaggi per i diversi modelli e discusso del perchè conviene utilizzare il modello di OpenAI rispetto Huggingface o altre soluzioni concorrenti. Per la discussione di questo problema, abbiamo creato un documento apposito che elenca specifiche, pro e contro dei vari modelli. \\
    \textbf{Conclusione:} L'azienda ha apprezzato molto il lavoro svolto e nei prossimi giorni creeranno un profilo su OpenAi e ci aggiungeranno come collaboratori. Inoltre, ci pagheranno dei token per effettuare i test.

    \subsection{Correttezza dei casi d'uso revisionati}
    \textbf{Domanda:} I casi d'uso individuati vanno bene? \\
    \textbf{Discussione:} Abbiamo revisionato i casi d'uso e modificato le cose che non andavano bene, e volevamo capire se andava aggiunto altro oppure modificato qualcosa. \\
    \textbf{Conclusione:} L'azienda ci conferma definitivamente che i casi d'uso individuati vanno bene.



    % ...

    \section{Conclusioni}  % TODO: inserire conclusioni della riunione
    \textbf{Prossimi Passi:} 
    Come prossimo passo bisogna capire bene quale tecnologia usare per il database e sceglierne uno, inoltre bisogna pensare a una interfaccia grafica per lo sviluppo della webapp andandone a definire le scelte delle varie tecnologie da discutere in una prossima riunione.
%%%%%%%%%%%%%%%%%%%%%%%%%%%%%%%%%%%%%%%%%%%%%%%%%%%%%%%%%%%%%%%%%%%%%%%%%%%%%%%%%%%%%%


% Sezione Firme %%%%%%%%%%%%%%%%%%%%%%%%%%%%%%%%%%%%%%%%%%%%%%%%%%%%%%%%%%%%%%%%%%%%%%
% non toccare, modifica invece la variabile isEsterno
\ifthenelse{\equal{\isEsterno}{1}}{
    \begin{table}[b]
        \begin{tabular}{@{}p{.5in}p{4in}@{}}
            Data:  & \hrulefill \\
                   &     		\\
                   &     		\\
            Firma: & \hrulefill \\
        \end{tabular}
        \end{table}
}{}
%%%%%%%%%%%%%%%%%%%%%%%%%%%%%%%%%%%%%%%%%%%%%%%%%%%%%%%%%%%%%%%%%%%%%%%%%%%%%%%%%%%%%%


\end{document} 
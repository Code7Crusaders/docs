%%% INTRO %%%%%%%%%%%%%%%%%%%%%%%%%%%%%%%%%%%%%%%%%%%%%%%%%%%%%%%%%%%%%%%%%%%%%%%%%%%%
% Template sia per verbali interni che esterni
% Segui i commenti "TODO" per ricordarti cosa modificare
% In caso di verbali esterni ricordati di settare isEsterno ad 1
%%%%%%%%%%%%%%%%%%%%%%%%%%%%%%%%%%%%%%%%%%%%%%%%%%%%%%%%%%%%%%%%%%%%%%%%%%%%%%%%%%%%%



%%% Settings %%%%%%%%%%%%%%%%%%%%%%%%%%%%%%%%%%%%%%%%%%%%%%%%%%%%%%%%%%%%%%%%%%%%%%%%%
\documentclass{article}

\usepackage{graphicx}  % serve per inserire immagini
\usepackage{fancyhdr}  % creazione header-footer
\usepackage{tabularx}  % serve per creare tabelle con colonne a larghezza variabile
\usepackage{ifthen}  % serve per mostrare cose diverse in base a condizioni
\usepackage{geometry}
\usepackage{setspace}
\usepackage{tikz}
\usepackage[italian]{babel}
\usepackage[hidelinks]{hyperref}

% TODO: setta a 1 se il verbale è esterno, 0 se è interno
\newcommand{\isEsterno}{1}

% Margini della pagina
\geometry{a4paper, margin=1in}

% Intestazione personalizzata
\pagestyle{fancy}
\fancyhf{}
\fancyhead[L]{Code7Crusaders - Software Development Team}
\fancyhead[R]{\thepage}

% Spaziatura delle righe
\setstretch{1.2}

\begin{document}
%%%%%%%%%%%%%%%%%%%%%%%%%%%%%%%%%%%%%%%%%%%%%%%%%%%%%%%%%%%%%%%%%%%%%%%%%%%%%%%%%%%%%%



%%% Sezione del titolo %%%%%%%%%%%%%%%%%%%%%%%%%%%%%%%%%%%%%%%%%%%%%%%%%%%%%%%%%%%%%%%
\begin{titlepage}

    \AddToHookNext{shipout/background}{
        \begin{tikzpicture}[remember picture,overlay]
        \node at (current page.center) {
            \includegraphics{../../img/background.png}
        };
        \end{tikzpicture}
    }

    \centering
    \vspace*{2cm}
    
    \includegraphics[width=0.3\textwidth]{../../img/logo/7Crusaders_logo.png} % logo
    \vspace{1cm}
    
    {\Huge \textbf{Code7Crusaders}}\\
    \vspace{0.5cm}
    {\Large Software Development Team}\\
    \vspace{2cm}
    
    {\large \textbf{Incontro del 14/11/2024 con Ergon}}\\ % TODO: inserire titolo del verbale
    \vspace{5cm}                           % esempio: Riunione Settimanale 04/11/2024
    
    
    \textbf{Membri del Team:}\\
    Enrico Cotti Cottini, Gabriele Di Pietro, Tommaso Diviesti \\
    Francesco Lapenna, Matthew Pan, Eddy Pinarello, Filippo Rizzolo \\
    \vspace{0.5cm}
    
    \vspace{1cm}
\end{titlepage}
%%%%%%%%%%%%%%%%%%%%%%%%%%%%%%%%%%%%%%%%%%%%%%%%%%%%%%%%%%%%%%%%%%%%%%%%%%%%%%%%%%%%%%



% Versioni %%%%%%%%%%%%%%%%%%%%%%%%%%%%%%%%%%%%%%%%%%%%%%%%%%%%%%%%%%%%%%%%%%%%%%%%%%%
\newpage
\begin{table}[h!]
\centering
\textbf{Versioni} \\ % Titolo sopra la tabella
\vspace{2mm} % Spazio tra il titolo e la tabella
\begin{tabular}{|c|c|c|c|c|}
    \hline
    \textbf{Ver.} & \textbf{Data} & \textbf{Autore} & \textbf{Verificatore} & \textbf{Descrizione} \\
    \hline
    0.1 & 20/11/2024 & Lapenna Francesco & Matthew Pan & Prima stesura del documento \\ 
    \hline                                  % TODO: inserire data, nomi e descrizione
\end{tabular}
\end{table}
%%%%%%%%%%%%%%%%%%%%%%%%%%%%%%%%%%%%%%%%%%%%%%%%%%%%%%%%%%%%%%%%%%%%%%%%%%%%%%%%%%%%%%



% Indice %%%%%%%%%%%%%%%%%%%%%%%%%%%%%%%%%%%%%%%%%%%%%%%%%%%%%%%%%%%%%%%%%%%%%%%%%%%%%
\newpage
\tableofcontents
%%%%%%%%%%%%%%%%%%%%%%%%%%%%%%%%%%%%%%%%%%%%%%%%%%%%%%%%%%%%%%%%%%%%%%%%%%%%%%%%%%%%%%



% Registro Presenze %%%%%%%%%%%%%%%%%%%%%%%%%%%%%%%%%%%%%%%%%%%%%%%%%%%%%%%%%%%%%%%%%%
\newpage
\section{Registro Presenze}
\textbf{Piattaforma della riunione:} Piattaforma Zoom \\
\textbf{Ora di Inizio} 15:00\\
\textbf{Ora di Fine} 16:00\\  % TODO: inserire orari ed eventualmente piattaforma
\\
\begin{tabular}{|c|c|c|}  % TODO: inserire ruoli e presenze
    \hline
    \textbf{Componente} & \textbf{Ruolo} & \textbf{Presenza}\\
    \hline
    Enrico Cotti Cottini & Verificatore & Presente \\ 
    \hline
    Gabriele Di Pietro & Responsabile & Presente\\ 
    \hline
    Tommaso Diviesti & Redattore & Presente \\ 
    \hline 
    Francesco Lapenna & Redattore& Presente \\ 
    \hline
    Matthew Pan & Verificatore & Presente\\ 
    \hline 
    Eddy Pinarello & Redattore & Presente \\ 
    \hline 
    Filippo Rizzolo & Amministratore& Presente \\ 
    \hline 
\end{tabular}
% Presenze Rappresentanti Azienda %%%%%%%%%%%%%%%%%%%%%%%%%%%%%%%%%%%%%%%%%%%%%%%%%%%%
% non toccare, modifica invece la variabile isEsterno
\ifthenelse{\equal{\isEsterno}{1}}{
    \\
    \newline
    \newline
    \begin{tabular}{|c|c|}  % TODO: eventualmente modificare nomi rappresentanti
        \hline
        \textbf{Nome} & \textbf{Ruolo}\\
        \hline
        Gianluca Carlesso & Rappresentante Azienda \\
        \hline
    \end{tabular}
}{}
%%%%%%%%%%%%%%%%%%%%%%%%%%%%%%%%%%%%%%%%%%%%%%%%%%%%%%%%%%%%%%%%%%%%%%%%%%%%%%%%%%%%%%



% Sezione Verbale %%%%%%%%%%%%%%%%%%%%%%%%%%%%%%%%%%%%%%%%%%%%%%%%%%%%%%%%%%%%%%%%%%%%
\newpage
\section{Verbale}
    % TODO: per ogni punto discusso / attività svolta
    % inserire una sottosezione, sintesi ed eventuali decisioni

    \subsection{Preferenza tra App Mobile o Web App}
    \textbf{Domanda:} Preferite un'app mobile o va bene una web app? \\
    \textbf{Discussione:} Sono state identificate due opzioni principali: Web app responsive e App mobile (consigliato l'uso di framework cross-platform come .NET MAUI).\\
    \textbf{Conclusione:} L'azienda ha mostrato preferenza per soluzioni flessibili. È stato suggerito React come tecnologia per lo sviluppo web e, in alternativa, un’app Android se necessario.

    \subsection{Richiesta di un Dataset di Esempio2}
    \textbf{Domanda:} Potete fornire un dataset di esempio? \\
    \textbf{Risposta:} Verrà fornito un database di bevande nei prossimi giorni.

    \subsection{Hardware per Modelli LLM}
    \textbf{Domanda:} Ci fornite una macchina per eseguire i modelli LLM? Quali specifiche hardware? \\
    \textbf{Risposta:} L'azienda fornirà una macchina e il team di sviluppo potrà definire le specifiche hardware necessarie.

    \subsection{Requisiti Utente e Software}
    \textbf{Domanda:} Quali sono i requisiti utente e software? \\
    \textbf{Risposta:} Per i requisiti obbligatori fare riferimento al capitolato. Requisiti opzionali: Inserire funzionalità che permettano all'utente di fornire feedback sulle risposte generate dal sistema.

    \subsection{PoC}
    \textbf{Domanda:} È sufficiente un'interfaccia da terminale che risponda per il PoC? \\
    \textbf{Risposta:} Per la PoC è sufficiente una soluzione terminale che risponda alle domande. Una volta completata la logica del modello LLM, si procederà allo sviluppo dell’interfaccia grafica..

    \subsection{Target di Riferimento}
    \textbf{Domanda:} Chi è il target? \\
    \textbf{Risposta:} Utenti finali non esperti, come proprietari di pub o ristoranti nel settore alimentare.
    Attori coinvolti:
    \begin{itemize}
        \item L'azienda fornitrice che si interfaccia con la software house.
        \item La software house che si interfaccia con gli utenti finali..
    \end{itemize}

    \subsection{Unit Testing per l'LLM}
    \textbf{Domanda:} È necessario fare unit testing delle risposte dell'LLM? \\
    \textbf{Risposta:} Non è possibile testare completamente le risposte dell'LLM con un altro modello. Verranno eseguiti test a livello umano per verificare la coerenza e la qualità delle risposte. L'LLM viene considerato un sistema "Black Box" poiché basato su modelli preaddestrati..

    % ...

    \subsection*{Conclusioni}  % TODO: inserire conclusioni della riunione
    \textbf{Prossimi Passi:} 
    \begin{enumerate}
        \item Ricevere il database di esempio dall'azienda
        \item Definire le specifiche hardware per la macchina dedicata all'LLM
        \item Lavorare su una PoC con interazione via terminale.
        \item Identificare un approccio per raccogliere feedback utenti sulle risposte.
    \end{enumerate}
%%%%%%%%%%%%%%%%%%%%%%%%%%%%%%%%%%%%%%%%%%%%%%%%%%%%%%%%%%%%%%%%%%%%%%%%%%%%%%%%%%%%%%



% Sezione Firme %%%%%%%%%%%%%%%%%%%%%%%%%%%%%%%%%%%%%%%%%%%%%%%%%%%%%%%%%%%%%%%%%%%%%%
% non toccare, modifica invece la variabile isEsterno
\ifthenelse{\equal{\isEsterno}{1}}{
    \begin{table}[b]
        \begin{tabular}{@{}p{.5in}p{4in}@{}}
            Data:  & \hrulefill \\
                   &     		\\
                   &     		\\
            Firma: & \hrulefill \\
        \end{tabular}
        \end{table}
}{}
%%%%%%%%%%%%%%%%%%%%%%%%%%%%%%%%%%%%%%%%%%%%%%%%%%%%%%%%%%%%%%%%%%%%%%%%%%%%%%%%%%%%%%


\end{document} 
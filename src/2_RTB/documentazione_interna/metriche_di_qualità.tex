\section{Metriche di qualità}
La qualità di processo è un criterio fondamentale ed è alla base di ogni prodotto che rispecchi lo stato dell'arte.
Per raggiungere tale obiettivo è necessario sfruttare delle pratiche rigorose che consentano lo svolgimento di ogni attività in maniera ottimale.
Valutando nel miglior modo possibile la qualità del prodotto e l'efficacia dei processi, sono state definite delle metriche riportate di seguito.
Lo scopo di questa sezione è quello di identificarne i parametri che le metriche devono rispettare per essere considerate accettabili o ottime. Esse sono state suddivise utilizzando lo standard \emph{ISO/IEC 12207:1995},
il quale separa i processi di ciclo di vita del Software in tre categorie:
\begin{enumerate}
    \item Processi di base e/o primari;
    \item Processi di supporto;
    \item Processi organizzativi;
\end{enumerate}

\subsection{Processi di base e/o primari}
\subsubsection{Fornitura}
Nella fase di \emph{Fornitura} si definiscono le procedure e le risorse necessarie per la consegna del prodotto.
Definiamo quindi le seguenti metriche:


\paragraph*{1PBM-PV Planed Value}
\begin{itemize}
    \item \textbf{Definizione:} il Planned Value (Valore Pianificato) rappresenta il valore del lavoro programmato per essere completato fino a un determinato momento. Si tratta del budget preventivato per lo sprint in corso.
    \item \textbf{Come si calcola:}
\end{itemize}
\begin{center}
   $PV = BAC \times LP$ 
\end{center}
dove:
\begin{itemize}[label=$\rightarrow$]
    \item $BAC$: Budget At Completion
    \item $LP$: Percentuale Lavoro Pianificato
\end{itemize}
\begin{itemize}
    \item \textbf{Valore ammissibile:}
\end{itemize}
\begin{center}
    $PV \geq 0$
\end{center}
\begin{itemize}
    \item \textbf{Valore ottimo:}
\end{itemize}
\begin{center}
    $PV \leq BAC$
\end{center}
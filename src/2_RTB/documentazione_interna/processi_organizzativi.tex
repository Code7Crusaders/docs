\section{Processi Organizzativi}

\subsection{Gestione dei Processi}
\subsubsection{Introduzione}

La gestione dei processi rappresenta una fase cruciale per il successo di un progetto, garantendo che venga completato in conformità agli obiettivi e ai requisiti predefiniti. Questa fase si concentra sulla pianificazione, organizzazione, monitoraggio e controllo delle attività coinvolte nel ciclo di vita del software, assicurando che il lavoro svolto rispetti gli standard di qualità e soddisfi le aspettative del cliente. Le principali attività di gestione dei processi sono le seguenti:

\begin{itemize}
    \item \textbf{Definizione dei processi:} Documentazione dei processi chiave che verranno adottati nel progetto, inclusi quelli relativi allo sviluppo del software, al controllo di versione, alla gestione dei cambiamenti e all'assicurazione della qualità.
    
    \item \textbf{Pianificazione dei processi:} Definizione degli obiettivi del progetto, delle fasi, delle risorse necessarie e delle scadenze. In questa fase vengono stabiliti anche i criteri di successo e redatto un piano di lavoro dettagliato.
    
    \item \textbf{Assegnazione delle risorse:} Allocazione dei membri del team alle attività specifiche del progetto, tenendo conto delle loro competenze e disponibilità.
    
    \item \textbf{Monitoraggio e controllo:} Monitoraggio continuo dei progressi rispetto al piano stabilito, comprendente il controllo di tempi, costi e qualità, nonché l'identificazione e la gestione dei rischi.
    
    \item \textbf{Gestione dei cambiamenti:} Valutazione e gestione delle modifiche richieste durante lo sviluppo del software, che potrebbero riguardare i requisiti, la pianificazione o la distribuzione delle risorse.
    
    \item \textbf{Assicurazione della qualità:} Implementazione di processi e procedure finalizzati a garantire che il prodotto software soddisfi i requisiti e le aspettative del cliente.
    
    \item \textbf{Comunicazione e coordinamento:} Facilitazione della comunicazione tra i membri del team e gli stakeholder. Questo assicura che tutte le parti coinvolte siano informate sullo stato del progetto e sulle decisioni prese.
    
    \item \textbf{Miglioramento continuo:} Analisi dei processi utilizzati nel progetto per identificare aree di miglioramento e implementare azioni correttive al fine di ottimizzare l’efficienza e la qualità complessiva del lavoro svolto.
\end{itemize}


\subsubsection{Pianificazione}
\subsubsection{4.1.2.1 Descrizione}

La pianificazione dei processi consiste nell'identificare, organizzare e controllare le attività necessarie per garantire il successo del progetto. Si tratta di un'attività strategica che fornisce una direzione chiara e una solida struttura gestionale lungo tutto il ciclo di vita del progetto. La pianificazione dei processi è fondamentale per assicurare che il software venga completato nei tempi previsti, rispettando il budget e soddisfacendo gli standard di qualità richiesti.

\subsubsection{4.1.2.2 Obiettivi}

L’obiettivo principale della pianificazione dei processi è quello di garantire l'esecuzione efficiente ed efficace del progetto, assicurando che siano rispettati gli obiettivi e i requisiti stabiliti. Inoltre, la pianificazione deve assicurare che ogni membro del team assuma almeno una volta ciascun ruolo, contribuendo così alla crescita e alla collaborazione all'interno del gruppo. Un altro scopo importante è quello di ridurre i rischi e affrontare le sfide in modo anticipato, permettendo al team di superare eventuali difficoltà che potrebbero sorgere durante lo sviluppo.

\subsubsection{4.1.2.3 Assegnazione dei ruoli}

Durante l'implementazione del progetto, i membri del team di 7Last ricopriranno ruoli distinti, ognuno responsabile delle attività ad esso assegnate. Ogni membro dovrà prendersi carico delle proprie responsabilità, come indicato nei ruoli specifici che seguono.

\textbf{Responsabile:}
\begin{itemize}
    \item Coordina il gruppo di lavoro.
    \item Pianifica e controlla le attività.
    \item Gestisce le risorse.
    \item Gestisce le comunicazioni con l’esterno.
    \item Redige il Piano di Progetto.
\end{itemize}

\textbf{Amministratore:}
\begin{itemize}
    \item Gestisce l’ambiente di lavoro.
    \item Gestisce le procedure e le norme.
    \item Gestisce la configurazione del prodotto.
    \item Redige le Norme di Progetto.
\end{itemize}

\textbf{Analista:}
\begin{itemize}
    \item Analizza i requisiti del progetto.
    \item Studia il dominio applicativo del problema.
    \item Redige l’Analisi dei Requisiti.
\end{itemize}

\textbf{Progettista:}
\begin{itemize}
    \item Progetta l’architettura del prodotto.
    \item Prende decisioni tecniche e tecnologiche.
    \item Redige la Specifica Tecnica.
\end{itemize}

\textbf{Programmatore:}
\begin{itemize}
    \item Scrive il codice del prodotto.
    \item Implementa le funzionalità richieste.
    \item Codifica le componenti dell’architettura del prodotto.
    \item Redige il Manuale Utente.
\end{itemize}

\textbf{Verificatore:}
\begin{itemize}
    \item Verifica che il lavoro svolto sia conforme alle norme e alle specifiche tecniche del progetto.
    \item Ricerca ed eventualmente segnala errori.
    \item Redige il Piano di Qualità.
\end{itemize}


\subsubsection{Metriche}
\subsubsection{Coordinamento}

\section{Processi Organizzativi}

\subsection{Gestione dei Processi}

\subsubsection{Introduzione}
La gestione dei processi rappresenta una fase cruciale per il successo di un progetto, garantendo che venga completato in conformità agli obiettivi e ai requisiti predefiniti. Questa fase si concentra sulla pianificazione, organizzazione, monitoraggio e controllo delle attività coinvolte nel ciclo di vita del software, assicurando che il lavoro svolto rispetti gli standard di qualità e soddisfi le aspettative del cliente. 

Le principali attività di gestione dei processi sono le seguenti:
\begin{itemize}
    \item \textbf{Definizione dei processi:} Documentazione dei processi chiave adottati nel progetto, inclusi quelli relativi allo sviluppo del software, controllo di versione, gestione dei cambiamenti e assicurazione della qualità.
    \item \textbf{Pianificazione dei processi:} Definizione degli obiettivi del progetto, delle fasi, delle risorse necessarie e delle scadenze. In questa fase vengono stabiliti i criteri di successo e redatto un piano di lavoro dettagliato.
    \item \textbf{Assegnazione delle risorse:} Allocazione dei membri del team alle attività specifiche, tenendo conto delle loro competenze e disponibilità.
    \item \textbf{Monitoraggio e controllo:} Controllo continuo dei progressi rispetto al piano stabilito, comprendente tempi, costi e qualità, oltre a gestione dei rischi.
    \item \textbf{Gestione dei cambiamenti:} Valutazione e gestione delle modifiche ai requisiti, alla pianificazione o alla distribuzione delle risorse.
    \item \textbf{Assicurazione della qualità:} Implementazione di procedure per garantire che il software soddisfi i requisiti e le aspettative del cliente.
    \item \textbf{Comunicazione e coordinamento:} Facilitazione della comunicazione tra membri del team e stakeholder per mantenere tutte le parti informate sullo stato del progetto.
    \item \textbf{Miglioramento continuo:} Analisi dei processi per identificare aree di miglioramento e ottimizzare l’efficienza e la qualità.
\end{itemize}

\subsubsection{Pianificazione}

\subsubsubsection{Descrizione}
La pianificazione dei processi rappresenta un elemento fondamentale nell'ambito della gestione di un progetto, poiché implica l'analisi, la definizione, l'organizzazione e il controllo accurato delle varie attività che devono essere svolte per garantire il raggiungimento degli obiettivi prefissati. Si tratta di un'attività di natura strategica che non solo assicura una chiara direzione operativa, ma fornisce anche una struttura gestionale ben definita e solida, in grado di guidare il progetto attraverso tutte le fasi del suo ciclo di vita, dalla concezione iniziale fino al completamento.

\subsubsubsection{Obiettivi}
L’obiettivo principale della pianificazione è assicurare l'esecuzione efficiente ed efficace del progetto, rispettando gli obiettivi e i requisiti stabiliti. Inoltre:
\begin{itemize}
    \item Ogni membro del team deve assumere almeno una volta ciascun ruolo, favorendo crescita e collaborazione.
    \item Ridurre e prevenire i rischi affrontando le sfide in modo anticipato, permettendo al team di superare eventuali difficoltà nei tempi previsti.
\end{itemize}

\subsubsection{Assegnazione dei Ruoli}

Durante l'implementazione del progetto, i membri del team di Code7Crusaders ricopriranno ruoli distinti. Ogni ruolo comporta specifiche responsabilità:
\begin{itemize}
    \item \textbf{Responsabile:}
    \begin{itemize}
        \item Coordina il gruppo di lavoro.
        \item Pianifica e controlla le attività.
        \item Gestisce le risorse e le comunicazioni esterne.
        \item Redige il \href{https://code7crusaders.github.io/docs/RTB/documentazione_interna/glossario.html#piano-di-progetto}{Piano di Progetto\textsuperscript{G}}.
    \end{itemize}
    \item \textbf{Amministratore:}
    \begin{itemize}
        \item Gestisce l’ambiente di lavoro e le procedure.
        \item Gestisce la configurazione del prodotto.
        \item Redige le \href{https://code7crusaders.github.io/docs/RTB/documentazione_interna/glossario.html#norme-di-progetto}{Norme di Progetto\textsuperscript{G}}.
    \end{itemize}
    \item \textbf{Analista:}
    \begin{itemize}
        \item Analizza i requisiti del progetto e il dominio applicativo.
        \item Redige l’\href{https://code7crusaders.github.io/docs/RTB/documentazione_interna/glossario.html#analisi-dei-requisiti}{Analisi dei Requisiti\textsuperscript{G}}.
    \end{itemize}
    \item \textbf{Progettista:}
    \begin{itemize}
        \item Progetta l’architettura del prodotto.
        \item Prende decisioni tecniche e tecnologiche.
        \item Redige la Specifica Tecnica.
    \end{itemize}
    \item \textbf{Programmatore:}
    \begin{itemize}
        \item Scrive il codice e implementa le funzionalità richieste.
        \item Redige il Manuale Utente.
    \end{itemize}
    \item \textbf{Verificatore:}
    \begin{itemize}
        \item Verifica che il lavoro svolto sia conforme alle norme e alle specifiche.
        \item Redige il \href{https://code7crusaders.github.io/docs/RTB/documentazione_interna/glossario.html#piano-di-qualifica}{Piano di Qualifica\textsuperscript{G}}.
    \end{itemize}
\end{itemize}

\subsubsection{Ticketing}
Il gruppo code7crusaders utilizza GitHub Projects per il ticketing. La roadmap è organizzata in tre colonne: \textit{To Do}, \textit{In Progress}, e \textit{Completed}. Ogni attività è classificata in base a:
\begin{itemize}
    \item \textbf{Priorità:} Bassa, media o alta.
    \item \textbf{Dimensione:} XS, S, M, L, XL.
    \item \textbf{Stima ore:} Numero di ore necessarie per completare l’attività.
\end{itemize}

\subsubsection{Gestione dei rischi}
\subsubsubsection{Struttura dei rischi}
I rischi sono classificati in tre categorie principali:
\begin{itemize}
    \item Rischi di natura tecnologica;
    \item Rischi legati alla comunicazione;
    \item Rischi relativi alla pianificazione.
\end{itemize}

Ogni rischio è identificato tramite un codice univoco con la seguente struttura:

\[
\textbf{R[Categoria][Indice] - [Nome]}
\]

Dove:
\begin{itemize}
    \item \textbf{Categoria}: indica il tipo di rischio e può assumere i seguenti valori:
    \begin{itemize}
        \item \textbf{T}: per i rischi tecnologici;
        \item \textbf{C}: per i rischi comunicativi;
        \item \textbf{P}: per i rischi di pianificazione.
    \end{itemize}
    \item \textbf{Indice}: un identificatore progressivo univoco all'interno della categoria di appartenenza;
    \item \textbf{Nome}: il nome descrittivo del rischio.
\end{itemize}

\subsection{Procedure Comunicative}

\subsubsection{Comunicazioni Asincrone}
Per assicurare una comunicazione efficace in modalità asincrona, sono stati identificati strumenti specifici per le interazioni interne al team e quelle con soggetti esterni. Di seguito si descrivono le piattaforme utilizzate in ciascun contesto.
\begin{itemize}
    \item \textbf{Interne:} Il gruppo 7Crusaders detiene un gruppo Whatsapp come canale di comunicazione asincrono, consentendo comunicazione semplice e veloce.
    \item \textbf{Esterne:} Gestite tramite e-mail e tramite la piattaforma Telegram.
\end{itemize}

\subsubsection{Comunicazioni Sincrone}
Per garantire un'efficace gestione della comunicazione, è fondamentale distinguere tra le modalità di interazione sincrona utilizzate internamente al team e quelle adottate per le relazioni esterne con l'azienda Ergon. Di seguito si riportano le piattaforme selezionate per ciascun contesto.
\begin{itemize}
    \item \textbf{Interne:} Viene adottata la piattaforma Discord per la velocità nell'effettuare riunioni in chiamata vocale.
    \item \textbf{Esterne:} L'azienda Ergon adotta Zoom come piattaforma di riunioni esterne.
\end{itemize}

\subsubsection{Riunioni Interne}
Le riunioni interne del gruppo Code7Crusaders si tengono ogni venerdì, utilizzando Discord come piattaforma di comunicazione. Questi incontri servono principalmente per monitorare i progressi delle attività in corso, discutere eventuali difficoltà riscontrate e pianificare i passi successivi. L’orario delle riunioni è fissato dalle 15:00 alle 16:00, salvo necessità particolari che richiedano un adattamento.
Nel caso in cui un membro non possa partecipare, è tenuto a informare tempestivamente il resto del team. Se necessario, i membri assenti potranno recuperare le informazioni rilevanti consultando il verbale della riunione.
Durante queste riunioni, il team lavora in modo collaborativo e dinamico: ogni membro condivide lo stato delle proprie attività, proponendo soluzioni a eventuali problemi emersi. L'approccio informale permette di discutere liberamente idee, priorità e obiettivi futuri, favorendo una comunicazione aperta ed efficace.

\subsubsection{Riunioni Esterne}
Durante lo sviluppo del progetto, è fondamentale organizzare incontri periodici con i committenti o con la proponente per valutare lo stato di avanzamento del lavoro, risolvere eventuali dubbi e discutere questioni rilevanti. La pianificazione e la gestione di questi incontri sono affidate al responsabile, che si occupa di convocarli e garantirne un’efficace organizzazione.
Il responsabile ha anche il compito di presentare i punti principali della discussione alla proponente o ai committenti, coinvolgendo i membri del gruppo direttamente interessati in base agli argomenti trattati. Questo metodo garantisce una comunicazione chiara e mirata, evitando dispersioni di tempo e favorendo la comprensione reciproca.
La partecipazione alle riunioni è considerata una priorità per tutti i membri del gruppo. Ogni membro si impegna a riorganizzare i propri impegni, quando possibile, per assicurare una presenza costante. Nel caso in cui un membro sia impossibilitato a partecipare per cause inderogabili, il responsabile si farà carico di informare prontamente i committenti o la proponente e, se necessario, proporrà il rinvio dell’incontro a una data più adeguata. Durante ogni incontro, sarà inoltre garantita una registrazione accurata delle informazioni discusse, per mantenerne traccia e facilitarne il recupero in futuro.

\subsection{Formazione}
\subsubsection{Introduzione}
La formazione è una componente essenziale per garantire che tutti i membri del team siano adeguatamente preparati per affrontare le sfide tecniche e gestionali del progetto. Questo processo si concentra sullo sviluppo delle competenze e delle conoscenze necessarie per utilizzare strumenti, tecnologie e metodologie specifiche del progetto, promuovendo così l’efficienza e la qualità del lavoro svolto.

\subsubsection{Metodo di formazione}

\textbf{Individuale:}
Ogni individuo del team dovrà compiere un processo di autoformazione per riuscire a svolgere al meglio il ruolo assegnato. La rotazione dei ruoli permetterà al nuovo occupante di un ruolo di apprendere le competenze necessarie da chi lo ha precedentemente svolto, nel caso avesse delle lacune. Questo metodo permette di avere una formazione continua e di garantire che ogni membro del team sia in grado di svolgere ogni ruolo.
\textbf{Di gruppo:}
Oltre alla formazione individuale, il team parteciperà a sessioni di formazione collettiva per condividere conoscenze, affrontare problematiche comuni e rafforzare la collaborazione. Questi incontri includeranno workshop, presentazioni tecniche e sessioni di revisione del lavoro, mirate a uniformare le competenze tra i membri del gruppo. La formazione di gruppo consente inoltre di sviluppare una visione condivisa del progetto e di affrontare sfide complesse in modo coordinato, promuovendo il lavoro di squadra e la comunicazione efficace.


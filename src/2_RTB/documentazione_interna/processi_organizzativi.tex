\section{Processi Organizzativi}

\subsection{Gestione dei Processi}

\subsubsection{Introduzione}
La gestione dei processi rappresenta una fase cruciale per il successo di un progetto, garantendo che venga completato in conformità agli obiettivi e ai requisiti predefiniti. Questa fase si concentra sulla pianificazione, organizzazione, monitoraggio e controllo delle attività coinvolte nel ciclo di vita del software, assicurando che il lavoro svolto rispetti gli standard di qualità e soddisfi le aspettative del cliente. 

Le principali attività di gestione dei processi sono le seguenti:
\begin{itemize}
    \item \textbf{Definizione dei processi:} Documentazione dei processi chiave adottati nel progetto, inclusi quelli relativi allo sviluppo del software, controllo di versione, gestione dei cambiamenti e assicurazione della qualità.
    \item \textbf{Pianificazione dei processi:} Definizione degli obiettivi del progetto, delle fasi, delle risorse necessarie e delle scadenze. In questa fase vengono stabiliti i criteri di successo e redatto un piano di lavoro dettagliato.
    \item \textbf{Assegnazione delle risorse:} Allocazione dei membri del team alle attività specifiche, tenendo conto delle loro competenze e disponibilità.
    \item \textbf{Monitoraggio e controllo:} Controllo continuo dei progressi rispetto al piano stabilito, comprendente tempi, costi e qualità, oltre a gestione dei rischi.
    \item \textbf{Gestione dei cambiamenti:} Valutazione e gestione delle modifiche ai requisiti, alla pianificazione o alla distribuzione delle risorse.
    \item \textbf{Assicurazione della qualità:} Implementazione di procedure per garantire che il software soddisfi i requisiti e le aspettative del cliente.
    \item \textbf{Comunicazione e coordinamento:} Facilitazione della comunicazione tra membri del team e stakeholder per mantenere tutte le parti informate sullo stato del progetto.
    \item \textbf{Miglioramento continuo:} Analisi dei processi per identificare aree di miglioramento e ottimizzare l’efficienza e la qualità.
\end{itemize}

\subsubsection{Pianificazione}

\paragraph{Descrizione}
La pianificazione dei processi consiste nell'identificare, organizzare e controllare le attività necessarie per il successo del progetto. È un'attività strategica che garantisce una direzione chiara e una solida struttura gestionale lungo il ciclo di vita del progetto.

\paragraph{Obiettivi}
L’obiettivo principale della pianificazione è assicurare l'esecuzione efficiente ed efficace del progetto, rispettando gli obiettivi e i requisiti stabiliti. Inoltre:
\begin{itemize}
    \item Ogni membro del team deve assumere almeno una volta ciascun ruolo, favorendo crescita e collaborazione.
    \item Ridurre i rischi e affrontare le sfide in modo anticipato, permettendo al team di superare eventuali difficoltà.
\end{itemize}

\subsubsection{Assegnazione dei Ruoli}

Durante l'implementazione del progetto, i membri del team di 7Last ricopriranno ruoli distinti. Ogni ruolo comporta specifiche responsabilità:
\begin{itemize}
    \item \textbf{Responsabile:}
    \begin{itemize}
        \item Coordina il gruppo di lavoro.
        \item Pianifica e controlla le attività.
        \item Gestisce le risorse e le comunicazioni esterne.
        \item Redige il Piano di Progetto.
    \end{itemize}
    \item \textbf{Amministratore:}
    \begin{itemize}
        \item Gestisce l’ambiente di lavoro e le procedure.
        \item Gestisce la configurazione del prodotto.
        \item Redige le Norme di Progetto.
    \end{itemize}
    \item \textbf{Analista:}
    \begin{itemize}
        \item Analizza i requisiti del progetto e il dominio applicativo.
        \item Redige l’Analisi dei Requisiti.
    \end{itemize}
    \item \textbf{Progettista:}
    \begin{itemize}
        \item Progetta l’architettura del prodotto.
        \item Prende decisioni tecniche e tecnologiche.
        \item Redige la Specifica Tecnica.
    \end{itemize}
    \item \textbf{Programmatore:}
    \begin{itemize}
        \item Scrive il codice e implementa le funzionalità richieste.
        \item Redige il Manuale Utente.
    \end{itemize}
    \item \textbf{Verificatore:}
    \begin{itemize}
        \item Verifica che il lavoro svolto sia conforme alle norme e alle specifiche.
        \item Redige il Piano di Qualità.
    \end{itemize}
\end{itemize}

\subsubsection{Ticketing}
Il gruppo code7crusaders utilizza GitHub Projects per il ticketing. La roadmap è organizzata in tre colonne: \textit{To Do}, \textit{In Progress}, e \textit{Completed}. Ogni attività è classificata in base a:
\begin{itemize}
    \item \textbf{Priorità:} Bassa, media o alta.
    \item \textbf{Dimensione:} XS, S, M, L, XL.
    \item \textbf{Stima ore:} Numero di ore necessarie per completare l’attività.
\end{itemize}

\subsection{Procedure Comunicative}

\subsubsection{Comunicazioni Asincrone}
\begin{itemize}
    \item \textbf{Interne:} Gestite tramite WhatsApp.
    \item \textbf{Esterne:} Gestite tramite Telegram.
\end{itemize}

\subsubsection{Comunicazioni Sincrone}
\begin{itemize}
    \item \textbf{Interne:} Gestite tramite Discord.
    \item \textbf{Esterne:} Gestite tramite Zoom.
\end{itemize}

\subsubsection{Riunioni Interne}
DA fare
\subsubsection{Riunioni Esterne}
DA fare

\subsection{Metriche e Miglioramento}

\subsubsection{Metriche}
DA fare

\subsubsection{Miglioramento}
DA fare

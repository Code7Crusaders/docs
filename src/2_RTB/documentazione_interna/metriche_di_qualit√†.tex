\section{Metriche di qualità}
La qualità di processo è un criterio fondamentale ed è alla base di ogni prodotto che rispecchi lo stato dell'arte.
Per raggiungere tale obiettivo è necessario sfruttare delle pratiche rigorose che consentano lo svolgimento di ogni attività in maniera ottimale.
Valutando nel miglior modo possibile la qualità del prodotto e l'efficacia dei processi, sono state definite delle metriche riportate di seguito.
Lo scopo di questa sezione è quello di identificarne i parametri che le metriche devono rispettare per essere considerate accettabili o ottime. Esse sono state suddivise utilizzando lo standard \emph{ISO/IEC 12207:1995},
il quale separa i processi di ciclo di vita del Software in tre categorie:
\begin{enumerate}
    \item Processi di base e/o primari;
    \item Processi di supporto;
    \item Processi organizzativi;
\end{enumerate}

\subsection{Processi di base e/o primari}
\subsubsection{Fornitura}
Nella fase di \emph{Fornitura} si definiscono le procedure e le risorse necessarie per la consegna del prodotto.
Definiamo quindi le seguenti metriche:


\paragraph*{1PBM-PV Planed Value}
\begin{itemize}
    \item \textbf{Definizione:} il Planned Value (Valore Pianificato) rappresenta il valore del lavoro programmato per essere completato fino a un determinato momento. Si tratta del budget preventivato per lo sprint in corso.
    \item \textbf{Come si calcola:}
\end{itemize}
\begin{center}
   $PV = BAC \times LP$ 
\end{center}
dove:
\begin{itemize}[label=$\rightarrow$]
    \item $BAC$: Budget At Completion
    \item $LP$: Percentuale Lavoro Pianificato
\end{itemize}
\begin{itemize}
    \item \textbf{Valore ammissibile:}
\end{itemize}
\begin{center}
    $PV \geq 0$
\end{center}
\begin{itemize}
    \item \textbf{Valore ottimo:}
\end{itemize}
\begin{center}
    $PV \leq BAC$
\end{center}

\paragraph*{2PBM-ETC Estimated to complete}
\begin{itemize}
    \item \textbf{Definizione:} l’Estimate to Complete (o Stima al Completamento) rappresenta una previsione del costo necessario per completare le attività rimanenti del progetto basata sulle performance attuali.
    \item \textbf{Come si calcola:}
\end{itemize}
\begin{center}
   $ETC = BAC - EV$ 
\end{center}
dove:
\begin{itemize}[label=$\rightarrow$]
    \item $BAC$: Budget At Completion
    \item $EV$: Earned Value
\end{itemize}
\begin{itemize}
    \item \textbf{Valore ammissibile:}
\end{itemize}
\begin{center}
    $ETC \geq 0$
\end{center}
\begin{itemize}
    \item \textbf{Valore ottimo:}
\end{itemize}
\begin{center}
    $ETC \leq EAC$
\end{center}

\paragraph*{3PBM-EAC Estimate at Completition}
\begin{itemize}
    \item \textbf{Definizione:} l’Estimate at Completion (o Stima da Completare) rappresenta una previsione aggiornata del costo totale del progetto basata sulle performance attuali, calcolata in base ai costi effettivamente sostenuti e ai costi stimati per completare il lavoro rimanente.
    \item \textbf{Come si calcola:}
\end{itemize}
\begin{center}
   $EAC = AC + (BAC - EV)$ 
\end{center}
dove:
\begin{itemize}[label=$\rightarrow$]
    \item $BAC$: Budget At Completion
    \item $EV$: Earned Value
    \item $AC$: Actual Cost
\end{itemize}
\begin{itemize}
    \item \textbf{Valore ammissibile:}
\end{itemize}
\begin{center}
    $EAC \leq BAC + 10\%$
\end{center}
\begin{itemize}
    \item \textbf{Valore ottimo:}
\end{itemize}
\begin{center}
    $EAC \leq BAC$
\end{center}

\paragraph*{4PBM-EV Earned Value}
\begin{itemize}
    \item \textbf{Definizione:} l’Earned Value (Valore Guadagnato) rappresenta il valore del lavoro effettivamente completato fino al periodo in analisi.
    \item \textbf{Come si calcola:}
\end{itemize}
\begin{center}
   $EV = AC \times LC$ 
\end{center}
dove:
\begin{itemize}[label=$\rightarrow$]
    \item $LC$: Percentuale di lavoro completato
    \item $AC$: Actual Cost
\end{itemize}
\begin{itemize}
    \item \textbf{Valore ammissibile:}
\end{itemize}
\begin{center}
    $EV \geq 0$
\end{center}
\begin{itemize}
    \item \textbf{Valore ottimo:}
\end{itemize}
\begin{center}
    $EV \leq EAC$
\end{center}

\paragraph*{5PBM-AC}
\begin{itemize}
    \item \textbf{Definizione:} l’Actual Cost (Costo Effettivo) rappresenta il costo effettivamente sostenuto per completare il lavoro fino al periodo in analisi.
    \item \textbf{Come si calcola:} Si ottiene sommando tutti i costi effettivi sostenuti fino a quella data.
\end{itemize}
\begin{itemize}
    \item \textbf{Valore ammissibile:}
\end{itemize}
\begin{center}
    $AC \geq 0$
\end{center}
\begin{itemize}
    \item \textbf{Valore ottimo:}
\end{itemize}
\begin{center}
    $AC \leq EAC$
\end{center}

\paragraph*{6PBM-SV Scheduled Variance}
\begin{itemize}
    \item \textbf{Definizione:} la Schedule Variance (o Variazione di Programma) rappresenta la differenza tra il valore del lavoro effettivamente completato e il valore del lavoro pianificato, calcolata in percentuale.
    \item \textbf{Come si calcola:}
\end{itemize}
\begin{center}
   $SV = (EV-PV)/EV$ 
\end{center}
dove:
\begin{itemize}[label=$\rightarrow$]
    \item $EV$: Earned Value
    \item $PV$: Planned Value
\end{itemize}
\begin{itemize}
    \item \textbf{Valore ammissibile:}
\end{itemize}
\begin{center}
    $SV \geq -10\%$
\end{center}
\begin{itemize}
    \item \textbf{Valore ottimo:}
\end{itemize}
\begin{center}
    $SV \geq 0\%$
\end{center}

\paragraph*{7PBM-CV Cost Variance}
\begin{itemize}
    \item \textbf{Definizione:} la Cost Variance (o Variazione dei Costi) rappresenta la differenza tra il valore del lavoro effettivamente completato e il costo effettivamente sostenuto per completarlo, calcolata in percentuale
    \item \textbf{Come si calcola:}
\end{itemize}
\begin{center}
   $CV = (EV - AC)/EV$ 
\end{center}
dove:
\begin{itemize}[label=$\rightarrow$]
    \item $EV$: Earned Value
    \item $AC$: Actual Cost
\end{itemize}
\begin{itemize}
    \item \textbf{Valore ammissibile:}
\end{itemize}
\begin{center}
    $CV \geq -10\%$
\end{center}
\begin{itemize}
    \item \textbf{Valore ottimo:}
\end{itemize}
\begin{center}
    $CV \geq 0\%$
\end{center}

\paragraph*{8PBM-CPI Cost Performance Index}
\begin{itemize}
    \item \textbf{Definizione:} il Cost Performance Index rappresenta il rapporto tra il valore del lavoro effettivamente completato e i costi sostenuti per completarlo.
    \item \textbf{Come si calcola:}
\end{itemize}
\begin{center}
   $CPI = EV/AC$ 
\end{center}
dove:
\begin{itemize}[label=$\rightarrow$]
    \item $EV$: Earned Value
    \item $AC$: Actual Cost
\end{itemize}
\begin{itemize}
    \item \textbf{Valore ammissibile:}
\end{itemize}
\begin{center}
    $CPI \geq 0.8$
\end{center}
\begin{itemize}
    \item \textbf{Valore ottimo:}
\end{itemize}
\begin{center}
    $CPI \geq 1$
\end{center}

\paragraph*{9PBM-SPI Scheduled Performance Index}
\begin{itemize}
    \item \textbf{Definizione:} lo Schedule Performance Index rappresenta l’efficienza con cui il progetto sta rispettando il programma.
    \item \textbf{Come si calcola:}
\end{itemize}
\begin{center}
   $SPI = EV/PV$ 
\end{center}
dove:
\begin{itemize}[label=$\rightarrow$]
    \item $EV$: Earned Value
    \item $PV$: Planned Value
\end{itemize}
\begin{itemize}
    \item \textbf{Valore ammissibile:}
\end{itemize}
\begin{center}
    $SPI \geq 0.8$
\end{center}
\begin{itemize}
    \item \textbf{Valore ottimo:}
\end{itemize}
\begin{center}
    $SPI \geq 1$
\end{center}

\paragraph*{10PBM-OTDR On-Time Delivery Rate}
\begin{itemize}
    \item \textbf{Definizione:} l’On-Time Delivery Rate (o Tasso di Consegna nei Tempi) rappresenta la percentuale di attività completate entro la data di scadenza.
    \item \textbf{Come si calcola:}
\end{itemize}
\begin{center}
   $OTDR = AP/AT$ 
\end{center}
dove:
\begin{itemize}[label=$\rightarrow$]
    \item $AP$: Attività completate entro la data di scadenza
    \item $AT$: Attività Totali
\end{itemize}
\begin{itemize}
    \item \textbf{Valore ammissibile:}
\end{itemize}
\begin{center}
    $OTDR \geq 90\%$
\end{center}
\begin{itemize}
    \item \textbf{Valore ottimo:}
\end{itemize}
\begin{center}
    $OTDR \geq 95\%$
\end{center}

\subsubsection{Sviluppo}
Nella fase di \emph{sviluppo} si realizza il prodotto software, seguendo le specifiche definite in fase di progettazione.
\paragraph{Analisi dei Requisiti}
Questa fase consiste nell'esaminare le richieste del proponente e nel definire i requisiti che il prodotto dovrà soddisfare. Definiamo quindi le seguenti metriche:


\paragraph{Progettazione}
Questa fase consiste nel definire l'architettura  del prodotto software, in modo da soddisfare i requisiti in fase di analisi. Definiamo quindi le seguenti metriche:

\paragraph{Codifica}
Queste metriche aiutano a valutare la qualità del codice, la complessità e la manutenibilità

\subsection{Processi di Supporto}
I \emph{processi di supporto} si affiancano ai processi primari per garantire il corretto svolgimento delle attività.
\subsubsection{Documentazione}
La \emph{Documentazione} è un aspetto fondamentale per la comprensione del prodotto e per la sua manutenibilità. A livello pratico consiste nella redazione di manuali e documenti tecnici
che descrivano il funzionamento del prodotto e le scelte progettuali adottate. Per valutare la qualità di tale processo, sono state definite le seguenti metriche:

\subsubsection{Gestione della Qualità}
La \emph{gestione della qualità} è un processo che si occupa di definire una metodologia per garantire la qualità del prodotto. Per valutare la qualità di tale processo, sono state definite le seguenti metriche.

\subsubsection{Verifica}
La \emph{verifica} è un processo che si occupa di controllare che il prodotto soddisfi i requisiti
stabiliti e sia pienamente funzionante. Per valutare la qualità di tale processo, sono state
definite le seguenti metriche.

\subsubsection{Risoluzione dei problemi}
La \emph{risoluzione dei problemi} è un processo che mira a identificare, analizzare e risolvere le
varie problematiche che possono emergere durante lo sviluppo. La gestione dei rischi,
in particolare, si occupa di identificare, analizzare e gestire i rischi che possono insorgere
durante lo svolgimento del progetto. Per valutare la qualità di tale processo, sono state
definite le seguenti metriche.

\subsection{Processi organizzativi}
I \emph{Processi organizzativi} sono processi che si occupano di definire le linee guida e le procedure da seguire per garantire un'efficace gestione e coordinazione del progetto.

\subsubsection{Pianificazione}
La \emph{Pianificazione} è un processo che si occupa di definire le attività da svolgere e le risorse temporali e umane necessarie per il loro svolgimento. Per valutare la qualità di tale processo sono state definite le seguenti metriche:

\section{Processi primari}
% Suddivoso in due parti:
% A: Fornitura
% B: Sviluppo


%--------------PARTE A: FORNITURA------------------------------------------------------------------------------------------
\subsection{Fornitura}
\subsubsection{Introduzione}
Il processo di fornitura rappresenta un percorso ben definito che stabilisce un contratto tra fornitore e cliente, 
accompagnando la creazione e la consegna del software. 
Fondamentale per garantire che il software risponda ai requisiti del cliente, rispetti i tempi e i costi, 
e soddisfi gli standard di qualità, il processo include anche un continuo dialogo tra le parti per chiarire le necessità, 
risolvere eventuali difficoltà tecniche e stabilire le basi per il corretto sviluppo del prodotto, 
attraverso un'accurata definizione dei requisiti e dei vincoli tecnologici.

Il processo di fornitura si articola nelle seguenti fasi principali:

\begin{enumerate}
    \item \textbf{Preparazione della proposta}  
    Questa fase iniziale si concentra sulla raccolta delle informazioni necessarie e sulla stesura di una proposta formale per il cliente. Include:  
    \begin{itemize}
        \item Analisi delle esigenze del cliente.  
        \item Studio di fattibilità.  
        \item Elaborazione della proposta di candidatura.  
    \end{itemize}
    
    \item \textbf{Pianificazione}  
    Qui si stabilisce l’organizzazione e la programmazione delle attività del progetto, con particolare attenzione a:  
    \begin{itemize}
        \item Definizione delle milestone.  
        \item Creazione del piano di progetto.  
        \item Assegnazione di compiti e risorse.  
    \end{itemize}
    
    \item \textbf{Esecuzione}  
    Durante questa fase si procede con la realizzazione pratica del progetto, che comprende:  
    \begin{itemize}
        \item Sviluppo del software.  
        \item Test e verifiche.  
        \item Preparazione della documentazione.  
    \end{itemize}
    
    \item \textbf{Revisione}  
    Questa fase consiste nel valutare approfonditamente il lavoro svolto per verificarne la conformità agli s
    tandard di qualità e ai requisiti contrattuali. Le attività principali sono:  
    \begin{itemize}
        \item Revisione del codice.  
        \item Esecuzione dei test di accettazione.  
        \item Risoluzione di eventuali discrepanze.  
    \end{itemize}
    
    \item \textbf{Consegna}  
    Infine, il prodotto finale viene consegnato al cliente. Questa fase comprende:  
    \begin{itemize}
        \item Consegna del software.  
        \item Formazione del personale.  
    \end{itemize}
\end{enumerate}

\subsubsection{Contatti con l’azienda proponente}
7Crusaders dispone di un indirizzo email(code7crusaders@gmail.com) e un canale Discord per le riunioni telematiche. 
Gli incontri online si svolgeranno settimanalmente, con la possibilità di pianificare riunioni aggiuntive su richiesta del team.
Ad ogni incontro settimanale verrà redatto un verbale che riporterà gli argomenti discussi e le scelte intraprese.
Per ogni meeting con l’azienda proponente sarà preparato un verbale che riepilogherà i punti principali discussi. 
Tutti i verbali interni e esterni per discussioni durante lo svolgimento dell'RTB saranno accessibili al seguente link: \url{https://code7crusaders.github.io/docs/RTB/index.html}.
Inoltre per una comunicazione più rapida e informale, il team e l'azienda utilizzeranno Telegram.



\subsubsection{Piano di Progetto}
Il \textbf{Piano di Progetto v1.0}, redatto dal Responsabile, descrive in dettaglio il processo di sviluppo del progetto. Esso funge da guida fondamentale per il team al fine di:
\begin{itemize}
    \item mantenere l’allineamento con gli obiettivi;
    \item gestire le risorse in modo efficace;
    \item affrontare e mitigare eventuali criticità durante le varie fasi.
\end{itemize}

Si articola nelle seguenti sezioni:
\begin{itemize}
    \item \textbf{Analisi dei rischi}: identifica, valuta e gestisce i rischi potenziali che possono influenzare il successo del progetto. I rischi sono classificati in:
    \begin{itemize}
        \item tecnologici;
        \item di comunicazione;
        \item di pianificazione.
    \end{itemize}
    Per ciascun rischio vengono definiti segnali di manifestazione, probabilità, impatto e strategie di mitigazione.

    \item \textbf{Modello di sviluppo}: descrive l’approccio metodologico scelto, nel nostro caso il framework \textit{agile Scrum}. Include:
    \begin{itemize}
        \item gli eventi principali del framework;
        \item le pratiche adottate dal team.
    \end{itemize}

    \item \textbf{Pianificazione}: include una roadmap dettagliata che descrive:
    \begin{itemize}
        \item le attività necessarie per raggiungere gli obiettivi di ogni sprint;
        \item la distribuzione temporale delle risorse.
    \end{itemize}

    \item \textbf{Preventivo}: fornisce una stima delle ore produttive disponibili, distribuite tra:
    \begin{itemize}
        \item i ruoli assegnati ai membri del team;
        \item ogni sprint pianificato.
    \end{itemize}
    Inoltre, include il costo stimato di ogni sprint.

    \item \textbf{Consuntivo}: analizza a posteriori la ripartizione effettiva delle ore e dei costi. Contiene:
    \begin{itemize}
        \item una retrospettiva sulle discrepanze rispetto al preventivo;
        \item eventuali miglioramenti nella pianificazione futura.
    \end{itemize}
\end{itemize}

\subsubsection{Piano di Qualifica}
Il \textbf{Piano di Qualifica v1.0}, redatto dall’Amministratore, descrive le strategie e gli approcci adottati per garantire la qualità del prodotto o servizio sviluppato. Si compone delle seguenti sezioni:
\begin{itemize}
    \item \textbf{Qualità di processo}: specifica gli standard e le procedure seguite per garantire la qualità dei processi di sviluppo. Include:
    \begin{itemize}
        \item metodologie utilizzate;
        \item criteri per la misurazione e il miglioramento dei processi.
    \end{itemize}

    \item \textbf{Qualità di prodotto}: descrive gli standard e le specifiche che il prodotto deve soddisfare per essere considerato di qualità. Include:
    \begin{itemize}
        \item metriche e criteri di valutazione;
        \item specifiche tecniche richieste.
    \end{itemize}

    \item \textbf{Specifiche dei test}: fornisce una descrizione dettagliata dei test pianificati durante lo sviluppo per verificare che i requisiti siano soddisfatti.

    \item \textbf{Cruscotto delle metriche}: presenta un resoconto delle attività di valutazione svolte, utile per:
    \begin{itemize}
        \item monitorare l’andamento del progetto rispetto agli obiettivi prefissati;
        \item identificare azioni correttive necessarie.
    \end{itemize}
\end{itemize}
\subsubsection{Glossario}
Il Glossario rappresenta un riferimento completo che raccoglie e definisce i termini tecnici utilizzati nel progetto. Questo documento garantisce una comprensione uniforme della terminologia specifica del settore, riducendo il rischio di equivoci e favorendo una comunicazione chiara. Inoltre, contribuisce a migliorare la coerenza e la qualità della documentazione prodotta dal team.

\subsubsection{Strumenti}
Di seguito sono elencati gli strumenti software utilizzati nel processo di fornitura:
\begin{itemize}
    \item \textbf{Discord}: piattaforma utilizzata per le riunioni interne.
    \item \textbf{Google Meet}: utilizzato per le riunioni formali online con l'azienda proponente.
    \item \textbf{Telegram}: piattaforma utilizzata come metodo informale per comunicare con l'azienda proponente.
    \item \textbf{LaTeX}: sistema per la creazione di documenti e slide di presentazione.
    \item \textbf{GitHubProject}: sistema di ticketing e roadmap integrato nella piattaforma di GitHub. 
\end{itemize}








%--------------PARTE B: SVILUPPO------------------------------------------------------------------------------------------

\subsection{Sviluppo}
\subsubsection{Introduzione}

AAAAAAAAAAAAA
\subsubsection{Analisi dei Requisiti}

\subsubsubsection{Descrizione}
Il documento di analisi dei requisiti ha l'obiettivo di identificare,
descrivere e documentare in modo completo le necessità, 
le funzionalità e le prestazioni che il sistema software deve garantire. 
Questo rappresenta il fondamento del processo di sviluppo del software, 
fornendo un riferimento chiaro e dettagliato per la progettazione.

Attraverso questa attività, si mira a comprendere a fondo le esigenze degli stakeholder, 
come utenti finali e clienti, assicurandosi che il prodotto software finale risponda 
pienamente alle loro aspettative e necessità. L’analisi dei requisiti comprende 
generalmente la raccolta e la documentazione dei requisiti funzionali, 
qualitativi e vincolanti, la definizione dei casi d’uso, 
nonché la prioritizzazione e la tracciabilità dei requisiti lungo tutto il ciclo
di vita del software.

\subsubsubsection{Documento}
Gli analisti hanno il compito di redigere l’Analisi dei Requisiti, comprendendo le seguenti
sezioni:

\begin{itemize}
    \item \textbf{Introduzione}:  
    Presenta lo scopo del documento, fornendo una base per la progettazione, implementazione e verifica del sistema.

    \item \textbf{Descrizione del prodotto}:  
    \begin{itemize}
        \item \textbf{Obiettivi del prodotto}: principali scopi operativi del sistema.
        \item \textbf{Architettura del prodotto}: panoramica dei componenti e delle loro interazioni.
        \item \textbf{Funzionalità del prodotto}: capacità operative principali.
        \item \textbf{Caratteristiche degli utenti}: profili degli utenti e relative esigenze.
    \end{itemize}

    \item \textbf{Casi d’uso}:  
    \begin{itemize}
        \item \textbf{Elenco dei casi d’uso}: scenari principali che descrivono le funzionalità.
    \end{itemize}

    \item \textbf{Requisiti}:  
    \begin{itemize}
        \item \textbf{Requisiti funzionali}: operazioni e servizi che il sistema deve offrire.
        \item \textbf{Requisiti qualitativi}: standard di qualità come prestazioni e usabilità.
        \item \textbf{Requisiti di vincolo}: limitazioni esterne e tecniche.
    \end{itemize}
\end{itemize}


\subsubsubsection{Codifica dei casi d'uso}
I casi d'uso sono codificati utilizzando la seguente notazione:

\begin{itemize}
    \item \textbf{UC[ID-Principale][ID-Sottocaso]}: Identificativo univoco del caso d'uso, composto da un ID principale che identifica il caso principale e, se necessario, da un ID del sottocaso.
    \item \textbf{Titolo}: Breve descrizione del caso d'uso.
    \item \textbf{Attori}: Elenco degli attori coinvolti nel caso d'uso.
    \item \textbf{Precondizioni}: Condizioni che devono essere vere prima che il caso d'uso possa iniziare.
    \item \textbf{Postcondizioni}: Condizioni che devono essere vere dopo che il caso d'uso è stato completato con successo.
    \item \textbf{Scenario principale}: Descrizione dettagliata del flusso di eventi principale del caso d'uso.
    \item \textbf{Generalizzaioni}: Eventuali casi d'uso generalizzati.
    \item \textbf{Estensioni}: Eventuali casi d'uso estesi.
\end{itemize}

\subsubsubsection{Diagrammi Casi D'uso}
I diagrammi dei casi d’uso rappresentano visivamente le interazioni tra attori e sistema,
illustrando i vari scenari di utilizzo. Ogni caso d’uso descrive una sequenza di 
azioni necessarie per raggiungere un obiettivo specifico, aiutando a identificare i 
requisiti funzionali e a chiarire le aspettative degli utenti. Questi diagrammi 
facilitano la comunicazione tra sviluppatori e stakeholder, garantendo che tutte le 
funzionalità richieste siano considerate e implementate correttamente. 
Di seguito sono elencati i principali componenti di un diagramma dei casi d’uso.

\begin{itemize}
    \item \textbf{Attori}: 
    \item \textbf{Sistema}: 
    \item \textbf{Casi d’uso}:
    \item \textbf{Sottocasi d’uso}: 
    \item \textbf{Relazioni tra Attori e Casi d’Uso}:
    \item \textbf{Relazioni tra Attori}:
    \item \textbf{Relazioni tra Casi d’Uso}:
\end{itemize}

\subsubsubsection{Requisiti}

I requisiti sono classificati in tre categorie principali:  
\begin{itemize}
    \item \textbf{Funzionali}: riguardano l'usabilità del prodotto finale;  
    \item \textbf{Di qualità}: includono gli strumenti e la documentazione da fornire;  
    \item \textbf{Di vincolo}: fanno riferimento alle tecnologie da utilizzare.
\end{itemize}
Ciascun requisito è indicato da:
\begin{itemize}
    \item \textbf{Codice Identificativo}: codice univoco che identifica il requisito;
    \item \textbf{Descrizione}: breve spiegazione del requisito;
    \item \textbf{Fonte}: origine del requisito (es. capitolato, interno, ecc..);
    \item \textbf{Priorità}: importanza del requisito rispetto agli altri;
\end{itemize} 

\subsubsubsection{Fonti dei requisiti}
I requisiti sono stati identificati a partire dalle seguenti fonti:
\begin{itemize}
    \item \textbf{Capitolato}: Requisiti individuati tramite analisi del capitolato;
    \item \textbf{interno}: requisiti individuati durante riunioni interne al gruppo di lavoro;
    \item \textbf{Esterno}: requisiti individuati durante incontri con il proponente;
    \item \textbf{Piano di Qualifica}: Requisiti necessari per rispettare standard di qualità definiti nel documento Piano di Qualifica;
    \item \textbf{Norme di Progetto}: Requisiti necessari per rispettare le norme di progetto definite nel documento Norme di Progetto;
\end{itemize}

\subsubsubsection{Codifica dei requisiti}
I requisiti sono codificati come segue: \textbf{R[Tipo][Importanza][Numero]}
\newline
Dove \textbf{Tipo} può essere:
\begin{itemize}
    \item \textbf{F (funzionale)}
    \item \textbf{Q (di qualità)}
    \item \textbf{V (di vincolo)}
\end{itemize}
\textbf{Importanza} può essere:
\begin{itemize}
    \item \textbf{O (obbligatorio)}
    \item \textbf{D (desiderabile)}
    \item \textbf{F (facoltativo )}
\end{itemize}
\textbf{Numero} è un numero identificativo univoco del requisito.


\subsubsection{Progettazione}
\subsubsection{Codifica}

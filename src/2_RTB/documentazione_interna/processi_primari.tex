\section{Processi primari}
% Suddivoso in due parti:
% A: Fornitura
% B: Sviluppo


%--------------PARTE A: FORNITURA------------------------------------------------------------------------------------------
\subsection{Fornitura}
\subsubsection{Introduzione}
Il processo di fornitura rappresenta un percorso ben definito che stabilisce un contratto tra fornitore e cliente, 
accompagnando la creazione e la consegna del software. 
Fondamentale per garantire che il software risponda ai requisiti del cliente, rispetti i tempi e i costi, 
e soddisfi gli standard di qualità, il processo include anche un continuo dialogo tra le parti per chiarire le necessità, 
risolvere eventuali difficoltà tecniche e stabilire le basi per il corretto sviluppo del prodotto, 
attraverso un'accurata definizione dei requisiti e dei vincoli tecnologici.

Il processo di fornitura si articola nelle seguenti fasi principali:

\begin{enumerate}
    \item \textbf{Preparazione della proposta}  
    Questa fase iniziale si concentra sulla raccolta delle informazioni necessarie e sulla stesura di una proposta formale per il cliente. Include:  
    \begin{itemize}
        \item Analisi delle esigenze del cliente.  
        \item Studio di fattibilità.  
        \item Elaborazione della proposta di candidatura.  
    \end{itemize}
    
    \item \textbf{Pianificazione}  
    Qui si stabilisce l’organizzazione e la programmazione delle attività del progetto, con particolare attenzione a:  
    \begin{itemize}
        \item Definizione delle milestone.  
        \item Creazione del piano di progetto.  
        \item Assegnazione di compiti e risorse.  
    \end{itemize}
    
    \item \textbf{Esecuzione}  
    Durante questa fase si procede con la realizzazione pratica del progetto, che comprende:  
    \begin{itemize}
        \item Sviluppo del software.  
        \item Test e verifiche.  
        \item Preparazione della documentazione.  
    \end{itemize}
    
    \item \textbf{Revisione}  
    Questa fase consiste nel valutare approfonditamente il lavoro svolto per verificarne la conformità agli s
    tandard di qualità e ai requisiti contrattuali. Le attività principali sono:  
    \begin{itemize}
        \item Revisione del codice.  
        \item Esecuzione dei test di accettazione.  
        \item Risoluzione di eventuali discrepanze.  
    \end{itemize}
    
    \item \textbf{Consegna}  
    Infine, il prodotto finale viene consegnato al cliente. Questa fase comprende:  
    \begin{itemize}
        \item Consegna del software.  
        \item Formazione del personale.  
    \end{itemize}
\end{enumerate}

\subsubsection{Contatti con l’azienda proponente}
7Crusaders dispone di un indirizzo email(code7crusaders@gmail.com) e un canale Discord per le riunioni telematiche. 
Gli incontri online si svolgeranno settimanalmente, con la possibilità di pianificare riunioni aggiuntive su richiesta del team.
Ad ogni incontro settimanale verrà redatto un verbale che riporterà gli argomenti discussi e le scelte intraprese.
Per ogni meeting con l’azienda proponente sarà preparato un verbale che riepilogherà i punti principali discussi. 
Tutti i verbali interni e esterni per discussioni durante lo svolgimento dell'RTB saranno accessibili al seguente link: \url{https://code7crusaders.github.io/docs/RTB/index.html}.
Inoltre per una comunicazione più rapida e informale, il team e l'azienda utilizzeranno Telegram.



\subsubsection{Piano di Qualifica}

\subsubsection{Piano di Progetto}

\subsubsection{Glossario}

\subsubsection{Strumenti}
Di seguito sono elencati gli strumenti software utilizzati nel processo di fornitura:
\begin{itemize}
    \item \textbf{Discord}: piattaforma utilizzata per le riunioni interne.
    \item \textbf{Google Meet}: utilizzato per le riunioni formali online con l'azienda proponente.
    \item \textbf{Telegram}: piattaforma utilizzata come metodo informale per comunicare con l'azienda proponente.
    \item \textbf{LaTeX}: sistema per la creazione di documenti e slide di presentazione.
\end{itemize}








%--------------PARTE B: SVILUPPO------------------------------------------------------------------------------------------

\subsection{Sviluppo}
\subsubsection{Introduzione}
\subsubsection{Analisi dei Requisiti}
\subsubsection{Progettazione}
\subsubsection{Codifica}
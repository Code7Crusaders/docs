\section{Processi primari}
% Suddivoso in due parti:
% A: Fornitura
% B: Sviluppo



%--------------PARTE A: FORNITURA------------------------------------------------------------------------------------------
\subsection{Fornitura}
\subsubsection{Introduzione}
Il processo di fornitura rappresenta un percorso ben definito che stabilisce un contratto tra fornitore e cliente, 
accompagnando la creazione e la consegna del software. 
Fondamentale per garantire che il software risponda ai requisiti del cliente, rispetti i tempi e i costi, 
e soddisfi gli standard di qualità, il processo include anche un continuo dialogo tra le parti per chiarire le necessità, 
risolvere eventuali difficoltà tecniche e stabilire le basi per il corretto sviluppo del prodotto, 
attraverso un'accurata definizione dei requisiti e dei vincoli tecnologici.
Il processo di fornitura si articola nelle seguenti fasi principali:

\begin{enumerate}
    \item \textbf{Preparazione della proposta}  
    Questa fase iniziale si concentra sulla raccolta delle informazioni necessarie e sulla stesura di una proposta formale per il cliente. Include:  
    \begin{itemize}
        \item Analisi delle esigenze del cliente.  
        \item Studio di fattibilità.  
        \item Elaborazione della proposta di candidatura.  
    \end{itemize}
    
    \item \textbf{Pianificazione}  
    Qui si stabilisce l’organizzazione e la programmazione delle attività del progetto, con particolare attenzione a:  
    \begin{itemize}
        \item Definizione delle milestone.  
        \item Creazione del piano di progetto.  
        \item Assegnazione di compiti e risorse.  
    \end{itemize}
    
    \item \textbf{Esecuzione}  
    Durante questa fase si procede con la realizzazione pratica del progetto, che comprende:  
    \begin{itemize}
        \item Sviluppo del software.  
        \item Test e verifiche.  
        \item Preparazione della documentazione.  
    \end{itemize}
    
    \item \textbf{Revisione}  
    Questa fase consiste nel valutare approfonditamente il lavoro svolto per verificarne la conformità agli s
    tandard di qualità e ai requisiti contrattuali. Le attività principali sono:  
    \begin{itemize}
        \item Revisione del codice.  
        \item Esecuzione dei test di accettazione.  
        \item Risoluzione di eventuali discrepanze.  
    \end{itemize}
    
    \item \textbf{Consegna}  
    Infine, il prodotto finale viene consegnato al cliente. Questa fase comprende:  
    \begin{itemize}
        \item Consegna del software.  
        \item Formazione del personale.  
    \end{itemize}
\end{enumerate}

\subsubsection{Contatti con l’azienda proponente}
7Crusaders dispone di un indirizzo email(code7crusaders@gmail.com) e un canale Discord per le riunioni telematiche. 
Gli incontri online si svolgeranno settimanalmente, con la possibilità di pianificare riunioni aggiuntive su richiesta del team.
Ad ogni incontro settimanale verrà redatto un verbale che riporterà gli argomenti discussi e le scelte intraprese.
Per ogni meeting con l’azienda proponente sarà preparato un verbale che riepilogherà i punti principali discussi. 
Tutti i verbali interni e esterni per discussioni durante lo svolgimento dell'\href{https://code7crusaders.github.io/docs/RTB/documentazione_interna/glossario.html#rtb-requirements-and-technology-baseline}{RTB\textsuperscript{G}} saranno accessibili al seguente link: \url{https://code7crusaders.github.io/docs/RTB/index.html}.
Inoltre per una comunicazione più rapida e informale, il team e l'azienda utilizzeranno Telegram.

\subsubsection{Analisi dei Requisiti}

L'\href{https://code7crusaders.github.io/docs/RTB/documentazione_interna/glossario.html#analisi-dei-requisiti}{\textbf{Analisi dei Requisiti\textsuperscript{G}} v1.0}, redatto dagli Analisti, rappresenta un documento fondamentale per lo sviluppo del sistema software. Il suo obiettivo principale è definire in dettaglio le funzionalità necessarie affinché il prodotto soddisfi pienamente le richieste della Proponente.
Il documento comprende i seguenti elementi essenziali:
\begin{itemize}
    \item \textbf{Definizione degli attori}: entità o persone che interagiscono con il sistema.
    \item \textbf{Definizione dei casi d’uso}: rappresentazione narrativa di scenari specifici che descrivono come gli attori interagiscono con il sistema. I casi d’uso offrono una visione chiara delle azioni eseguibili all’interno del sistema e delle interazioni degli utenti con esso. All’interno di ciascun caso d’uso, viene fornito:
    \begin{itemize}
        \item un elenco preciso delle azioni intraprese dall’attore per attivare il caso d’uso;
        \item una base per facilitare l’estrazione dei requisiti corrispondenti.
    \end{itemize}
    \item \textbf{Definizione di requisiti}: individuazione e categorizzazione dei requisiti in:
    \begin{itemize}
        \item \textbf{Requisiti funzionali}: specificano le operazioni che il sistema deve essere in grado di eseguire;
        \item \textbf{Requisiti di qualità}: si concentrano sulla definizione degli standard e degli attributi che il software deve possedere per garantire prestazioni, affidabilità, usabilità e sicurezza ottimali;
        \item \textbf{Requisiti di vincolo}: delineano vincoli e limitazioni che il sistema deve rispettare, includendo restrizioni tecnologiche, normative o di risorse.
    \end{itemize}
\end{itemize}

\subsubsection{Piano di Progetto}
Il \href{https://code7crusaders.github.io/docs/RTB/documentazione_interna/glossario.html#piano-di-progetto}{\textbf{Piano di Progetto\textsuperscript{G}} v1.0}, redatto dal Responsabile, descrive in dettaglio il processo di sviluppo del progetto. Esso funge da guida fondamentale per il team al fine di:
\begin{itemize}
    \item mantenere l’allineamento con gli obiettivi;
    \item gestire le risorse in modo efficace;
    \item affrontare e mitigare eventuali criticità durante le varie fasi.
\end{itemize}

Si articola nelle seguenti sezioni:
\begin{itemize}
    \item \textbf{Analisi dei rischi}: identifica, valuta e gestisce i rischi potenziali che possono influenzare il successo del progetto. I rischi sono classificati in:
    \begin{itemize}
        \item tecnologici;
        \item di comunicazione;
        \item di pianificazione.
    \end{itemize}
    Per ciascun rischio vengono definiti segnali di manifestazione, probabilità, impatto e strategie di mitigazione.

    \item \textbf{Modello di sviluppo}: descrive l’approccio metodologico scelto, nel nostro caso il framework \textit{agile Scrum}. Include:
    \begin{itemize}
        \item gli eventi principali del framework;
        \item le pratiche adottate dal team.
    \end{itemize}

    \item \textbf{Pianificazione}: include una roadmap dettagliata che descrive:
    \begin{itemize}
        \item le attività necessarie per raggiungere gli obiettivi di ogni sprint;
        \item la distribuzione temporale delle risorse.
    \end{itemize}

    \item \textbf{Preventivo}: fornisce una stima delle ore produttive disponibili, distribuite tra:
    \begin{itemize}
        \item i ruoli assegnati ai membri del team;
        \item ogni sprint pianificato.
    \end{itemize}
    Inoltre, include il costo stimato di ogni sprint.

    \item \textbf{Consuntivo}: analizza a posteriori la ripartizione effettiva delle ore e dei costi. Contiene:
    \begin{itemize}
        \item una retrospettiva sulle discrepanze rispetto al preventivo;
        \item eventuali miglioramenti nella pianificazione futura.
    \end{itemize}
\end{itemize}

\subsubsection{Piano di Qualifica}
Il \href{https://code7crusaders.github.io/docs/RTB/documentazione_interna/glossario.html#piano-di-qualifica}{\textbf{Piano di Qualifica\textsuperscript{G}} v1.0}, redatto dall’Amministratore, descrive le strategie e gli approcci adottati per garantire la qualità del prodotto o servizio sviluppato. Si compone delle seguenti sezioni:
\begin{itemize}
    \item \textbf{Qualità di processo}: specifica gli standard e le procedure seguite per garantire la qualità dei processi di sviluppo. Include:
    \begin{itemize}
        \item metodologie utilizzate;
        \item criteri per la misurazione e il miglioramento dei processi.
    \end{itemize}

    \item \textbf{Qualità di prodotto}: descrive gli standard e le specifiche che il prodotto deve soddisfare per essere considerato di qualità. Include:
    \begin{itemize}
        \item metriche e criteri di valutazione;
        \item specifiche tecniche richieste.
    \end{itemize}

    \item \textbf{Specifiche dei test}: fornisce una descrizione dettagliata dei test pianificati durante lo sviluppo per verificare che i requisiti siano soddisfatti.

    \item \textbf{Cruscotto delle metriche}: presenta un resoconto delle attività di valutazione svolte, utile per:
    \begin{itemize}
        \item monitorare l’andamento del progetto rispetto agli obiettivi prefissati;
        \item identificare azioni correttive necessarie.
    \end{itemize}
\end{itemize}
\subsubsection{Glossario}
Il \href{https://code7crusaders.github.io/docs/RTB/documentazione_interna/glossario.html#glossario}{Glossario\textsuperscript{G}} rappresenta un riferimento completo che raccoglie e definisce i termini tecnici utilizzati nel progetto. Questo documento garantisce una comprensione uniforme della terminologia specifica del settore, riducendo il rischio di equivoci e favorendo una comunicazione chiara. Inoltre, contribuisce a migliorare la coerenza e la qualità della documentazione prodotta dal team.

\subsubsection{Strumenti}
Di seguito sono elencati gli strumenti software utilizzati nel processo di fornitura:
\begin{itemize}
    \item \textbf{Discord}: piattaforma utilizzata per le riunioni interne.
    \item \textbf{Google Meet}: utilizzato per le riunioni formali online con l'azienda proponente.
    \item \textbf{Telegram}: piattaforma utilizzata come metodo informale per comunicare con l'azienda proponente.
    \item \textbf{LaTeX}: sistema per la creazione di documenti e slide di presentazione.
    \item \textbf{GitHubProject}: sistema di ticketing e roadmap integrato nella piattaforma di GitHub. 
\end{itemize}

\subsubsection{Metriche}
\begin{table}[h!]
    \centering
    \renewcommand{\arraystretch}{1.5}
    \begin{tabular}{|m{5cm}|m{5cm}|}
        \hline
        \textbf{Metrica} & \textbf{Nome} \\ \hline
        1PBM-PV         & Planned Value \\ \hline
        2PBM-ETC        & Estimated to Complete \\ \hline
        3PBM-EAC        & Estimate at Completion \\ \hline
        4PBM-EV         & Earned Value \\ \hline
        5PBM-AC         & Actual Cost \\ \hline
        6PBM-SV         & Scheduled Variance \\ \hline
        7PBM-CV         & Cost Variance \\ \hline
        8PBM-CPI        & Cost Performance Index \\ \hline
        9PBM-SPI        & Scheduled Performance Index \\ \hline
        10PBM-OTDR      & On-Time Delivery Rate \\ \hline
    \end{tabular}
    \caption{Metriche di fornitura}
    \label{tab:metriche}
\end{table}







%--------------PARTE B: SVILUPPO------------------------------------------------------------------------------------------

\subsection{Sviluppo}
\subsubsection{Introduzione}
Il processo di sviluppo ha l’obiettivo fondamentale di identificare e pianificare con precisione i compiti e le attività che il team deve eseguire per realizzare il prodotto software richiesto. Questo processo non si limita alla semplice suddivisione delle mansioni, ma prevede anche l’assegnazione di ruoli specifici a ciascun membro del team, in modo da valorizzare al meglio le competenze individuali e garantire un flusso di lavoro armonioso e produttivo.  
Per assicurare che il software sviluppato soddisfi pienamente le aspettative e le necessità del committente, il gruppo \textbf{Code7Crusaders} definisce in maniera dettagliata gli obiettivi di sviluppo e design. Questo processo prevede la redazione di linee guida chiare e la realizzazione di un piano di lavoro che consenta di monitorare costantemente l’avanzamento delle attività e di apportare eventuali correzioni. In particolare, il prodotto finale deve soddisfare le richieste del committente, come descritto nell’\href{https://code7crusaders.github.io/docs/RTB/documentazione_interna/glossario.html#analisi-dei-requisiti}{Analisi dei Requisiti\textsuperscript{G}}. 
Gli obiettivi di sviluppo definiti dal team devono essere rispettati, e il software deve superare con successo tutte le fasi di verifica e validazione\textsuperscript{G}. Il processo include la pianificazione accurata delle attività, la definizione delle specifiche di design e lo sviluppo delle funzionalità richieste. Una volta implementato il prodotto, viene avviata una fase di test che garantisce la qualità complessiva e l’aderenza ai requisiti stabiliti. Parallelamente, viene prodotta un’adeguata documentazione che facilita la tracciabilità e il mantenimento futuro del software, contribuendo a preservare il valore del progetto nel tempo.  

\subsubsection{Analisi dei Requisiti}

\subsubsubsection{Descrizione}
L'\href{https://code7crusaders.github.io/docs/RTB/documentazione_interna/glossario.html#analisi-dei-requisiti}{\textbf{Analisi dei Requisiti\textsuperscript{G}} v1.0} è un documento redatto dagli Analisti che comprende i seguenti aspetti:
\begin{itemize}
    \item \textbf{Introduzione}: descrive l’obiettivo del documento, il fine del prodotto e i riferimenti utilizzati per la sua stesura;
    \item \textbf{Descrizione del prodotto}: illustra le funzionalità attese del prodotto e le caratteristiche principali degli utenti;
    \item \textbf{Attori}: definisce i soggetti che utilizzeranno il sistema finale;
    \item \textbf{Casi d’uso}: identifica gli attori e descrive tutte le possibili interazioni con il sistema;
    \item \textbf{Requisiti}: raccoglie le caratteristiche essenziali da soddisfare e le fonti da cui queste sono state derivate.
\end{itemize}

\subsubsubsection{Scopo}
Lo scopo principale dell'\href{https://code7crusaders.github.io/docs/RTB/documentazione_interna/glossario.html#analisi-dei-requisiti}{\textbf{Analisi dei Requisiti\textsuperscript{G}} v1.0} è quello di specificare in modo completo e preciso le funzionalità e le caratteristiche che il prodotto software deve offrire. Tale analisi consente di comprendere appieno:
\begin{itemize}
    \item le necessità degli utenti;
    \item gli obiettivi principali del sistema;
    \item il contesto operativo in cui il sistema sarà utilizzato.
\end{itemize}

Gli obiettivi fondamentali di questa attività includono:
\begin{itemize}
    \item Individuare e chiarire le finalità e le aspettative legate al prodotto da sviluppare;
    \item Fornire ai Progettisti una base dettagliata per definire l’architettura e il design del sistema;
    \item Offrire un supporto per la pianificazione del progetto utilizzando i requisiti identificati;
    \item Facilitare lo scambio di informazioni tra il team di sviluppo e la Proponente;
    \item Servire come riferimento per la fase di verifica del sistema.
\end{itemize}


\subsubsubsection{Codifica dei casi d'uso}
I casi d'uso sono codificati utilizzando la seguente notazione:

\begin{itemize}
    \item \textbf{UC[ID-Principale][ID-Sottocaso]}: Identificativo univoco del caso d'uso, composto da un ID principale che identifica il caso principale e, se necessario, da un ID del sottocaso.
    \item \textbf{Titolo}: Breve descrizione del caso d'uso.
    \item \textbf{Attori}: Elenco degli attori coinvolti nel caso d'uso.
    \item \textbf{Precondizioni}: Condizioni che devono essere vere prima che il caso d'uso possa iniziare.
    \item \textbf{Postcondizioni}: Condizioni che devono essere vere dopo che il caso d'uso è stato completato con successo.
    \item \textbf{Scenario principale}: Descrizione dettagliata del flusso di eventi principale del caso d'uso.
    \item \textbf{Generalizzaioni}: Eventuali casi d'uso generalizzati.
    \item \textbf{Estensioni}: Eventuali casi d'uso estesi.
\end{itemize}

\subsubsubsection{Diagrammi Casi D'uso}
I diagrammi dei casi d’uso rappresentano visivamente le interazioni tra attori e sistema,
illustrando i vari scenari di utilizzo. Ogni caso d’uso descrive una sequenza di 
azioni necessarie per raggiungere un obiettivo specifico, aiutando a identificare i 
requisiti funzionali e a chiarire le aspettative degli utenti. Questi diagrammi 
facilitano la comunicazione tra sviluppatori e stakeholder, garantendo che tutte le 
funzionalità richieste siano considerate e implementate correttamente. 
Di seguito sono elencati i principali componenti di un diagramma dei casi d’uso.

\begin{itemize}
    \item \textbf{Attori}: I soggetti che interagiscono con il sistema, rappresentati come uomini stilizzati possono essere persone, altri applicativi o dispositivi che utilizzano le funzionalità del sistema. \ref{fig:attore}
\end{itemize}

\begin{figure}[H]
    \centering
    \includegraphics{../../img/Attore.png}
    \caption{Esempio di attore}
    \label{fig:attore}
\end{figure}

\begin{itemize}
    \item \textbf{Sistema}: Indica il contesto del sistema software, indicando funzionalità interne al contesto definito. \ref{fig:sistema} 
\end{itemize}

\begin{figure}[H]
    \centering
    \includegraphics{../../img/sistema.png}
    \caption{Esempio di sistema}
    \label{fig:sistema}
\end{figure}

\begin{itemize}
    \item \textbf{Casi d’uso}: funzionalità offerte dal sistema che soddisfano le necessità di un Attore. Ogni caso d’uso descrive una sequenza specifica di interazioni tra gli attori e il sistema. \ref{fig:caso_uso}
\end{itemize}

\begin{figure}[H]
    \centering
    \includegraphics{../../img/CasoUso.png}
    \caption{Esempio di caso d'uso}
    \label{fig:caso_uso}
\end{figure}

\begin{itemize}
    \item \textbf{Sottocasi d’uso}: Scenari specifici che si verificano all'interno del caso d’uso principale. \ref{fig:sottocaso_uso}
\end{itemize}

\begin{figure}[H]
    \centering
    \includegraphics{../../img/Sottocasouso.png}
    \caption{Esempio di sottocaso d'uso}
    \label{fig:sottocaso_uso}
\end{figure}

\begin{itemize}
    \item \textbf{Relazioni tra Attori e Casi d’Uso}: 
    \begin{itemize}
        \item \textbf{Associazione}: Collegamento tra un attore e un caso d’uso, indicando che l’attore è coinvolto nel caso d’uso. \ref{fig:associazione}
    \end{itemize}
\end{itemize}

\begin{figure}[H]
    \centering
    \includegraphics{../../img/associazione.png}
    \caption{Esempio di associazione}
    \label{fig:associazione}
\end{figure}

\begin{itemize}
    \item \textbf{Relazioni tra Attori}:
    \begin{itemize}
        \item \textbf{Generalizzazione}: Un attore eredita le funzionalità di un altro attore. Una relazione padre figlio dove il figlio eredita almeno una funzionalità del padre. Utilizzata nel caso in cui due attori condividano funzionalità. \ref{fig:generalizzazione_casiuso}
    \end{itemize}
\end{itemize}

\begin{figure}[H]
    \centering
    \includegraphics{../../img/Generalizzazione_casiuso.png}
    \caption{Esempio di Generalizzaione casi d'uso}
    \label{fig:generalizzazione_casiuso}
\end{figure}

\begin{itemize}
    \item \textbf{Relazioni tra Casi d’Uso}:
    \begin{itemize}
        \item \textbf{Inclusione}: Indica che un caso d’uso include un altro caso d’uso. Questo significa che durante l'esecuzione di un caso d'uso si eseguono anche i casi d'uso inclusi. Questo per evitare la ripetizione di funzionalità uguali in più casi d'uso. \ref{fig:inclusione}
    \end{itemize}
    

    \begin{figure}[H]
        \centering
        \includegraphics{../../img/include.png}
        \caption{Esempio di inclusione}
        \label{fig:inclusione}
    \end{figure}

    
    \begin{itemize}
        \item \textbf{Estensione}: Un caso d'uso esteso aggiunge funzionalità al caso d'uso principale, ma viene attivato solo in specifiche circostanze. Quando ciò accade, il flusso del caso d'uso principale si interrompe temporaneamente per consentire l'esecuzione del caso d'uso esteso. \ref{fig:estensione}
    \end{itemize}

    \begin{figure}[H]
        \centering
        \includegraphics{../../img/estensione.png}
        \caption{Esempio di estensione}
        \label{fig:estensione}
    \end{figure}
    
    \begin{itemize}
        \item \textbf{Generalizzazione}: Un caso d'uso eredita le funzionalità di un altro caso d'uso. Una relazione padre figlio dove il figlio eredita almeno una funzionalità del padre. Utilizzata nel caso in cui due casi d'uso condividano funzionalità. \ref{fig:generalizzazione_attore}
    \end{itemize}

    \begin{figure}[H]
        \centering
        \includegraphics{../../img/Generalizzazione_Attori.png}
        \caption{Esempio di Generalizzaione attore}
        \label{fig:generalizzazione_attore}
    \end{figure}

\end{itemize}

\subsubsubsection{Requisiti}

I requisiti sono classificati in tre categorie principali:  
\begin{itemize}
    \item \textbf{Funzionali}: riguardano l'usabilità del prodotto finale;  
    \item \textbf{Di qualità}: includono gli strumenti e la documentazione da fornire;  
    \item \textbf{Di vincolo}: fanno riferimento alle tecnologie da utilizzare.
\end{itemize}
Ciascun requisito è indicato da:
\begin{itemize}
    \item \textbf{Codice Identificativo}: codice univoco che identifica il requisito;
    \item \textbf{Descrizione}: breve spiegazione del requisito;
    \item \textbf{Fonte}: origine del requisito (es. capitolato, interno, ecc..);
    \item \textbf{Priorità}: importanza del requisito rispetto agli altri;
\end{itemize} 

\subsubsubsection{Fonti dei requisiti}
I requisiti sono stati identificati a partire dalle seguenti fonti:
\begin{itemize}
    \item \textbf{Capitolato}: Requisiti individuati tramite analisi del capitolato;
    \item \textbf{interno}: requisiti individuati durante riunioni interne al gruppo di lavoro;
    \item \textbf{Esterno}: requisiti individuati durante incontri con il proponente;
    \item \href{https://code7crusaders.github.io/docs/RTB/documentazione_interna/glossario.html#piano-di-qualifica}{\textbf{Piano di Qualifica}\textsuperscript{G}}: Requisiti necessari per rispettare standard di qualità definiti nel documento \href{https://code7crusaders.github.io/docs/RTB/documentazione_interna/glossario.html#piano-di-qualifica}{Piano di Qualifica\textsuperscript{G}};
    \item \href{https://code7crusaders.github.io/docs/RTB/documentazione_interna/glossario.html#norme-di-progetto}{\textbf{Norme di Progetto}\textsuperscript{G}}: Requisiti necessari per rispettare le norme di progetto definite nel documento \href{https://code7crusaders.github.io/docs/RTB/documentazione_interna/glossario.html#norme-di-progetto}{Norme di Progetto\textsuperscript{G}};
\end{itemize}

\subsubsubsection{Codifica dei requisiti}
I requisiti sono codificati come segue: \textbf{R[Tipo][Importanza][Numero]}
\newline
Dove \textbf{Tipo} può essere:
\begin{itemize}
    \item \textbf{F (funzionale)}
    \item \textbf{Q (di qualità)}
    \item \textbf{V (di vincolo)}
\end{itemize}
\textbf{Importanza} può essere:
\begin{itemize}
    \item \textbf{O (obbligatorio)}
    \item \textbf{D (desiderabile)}
    \item \textbf{F (facoltativo )}
\end{itemize}
\textbf{Numero} è un numero identificativo univoco del requisito.
\subsubsubsection{Metriche}
\begin{table}[h!]
    \centering
    \renewcommand{\arraystretch}{1.5}
    \begin{tabular}{|>{\centering\arraybackslash}m{5cm}|>{\centering\arraybackslash}m{5cm}|}
        \hline
        \textbf{Metrica} & \textbf{Nome} \\
        \hline
        11PBM-PRO & Percentuale Requisiti Obbligatori \\
        \hline
        12PBM-PRD & Percentuale Requisiti Desiderabili \\
        \hline
        13PBM-PRF & Percentuale Requisiti Facoltativi \\
        \hline
    \end{tabular}
    \caption{Percentuali requisiti}
    \label{tab:percentuali_requisiti}
\end{table}

%--------------------------DA AMPLIARE--------------------------------------------
\subsubsection{Progettazione}
\subsubsubsection{Introduzione}
La fase di progettazione riveste un ruolo cruciale nel definire la struttura principale del progetto, 
basandosi sui requisiti individuati durante l'analisi e descritti 
nell’\href{https://code7crusaders.github.io/docs/RTB/documentazione_interna/glossario.html#analisi-dei-requisiti}
{Analisi dei Requisiti\textsuperscript{G}}. 
Questa attività è affidata ai progettisti, i quali elaborano un piano dettagliato per implementare tutti i requisiti specificati.  
Per questa fase del ciclo di vita del software, il nostro gruppo si pone i seguenti obiettivi:
\begin{itemize}
    \item Trasformare i requisiti in specifiche tecniche dettagliate che coprano tutti gli aspetti del sistema.
    \item Garantire una struttura facilmente comprensibile per agevolare la manutenzione futura.
    \item Ottenere l’approvazione per il passaggio alla fase di sviluppo.
\end{itemize}
Il processo di progettazione si articola in tre livelli principali:
\begin{itemize}
    \item \textbf{Design dell’interfaccia}: questa fase si concentra su un livello di astrazione elevato rispetto al 
    funzionamento interno del sistema. Durante la progettazione dell’interfaccia, l’attenzione è rivolta alle 
    tecnologie da utilizzare nella fase di sviluppo del software, portando alla creazione di 
    un \href{https://code7crusaders.github.io/docs/RTB/documentazione_interna/glossario.html#poc-proof-of-concept}
    {Proof of Concept\textsuperscript{G}}.
    \item \textbf{Progettazione architetturale}: si definisce la struttura generale del sistema a un alto livello, senza 
    entrare nei dettagli interni dei componenti principali. In questa fase vengono anche definiti i test di integrazione.
    \item \textbf{Progettazione dettagliata}: si specificano gli elementi interni di ciascun componente principale, 
    incluse le specifiche architetturali del prodotto. Si producono inoltre i diagrammi delle classi e si definiscono 
    i test di unità per ogni componente. Questa fase culmina nella creazione 
    della \href{https://code7crusaders.github.io/docs/RTB/documentazione_interna/glossario.html#pb-product-baseline}
    {Product Baseline\textsuperscript{G}}.
\end{itemize}


\subsubsubsection{Specifica Tecnica}
La specifica tecnica è un documento che funge da fondamento per l'intero processo di sviluppo, 
guidando il team nelle scelte architetturali e tecnologiche, garantendo un approccio metodico e strutturato.

{Elementi Chiave della Specifica Tecnica}
\begin{itemize}
    \item \textbf{Architettura del sistema}: questa sezione definisce la struttura complessiva del sistema software, identificando i componenti principali, i moduli e le relative interfacce. Include inoltre dettagli sull'organizzazione logica e fisica del sistema, come la suddivisione in livelli o strati e le relazioni tra le diverse parti.
    \item \textbf{Tecnologie adottate}: vengono specificate le tecnologie scelte per lo sviluppo, inclusi linguaggi di programmazione, framework, librerie e tool di supporto. Questa sezione comprende anche l'ambiente di sviluppo e gli strumenti di gestione del progetto.
    \item \textbf{Struttura dei dati}: si descrive l'organizzazione e la gestione dei dati all'interno del sistema, comprendendo database, file system, protocolli di accesso e tecniche di persistenza adottate.
    \item \textbf{Interfacce esterne}: vengono identificate e descritte le interfacce con altri sistemi o applicazioni esterne. Questo include i formati dei dati scambiati e i protocolli di comunicazione utilizzati per garantire un'integrazione efficace.
    \item \textbf{Design pattern}: si presentano i design pattern adottati durante lo sviluppo, fornendo soluzioni consolidate a problematiche comuni e ricorrenti.
    \item \textbf{Pianificazione e risorse}: questa sezione fornisce una pianificazione dettagliata delle attività necessarie per implementare la specifica tecnica. Vengono inclusi stime di tempi, costi e risorse richieste, sia umane che tecnologiche.
    \item \textbf{Procedure di testing e validazione}: vengono fornite indicazioni sulle procedure e sugli strumenti necessari per testare e validare il sistema, assicurandosi che soddisfi i requisiti funzionali e le aspettative del cliente.
    \item \textbf{Requisiti tecnici}: viene riportato un elenco completo e dettagliato dei requisiti tecnici che il sistema deve soddisfare, inclusi parametri relativi a funzionalità, prestazioni e caratteristiche richieste.
\end{itemize}



\subsubsubsection{Qaulità dell'architettura}
La qualità dell'architettura è un aspetto fondamentale per garantire il successo di un sistema software. Essa si misura attraverso una serie di caratteristiche chiave che ne determinano l'efficacia, la robustezza e l'adattabilità. Di seguito vengono elencati i principali attributi:
\begin{itemize}
    \item \textbf{Scalabilità}: rappresenta la capacità del sistema di adattarsi a un incremento del carico di lavoro o delle risorse senza compromettere le prestazioni o la qualità del servizio. Un'architettura scalabile è in grado di rispondere dinamicamente alle variazioni delle richieste degli utenti e alle risorse disponibili.
    \item \textbf{Flessibilità}: indica l'abilità del sistema di adattarsi a modifiche nei requisiti o nell'ambiente operativo senza richiedere interventi significativi. Una buona flessibilità si ottiene tramite componenti ben separati e interfacce standardizzate, che agevolano l'aggiunta di nuove funzionalità o modifiche esistenti.
    \item \textbf{Manutenibilità}: definisce la facilità con cui il software può essere compreso, modificato e corretto durante il suo ciclo di vita. Un'architettura manutenibile è caratterizzata da un codice modulare, ben documentato e organizzato, facilitando l'identificazione e la risoluzione dei problemi.
    \item \textbf{Testabilità}: riguarda la facilità di verificare che il sistema soddisfi i requisiti funzionali e non funzionali. Un'architettura testabile prevede componenti isolati e interfacce ben definite, che consentono l'automazione dei test e garantiscono una verifica continua della qualità.
    \item \textbf{Affidabilità}: assicura il funzionamento corretto del sistema sia in condizioni normali che in situazioni anomale. Un'architettura affidabile integra meccanismi di gestione degli errori, recupero e ripristino, che permettono al sistema di continuare a operare anche in caso di guasti o interruzioni.
    \item \textbf{Performance}: misura la capacità del sistema di fornire risposte rapide e tempi di elaborazione ottimizzati anche sotto carichi di lavoro elevati. Le architetture performanti includono ottimizzazioni del codice, una gestione efficiente delle risorse e accessi ai dati ottimizzati.
    \item \textbf{Usabilità}: si riferisce alla facilità di utilizzo e comprensione del sistema da parte dell'utente finale. Un'architettura usabile prevede interfacce intuitive, una navigazione chiara e una presentazione delle informazioni coerente, garantendo così un'esperienza utente positiva.
    \item \textbf{Compatibilità}: indica la capacità del sistema di interagire senza difficoltà con altre piattaforme, sistemi o tecnologie. Una buona compatibilità si ottiene grazie all'adozione di standard aperti, interfacce ben definite e protocolli di comunicazione standardizzati.
    \item \textbf{Documentazione}: comprende la fornitura di documenti completi e dettagliati relativi all'architettura del sistema. Tra questi vi sono diagrammi, specifiche tecniche, manuali utente e guide per gli sviluppatori. Una documentazione accurata facilita la comprensione, la manutenzione e l'evoluzione del sistema.
    \item \textbf{Safety}: rappresenta la capacità del sistema di garantire la sicurezza dei dati e delle informazioni sensibili, proteggendoli anche in caso di malfunzionamenti o situazioni impreviste.
\end{itemize}


\subsubsection{Diagrammi UML}
L'utilizzo dei diagrammi UML (Unified Modeling Language) nella progettazione del sistema software offre numerosi vantaggi. Di seguito sono elencati i principali aspetti positivi:
\begin{itemize}
    \item \textbf{Chiarezza visiva}: i diagrammi UML forniscono una rappresentazione grafica chiara e intuitiva delle relazioni e delle interazioni tra i componenti del sistema software. Questo facilita la comprensione del sistema da parte degli stakeholder, sia tecnici che non tecnici.
    \item \textbf{Standardizzazione}: UML rappresenta uno standard internazionale ampiamente riconosciuto per la modellazione dei sistemi software. Grazie a questa standardizzazione, i diagrammi UML risultano facilmente comprensibili e interpretabili dai professionisti del settore a livello globale.
    \item \textbf{Comunicazione efficace}: i diagrammi UML forniscono un linguaggio visivo comune che facilita la comunicazione tra sviluppatori, stakeholder tecnici e non tecnici e altri membri del team di progetto. Questo consente una condivisione chiara delle informazioni e una comunicazione più efficace.
    \item \textbf{Supporto all'analisi e alla progettazione}: durante le fasi di analisi e progettazione del ciclo di sviluppo del software, i diagrammi UML possono essere utilizzati per visualizzare i requisiti, analizzare le relazioni tra i componenti e descrivere le interazioni all'interno del sistema.
    \item \textbf{Modellazione dei requisiti}: UML consente di modellare i requisiti funzionali e non funzionali in modo chiaro e strutturato. I diagrammi, come quelli dei casi d'uso e delle sequenze, permettono di identificare scenari, vincoli e comportamenti del sistema che guidano lo sviluppo.
    \item \textbf{Facilità di manutenzione}: grazie alla loro chiarezza, i diagrammi UML semplificano la comprensione dell'architettura del sistema, rendendo più agevole l'identificazione di problemi o aree di miglioramento. Questo facilita le attività di manutenzione ed evoluzione del software nel tempo.
    \item \textbf{Supporto ai principi di progettazione}: UML è un valido strumento per applicare e comunicare principi di progettazione consolidati, come l'incapsulamento, l'ereditarietà e il polimorfismo, che contribuiscono alla creazione di un'architettura software robusta e modulare.
    \item \textbf{Documentazione tecnica}: i diagrammi UML possono essere utilizzati per generare una documentazione tecnica dettagliata del sistema software. Questa documentazione rappresenta una guida completa sia per gli sviluppatori che per gli utenti finali e gli stakeholder coinvolti.
\end{itemize}


\subsubsection{Desing Pattern}
I \textit{design pattern} rappresentano soluzioni progettuali generiche e riutilizzabili per risolvere problemi comuni che emergono durante lo sviluppo del software. Si tratta di modelli architetturali, paradigmi o concetti consolidati, descritti spesso tramite codice o diagrammi, che forniscono una guida chiara per affrontare specifiche problematiche di progettazione in maniera efficiente ed efficace.

La documentazione relativa ai \textit{design pattern} svolge un ruolo fondamentale come punto di riferimento per l'intero team di sviluppo. Essa permette agli sviluppatori di comprendere rapidamente come ciascun pattern è stato implementato all'interno del codice. Questo è particolarmente utile durante:
\begin{itemize}
    \item le \textbf{fasi di progettazione}, per identificare soluzioni standard e ben consolidate;
    \item lo \textbf{sviluppo}, per garantire coerenza e chiarezza nell'implementazione;
    \item la \textbf{manutenzione del software}, per ottenere una visione chiara della struttura architetturale e dei pattern adottati.
\end{itemize}
L'utilizzo dei \textit{design pattern} migliora la leggibilità del codice, ne facilita la manutenzione e promuove l'adozione di soluzioni standardizzate all'interno del sistema.

\subsubsection{Test}
L'attività di \textbf{testing} è una fase critica del processo di sviluppo software, finalizzata a garantire che il prodotto finale soddisfi i requisiti funzionali, prestazionali e qualitativi stabiliti. Tale fase include diverse attività fondamentali, tra cui:
\begin{itemize}
    \item \textbf{Pianificazione dei test}: definizione degli obiettivi, della strategia e delle risorse necessarie per eseguire i test;
    \item \textbf{Progettazione dei casi di test}: sviluppo di scenari di test dettagliati basati sui requisiti del sistema;
    \item \textbf{Esecuzione dei test}: applicazione dei casi di test per verificare il comportamento del software;
    \item \textbf{Analisi dei risultati}: valutazione degli output dei test per individuare difetti, errori o anomalie.
\end{itemize}

\subsubsection{Metriche}
\begin{table}[h!]
    \centering
    \caption{Metriche inerenti la Progettazione}
    \label{tab:metriche_progettazione}
    \begin{tabular}{|c|l|}
        \hline
        \textbf{Codice} & \textbf{Nome Esteso} \\ 
        \hline
        14M-PG         & Profondità delle Gerarchie \\ 
        \hline
    \end{tabular}
\end{table}






\subsubsection{Codifica}
\subsubsubsection{Descrizione e Scopo}
L'attività di sviluppo del codice è affidata ai Programmatori, responsabili della conversione delle scelte progettuali in codice sorgente funzionante. Gli sviluppatori operano in un contesto organizzato, seguendo scrupolosamente le linee guida e gli standard definiti durante la progettazione dell'architettura software. Tale approccio garantisce uniformità nell'implementazione e l'applicazione delle migliori pratiche, promuovendo la realizzazione di codice efficiente, affidabile e facilmente manutenibile. 
Per mantenere elevati standard qualitativi, i Programmatori devono attenersi alle metriche specificate nel Piano di Qualifica v1.0.

\subsubsubsection{Aspettative}
La fase di implementazione è finalizzata alla realizzazione di un software che risponda completamente ai requisiti e agli obiettivi stabiliti dalla Committente. Il codice prodotto deve soddisfare le seguenti caratteristiche fondamentali:
\begin{itemize}
    \item Rispetto delle specifiche fornite;
    \item Semplicità e leggibilità;
    \item Efficienza in termini di prestazioni;
    \item Copertura di test per assicurare il corretto funzionamento.
\end{itemize}

\subsubsubsection{Norme di codifica}
Le seguenti linee guida sono state definite per garantire la qualità e la manutenibilità del codice:

\begin{itemize}
  \item \textbf{Nomi significativi}: I nomi di variabili, costanti, classi e metodi devono essere chiari e descrittivi, per facilitare la comprensione del codice.
  \item \textbf{Commenti}: Il codice deve essere scritto in modo chiaro e facilmente comprensibile. I commenti devono essere usati solo quando strettamente necessario, per spiegare logiche complesse o decisioni particolari.
  \item \textbf{Indentazione e formattazione consistente}: Il codice deve essere correttamente indentato per favorire la leggibilità. È importante utilizzare spazi o tabulazioni in modo uniforme.
  \item \textbf{Gestione delle eccezioni}: Le eccezioni e gli errori devono essere gestiti correttamente, fornendo messaggi utili e trattando adeguatamente le situazioni anomale.
  \item \textbf{Riutilizzo del codice}: Scrivere il codice in modo modulare, per favorire il riuso in altre parti del programma o in progetti futuri.
  \item \textbf{Testabilità}: Le funzioni devono essere piccole e focalizzate, per rendere il codice facilmente testabile e verificabile.
  \item \textbf{Sicurezza}: Il codice deve essere sicuro, prevenendo vulnerabilità e proteggendo da potenziali attacchi.
  \item \textbf{Performance}: Ottimizzare il codice per ottenere buone prestazioni, utilizzando in modo efficiente le risorse disponibili e minimizzando il tempo di esecuzione.
  \item \textbf{Compatibilità}: Garantire che il codice funzioni correttamente su diverse piattaforme, sistemi operativi e browser, per una fruizione omogenea da parte degli utenti.
\end{itemize}

\subsubsubsection{Metriche}
\begin{table}[h!]
    \centering
    \renewcommand{\arraystretch}{1.5}
    \begin{tabular}{|l|l|}
        \hline
        \textbf{Metrica} & \textbf{Nome} \\
        \hline
        15RPM-PRM & Parametri per Metodo \\
        \hline
        16RPM-CPC & Campi per Classe \\
        \hline
        17RPM-LCPM & Linea Di Commento Per Metodo \\
        \hline
        18RPM-CCM & Complessità Ciclomatica Metrica \\
        \hline
    \end{tabular}
    \caption{Metriche di Codifica}
    \label{tab:metriches_codifica}
\end{table}


\subsubsubsection{Strumenti}
Visual Studio Code è un ambiente di sviluppo integrato (IDE) creato da Microsoft, scelto dal team come strumento principale per lo sviluppo software. Questo IDE offre funzionalità avanzate che facilitano la scrittura, la modifica e il debugging del codice, supportando una vasta gamma di linguaggi di programmazione e framework.


\section{Processi di Supporto}
% Suddivoso in parti:
% Documentazione
% Gestione della configurazione
% Gestione della qualità 
% Verifica 
% Validazione

\subsection{Documentazione}
\subsubsection{Introduzione}
La documentazione software si riferisce al testo che accompagna un programma, 
descrivendo il prodotto sia per gli sviluppatori che per gli utilizzatori. 
Essa ha l'obiettivo di supportare i membri del team durante lo sviluppo, monitorando i 
processi e documentando tutte le attività, per facilitare anche la manutenzione del software 
e migliorare la qualità del prodotto finale.

In base a quanto sopra, la documentazione svolge un ruolo cruciale nel ciclo di vita del software. 
Le aspettative nei suoi confronti includono:
\begin{itemize}
    \item Definizione di regole chiare e concise per la redazione dei documenti.
    \item Adozione di una struttura uniforme e standard per tutti i documenti nel ciclo di vita del software, per garantire omogeneità.
\end{itemize}

\subsection{Ciclo di Vita del Documento}

Il ciclo di vita di un documento software si articola in tre fasi principali:
\begin{itemize}
    \item \textbf{Redazione}: la fase di creazione del documento, che viene suddivisa tra i membri del gruppo e supportata dall'uso di un sistema di versionamento.
    \item \textbf{Verifica}: una volta completata la stesura, il documento passa alla fase di verifica, 
    che può essere effettuata su parti del documento o su tutto il contenuto. 
    Ogni sezione deve essere verificata da una persona distinta dal redattore della sezione stessa.
    \item \textbf{Approvazione}: il documento, una volta completato e verificato, viene approvato dal Responsabile di Progetto.
\end{itemize}

\subsection{Template}

Il gruppo ha deciso di utilizzare un template semplice, creato con Latex. Questo modello è stato 
standardizzato e viene utilizzato per la redazione di tutti i documenti ufficiali.

\subsection{Documenti Prodotti}

I documenti generati durante il ciclo di vita del software sono suddivisi in due categorie principali:

\subsubsection{Formali}
I documenti formali sono quelli con un nome univoco e utilizzati per regolare le attività interne al gruppo 
durante tutto il ciclo di vita del software. 
Sono versionati e approvati dal Responsabile di Progetto. 
Questi documenti si suddividono in:
\begin{itemize}
    \item \textbf{Ad uso interno}: destinati esclusivamente ai membri del gruppo, come ad esempio:
    \begin{itemize}
        \item Norme di progetto
        \item Verbali interni
    \end{itemize}
    \item \textbf{Ad uso esterno}: destinati a enti esterni come il committente o il proponente, e consegnati 
    nell'ultima versione approvata. Tra questi:
    \begin{itemize}
        \item Analisi dei Requisiti
        \item Piano di Progetto
        \item Piano di Qualifica
        \item Glossario
        \item Verbali esterni
    \end{itemize}
\end{itemize}

\subsubsection{Informali}
I documenti informali comprendono:
\begin{itemize}
    \item Documenti non ancora approvati dal Responsabile di Progetto.
    \item Bozze e appunti brevi.
    \item Documenti che non necessitano di essere versionati.
\end{itemize}
Questi documenti sono gestiti in una sezione separata, dove il gruppo ha creato un Google Drive condiviso per facilitarne la gestione.

\subsection{Struttura del Documento}

Tutti i documenti ufficiali seguono una struttura rigida che deve essere rispettata. La struttura include:

\subsubsection{Prima Pagina}
La prima pagina include:
\begin{itemize}
    \item Il titolo del gruppo, seguito dal nome esteso dell’acronimo.
    \item L'indirizzo email del gruppo (\texttt{code7crusaders@gmail.com}).
    \item Il nome del documento.
    \item Le informazioni sul documento, che elencano i redattori, i verificatori, l’amministratore, i destinatari e la versione.
\end{itemize}

\subsubsection{Registro dei Cambiamenti - Changelog}
Il registro dei cambiamenti tiene traccia della storia del documento. In questa sezione sono inclusi:
\begin{itemize}
    \item La versione del documento.
    \item La data di ogni modifica.
    \item L'autore che ha effettuato la modifica.
    \item Il verificatore delle modifiche.
    \item Una breve descrizione delle modifiche.
\end{itemize}

\subsubsection{Indice}
Ogni documento include un indice subito dopo il registro dei cambiamenti. 
Questo indice aiuta a navigare nel documento, rendendo più facile la ricerca di sezioni specifiche.

\subsubsection{Verbali}
I verbali sono documenti speciali con una struttura diversa rispetto agli altri. Non includono il registro dei cambiamenti né l'indice.
La struttura dopo la prima pagina prevede:
\begin{itemize}
    \item \textbf{Partecipanti}: orario di inizio e fine dell'incontro, seguito da una tabella con i nomi e le durate di presenza dei partecipanti.
    \item \textbf{Sintesi ed elaborazione incontro}: un riassunto degli argomenti trattati e una sezione per eventuali dubbi o indicazioni per i prossimi incontri.
\end{itemize}
I verbali sono suddivisi in interni (tra i membri del gruppo) ed esterni (con l'azienda o il committente).

\subsection{Norme Tipografiche}

\subsubsection{Nome del File}
I file devono avere nomi coerenti, con la lettera iniziale maiuscola e le restanti lettere minuscole, 
eccetto i verbali, che sono nominati con la data dell'incontro in formato GG-MM-AAAA.

\subsubsection{Stile del Testo}
Lo stile del testo nei documenti ufficiali include:
\begin{itemize}
    \item \textbf{Grassetto}: per titoli e parole di rilevanza.
    \item \underline{Sottolineato}: solo per i link e parole del glossario.
\end{itemize}

\subsubsection{Glossario}
Il glossario è un documento contenente termini e definizioni utili per comprendere meglio il linguaggio tecnico, evitando ambiguità. 
I termini sono registrati in ordine alfabetico.

\subsection{Elementi Grafici}

\subsubsection{Tabelle}
Le tabelle nei documenti ufficiali devono avere un titolo che descriva il contenuto e devono essere centrate orizzontalmente. 
Le righe della tabella sono alternate in colori per migliorare la leggibilità. 

\subsubsection{Immagini}
Le immagini devono essere centrate orizzontalmente. Anche i diagrammi UML sono trattati come immagini.

\subsection{Strumenti}

\subsubsection{Versionamento}
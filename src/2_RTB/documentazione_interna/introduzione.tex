\section{Introduzione}

\subsection{Scopo del documento}
Questo documento ha lo scopo di definire le regole e le procedure che ogni membro del team deve seguire durante 
lo sviluppo del progetto. In particolare, mira a stabilire il \textit{Way of Working} del gruppo. 

La sua redazione inizia nelle prime fasi del progetto e continua anche durante le fasi successive, 
per essere costantemente aggiornato e adattato alle esigenze del team. 

Il processo seguirà le linee guida dello standard ISO/IEC 12207:1995, suddivise in:
\begin{itemize}
    \item Processi primari
    \item Processi di supporto
    \item Processi organizzativi
\end{itemize}


\subsection{Scopo del progetto}
Il progetto si propone di sviluppare un Assistente Virtuale intelligente per aziende che operano nel 
settore della vendita multiprodotto. Questo assistente avrà il compito di semplificare l'accesso alle 
informazioni sui prodotti disponibili, rispondendo alle domande più frequenti poste dai clienti in modo rapido ed efficace.

Grazie all'uso di tecnologie avanzate come il \href{https://code7crusaders.github.io/docs/RTB/documentazione_interna/glossario.html#machine-learning}{Machine Learning\textsuperscript{G}} e il \href{https://code7crusaders.github.io/docs/RTB/documentazione_interna/glossario.html#natural-language-processing-nlp}{Natural Language Processing\textsuperscript{G}}, 
il sistema sarà in grado di analizzare i dati contenuti nei cataloghi aziendali e negli archivi digitali, 
fornendo risposte precise e personalizzate.

L’obiettivo principale è ridurre la dipendenza dagli specialisti aziendali, che attualmente rappresentano 
l’unico canale di accesso per ottenere dettagli approfonditi sui prodotti. Questo migliorerà l’efficienza 
operativa, ottimizzerà le risorse e offrirà una migliore esperienza ai clienti, che potranno interagire con 
il sistema in modo intuitivo e diretto attraverso piattaforme digitali come siti web o chatbot.

In sintesi, il progetto intende rendere l'accesso alle informazioni aziendali più semplice, veloce e scalabile, 
migliorando al contempo la qualità del servizio offerto ai clienti.


\subsection{Glossario}
Per evitare ambiguità e facilitare la comprensione del documento, si farà uso di un glossario, 
contenente la definizione dei termini tecnici e degli acronimi utilizzati, 
che sarà incluso all'interno del file \textit{glossario}.


\subsection{Riferimenti}
\subsubsection{Normativi}
\begin{itemize}
	\item \textbf{Capitolato C7:} \\ \url{https://www.math.unipd.it/~tullio/IS-1/2024/Progetto/C7.pdf}
	\item \textbf{ISO/IEC 12207:1995} \\ \url{https://www.math.unipd.it/~tullio/IS-1/2009/Approfondimenti/ISO_12207-1995.pdf}
\end{itemize}

\subsubsection{Informativi}
\begin{itemize}
    \item\textbf{Glossario RTB}\\ \url{https://code7crusaders.github.io/docs/RTB/documentazione_interna/glossario.html}
    \item\textbf{Documentazione Git}\\ \url{https://git-scm.com/docs}
    \item\textbf{Documentazione Latex}\\ \url{https://www.latex-project.org/help/documentation/}

\end{itemize}

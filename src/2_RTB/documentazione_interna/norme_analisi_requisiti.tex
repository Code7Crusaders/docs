\section{Analisi dei requisiti}

\subsection{Descrizione}
Il documento di analisi dei requisiti ha l'obiettivo di identificare,
descrivere e documentare in modo completo le necessità, 
le funzionalità e le prestazioni che il sistema software deve garantire. 
Questo rappresenta il fondamento del processo di sviluppo del software, 
fornendo un riferimento chiaro e dettagliato per la progettazione.

Attraverso questa attività, si mira a comprendere a fondo le esigenze degli stakeholder, 
come utenti finali e clienti, assicurandosi che il prodotto software finale risponda 
pienamente alle loro aspettative e necessità. L’analisi dei requisiti comprende 
generalmente la raccolta e la documentazione dei requisiti funzionali, 
qualitativi e vincolanti, la definizione dei casi d’uso, 
nonché la prioritizzazione e la tracciabilità dei requisiti lungo tutto il ciclo
di vita del software.

\subsection{Documento}
Gli analistiG hanno il compito di redigere l’Analisi dei Requisiti , comprendendo le seguenti
sezioni:



\subsubsection{Codifica dei casi d'uso}
I casi d'uso sono codificati utilizzando la seguente notazione:

\begin{itemize}
    \item \textbf{UC[ID-Principale][ID-Sottocaso]}: Identificativo univoco del caso d'uso, composto da un ID principale che identifica il caso principale e, se necessario, da un ID del sottocaso.
    \item \textbf{Titolo}: Breve descrizione del caso d'uso.
    \item \textbf{Attori}: Elenco degli attori coinvolti nel caso d'uso.
    \item \textbf{Precondizioni}: Condizioni che devono essere vere prima che il caso d'uso possa iniziare.
    \item \textbf{Postcondizioni}: Condizioni che devono essere vere dopo che il caso d'uso è stato completato con successo.
    \item \textbf{Scenario principale}: Descrizione dettagliata del flusso di eventi principale del caso d'uso.
    \item \textbf{Generalizzaioni}: Eventuali casi d'uso generalizzati.
    \item \textbf{Estensioni}: Eventuali casi d'uso estesi.
\end{itemize}
\section{Standard per la qualità}

Abbiamo scelto di utilizzare standard internazionali per valutare e migliorare la qualità dei processi e del software. In particolare, seguiremo lo standard ISO/IEC 12207:1995, che fornisce linee guida per organizzare i processi in categorie principali, di supporto e organizzative.
Per quanto riguarda la qualità del software, adotteremo lo standard ISO/IEC 25010:2023, il quale offre una struttura chiara e completa per identificare e classificare le metriche di qualità. Abbiamo deciso di limitarci a questi due standard, poiché il precedente ISO/IEC 9126:2001 è stato ufficialmente ritirato e sostituito dallo standard ISO/IEC 25010:2023.

\subsection{Caratteristiche del sistema ISO/IEC 25010:2023}

\subsubsection{Appropriatezza funzionale}
\begin{itemize}
    \item \textbf{Completezza:} Il software deve soddisfare tutti i requisiti definiti e le aspettative degli utenti.
    \item \textbf{Correttezza:} Il funzionamento del software deve essere accurato e coerente con le specifiche.
    \item \textbf{Appropriatezza:} Il prodotto deve risultare idoneo allo scopo previsto e al contesto operativo in cui viene utilizzato.
\end{itemize}

\subsubsection{Performance}
\begin{itemize}
    \item \textbf{Tempo:} Il software deve rispettare le tempistiche di sviluppo e consegna stabilite.
    \item \textbf{Risorse:} L'utilizzo delle risorse di sistema deve essere ottimizzato ed efficiente.
    \item \textbf{Capacità:} Il software deve essere in grado di gestire carichi di lavoro attesi senza degradare le prestazioni.
\end{itemize}

\subsubsection{Compatibilità}
\begin{itemize}
    \item \textbf{Coesistenza:} Il prodotto deve operare correttamente in presenza di altri software o sistemi.
    \item \textbf{Interoperabilità:} Deve essere possibile scambiare informazioni e collaborare con altri software o sistemi.
\end{itemize}

\subsubsection{Usabilità}
\begin{itemize}
    \item \textbf{Riconoscibilità:} L’interfaccia utente deve essere intuitiva e facilmente comprensibile.
    \item \textbf{Apprendibilità:} Gli utenti devono poter imparare a utilizzare il software in modo rapido.
    \item \textbf{Operabilità:} Il software deve risultare semplice da utilizzare e da controllare.
    \item \textbf{Gestione degli errori:} Deve essere in grado di individuare e gestire gli errori in modo efficace.
    \item \textbf{Estetica:} L’interfaccia utente deve essere gradevole e visivamente accattivante.
    \item \textbf{Accessibilità:} Deve garantire l’accesso anche agli utenti con disabilità.
\end{itemize}

\subsubsection{Affidabilità}
\begin{itemize}
    \item \textbf{Maturità:} Il software deve garantire stabilità, robustezza e affidabilità.
    \item \textbf{Disponibilità:} Deve essere accessibile e operativo ogni volta che è necessario.
    \item \textbf{Tolleranza ai guasti:} Il software deve saper gestire errori o condizioni inaspettate senza interruzioni gravi.
    \item \textbf{Recuperabilità:} Deve poter ripristinare i dati e le funzionalità in caso di errore o guasto.
\end{itemize}

\subsubsection{Sicurezza}
\begin{itemize}
    \item \textbf{Riservatezza:} Il software deve proteggere i dati sensibili e garantire la privacy.
    \item \textbf{Integrità:} Deve assicurare che i dati siano accurati e completi.
    \item \textbf{Non ripudio:} Deve garantire che le transazioni non possano essere negate.
    \item \textbf{Autenticazione:} Deve verificare l'identità degli utenti e limitare l’accesso ai soli utenti autorizzati.
    \item \textbf{Autenticità:} Deve permettere di verificare la provenienza dei dati e delle informazioni.
\end{itemize}

\subsubsection{Manutenibilità}
\begin{itemize}
    \item \textbf{Modularità:} Il software deve essere strutturato in moduli indipendenti per semplificarne la gestione.
    \item \textbf{Riutilizzabilità:} Parti del software devono poter essere riutilizzate in contesti o progetti differenti.
    \item \textbf{Analizzabilità:} Il codice e la documentazione devono consentire una facile identificazione dei problemi.
    \item \textbf{Modificabilità:} Il software deve poter essere aggiornato e migliorato senza difficoltà.
    \item \textbf{Testabilità:} Il prodotto deve consentire l’efficace esecuzione di test per validarne il funzionamento.
\end{itemize}

\subsubsection{Portabilità}
\begin{itemize}
    \item \textbf{Adattabilità:} Il software deve essere in grado di funzionare correttamente in ambienti diversi, adattandosi a nuove tecnologie o requisiti specifici.
    \item \textbf{Installabilità:} Deve essere possibile installare e configurare il software in modo semplice e rapido.
    \item \textbf{Sostituibilità:} Il prodotto deve poter essere facilmente rimpiazzato da versioni aggiornate o soluzioni alternative senza impatti significativi.
\end{itemize}

\subsection{Suddivisione secondo standard ISO/IEC 12207:1995}

\subsubsection{Processi primari}
I processi primari sono fondamentali per lo sviluppo del software e includono:
\begin{itemize}
    \item \textbf{Acquisizione:} Comprende tutte le attività necessarie per ottenere un prodotto software o servizio. Queste attività includono la stesura delle richieste, la scelta del fornitore e la gestione del contratto.
    \item \textbf{Fornitura:} Si riferisce alle attività svolte per consegnare il prodotto o servizio al cliente. Tra queste vi sono la preparazione delle offerte, la negoziazione contrattuale e la consegna finale.
    \item \textbf{Sviluppo:} Riguarda la progettazione e la creazione del software, partendo dall'analisi dei requisiti fino alla fase di implementazione, test e integrazione.
    \item \textbf{Operatività:} Insieme di attività necessarie per garantire il funzionamento del software in un ambiente reale, come la gestione delle operazioni quotidiane, il monitoraggio delle prestazioni e il backup dei dati.
    \item \textbf{Manutenzione:} Include le operazioni di aggiornamento o modifica del software dopo la sua consegna, sia per correggere difetti che per ottimizzarne le prestazioni.
\end{itemize}

\subsubsection{Processi di supporto}
I processi di supporto forniscono assistenza ai processi primari e comprendono:
\begin{itemize}
    \item \textbf{Documentazione:} Creazione e gestione dei documenti necessari per supportare il ciclo di vita del software.
    \item \textbf{Gestione della configurazione:} Monitoraggio e controllo delle modifiche al software per garantire la coerenza e la tracciabilità delle versioni.
    \item \textbf{Assicurazione della qualità:} Attività di controllo e verifica per assicurare che processi e prodotti rispettino gli standard prefissati, includendo pianificazione della qualità, ispezioni e revisioni.
    \item \textbf{Verifica:} Controllo dei prodotti per accertarsi che soddisfino i requisiti stabiliti.
    \item \textbf{Validazione:} Garantisce che il software risponda alle reali esigenze degli utenti finali.
    \item \textbf{Revisioni congiunte con il cliente:} Sessioni di verifica periodica con tutte le parti coinvolte per monitorare i progressi del progetto.
    \item \textbf{Audit:} Ispezioni formali per garantire il rispetto delle procedure, degli standard e dei requisiti.
    \item \textbf{Risoluzione dei problemi:} Attività dedicate all'identificazione e risoluzione tempestiva di eventuali malfunzionamenti o problematiche.
\end{itemize}

\subsubsection{Processi organizzativi}
Questi processi forniscono supporto all’intera organizzazione e comprendono:
\begin{itemize}
    \item \textbf{Gestione:} Attività di pianificazione, monitoraggio e controllo delle operazioni del progetto.
    \item \textbf{Infrastruttura:} Gestione delle risorse e infrastrutture necessarie per supportare tutte le fasi del ciclo di vita del software.
    \item \textbf{Miglioramento:} Identificazione e implementazione di azioni mirate al miglioramento continuo dei processi organizzativi.
    \item \textbf{Formazione:} Programmi di formazione e aggiornamento delle competenze per il personale coinvolto nello sviluppo e nella gestione del software.
\end{itemize}

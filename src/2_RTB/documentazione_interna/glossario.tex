\documentclass[italian,12pt]{article}

%--------------variabili------------------%
\def\Title{Glossario}
\def\Author{Code7Crusaders}
\def\Version{v0.1}
%-----------------------------------------%

%pacchetti extra da scaricare dblfloatfix, fancyhdr
\usepackage{fancyhdr}%creazione header-footer
\usepackage{graphicx} %serve per inserire immagini
%\usepackage{dblfloatfix} %serve per posizionare gli elementi dove si vuole
\usepackage[hidelinks]{hyperref} %serve per i link
\usepackage{tikz}
% \usepackage{tgadventor} % font
\usepackage[useregional=numeric,showseconds=true,showzone=false]{datetime2}
\usepackage{caption}
\usepackage{geometry}
\usepackage{setspace}
\usepackage{eurosym}
\usepackage[italian]{babel}
\usepackage[hidelinks]{hyperref}
\usepackage{tabularx}
\usepackage{longtable}
\usepackage{float}

\linespread{1.2}
\geometry{a4paper, margin=1in}

\renewcommand{\contentsname}{Indice}
\renewcommand\familydefault{\sfdefault}

\renewcommand{\thesection}{}
\renewcommand{\thesubsection}{}

\begin{document}

\begin{titlepage} 

    \AddToHookNext{shipout/background}{
    \begin{tikzpicture}[remember picture,overlay]
    \node at (current page.center) {
    \includegraphics{../../img/background.png}
    };
    \end{tikzpicture}
    }

    \centering
    \vspace*{2cm}
    
    \includegraphics[width=0.3\textwidth]{../../img/logo/7Crusaders_logo.png} % Aggiungi il logo qui
    \vspace{1cm}
    
    {\Huge \textbf{Code7Crusaders}}\\
    \vspace{0.5cm}
    {\Large Software Development Team}\\
    \vspace{2cm}
    
    \large \textbf{Glossario}
    \vspace{3.9cm}

    \textbf{Membri del Team:}\\
    Enrico Cotti Cottini, Gabriele Di Pietro, Tommaso Diviesti \\
    Francesco Lapenna, Matthew Pan, Eddy Pinarello, Filippo Rizzolo \\
    \vspace{0.5cm}
    
    \vspace{1cm}
\end{titlepage}



\newpage
%------------------------Versioni
\begin{center}
    \textbf{Versioni}
    \\
    \vspace{0.3cm}
    \begin{tabular}{|c|c|c|c|c|}
        \hline
        \textbf{Ver} & \textbf{Data} & \textbf{Redattore} & \textbf{Verificatore} & \textbf{descrizione}\\
        0.2 & 18/11/2024 & Enrico Cotti Cottini &  & Aggiunta di nuovi termini \\
        \hline
        0.1 & 05/11/2024 & Gabriele Di Pietro & Filippo Rizzolo & Prima stesura del documento \\
        \hline
    \end{tabular}
\end{center}
%----------------

\newpage
\tableofcontents
\newpage

% per mettere lettere
% \section{Lettera}

% entry di glossario
% \subsection{Parola}
% Entry glossatio

% fine lettera 
% \newpage

\section{A}

\subsection{Analisi dei Requisiti}
Processo di identificazione e definizione delle necessità e delle aspettative degli stakeholder per un progetto. Serve come base per la progettazione e lo sviluppo del prodotto.

\newpage

\section{D}

\subsection{Diagrammi UML}
Diagrammi standardizzati utilizzati per modellare e visualizzare il design di sistemi software. Aiutano a descrivere la struttura, il comportamento e le interazioni tra componenti del sistema.

\subsection{Documentazione Formale e Informale}
La documentazione formale include materiali ufficiali come norme e piani di progetto, mentre quella informale comprende appunti e bozze non ancora approvate. Entrambe supportano lo sviluppo del progetto.

\newpage

\section{G}

\subsection{GitHub Actions}
Una piattaforma di automazione che permette di eseguire flussi di lavoro direttamente nei repository GitHub. Supporta azioni come compilazioni, test e deployment automatizzati, migliorando l'efficienza nello sviluppo software.

\subsection{Glossario}
Elenco strutturato di termini tecnici o specializzati, ognuno corredato dalla propria definizione o spiegazione. Questo strumento aiuta a migliorare la comunicazione tra le varie parti coinvolte in un progetto, riducendo le ambiguità e garantendo una comprensione condivisa dei termini utilizzati in un determinato contesto.

\newpage

\section{L}

\subsection{LLM (Large Language Model)}
Un modello di intelligenza artificiale addestrato su grandi quantità di dati testuali per comprendere, generare e contestualizzare il linguaggio naturale. È utilizzato in applicazioni come chatbot, traduttori e analisi del linguaggio.

\newpage

\section{M}

\subsection{Machine Learning}
Una branca dell’intelligenza artificiale che utilizza algoritmi per apprendere dai dati e migliorare le prestazioni senza essere esplicitamente programmata. Trova applicazione in ambiti come la previsione, il riconoscimento e la personalizzazione.

\subsection{Milestone}
Un obiettivo intermedio significativo nel ciclo di vita del progetto, utilizzato per monitorare i progressi e verificare il completamento di specifiche fasi o attività.

\newpage

\section{N}

\subsection{Natural Language Processing (NLP)}
Tecnologia che combina linguistica e intelligenza artificiale per consentire ai computer di comprendere, analizzare e generare linguaggio naturale. Viene usata in chatbot, traduttori automatici e analisi testuale.

\subsection{Norme di Progetto}
Regole e linee guida stabilite all'interno di un progetto per garantire coerenza e qualità nelle attività svolte. Definiscono standard e procedure, come documentazione, gestione delle versioni e criteri di codifica, per assicurare uniformità nell'approccio e nel risultato finale.

\newpage

\section{P}

\subsection{PB (Product Baseline)}
Una revisione che documenta lo stato del progetto al termine dello sviluppo, descrivendo il prodotto finale in termini di specifiche tecniche, funzionalità e conformità ai requisiti iniziali.

\subsection{Piano di Progetto}
Documento formale che delinea in dettaglio la pianificazione, la esecuzione, il monitoraggio e il controllo di tutte le attività coinvolte nella realizzazione di un progetto. Questo documento fornisce una roadmap chiara e organizzata, comprensiva di obiettivi, risorse, scadenze e strategie di gestione dei rischi.

\subsection{Piano di Qualifica}
Documento che stabilisce gli standard di qualità, i processi e le attività di testing che saranno implementati durante lo sviluppo di un progetto. Contiene una descrizione dettagliata delle strategie di testing, delle metriche di valutazione e dei criteri di accettazione del prodotto finale.

\subsection{PoC (Proof of Concept)}
Una dimostrazione pratica per validare la fattibilità di un'idea, tecnologia o progetto. Permette di identificare eventuali rischi o limitazioni prima di procedere con lo sviluppo completo.

\newpage

\section{R}

\subsection{Roadmap di GitHub}
Strumento visuale per pianificare e monitorare il progresso delle attività di un progetto all'interno di GitHub. Aiuta a gestire priorità, scadenze e assegnazione dei compiti.

\subsection{RTB (Requirements and Technology Baseline)}
Una revisione formale che stabilisce i requisiti funzionali e tecnici di un progetto, definendo il punto di riferimento per lo sviluppo e la verifica delle funzionalità richieste.

\newpage

\section{T}

\subsection{Template di LaTeX}
Un modello predefinito per la creazione di documenti in LaTeX. Consente di standardizzare la struttura e il formato dei documenti, semplificando la redazione e migliorando la coerenza stilistica.

\newpage

\section{V}

\subsection{Versionamento}
Processo di gestione delle modifiche ai documenti o al codice sorgente mediante strumenti che tracciano e memorizzano le varie versioni. Garantisce trasparenza e facilita la collaborazione tra membri del team.

\end{document}

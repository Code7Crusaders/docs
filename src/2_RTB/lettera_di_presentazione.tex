\documentclass{article}
\usepackage{graphicx}
\usepackage{fancyhdr}
\usepackage{geometry}
\usepackage{setspace}
\usepackage{eurosym}
\usepackage{tikz}
\usepackage[italian]{babel}
\usepackage[hidelinks]{hyperref}

% Margini della pagina
\geometry{a4paper, margin=1in}

% Intestazione personalizzata
\pagestyle{fancy}
\fancyhf{}
\fancyhead[L]{Code7Crusaders - Software Development Team}
\fancyhead[R]{\thepage}

% Spaziatura delle righe
\setstretch{1.2}

\begin{document}

% Sezione del titolo
\begin{titlepage}

    \AddToHookNext{shipout/background}{
    \begin{tikzpicture}[remember picture,overlay]
    \node at (current page.center) {
    \includegraphics{../img/background.png}
    };
    \end{tikzpicture}
    }  
  
    \centering
    \vspace*{2cm}
    
    \includegraphics[width=0.3\textwidth]{../img/logo/7Crusaders_logo.png} % Aggiungi il logo qui
    \vspace{1cm}
    
    {\Huge \textbf{Code7Crusaders}}\\
    \vspace{0.5cm}
    {\Large Software Development Team}\\
    \vspace{2cm}
    
    {\large \textbf{Lettera di Presentazione}}\\
    \vspace{5cm}

    \textbf{Membri del Team:}\\
    Enrico Cotti Cottini, Gabriele Di Pietro, Tommaso Diviesti \\
    Francesco Lapenna, Matthew Pan, Eddy Pinarello, Filippo Rizzolo \\
    \vspace{0.5cm}
    
    \vspace{1cm}
\end{titlepage}

Ai professori Tullio Vardanega e Riccardo Cardin.
Con il presente documento, il gruppo \textit{Code7Crusaders} desidera annunciare la propria intenzione di sostenere
la revisione RTB (\textit{Requirements and Tecnology Baseline}) del progetto
\begin{center}
    \textit{LLM: Assistente Virtuale}
\end{center}
proposto dall'azienda \textbf{Ergon Informatica}.\\
La completa documentazione inerente al progetto è visibile attraverso il seguente link:
\begin{center}
    \texttt{https://code7crusaders.github.io}
\end{center}
Nello specifico è presente una release \textit{RTB} all'interno della quale sono visibili i documenti sviluppati finora, tra cui:
\begin{itemize}
    \item \textbf{Documenti Esterni:}
    \begin{itemize}
        \item \texttt{Analisi dei Requisiti v1.0;}
        \item \texttt{Piano di Progetto v1.0;}
        \item \texttt{Piano di Qualifica v1.0;}
    \end{itemize}
    \item \textbf{Documenti Interni:}
    \begin{itemize}
        \item \texttt{Norme di Progetto v1.0;}
        \item \texttt{Glossario v1.0;}
    \end{itemize}
    \item \textbf{Verbali Esterni}
    \item \textbf{Verbali Interni}
\end{itemize}

Di seguito viene fornito il link al repository GitHub del gruppo contenente il \textbf{Proof of Concept} del progetto:
\begin{center}
    \texttt{inserire qui link o + link}
\end{center}
Il costo di realizzazione del progetto ... aggiungere roba se è aumentato altrimenti dire che è uguale all'altro\\
\newpage
Di seguito vengono riportati i nomi dei componenti del gruppo \textbf{Code7Crusaders}:\\
\vspace{2cm}
\begin{tabular}{|c|c|}
    \hline
    \textbf{Nome} & \textbf{Matricola} \\
    \hline
    Enrico Cotti Cottini &  \\ 
    \hline
    Gabriele Di Pietro & 2010000 \\ 
    \hline
    Tommaso Diviesti &  \\ 
    \hline %linea di fine
    Francesco Lapenna &  \\ 
    \hline
    Matthew Pan & \\ 
    \hline %linea di fine
    Eddy Pinarello & \\ 
    \hline %linea di fine
    Filippo Rizzolo & \\ 
    \hline %linea di fine
\end{tabular}
\\
\vspace{3cm}
Nell'attesa di un cortese riscontro, porgiamo distinti saluti, \textbf{Code7Crusaders}

\begin{table}[b]
	\begin{tabular}{@{}p{2in}p{3in}@{}}
			   &     		\\
			   &     		\\
		\textbf{Firma Responsabile:} & \hrulefill \\
	\end{tabular}
\end{table}


\end{document} 



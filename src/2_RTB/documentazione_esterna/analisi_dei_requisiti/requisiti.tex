\section{Requisiti}
In questa sezione vengono presentati i requisiti emersi durante l'attività di analisi, 
condotta a partire dai casi d'uso, dall'esame del capitolato d'appalto e dagli incontri, 
sia interni che con il proponente. 

\subsection{Classificazione dei requisiti}
I requisiti sono classificati in tre categorie principali:  
\begin{itemize}
	\item \textbf{Funzionali}: riguardano l'usabilità del prodotto finale;  
	\item \textbf{Di qualità}: includono gli strumenti e la documentazione da fornire;  
	\item \textbf{Di vincolo}: fanno riferimento alle tecnologie da utilizzare.
\end{itemize}
Ciascun requisito è indicato da:
\begin{itemize}
	\item \textbf{Codice Identificativo}: codice univoco che identifica il requisito;
	\item \textbf{Descrizione}: breve spiegazione del requisito;
	\item \textbf{Fonte}: origine del requisito (es. capitolato, interno, ecc.);
	\item \textbf{Priorità}: importanza del requisito rispetto agli altri;
\end{itemize} 

\subsection{Fonti dei requisiti}
I requisiti sono stati identificati a partire dalle seguenti fonti:
\begin{itemize}
	\item \textbf{Capitolato}: requisiti individuati tramite analisi del capitolato;
	\item \textbf{Interno}: requisiti individuati durante riunioni interne al gruppo di lavoro;
	\item \textbf{Esterno}: requisiti individuati durante incontri con il proponente;
	\item \href{https://code7crusaders.github.io/docs/RTB/documentazione_interna/glossario.html#piano-di-qualifica}{\textbf{Piano di Qualifica}\textsuperscript{G}}: requisiti necessari per rispettare standard di qualità definiti nel documento \href{https://code7crusaders.github.io/docs/RTB/documentazione_interna/glossario.html#piano-di-qualifica}{Piano di Qualifica\textsuperscript{G}};
	\item \href{https://code7crusaders.github.io/docs/RTB/documentazione_interna/glossario.html#norme-di-progetto}{\textbf{Norme di Progetto}\textsuperscript{G}}: requisiti necessari per rispettare le norme di progetto definite nel documento \href{https://code7crusaders.github.io/docs/RTB/documentazione_interna/glossario.html#norme-di-progetto}{Norme di Progetto\textsuperscript{G}}\href{https://code7crusaders.github.io/docs/RTB/documentazione_interna/glossario.html#norme-di-progetto}{norme di progetto\textsuperscript{G}} definite nel documento Norme di Progetto;
\end{itemize}
\newpage
\subsection{Codifica dei requisiti}
I requisiti sono codificati come segue: \textbf{R[Tipo][Importanza][Numero]}
\newline
Dove \textbf{Tipo} può essere:
\begin{itemize}
	\item \textbf{F (funzionale)}
	\item \textbf{Q (di qualità)}
	\item \textbf{V (di vincolo)}
\end{itemize}
\textbf{Importanza} può essere:
\begin{itemize}
	\item \textbf{O (obbligatorio)}
	\item \textbf{D (desiderabile)}
	\item \textbf{F (facoltativo)}
\end{itemize}
\textbf{Numero} è un numero identificativo univoco del requisito.

\textbf{Esempio}:
\begin{itemize}
	\item \textbf{RFO1}: requisito funzionale obbligatorio numero 1
	\item \textbf{RQD2}: requisito di qualità desiderabile numero 2
	\item \textbf{RVF3}: requisito di vincolo facoltativo numero 3
\end{itemize}

\pagebreak
\subsection{Requisiti funzionali}
\begin{longtable}{|>{\centering\arraybackslash}m{0.10\textwidth}|>{\centering\arraybackslash}m{0.20\textwidth}|>{\arraybackslash}m{0.6\textwidth}|}
	\hline
	\textbf{Codice} & \textbf{Fonte} & \textbf{Descrizione}\\\hline
	\endfirsthead
	\hline
	\textbf{Codice} & \textbf{Fonte} & \textbf{Descrizione}\\\hline
	\endhead
	\hline
	\textbf{RFO1} & Capitolato & Il sistema riceve in ingresso i dati semantici aziendali da cui apprendere la conoscenza. I documenti accettati in ingresso dal componente dovranno rispettare i formati previsti (\texttt{.pdf, .txt}) \\
	\hline
	\textbf{RFO2} & Capitolato & I testi recuperati dai documenti verranno suddivisi in blocchi, ovvero pezzi più piccoli di dati che rappresentano una piccola porzione del contesto\\
	\hline
	\textbf{RFO3} & Capitolato & \href{https://code7crusaders.github.io/docs/RTB/documentazione_interna/glossario.html#embedding}{Embedding\textsuperscript{G}} model è il componente che riceverà in ingresso i blocchi di testo e lì trasformerà in rappresentazioni vettoriali\\
	\hline
	\textbf{RFO4} & Capitolato & I vettori generati verranno memorizzati all’interno di un \href{https://code7crusaders.github.io/docs/RTB/documentazione_interna/glossario.html#database-vettoriale}{database vettoriale\textsuperscript{G}} e opportunamente indicizzati\\
	\hline
	\textbf{RFO5} & Capitolato, Esterno & Da un’interfaccia utente della web app, viene catturata una domanda da parte dell’utente\\
	\hline
	\textbf{RFO6} & Capitolato & La domanda viene inoltrata al sistema attraverso delle \href{https://code7crusaders.github.io/docs/RTB/documentazione_interna/glossario.html#api-rest-representational-state-transfer}{API REST\textsuperscript{G}} risiedenti in un Web Server\\
	\hline
	\textbf{RFO7} & Capitolato & La query ricevuta viene gestita dall’\href{https://code7crusaders.github.io/docs/RTB/documentazione_interna/glossario.html#embedding}{Embedding\textsuperscript{G}} Model che trasforma la domanda in rappresentazione vettoriale \\
	\hline
	\textbf{RFO8} & Capitolato & La rappresentazione vettoriale viene utilizzata per effettuare una ricerca all’interno del \href{https://code7crusaders.github.io/docs/RTB/documentazione_interna/glossario.html#database-vettoriale}{database vettoriale\textsuperscript{G}} da dove vengono reperiti i vettori più simili\\
	\hline
	\textbf{RFO9} & Capitolato & Sia la domanda sia i risultati della ricerca nel \href{https://code7crusaders.github.io/docs/RTB/documentazione_interna/glossario.html#database-vettoriale}{database vettoriale\textsuperscript{G}}, vengono inviati al sistema \href{https://code7crusaders.github.io/docs/RTB/documentazione_interna/glossario.html#llm-large-language-model}{LLM\textsuperscript{G}} che costruirà la risposta utilizzando il contesto fornito\\
	\hline
	\textbf{RFO10} & Capitolato & Attraverso \href{https://code7crusaders.github.io/docs/RTB/documentazione_interna/glossario.html#api-rest-representational-state-transfer}{API REST\textsuperscript{G}}, il sistema inoltra la riposta all'account dell’utente\\
	\hline
	\textbf{RFO11} & Interno & Il sistema deve permettere all'utente registrato di effettuare una conversazione con il bot\\
	\hline
	\textbf{RFO12} & Interno & L'utente deve essere in grado di ottenere informazioni riguardo un prodotto o una serie di prodotti attraverso la conversazione con il bot\\
	\hline
	\textbf{RFO13} & Interno & L'utente deve essere in grado di salvare una conversazione avviata\\
	\hline
	\textbf{RFO14} & Interno & L'utente deve essere in grado di visualizzare le conversazioni precedentemente salvate\\
	\hline
	\textbf{RFO15} & Interno & L'utente deve essere in grado di visualizzare e riprendere una intera conversazione singola salvata\\
	\hline
	\textbf{RFO16} & Interno & L'utente o L'amministratore devono poter accedere al sistema inserendo Username e Password\\
	\hline
	\textbf{RFO17} & Interno & Il sistema deve bloccare utenti non registrati\\
	\hline
	\textbf{RFO18} & Interno & Il sistema deve bloccare qualsiasi tentativo di rottura del sistema (SQL Injection...)\\
	\hline
	\textbf{RFO19} & Interno, Esterno & L'utente deve essere in grado di dare un feedback sulla qualità della conversazione dopo averla provata\\
	\hline
	\textbf{RFO20} & Esterno & L'amministratore deve essere in grado di creare \href{https://code7crusaders.github.io/docs/RTB/documentazione_interna/glossario.html#template}{template\textsuperscript{G}} di domande e risposta\\
	\hline
	\textbf{RFO21} & Esterno & L'amministratore deve essere in grado di modificare \href{https://code7crusaders.github.io/docs/RTB/documentazione_interna/glossario.html#template}{template\textsuperscript{G}} di domande e risposta\\
	\hline
	\textbf{RFO22} & Interno & L'amministratore deve essere in grado di eliminare un \href{https://code7crusaders.github.io/docs/RTB/documentazione_interna/glossario.html#template}{template\textsuperscript{G}} precedentemente creato\\
	\hline
	\textbf{RFF23} & Interno, Esterno & Il sistema deve poter fermare la creazione di un formato \href{https://code7crusaders.github.io/docs/RTB/documentazione_interna/glossario.html#template}{template\textsuperscript{G}} invalido\\
	\hline
	\textbf{RFO24} & Interno & L'amministratore deve aver accesso ad una dashboard che gli permetta di monitorare le prestazioni di sistema\\
	\hline
	\textbf{RFO25} & Interno & L'amministratore deve poter Visuallizzare i feedback dati dagli utenti\\
	\hline
	\textbf{RFO26} & Esterno & L'amministratore deve poter importare dati da un documento esterno\\
	\hline
	\textbf{RFO27} & Esterno & Il sistema deve poter fermare l'importazione dati di file non compatibili\\
	\hline
	\textbf{RFO28} & Interno & L'utente deve poter eliminare una conversazione precedentemente effettuata\\
	\hline
	\textbf{RFD29} & Esterno & L'utente deve poter mandare richieste di assistenza per poter parlare con un operatore umano\\
	\hline
	\textbf{RFO30} & Esterno & L'amministratore deve poter visualizzare le richieste di assistenza ricevute da parte dell'utente\\
	\hline
	\textbf{RFD31} & Interno & L'amministratore deve poter segnalare ad altri amministratori che una richiesta è stata presa in carico\\
	\hline
	\textbf{RFD32} & Esterno, Interno & L'ammistratore deve essere in grado di poter rispondere all'utente tramite contatto via e-mail\\
	\hline
	\textbf{RFO33} & Interno & L'amministratore deve essere in grado di visualizzare l'utilizzo generale degli utenti del servizio\\
	\hline
	\textbf{RFD34} & Interno & L'amministratore deve essere in grado di visualizzare i costi del sistema\\
	\hline
	\textbf{RFO35} & Interno, Esterno & Ogni messaggio lasciato dal bot può essere valutato pertanto è necessario che questo dato venga salvato in modo (positivo/negativo)\\
	\hline 
	\textbf{RFO36} & Interno & Un Amministratore deve poter modificare o aggiungere un \href{https://code7crusaders.github.io/docs/RTB/documentazione_interna/glossario.html#template}{template\textsuperscript{G}} pertanto è necessario garantire la persistenza dell'ultima modifica.\\
	\hline
	\textbf{RFD37} & Interno & Ogni Prodotto può essere mostrato con una propria immagine\\
	\hline
	\textbf{RFO38} & Interno (Analisi Vettoriale) & I Documenti inseriti da Amministratore, devono essere spezzattati ed embeddati tramite il modello di embedding (Openai embedded large), per essere vettorializzati\\
	\hline
	\textbf{RFD39} & Interno (Analisi Backend) & Le metriche delle run del chatbot devono essere esportatabili in JSON\\
	\hline
	\textbf{RFD40} & Interno (Analisi Backend) & Le metriche della run devono includere i seguenti valori: ID univoco della run, nome assegnato alla sessione, dati di input elaborati dal modello, timestamp di avvio e completamento dell'esecuzione, eventuali errori incontrati, risultato generato dal modello, numero totale di token utilizzati e stima dei costi basata sul consumo di token.\\
	\hline
	\textbf{RFO41} & Interno & Il bot per rispondere ad una domanda deve ricordarsi i messaggi precedenti nella singola conversazione\\
	\hline
    \textbf{RFD42} & Interno & Il sistema deve notificare l'utente quando la memoria per le chat salvate è piena e non è possibile salvare ulteriori conversazioni\\
    \hline
	\caption{Requisiti funzionali}
\end{longtable}

\pagebreak
\subsection{Requisiti qualitativi}
\begin{longtable}{|>{\centering\arraybackslash}m{0.10\textwidth}|>{\centering\arraybackslash}m{0.20\textwidth}|>{\centering\arraybackslash}m{0.6\textwidth}|}
	\hline
	\textbf{Codice} & \textbf{Fonte} & \textbf{Descrizione}\\\hline
	\endfirsthead
	\hline
	\textbf{Codice} & \textbf{Fonte} & \textbf{Descrizione}\\\hline
	\endhead
	\hline
	\textbf{RQO1} & Capitolato, \href{https://code7crusaders.github.io/docs/RTB/documentazione_interna/glossario.html#piano-di-qualifica}{Piano di Qualifica\textsuperscript{G}} & Schema di progettazione della base di dati \\
	\hline
	\textbf{RQO2} & Capitolato, \href{https://code7crusaders.github.io/docs/RTB/documentazione_interna/glossario.html#piano-di-qualifica}{Piano di Qualifica\textsuperscript{G}} & Codice prodotto in formato sorgente reso disponibile tramite repository pubblici \\
	\hline
	\textbf{RQO3} & \href{https://code7crusaders.github.io/docs/RTB/documentazione_interna/glossario.html#piano-di-qualifica}{Piano di Qualifica\textsuperscript{G}} & Documentazione riassuntiva delle metriche e dei risultati\\
	\hline
	\textbf{RQO4} & \href{https://code7crusaders.github.io/docs/RTB/documentazione_interna/glossario.html#piano-di-qualifica}{Piano di Qualifica\textsuperscript{G}} & Il software deve essere testato con una copertura di codice minima dell'80\% e una copertura dei rami dell'80\%, con un obiettivo ottimale del 100\% \\
	\hline
	\textbf{RQO5} & \href{https://code7crusaders.github.io/docs/RTB/documentazione_interna/glossario.html#piano-di-qualifica}{Piano di Qualifica\textsuperscript{G}} & Il 90\% dei test deve essere superato come requisito minimo, mentre l'obiettivo ottimale è il 100\% \\
	\hline
	\textbf{RQO6} & \href{https://code7crusaders.github.io/docs/RTB/documentazione_interna/glossario.html#piano-di-qualifica}{Piano di Qualifica\textsuperscript{G}} & La metodologia di sviluppo deve seguire il paradigma del Test Driven Development (TDD), garantendo che il codice venga scritto partendo dai test \\
	\hline
	\caption{Requisiti qualitativi}
\end{longtable}


\pagebreak
\subsection{Requisiti di vincolo}
\begin{longtable}{|>{\centering\arraybackslash}m{0.10\textwidth}|>{\centering\arraybackslash}m{0.20\textwidth}|>{\centering\arraybackslash}m{0.60\textwidth}|}
	\hline
	\textbf{Codice} & \textbf{Fonte} & \textbf{Descrizione}\\\hline
	\endfirsthead
	\hline
	\textbf{Codice} & \textbf{Fonte} & \textbf{Descrizione}\\\hline
	\endhead
	\hline
	\textbf{RVO1} & Capitolato & Il chatbot deve rispondere con il contesto dato dai file di allenamento (pdf o file di testo inseriti)\\
	\hline
	\textbf{RVO2} & Capitolato & \href{https://code7crusaders.github.io/docs/RTB/documentazione_interna/glossario.html#llm-large-language-model}{LLM\textsuperscript{G}} deve essere integrato tramite API\\
	\hline
	\textbf{RVO3} & Interno (Analisi dei modelli) & LLM utilizzato deve essere quello di OpenAI\\
	\hline
	\textbf{RVO4} & Capitolato & Deve essere usato un database relazionale\\
	\hline
	\textbf{RVO5} & Interno (Analisi del Database) & Deve essere gestito il salvataggio delle chat precedenti con tutti i messaggi in esse tramite un database relazionale con PostgreSQL\\
	\hline
	\textbf{RVO6} & Capitolato & Deve essere implementato un database vettoriale\\
	\hline
	\textbf{RVO7} & Interno (Analisi dei modelli) & Deve essere implementato un database vettoriale FAISS per poter rendere possibile la ricerca con contesto dall'LLM\\
	\hline
	\textbf{RVO8} & Capitolato & Deve essere implementato un \href{https://code7crusaders.github.io/docs/RTB/documentazione_interna/glossario.html#embedding}{embedding\textsuperscript{G}} model\\
	\hline
	\textbf{RVO9} & Interno (Analisi dei modelli) & L'embedding model deve essere quello di OpenAI\\
	\hline
	\textbf{RVO10} & Capitolato & Deve essere implementata una WebApp che permetta di comunicare con il chatbot\\
	\hline
	\textbf{RVO11} & Interno (Analisi \href{https://code7crusaders.github.io/docs/RTB/documentazione_interna/glossario.html#frontend}{Frontend\textsuperscript{G}}) & L’interfaccia deve essere costruita utilizzando componenti funzionali React.\\
	\hline
	\textbf{RVO12} & Interno & Si deve creare un backend che gestisca le chiamate HTTP, il database vettoriale e il database relazionale con Flask.\\
	\hline
	\textbf{RVO13} & Interno & La gestione dello stato locale deve essere implementata tramite useState.\\
	\hline
	\textbf{RVO14} & Interno & La WebApp deve utilizzare React Router per gestire la navigazione tra le pagine.\\
	\hline
	\textbf{RVO15} & Interno & Gli stili devono essere gestiti tramite CSS inline o con className per garantire modularità.\\
	\hline
	\textbf{RVO16} & Interno & La comunicazione tra componenti deve essere gestita inviando funzioni come \href{https://code7crusaders.github.io/docs/RTB/documentazione_interna/glossario.html#props}{props\textsuperscript{G}}.\\
	\hline
	\textbf{RVO17} & Interno & La WebApp deve essere responsiva e adattarsi dinamicamente alle dimensioni della finestra.\\
	\hline
	\textbf{RVO18} & Interno (Analisi Vettoriale) & La gestione dei blocchi di testo vettorializzati deve essere gestita tramite Faiss\\
	\hline
	\textbf{RVD19} & Interno (Analisi Backend) & Le metriche delle run del chatbot devono essere recuperate tramite Langsmith\\
	\hline
	\textbf{RVO20} & Interno (Analisi Backend) & Bisogna usare la libreria LangChain per la interazione con i modelli LLM e Embedding\\
	\hline
	\caption{Requisiti di vincolo}
\end{longtable}


\pagebreak
\subsection{Tracciamento}
\subsubsection{Requisito - Fonte}
\begin{longtable}{|>{\centering\arraybackslash}m{0.40\textwidth}|>{\centering\arraybackslash}m{0.4\textwidth}|}
	\hline
	\textbf{Requisito} & \textbf{Fonte} 
	\endfirsthead
	\hline
	\textbf{Requisito} & \textbf{Fonte} 
	\endhead
	\hline
	\textbf{RFO1}            & Capitolato\\\hline
	\textbf{RFO2}            & Capitolato\\\hline
	\textbf{RFO3}            & Capitolato\\\hline
	\textbf{RFO4}            & Capitolato\\\hline
	\textbf{RFO5}            & Capitolato, Esterno\\\hline
	\textbf{RFO6}            & Capitolato\\\hline
	\textbf{RFO7}            & Capitolato\\\hline
	\textbf{RFO8}            & Capitolato\\\hline
	\textbf{RFO9}            & Capitolato\\\hline
	\textbf{RFO10}            & Capitolato\\\hline
	\textbf{RFO11}            & Interno\\\hline
	\textbf{RFO12}            & Interno\\\hline
	\textbf{RFO13}            & Interno\\\hline
	\textbf{RFO14}            & Interno\\\hline
	\textbf{RFO15}            & Interno\\\hline
	\textbf{RFO16}            & Interno\\\hline
	\textbf{RFO17}            & Interno\\\hline
	\textbf{RFO18}            & Interno\\\hline
	\textbf{RFO19}            & Interno, Esterno\\\hline
	\textbf{RFO20}            & Esterno\\\hline
	\textbf{RFO21}            & Esterno\\\hline
	\textbf{RFO22}            & Interno\\\hline
	\textbf{RFF23}            & Interno, Esterno\\\hline
	\textbf{RFO24}            & Interno\\\hline
	\textbf{RFO25}            & Interno\\\hline
	\textbf{RFO26}            & Esterno\\\hline
	\textbf{RFO27}            & Esterno\\\hline
	\textbf{RFO28}            & Interno\\\hline
	\textbf{RFD29}            & Esterno\\\hline
	\textbf{RFO30}            & Esterno\\\hline
	\textbf{RFD31}            & Interno\\\hline
	\textbf{RFD32}            & Esterno, Interno\\\hline
	\textbf{RFO33}            & Interno\\\hline
	\textbf{RFD34}            & Interno\\\hline
	\textbf{RFO35}            & Interno, esterno\\\hline
	\textbf{RFO36}            & Interno\\\hline
	\textbf{RFD37}            & Interno\\\hline	
	\textbf{RFO38}            & Interno (Analisi Vettoriale) \\\hline
	\textbf{RFD39}            & Interno (Analisi Backend) \\\hline
	\textbf{RFD40}            & Interno (Analisi Backend) \\\hline
	\textbf{RFO41}            & Interno \\\hline
	\textbf{RFD42}            & Interno \\\hline
    
	\textbf{RQO1}            & Capitolato, \href{https://code7crusaders.github.io/docs/RTB/documentazione_interna/glossario.html#piano-di-qualifica}{Piano di Qualifica\textsuperscript{G}}\\\hline
	\textbf{RQO2}            & Capitolato, \href{https://code7crusaders.github.io/docs/RTB/documentazione_interna/glossario.html#piano-di-qualifica}{Piano di Qualifica\textsuperscript{G}}\\\hline
	\textbf{RQO3}            & \href{https://code7crusaders.github.io/docs/RTB/documentazione_interna/glossario.html#piano-di-qualifica}{Piano di Qualifica\textsuperscript{G}}\\\hline
    \textbf{RQO4}            & \href{https://code7crusaders.github.io/docs/RTB/documentazione_interna/glossario.html#piano-di-qualifica}{Piano di Qualifica\textsuperscript{G}}\\\hline
	\textbf{RQO5}            & \href{https://code7crusaders.github.io/docs/RTB/documentazione_interna/glossario.html#piano-di-qualifica}{Piano di Qualifica\textsuperscript{G}}\\\hline
	\textbf{RQO6}            & \href{https://code7crusaders.github.io/docs/RTB/documentazione_interna/glossario.html#piano-di-qualifica}{Piano di Qualifica\textsuperscript{G}}\\\hline

    \textbf{RVO1}			 & Capitolato \\\hline
	\textbf{RVO2}			 & Capitolato \\\hline
	\textbf{RVO3}			 & Interno (Analisi dei modelli)\\\hline
	\textbf{RVO4}			 & Capitolato \\\hline
	\textbf{RVO5}			 & Interno (Analisi Database) \\\hline
	\textbf{RVO6}			 & Capitolato \\\hline
	\textbf{RVO7}			 & Interno (Analisi dei modelli) \\\hline
	\textbf{RVO8}			 & Capitolato \\\hline
	\textbf{RVF9}			 & Interno (Analisi dei modelli) \\\hline
	\textbf{RVF10}			 & Capitolato \\\hline
	\textbf{RVD11}			 & Interno (Analisi \href{https://code7crusaders.github.io/docs/RTB/documentazione_interna/glossario.html#frontend}{Frontend\textsuperscript{G}})\\\hline
	\textbf{RVO12}			 & Interno\\\hline
	\textbf{RVO13}			 & Interno\\\hline
	\textbf{RVO14}			 & Interno\\\hline
	\textbf{RVO16}           & Interno  \\\hline
	\textbf{RVO15}           & Interno  \\\hline
	\textbf{RVO17}           & Interno  \\\hline
	\textbf{RVO18}           & Interno (Analisi Vettoriale)\\\hline
	\textbf{RVD19}           & Interno (Analisi Backend)\\\hline
	\textbf{RVO20}           & Interno (Analisi Backend)\\\hline
	\caption{Requisito - Fonte}
\end{longtable}
\pagebreak
\subsection{Caso d'uso - Requisito}
\begin{longtable}{|>{\centering\arraybackslash}m{0.40\textwidth}|>{\centering\arraybackslash}m{0.4\textwidth}|}
	\hline
	\textbf{Caso d'uso} & \textbf{Requisito}\\
	\endfirsthead
	\hline
	U.C.1 & RFO5, RFO11 \\
	\hline
	U.C.2 & RFO2, RFO6, RFO7, RFO8, RFO9, RFO10, RFO12, RFD37, RFO41 \\
	\hline
	U.C.2.1 & RFO8, RFO9 \\
	\hline
	U.C.3 & RFO5, RFF36 \\
	\hline
	U.C.4 & RFO14\\
	\hline
	U.C.5 & RFO14\\
	\hline
	U.C.6 & RFO16\\
	\hline
	U.C.6.1 & RFO16\\
	\hline
	U.C.6.2 & RFO16\\
	\hline
	U.C.6.3 & RFO17\\
	\hline
	U.C.6.4 & RFO17\\
	\hline
	U.C.6.5 & RFO17, RFO18\\
	\hline
	U.C.7 & RFO16\\
	\hline
	U.C.7.1 & RFO16\\
	\hline
	U.C.7.2 & RFO16\\
	\hline
	U.C.7.3 & RFO17, RFO18\\
	\hline
	U.C.7.4 & RFO17\\
	\hline
	U.C.7.5 & RFO17\\
	\hline
	U.C.7.6 & RFO17\\
	\hline
	U.C.8 & RFO13\\
	\hline
	U.C.8.1 & RFD42 \\
	\hline
	U.C.9 & RFO19, RFO36\\
	\hline
	U.C.10 & RFO20, RFO36\\
	\hline
	U.C.11 & RFO21, RFO36\\
	\hline
	U.C.12 & RFO22, RFO36\\
	\hline
	U.C.13 & RFF23, RFO36 \\
	\hline
	U.C.14 & RFO24, RFO33, RFD34, RFD40\\
	\hline
	U.C.15 & RFO25\\
	\hline
	U.C.16 & RFO1, RFO3, RFO4, RFO26, RFO38 \\
	\hline
	U.C.16.1 & RFO1, RFO27\\
	\hline
	U.C.17 & RFD39, RFD40\\
	\hline
	U.C.18 & RFO28\\
	\hline
	U.C.19 & RFO15\\
	\hline
	U.C.20 & RFO30, RFD32\\
	\hline
	U.C.21 & RFD31\\
	\hline
	U.C.22 & RFD29\\
	\hline
	\caption{Caso d'uso - Requisito}
\end{longtable}
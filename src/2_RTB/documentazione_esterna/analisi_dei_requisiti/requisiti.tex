\section{Requisiti}
In questa sezione vengono presentati i requisiti emersi durante l'attività di analisi, 
condotta a partire dai casi d'uso, dall'esame del capitolato d'appalto e dagli incontri, 
sia interni che con il proponente. 

\subsection{Classificazione dei requisiti}
I requisiti sono classificati in tre categorie principali:  
\begin{itemize}
    \item \textbf{Funzionali}: riguardano l'usabilità del prodotto finale;  
    \item \textbf{Di qualità}: includono gli strumenti e la documentazione da fornire;  
    \item \textbf{Di vincolo}: fanno riferimento alle tecnologie da utilizzare.
\end{itemize}
Per ciascun requisito è indicato:
\begin{itemize}
    \item \textbf{Codice Identificativo}: codice univoco che identifica il requisito;
    \item \textbf{Descrizione}: breve spiegazione del requisito;
    \item \textbf{Fonte}: origine del requisito (es. capitolato, interno, ecc..);
    \item \textbf{Priorità}: importanza del requisito rispetto agli altri;
\end{itemize} 

\subsection{Fonti dei requisiti}
I requisiti sono stati identificati a partire dalle seguenti fonti:
\begin{itemize}
    \item \textbf{Capitolato}: Requisiti individuati tramite analisi del capitolato;
    \item \textbf{interno}: requisiti individuati durante riunioni interne al gruppo di lavoro;
    \item \textbf{Esterno}: requisiti individuati durante incontri con il proponente;
    \item \textbf{Piano di Qualifica}: Requisiti necessari per rispettare standard di qualità definiti nel documento Piano di Qualifica;
    \item \textbf{Norme di Progetto}: Requisiti necessari per rispettare le norme di progetto definite nel documento Norme di Progetto;
\end{itemize}

\subsection{Codifica dei requisiti}
I requisiti sono codificati come segue: \textbf{R[Tipo][Importanza][Numero]}
\newline
Dove \textbf{Tipo} può essere:
\begin{itemize}
    \item \textbf{F (funzionale)}
    \item \textbf{Q (di qualità)}
    \item \textbf{V (di vincolo)}
\end{itemize}
\textbf{Importanza} può essere:
\begin{itemize}
    \item \textbf{O (obbligatorio)}
    \item \textbf{D (desiderabile)}
    \item \textbf{F (facoltativo )}
\end{itemize}
\textbf{Numero} è un numero identificativo univoco del requisito.

\textbf{Esempio}:
\begin{itemize}
    \item RFO1: Requisito funzionale obbligatorio numero 1
    \item RQD2: Requisito di qualità desiderabile numero 2
    \item RVF3: Requisito di vincolo facoltativo numero 3
\end{itemize}

\pagebreak
\subsection{Requisiti funzionali}
\begin{longtable}{|>{\centering\arraybackslash}m{0.10\textwidth}|>{\centering\arraybackslash}m{0.20\textwidth}|>{\centering\arraybackslash}m{0.6\textwidth}|}
	\hline
	\textbf{Codice} & \textbf{Fonte} & \textbf{Descrizione}\\\hline
	\endfirsthead
	\hline
	\textbf{Codice} & \textbf{Fonte} & \textbf{Descrizione}\\\hline
	\endhead
	\hline
	RFO1            & Capitolato    & Requisito funzionale obbligatorio numero 1
	\\\hline
	\caption{Requisiti funzionali}
\end{longtable}

\pagebreak
\subsubsection{Requisiti Qualitativi}
\begin{longtable}{|>{\centering\arraybackslash}m{0.10\textwidth}|>{\centering\arraybackslash}m{0.20\textwidth}|>{\centering\arraybackslash}m{0.6\textwidth}|}
	\hline
	\textbf{Codice} & \textbf{Fonte} & \textbf{Descrizione}\\\hline
	\endfirsthead
	\hline
	\textbf{Codice} & \textbf{Fonte} & \textbf{Descrizione}\\\hline
	\endhead
	\hline
	RQD2            & Interno    & Requisito di qualità desiderabile numero 2
	\\\hline
	\caption{Requisiti Qualitativi}
\end{longtable}

\pagebreak
\subsubsection{Requisiti di vincolo}
\begin{longtable}{|>{\centering\arraybackslash}m{0.10\textwidth}|>{\centering\arraybackslash}m{0.20\textwidth}|>{\centering\arraybackslash}m{0.6\textwidth}|}
	\hline
	\textbf{Codice} & \textbf{Fonte} & \textbf{Descrizione}\\\hline
	\endfirsthead
	\hline
	\textbf{Codice} & \textbf{Fonte} & \textbf{Descrizione}\\\hline
	\endhead
	\hline
	RVF3            & Esterno    & Requisito di vincolo facoltativo numero 3
	\\\hline
	\caption{Requisiti di vincolo}
\end{longtable}

\subsubsection{Requisiti sistema operativo}

\subsubsection{Requisiti di prestazione}

\subsubsection{Requisiti di sicurezza}

\pagebreak
\subsection{Tracciamento}
\subsubsection{Requisito - Fonte}
\begin{longtable}{|>{\centering\arraybackslash}m{0.40\textwidth}|>{\centering\arraybackslash}m{0.4\textwidth}|}
	\hline
	\textbf{Requisito} & \textbf{Fonte} 
	\endfirsthead
	\hline
	\textbf{Requisito} & \textbf{Fonte} 
	\endhead
	\hline
	RFO1            & Capitolato    
	\\\hline
    RQD2            & Interno
    \\\hline
    RVF3            & Esterno
    \\\hline
	\caption{Requisito - Fonte}
\end{longtable}



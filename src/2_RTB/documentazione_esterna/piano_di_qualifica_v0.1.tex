%%% Settings %%%%%%%%%%%%%%%%%%%%%%%%%%%%%%%%%%%%%%%%%%%%%%%%%%%%%%%%%%%%%%%%%%%%%%%%%
\documentclass{article}

\usepackage{graphicx}  % serve per inserire immagini
\usepackage{fancyhdr}  % creazione header-footer
\usepackage{tabularx}  % serve per creare tabelle con colonne a larghezza variabile
\usepackage{ifthen}  % serve per mostrare cose diverse in base a condizioni
\usepackage{geometry}
\usepackage{setspace}
\usepackage{tikz}
\usepackage[italian]{babel}
\usepackage[hidelinks]{hyperref}
\usepackage{pgfgantt}  % per i diagrammi di Gantt
\usepackage{eurosym}
\usepackage{float}

% setta a 1 se il verbale è esterno, 0 se è interno
\newcommand{\isEsterno}{1}

% Margini della pagina
\geometry{a4paper, margin=1in}

% Intestazione personalizzata
\pagestyle{fancy}
\fancyhf{}
\fancyhead[L]{Code7Crusaders - Software Development Team}
\fancyhead[R]{\thepage}

% Spaziatura delle righe
\setstretch{1.2}

\begin{document}
\setcounter{secnumdepth}{5} % Permette la numerazione fino a \subparagraph
%%%%%%%%%%%%%%%%%%%%%%%%%%%%%%%%%%%%%%%%%%%%%%%%%%%%%%%%%%%%%%%%%%%%%%%%%%%%%%%%%%%%%%



%%% Sezione del titolo %%%%%%%%%%%%%%%%%%%%%%%%%%%%%%%%%%%%%%%%%%%%%%%%%%%%%%%%%%%%%%%
\begin{titlepage}

    \AddToHookNext{shipout/background}{
        \begin{tikzpicture}[remember picture,overlay]
        \node at (current page.center) {
            \includegraphics{../../img/background.png}
        };
        \end{tikzpicture}
    }

    \centering
    \vspace*{2cm}
    
    \includegraphics[width=0.3\textwidth]{../../img/logo/7Crusaders_logo.png} % logo
    \vspace{1cm}
    
    {\Huge \textbf{Code7Crusaders}}\\
    \vspace{0.5cm}
    {\Large Software Development Team}\\
    \vspace{2cm}
    
    {\large \textbf{Piano di Progetto}}\\
    \vspace{5cm}
    
    
    \textbf{Membri del Team:}\\
    Enrico Cotti Cottini, Gabriele Di Pietro, Tommaso Diviesti \\
    Francesco Lapenna, Matthew Pan, Eddy Pinarello, Filippo Rizzolo \\
    \vspace{0.5cm}
    
    \vspace{1cm}
\end{titlepage}
%%%%%%%%%%%%%%%%%%%%%%%%%%%%%%%%%%%%%%%%%%%%%%%%%%%%%%%%%%%%%%%%%%%%%%%%%%%%%%%%%%%%%%



% Versioni %%%%%%%%%%%%%%%%%%%%%%%%%%%%%%%%%%%%%%%%%%%%%%%%%%%%%%%%%%%%%%%%%%%%%%%%%%%
% \newpage
\begin{table}[h!]
\centering
\textbf{Versioni} \\ % Titolo sopra la tabella
\vspace{2mm} % Spazio tra il titolo e la tabella
\begin{tabular}{|c|c|c|c|c|}
    \hline
    \textbf{Ver.} & \textbf{Data} & \textbf{Autore} & \textbf{Verificatore} & \textbf{Descrizione} \\
    \hline
    0.3 & 16/12/2024 & Gabriele Di Pietro & Nome Verificatore & Stesura sezione 5 \\
    0.2 & 10/12/2024 & Gabriele Di Pietro & Nome Verificatore & Aggiunte tabelle \\
    0.1 & 05/12/2024 & Gabriele Di Pietro & Nome Verificatore & Prima stesura del documento \\  
    \hline
\end{tabular}
\caption{Versioni del documento}
\label{tab:versioni}
\end{table}
%%%%%%%%%%%%%%%%%%%%%%%%%%%%%%%%%%%%%%%%%%%%%%%%%%%%%%%%%%%%%%%%%%%%%%%%%%%%%%%%%%%%%%



% Indice %%%%%%%%%%%%%%%%%%%%%%%%%%%%%%%%%%%%%%%%%%%%%%%%%%%%%%%%%%%%%%%%%%%%%%%%%%%%%
% \newpage
\tableofcontents
%%%%%%%%%%%%%%%%%%%%%%%%%%%%%%%%%%%%%%%%%%%%%%%%%%%%%%%%%%%%%%%%%%%%%%%%%%%%%%%%%%%%%%



% Sezione Introduzione %%%%%%%%%%%%%%%%%%%%%%%%%%%%%%%%%%%%%%%%%%%%%%%%%%%%%%%%%%%%%%%
% \newpage
\section{Introduzione}
\subsection{Obiettivo del Documento}
Il documento ha lo scopo di definire le strategie di verifica e validazione per assicurare il corretto funzionamento e uno standard di qualità
dello strumento sviluppato e le attività che lo accompagnano. Sarà sottoposto a revisioni continue, così da poter seguire l'evoluzione del progetto.

\subsection{Glossario}
Il Glossario è uno strumento utilizzato per risolvere eventuali dubbi su termini specifici utilizzati nella redazione del documento. Esso conterrà la definizione dei 
termini evidenziati e sarà consultabile al seguente \href{https://code7crusaders.github.io/docs/RTB/documentazione_interna/glossario.html}{link}. I termini presenti in tale documento
saranno evidenziatida una 'G' al pedice.

\subsection{Riferimenti}
\subsubsection{Normativi}
\begin{itemize}
    \item \textbf{Regolamento del progetto} \\ \texttt{\url{https://www.math.unipd.it/~tullio/IS-1/2024/Dispense/PD1.pdf}}
    \item \textbf{{Norme del Progetto}} \\ \texttt{\url{inserire norme di progetto}}
\end{itemize}
\subsubsection{Informativi}
\begin{itemize}
    \item \textbf{Standard ISO/IEC 25010} \\ \texttt{\url{https://iso25000.com/index.php/en/iso-25000-standards/iso-25010}}
    \item \textbf{Standard ISO/IEC 12207:1995} \\ \texttt{\url{https://www.math.unipd.it/~tullio/IS-1/2009/Approfondimenti/ISO_12207-1995.pdf}}
    \item \textbf{Qualità di prodotto} \\ \texttt{\url{https://www.math.unipd.it/~tullio/IS-1/2024/Dispense/T07.pdf}}
    \item \textbf{Qualità di processo} \\ \texttt{\url{https://www.math.unipd.it/~tullio/IS-1/2024/Dispense/T08.pdf}}
    \item \textbf{Verifica e validazione}
    \begin{itemize}
        \item Introduzione \\ \texttt{\url{https://www.math.unipd.it/~tullio/IS-1/2024/Dispense/T09.pdf}}
        \item Analisi Statica \\ \texttt{\url{https://www.math.unipd.it/~tullio/IS-1/2024/Dispense/T10.pdf}}
        \item Analisi Dinamica \\ \texttt{\url{https://www.math.unipd.it/~tullio/IS-1/2024/Dispense/T11.pdf}}
    \end{itemize}
    \item \textbf{Capitolato d'appalto C7} \\ \texttt{\url{https://www.math.unipd.it/~tullio/IS-1/2024/Progetto/C7.pdf}}
    \item \textbf{Verbali esterni} \\ \texttt{\url{inserire link verbali}}
    \item \textbf{Verbali interni} \\ \texttt{\url{inserire link verbali}}
    \item \textbf{Analisi dei requisiti} \\ \texttt{\url{inserire link analisi dei req.}}
    \item \textbf{Glossario} \\ \texttt{\url{https://code7crusaders.github.io/docs/RTB/documentazione_interna/glossario.html}}
\end{itemize}
% \newpage
\section{Metriche di qualità}
La qualità di processo è un criterio fondamentale ed è alla base di ogni prodotto che
rispecchi lo stato dell’arte. Per raggiungere tale obiettivo è necessario sfruttare delle
pratiche rigorose che consentano lo svolgimento di ogni attività in maniera ottimale.
Al fine di valutare nel miglior modo possibile la qualità del prodotto e l’efficacia dei
processi, sono state definite delle metriche, meglio specificate nel documento Norme
di ProgettoG e qui di seguito riepilogate. Esse sono state suddivise utilizzando lo \textbf{Standard \texttt{ISO/IEC12207:1995}}, il quale separa i processi di ciclo di vita del software in processi di
base e/o primari, processi di supporto e processi organizzativi.
\subsection{Processi di base e/o primari}
\subsubsection{Fornitura} %Tabella
\begin{table}[H]
    \centering
    \renewcommand{\arraystretch}{1.5} % Aumenta lo spazio tra le righe
    \begin{tabular}{|c|l|c|c|}
        \hline
        \textbf{Codice} & \textbf{Nome} & \textbf{Ammissibile} & \textbf{Ottimale} \\
        \hline
        1PBM-PV & Planned Value & $PV \geq 0$ & $PV \leq BAC$ \\
        2PBM-ETC & Estimated to Complete & $ETC \geq 0$ & $ETC \leq EAC$ \\
        3PBM-EAC & Estimated at Completition & $EAC \leq BAC + 10\%$ & $EAC \leq BAC$ \\
        4PBM-EV & Earned Value & $EV \geq 0$ & $EV \leq EAC$ \\
        5PBM-AC & Actual Cost & $AC \geq 0$ & $AC \leq EAC$ \\
        6PBM-SV & Scheduled Variance & $SV \geq -10\%$ & $SV \geq 0\%$ \\
        7PBM-CV & Cost Variance & $CV \geq -10\%$ & $CV \geq 0\%$ \\
        8PBM-CPI & Cost Performance Index & $CPI \geq 0.8$ & $CPI \geq 1$ \\
        9PBM-SPI & Scheduled Performance Index & $SPI \geq 0.8$ & $SPI \geq 1$ \\
        10PBM-OTDR & On-Time Delivery Rate & $OTDR \geq 90\%$ & $OTDR \geq 95\%$ \\
        \hline
    \end{tabular}
    \label{tab:fornitura}
    \caption{Metriche di qualità per il processo di Fornitura}
\end{table}
% \newpage

\subsubsection{Sviluppo} %Non inserire nulla
\paragraph{Analisi dei requisiti}%Tabella
\textbf{(11PBM - 13PBM)}

\begin{table}[H]
    \centering
    \renewcommand{\arraystretch}{1.5} % Aumenta lo spazio tra le righe
    \begin{tabular}{|c|l|c|c|}
        \hline
        \textbf{Codice} & \textbf{Nome} & \textbf{Ammissibile} & \textbf{Ottimale} \\
        \hline
        11PBM-PRO & Percentuale Requisiti Obbligatori & $PRO = 100\%$ & $PRO = 100\%$ \\
        12PBM-PRO & Percentuale Requisiti Desiderabili & $PRD \geq 30\%$ & $PRD = 100\%$ \\
        13PBM-PRF & Percentuale Requisiti Facoltativi & $PRF \geq 0\%$ & $PRF = 100\%$ \\
        \hline
    \end{tabular}
    \label{tab:analisi_requisiti}
    \caption{Metriche di qualità per il processo di Analisi dei requisiti}
    \end{table}


\paragraph{Progettazione}%Tabella
\textbf{(14PBM)}

\begin{table}[H]
    \centering
    \renewcommand{\arraystretch}{1.5} % Aumenta lo spazio tra le righe
    \begin{tabular}{|c|l|c|c|}
        \hline
        \textbf{Codice} & \textbf{Nome} & \textbf{Ammissibile} & \textbf{Ottimale} \\
        \hline
        14PBM-PG & Profondità delle Gerarchie & $PG \leq 7$ & $PG \leq 5$ \\
        \hline
    \end{tabular}
    \label{tab:progettazione}
    \caption{Metriche di qualità per il processo di Progettazione}
\end{table}

\paragraph{Implementazione}%Tabella
\textbf{(15PBM - 18PBM)}

\begin{table}[H]
    \centering
    \renewcommand{\arraystretch}{1.5} % Aumenta lo spazio tra le righe
    \begin{tabular}{|c|l|c|c|}
        \hline
        \textbf{Codice} & \textbf{Nome} & \textbf{Ammissibile} & \textbf{Ottimale} \\
        \hline
        15PBM-PPM & Parametri per Metodo & $PPM \leq 7$ & $PPM \leq 5$\\
        16PBM-CPC & Campi per Classe & $CPC \leq 8$ & $CPC \leq 5$ \\
        17PBM-LCPM & Linea Di Commento Per Metodo & $LCPM \geq 50$ & $LCPM \geq 20$ \\
        18PBM-CCM & Complessità Ciclomatica Metrica & $CCM \leq 6$ & $CCM \leq 3$ \\
        \hline
    \end{tabular}
    \label{tab:codifica}
    \caption{Metriche di qualità per il processo di Codifica}
\end{table}

\paragraph{Verifica e Validazione}%Tabella
\textbf{(8PSM-CC - 12PSM-PTCP)}

\begin{table}[H]
    \centering
    \renewcommand{\arraystretch}{1.5} % Aumenta lo spazio tra le righe
    \begin{tabular}{|c|l|c|c|}
        \hline
        \textbf{Codice} & \textbf{Nome} & \textbf{Ammissibile} & \textbf{Ottimale} \\
        \hline
        8PSM-CC & Code Coverage & $CC \geq 80\%$ & $CC = 100\%$ \\
        9PSM-BC & Branch Coverage & $BC \geq 80\%$ & $BC = 100\%$ \\
        10PSM-SC & Statement Coverage & $SC \geq 80\%$ & $SC = 100\%$ \\
        11PSM-FD & Failure Density & $FD \leq 15\%$ & $FD = 0\%$ \\
        12PSM-PTCP & Passed Test Case Percentage & $PTCP \geq 90\%$ & $PTCP \geq 100\%$ \\
        \hline
    \end{tabular}
    \label{tab:verifica}
    \caption{Metriche di qualità per il processo di Verifica}
\end{table}
%%%%%%%%%%%%%%%%%NON SO PERCHÈ SIA BUGGATO E NON LA POSIZIONA NEL PUNTO GIUSTO MEttendo sotto il titolo
%% SE METTI DEL TESTO DOPO ANALISI DEI REQUISITI SI SISTEMA :(  per quel paragrafo quindi andrebbe aggiunto del testo per ogni paragrafo%%
% \newpage

\subsection{Processi di Supporto}
\subsubsection{Documentazione}%Tabella
\textbf{(1PSM-IG - 2PSM-CO)}

\begin{table}[H]
    \centering
    \renewcommand{\arraystretch}{1.5} % Aumenta lo spazio tra le righe
    \begin{tabular}{|c|l|c|c|}
        \hline
        \textbf{Codice} & \textbf{Nome} & \textbf{Ammissibile} & \textbf{Ottimale} \\
        \hline
        1PSM-IG & Indice di Gulpease & $IG \geq 50$ & $IG \geq 75$ \\
        2PSM-CO & Correttezza Ortografica & $CO = 0$ errori & $CO = 0$ errori \\ 
        \hline
    \end{tabular}
    \label{tab:documentazione}
    \caption{Metriche di qualità per il processo di Documentazione}
\end{table}

\subsubsection{Gestione della qualità}%Tabella
\textbf{(3PSM-FU - 7PSM-QMS)}
\begin{table}[H]
    \centering
    \renewcommand{\arraystretch}{1.5} % Aumenta lo spazio tra le righe
    \begin{tabular}{|c|l|c|c|}
        \hline
        \textbf{Codice} & \textbf{Nome} & \textbf{Ammissibile} & \textbf{Ottimale} \\
        \hline
        3PSM-FU & Facilità di Utilizzo & $FU \geq 3$ errori & $FU \geq 0$ errori \\
        4PSM-TA & Tempo di Apprendimento & $TA \leq 12$ minuti & $TA \leq 8$ minuti \\
        5PSM-TR & Tempo di Risposta & $TR \leq 8$ secondi & $TR \leq 4$ secondi \\
        6PSM-TE & Tempo di Elaborazione & $TE \leq 10$ secondi & $TE \leq 5$ secondi \\
        7PSM-QMS & Metriche di Qualità Soddisfatte & $QMS \geq 90\%$ & $QMS \geq 90\%$\\
        \hline
    \end{tabular}
    \label{tab:gestione_qualità}
    \caption{Metriche di qualità per il processo di Gestione della Qualità}
\end{table}

\subsubsection{Risoluzione dei Problemi}%Tabella
\textbf{(13PSM-RMR - 14PSM-NCR)}
\begin{table}[H]
    \centering
    \renewcommand{\arraystretch}{1.5} % Aumenta lo spazio tra le righe
    \begin{tabular}{|c|l|c|c|}
        \hline
        \textbf{Codice} & \textbf{Nome} & \textbf{Ammissibile} & \textbf{Ottimale} \\
        \hline
        13PSM-RMR & Risk Mitigation Rate & $RMR \geq 80\%$ & $RMR = 100\%$ \\
        14PSM-NCR & Richi non Calcolati & $NCR \leq 3$ & $NCR = 0$ \\
        \hline
    \end{tabular}
    \label{tab:risoluzione_problemi}
    \caption{Metriche di qualità per il processo di Risoluzione dei Problemi}
\end{table}
% \newpage
\subsection{Processi organizzativi}
\subsubsection{pianificazione} %Tabella
\textbf{(1POM-RSI)}
\begin{table}[H]
    \centering
    \renewcommand{\arraystretch}{1.5} % Aumenta lo spazio tra le righe
    \begin{tabular}{|c|l|c|c|}
        \hline
        \textbf{Codice} & \textbf{Nome} & \textbf{Ammissibile} & \textbf{Ottimale} \\
        \hline
        1POM-RSI & Requirements Stability Index & $RSI \geq 75\%$ & $RSI = 100\%$ \\
        \hline
    \end{tabular}
    \label{tab:pianificazione}
    \caption{Metriche di qualità per il processo di Pianificazione}
\end{table}
% \newpage

\section{Metodologie e Testing}
In questa sezione si illustrano le metodologie di \textit{Testing} adottate per garantire il rispetto dei vincoli individuati
nella sezione \textit{Requisiti} del documento Analisi dei Requisiti. I test sono suddivisi in cinque categorie:
\begin{enumerate}
    \item Test di unità
    \item Test di integrazione
    \item Test di Sistema
    \item Test di Regressione
    \item Test di Accettazione
\end{enumerate}
Verranno elencate le varie tipologie di test eseguite, indicando il codice del test, una breve descrizione di ciò che viene verificato e lo stato di avanzamento del test, espresso come segue.

\begin{table}[H]
    \centering
    \renewcommand{\arraystretch}{1.5}
\begin{tabular}{|c|c|}
    \hline
    \textbf{S} & Test Superato \\
    \hline
    \textbf{NS} & Test NON Superato \\
    \hline
    \textbf{NI} & Test NON Implementato \\
    \hline
\end{tabular}
\caption{Legenda per il Test}
\end{table}


\subsection{Test di Sistema} %tabellare
I test di sistema sono finalizzati alla verifica del soddisfacimento dei requisiti richiesti ed evidenziati nel documento
Analisi dei Requisiti. Questi test vengono effetuati sul sistema nel suo complesso, per vericare che il software funzioni correttamente
e che sia in grado di eseguire le operazioni richieste.

\subsection{Test di Accettazione} %tabellare
I test di Accettazione vengono effetuati per verificare che il Software soddisfi i requisiti richiesti e consentono di ultimare il processo di validazione finale.

\section{Cruscotto valutazione della qualità}


\subsection{Qualità processo di Fornitura}


\subsection{Qualità processo di Documentazione}


\subsection{Qualità del processo di gestione della qualità}


\subsection{Qualità del processo di gestione dei Rischi}


\subsection{Qualità del processo di pianificazione}


\section{Iniziative di automiglioramento per la qualità}
\subsection{Introduzione}
In questa sezione vengono riportate le iniziative prese per aumentare la qualità del prodotto e dei processi.
Ogni iniziativa è stata individuate grazie all'esperienza aquisita durante lo svolgimento del progetto e pian piano che abbiamo individuato un problema.
Dato che si tratta per tutti noi della prima esperienza con un progetto di questa portata, è stato necessario un grande numero di tentativi per comprendere come organizzarci e svolgere le varie attività.
Abbiamo capito quali sono i punti di forza e i punti deboli nel nostro lavoro individuando le aree in cui è possibile migliorare. Per ciascuna delle difficoltà riscontrata è indicata:
\begin{itemize}
    \item La fase del progetto in cui si è verificato il problema;
    \item Descrizione del problema;
    \item Contromisura adottata per risolvere il problema evidenziato;
\end{itemize}

\subsection{Problemi Rilevati ed iniziative adottate}
\subsubsection{Presentazioni del diario di Bordo}
\begin{itemize}
    \item \textbf{Fase del Progetto:} Iniziale;
    \item \textbf{Descrizione:} Ogni settimana è richiesta una presentazione che illustri l'operato svolto per quella settimana, in questo arco di tempo è necessario preparare delle slide ed esporle di persona, tuttavia questo per alcuni membri risulta essere difficoltoso a causa di lontananza e lavoro. Nonostante una tabella che riportasse i ruoli settimanali di ciascun membro.
    \item \textbf{Contromisura:} Abbiamo deciso che i membri che sono responsabili durante il periodo di \textit{vacanze natalizie} in cui non vi sono attività di diario di bordo esporranno al posto sostituiranno il responsabile che non può presentare durante quella settimana, ed eventualmente nel caso di diari di bordo online, esporranno loro per primi.
\end{itemize}
\subsubsection{Organizzazione delle riunioni}
\begin{itemize}
    \item \textbf{Fase del Progetto:} Intermedia;
    \item \textbf{Descrizione:} In realtà siamo rimasti stupiti che questo problema non si sia presentato fin da subito, ma durante il mese di dicembre ci sono stati problemi sulle sprint interne con molti membri assenti, questo ha portato ad un rallentamento del lavoro anche per la discrepanza di conoscenze che ogni membro ha. Ogni tanto per qualche membro non era chiaro in seguito dove recuperare determinate informazioni o se determinati documenti erano pronti oppure no. Il tutto è degenerato con il fatto che in pochi si autoassegnavano le \textit{issue}.
    \item \textbf{Contromisura:} Abbiamo deciso di rendere la stesura dei verbali un attività primaria da fare al momento della riunione e di completarla il più velocemente possibile, questo anche per aiutare chi deve presentare il diario di bordo. Inoltre per ogni issue creata abbiamo notificato tutti i membri tramite le varie piattaforme di comunicazione così da poter far capire a tutti il lavoro da svolgere per quella determinata sprint. Inoltre per evitare riunioni con pochi membri abbiamo preso in considerazione di essere più flessibili sull'orario delle riunioni settimanali.
\end{itemize}
%% Da aggiungere in caso se ne trovino altre
\subsection{Considerazioni Finali}
Fin da subito il nostro gruppo si è posto come obiettivo qullo di dotarsi di un \textit{Way of Working} preciso e ben definito, pianificando ogni singola attività e prevedendo tutte le possibili difficoltà durante lo svolgimento del progetto.
Questo per cercare di prevenire i problemi e di affrontarli con contromisure efficaci. Molti problemi ci sono stati precedentemente descritti in aula durante lo svolgimento del corso, altri gli abbiamo visti confrontandoci con altri gruppi nelle attività di diario di bordo,
altri ancora gli abbiamo affrontati durante il nostro lavorare, tuttavia grazie ai consigli e suggerimenti esterni che ci sono stati forniti, siamo riusciti a mettere in atto delle contromisure per risolverli o far si che non accadessero sin dall'inizio.
Questo ha migliorato notevolmente la qualità del nostro lavoro e ci ha permesso di svolgere le varie attività in modo efficente ed equo. Nonostante ciò siamo consapevoli che ci sono ancora molti aspetti su cui possiamo progredire e che ci sono molte iniziative di automiglioramento che possiamo adottare.
Siamo convinti che se continueremo a lavorare con lo stesso impegno e la stessa determinazione otterremo risultati di qualità superiore.

\end{document}

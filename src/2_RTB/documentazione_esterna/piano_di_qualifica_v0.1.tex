%%% Settings %%%%%%%%%%%%%%%%%%%%%%%%%%%%%%%%%%%%%%%%%%%%%%%%%%%%%%%%%%%%%%%%%%%%%%%%%
\documentclass{article}

\usepackage{graphicx}  % serve per inserire immagini
\usepackage{fancyhdr}  % creazione header-footer
\usepackage{tabularx}  % serve per creare tabelle con colonne a larghezza variabile
\usepackage{ifthen}  % serve per mostrare cose diverse in base a condizioni
\usepackage{geometry}
\usepackage{setspace}
\usepackage{tikz}
\usepackage[italian]{babel}
\usepackage[hidelinks]{hyperref}
\usepackage{pgfgantt}  % per i diagrammi di Gantt
\usepackage{eurosym}

% setta a 1 se il verbale è esterno, 0 se è interno
\newcommand{\isEsterno}{1}

% Margini della pagina
\geometry{a4paper, margin=1in}

% Intestazione personalizzata
\pagestyle{fancy}
\fancyhf{}
\fancyhead[L]{Code7Crusaders - Software Development Team}
\fancyhead[R]{\thepage}

% Spaziatura delle righe
\setstretch{1.2}

\begin{document}
\setcounter{secnumdepth}{5} % Permette la numerazione fino a \subparagraph
%%%%%%%%%%%%%%%%%%%%%%%%%%%%%%%%%%%%%%%%%%%%%%%%%%%%%%%%%%%%%%%%%%%%%%%%%%%%%%%%%%%%%%



%%% Sezione del titolo %%%%%%%%%%%%%%%%%%%%%%%%%%%%%%%%%%%%%%%%%%%%%%%%%%%%%%%%%%%%%%%
\begin{titlepage}

    \AddToHookNext{shipout/background}{
        \begin{tikzpicture}[remember picture,overlay]
        \node at (current page.center) {
            \includegraphics{../../img/background.png}
        };
        \end{tikzpicture}
    }

    \centering
    \vspace*{2cm}
    
    \includegraphics[width=0.3\textwidth]{../../img/logo/7Crusaders_logo.png} % logo
    \vspace{1cm}
    
    {\Huge \textbf{Code7Crusaders}}\\
    \vspace{0.5cm}
    {\Large Software Development Team}\\
    \vspace{2cm}
    
    {\large \textbf{Piano di Progetto}}\\
    \vspace{5cm}
    
    
    \textbf{Membri del Team:}\\
    Enrico Cotti Cottini, Gabriele Di Pietro, Tommaso Diviesti \\
    Francesco Lapenna, Matthew Pan, Eddy Pinarello, Filippo Rizzolo \\
    \vspace{0.5cm}
    
    \vspace{1cm}
\end{titlepage}
%%%%%%%%%%%%%%%%%%%%%%%%%%%%%%%%%%%%%%%%%%%%%%%%%%%%%%%%%%%%%%%%%%%%%%%%%%%%%%%%%%%%%%



% Versioni %%%%%%%%%%%%%%%%%%%%%%%%%%%%%%%%%%%%%%%%%%%%%%%%%%%%%%%%%%%%%%%%%%%%%%%%%%%
\newpage
\begin{table}[h!]
\centering
\textbf{Versioni} \\ % Titolo sopra la tabella
\vspace{2mm} % Spazio tra il titolo e la tabella
\begin{tabular}{|c|c|c|c|c|}
    \hline
    \textbf{Ver.} & \textbf{Data} & \textbf{Autore} & \textbf{Verificatore} & \textbf{Descrizione} \\
    \hline
    0.1 & 05/12/2024 & Gabriele Di Pietro & Nome Verificatore & Prima stesura del documento \\  
    \hline
\end{tabular}
\end{table}
%%%%%%%%%%%%%%%%%%%%%%%%%%%%%%%%%%%%%%%%%%%%%%%%%%%%%%%%%%%%%%%%%%%%%%%%%%%%%%%%%%%%%%



% Indice %%%%%%%%%%%%%%%%%%%%%%%%%%%%%%%%%%%%%%%%%%%%%%%%%%%%%%%%%%%%%%%%%%%%%%%%%%%%%
\newpage
\tableofcontents
%%%%%%%%%%%%%%%%%%%%%%%%%%%%%%%%%%%%%%%%%%%%%%%%%%%%%%%%%%%%%%%%%%%%%%%%%%%%%%%%%%%%%%



% Sezione Introduzione %%%%%%%%%%%%%%%%%%%%%%%%%%%%%%%%%%%%%%%%%%%%%%%%%%%%%%%%%%%%%%%
\newpage
\section{Introduzione}
\subsection{Obiettivo del Documento}
Il documento ha lo scopo di definire le strategie di verifica e validazione per assicurare il corretto funzionamento e uno standard di qualità
dello strumento sviluppato e le attività che lo accompagnano. Sarà sottoposto a revisioni continue, così da poter seguire l'evoluzione del progetto.

\subsection{Glossario}
Il Glossario è uno strumento utilizzato per risolvere eventuali dubbi su termini specifici utilizzati nella redazione del documento. Esso conterrà la definizione dei 
termini evidenziati e sarà consultabile al seguente \href{https://code7crusaders.github.io/docs/RTB/documentazione_interna/glossario.html}{link}. I termini presenti in tale documento
saranno evidenziatida una 'G' al pedice.

\subsection{Riferimenti}
\subsubsection{Normativi}
\begin{itemize}
    \item \textbf{Regolamento del progetto} \\ \texttt{\url{https://www.math.unipd.it/~tullio/IS-1/2024/Dispense/PD1.pdf}}
    \item \textbf{{Norme del Progetto}} \\ \texttt{\url{inserire norme di progetto}}
\end{itemize}
\subsubsection{Informativi}
\begin{itemize}
    \item \textbf{Standard ISO/IEC 25010} \\ \texttt{\url{https://iso25000.com/index.php/en/iso-25000-standards/iso-25010}}
    \item \textbf{Standard ISO/IEC 12207:1995} \\ \texttt{\url{https://www.math.unipd.it/~tullio/IS-1/2009/Approfondimenti/ISO_12207-1995.pdf}}
    \item \textbf{Qualità di prodotto} \\ \texttt{\url{https://www.math.unipd.it/~tullio/IS-1/2024/Dispense/T07.pdf}}
    \item \textbf{Qualità di processo} \\ \texttt{\url{https://www.math.unipd.it/~tullio/IS-1/2024/Dispense/T08.pdf}}
    \item \textbf{Verifica e validazione}
    \begin{itemize}
        \item Introduzione \\ \texttt{\url{https://www.math.unipd.it/~tullio/IS-1/2024/Dispense/T09.pdf}}
        \item Analisi Statica \\ \texttt{\url{https://www.math.unipd.it/~tullio/IS-1/2024/Dispense/T10.pdf}}
        \item Analisi Dinamica \\ \texttt{\url{ inserire appena le carica}}
    \end{itemize}
    \item \textbf{Capitolato d'appalto C7} \\ \texttt{\url{https://www.math.unipd.it/~tullio/IS-1/2024/Progetto/C7.pdf}}
    \item \textbf{Verbali esterni} \\ \texttt{\url{inserire link verbali}}
    \item \textbf{Verbali interni} \\ \texttt{\url{inserire link verbali}}
    \item \textbf{Analisi dei requisiti} \\ \texttt{\url{inserire link analisi dei req.}}
    \item \textbf{Glossario} \\ \texttt{\url{https://code7crusaders.github.io/docs/RTB/documentazione_interna/glossario.html}}
\end{itemize}

\section{Metriche di qualità}
La qualità di processo è un criterio fondamentale ed è alla base di ogni prodotto che
rispecchi lo stato dell’arte. Per raggiungere tale obiettivo è necessario sfruttare delle
pratiche rigorose che consentano lo svolgimento di ogni attività in maniera ottimale.
Al fine di valutare nel miglior modo possibile la qualità del prodotto e l’efficacia dei
processi, sono state definite delle metriche, meglio specificate nel documento Norme
di ProgettoG e qui di seguito riepilogate. Esse sono state suddivise utilizzando lo \textbf{Standard \texttt{ISO/IEC12207:1995}}, il quale separa i processi di ciclo di vita del software in processi di
base e/o primari, processi di supporto e processi organizzativi.
\subsection{Processi di base e/o primari}
\subsubsection{Fornitura} %Tabella
\subsubsection{Sviluppo} %Non inserire nulla
\paragraph{Analisi dei requisiti}%Tabella
\paragraph{Progettazione}%Tabella
\paragraph{Codifica} %Tabella
\subsection{Processi di Supporto}
\subsubsection{Documentazione}%Tabella
\subsubsection{Gestione della qualità}%Tabella
\subsubsection{Verifica}%Tabella
\subsubsection{Risoluzione dei Problemi}%Tabella
\subsection{Processi organizzativi}
\subsubsection{pianificazione} %Tabella
\newpage

\section{Metodologie e Testing}
In questa sezione si illustrano le metodologie di \textit{Testing} adottate per garantire il rispetto dei vincoli individuati
nella sezione \textit{Requisiti} del documento Analisi dei Requisiti. I test sono suddivisi in cinque categorie:
\begin{enumerate}
    \item Test di unità
    \item Test di integrazione
    \item Test di Sistema
    \item Test di Regressione
    \item Test di Accettazione
\end{enumerate}
Verranno elencate le varie tipologie di test eseguite, indicando il codice del test, una breve descrizione di ciò che viene verificato e lo stato di avanzamento del test, espresso come segue.

\begin{table}
    \centering
\begin{tabular}{|c|c|}
    \hline
    \textbf{S} & Test Superato \\
    \hline
    \textbf{NS} & Test NON Superato \\
    \hline
    \textbf{NI} & Test NON Implementato \\
    \hline
\end{tabular}
\caption{Esempio di tabella con legenda.}
\end{table}


\subsection{Test del Sistema} %tabellare


\subsection{Test di Accettazione} %tabellare


\section{Cruscotto valutazione della qualità}


\subsection{Qualità processo di Fornitura}


\subsection{Qualità processo di Documentazione}


\subsection{Qualità del processo di gestione della qualità}


\subsection{Qualità del processo di gestione dei Rischi}


\subsection{Qualità del processo di pianificazione}


\section{Iniziative di automiglioramento per la qualità}


\subsection{Introduzione}


\subsection{Problemi Rilevati ed iniziative adottate}


\subsection{Considerazioni Finali}



\end{document} 

\section{Introduzione}

\subsection{Scopo del documento}
Questo documento mira a offrire una panoramica dettagliata del prodotto, 
delineando i bisogni degli utenti in base alle diverse categorie individuate durante 
l'analisi del capitolato e gli incontri con il committente.
L'obiettivo è identificare chiaramente tutti i requisiti e gli attori coinvolti
nel sistema software, garantendo una descrizione accurata delle componenti del programma
e una visione strutturata delle attività da svolgere. 

I casi d’uso seguono una struttura logica ben definita e vengono descritti 
con precisione secondo i seguenti punti chiave:

\begin{itemize}
    \item \textbf{Descrizione}: Titolo del caso d’uso accompagnato da un breve commento esplicativo;
    \item \textbf{Attori coinvolti}: Soggetti che interagiscono con il sistema;
    \item \textbf{Precondizioni}: Stato del sistema prima dell’avvio del caso d’uso;
    \item \textbf{Postcondizioni}: Stato del sistema al termine dello scenario del caso d’uso;
    \item \textbf{Scenario principale}: Sequenza di azioni che collega le precondizioni ai risultati, descrivendo il flusso principale dello scenario.
\end{itemize}

\subsection{Scopo del prodotto}
Lo scopo del prodotto è realizzare un \textbf{Assistente Virtuale basato su LLM }, 
in particolare BLOOM, per supportare aziende produttrici di bevande nel fornire 
informazioni dettagliate e personalizzate sui loro prodotti. 
Il sistema si rivolge principalmente ai proprietari di locali, 
consentendo loro di ottenere risposte rapide e precise su caratteristiche, 
disponibilità e dettagli delle bevande, come se interagissero con uno specialista umano.

L’obiettivo è sostituire e migliorare il supporto degli specialisti tradizionali, 
rendendo le informazioni accessibili 24/7 tramite una \textbf{web app} intuitiva. 
Questa piattaforma permetterà agli utenti di formulare domande in linguaggio naturale 
e ricevere risposte pertinenti, basate sui dati forniti dalle aziende. 

Il sistema sarà progettato per garantire flessibilità e scalabilità, 
integrando dati relativi ai prodotti nei database aziendali e utilizzando 
le capacità avanzate di BLOOM per comprendere e generare risposte personalizzate. 
Questo approccio semplificherà il processo decisionale per i clienti, 
migliorando l’efficienza operativa delle aziende e offrendo 
un’esperienza utente fluida e moderna.

\subsection{Glossario}


\subsection{Maturità e miglioramenti}
Il prodotto ha raggiunto un livello di maturità tale da...

\subsection{Riferimenti}

\subsubsection{Riferimenti normativi}
I seguenti riferimenti normativi sono stati utilizzati...

\subsubsection{Riferimenti informativi}
I seguenti riferimenti informativi sono stati utilizzati...

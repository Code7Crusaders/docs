\section{Introduzione}

\subsection{Scopo del documento}
Questo documento mira a offrire una panoramica dettagliata del prodotto, 
delineando i bisogni degli utenti in base alle diverse categorie individuate durante 
l'analisi del capitolato e gli incontri con il committente.
L'obiettivo è identificare chiaramente tutti i requisiti e gli attori coinvolti
nel sistema software, garantendo una descrizione accurata delle componenti del programma
e una visione strutturata delle attività da svolgere. 

I casi d’uso seguono una struttura logica ben definita e vengono descritti 
con precisione secondo i seguenti punti chiave:

\begin{itemize}
    \item \textbf{Descrizione}: Titolo del caso d’uso accompagnato da un breve commento esplicativo;
    \item \textbf{Attori coinvolti}: Soggetti che interagiscono con il sistema;
    \item \textbf{Precondizioni}: Stato del sistema prima dell’avvio del caso d’uso;
    \item \textbf{Postcondizioni}: Stato del sistema al termine dello scenario del caso d’uso;
    \item \textbf{Scenario principale}: Sequenza di azioni che collega le precondizioni ai risultati, descrivendo il flusso principale dello scenario.
\end{itemize}

\subsection{Scopo del prodotto}
Il prodotto descritto in questo documento ha come obiettivo...

\subsection{Glossario}
Nel seguente glossario sono definiti i termini tecnici utilizzati nel documento...

\subsection{Maturità e miglioramenti}
Il prodotto ha raggiunto un livello di maturità tale da...

\subsection{Riferimenti}

\subsubsection{Riferimenti normativi}
I seguenti riferimenti normativi sono stati utilizzati...

\subsubsection{Riferimenti informativi}
I seguenti riferimenti informativi sono stati utilizzati...

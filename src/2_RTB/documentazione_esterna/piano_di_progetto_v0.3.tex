%%% Settings %%%%%%%%%%%%%%%%%%%%%%%%%%%%%%%%%%%%%%%%%%%%%%%%%%%%%%%%%%%%%%%%%%%%%%%%%
\documentclass{article}

\usepackage{graphicx}  % serve per inserire immagini
\usepackage{fancyhdr}  % creazione header-footer
\usepackage{tabularx}  % serve per creare tabelle con colonne a larghezza variabile
\usepackage{ifthen}  % serve per mostrare cose diverse in base a condizioni
\usepackage{geometry}
\usepackage{setspace}
\usepackage{tikz}
\usepackage[italian]{babel}
\usepackage[hidelinks]{hyperref}
\usepackage{pgfgantt}  % per i diagrammi di Gantt
\usepackage{eurosym}
\usepackage{float}

% setta a 1 se il verbale è esterno, 0 se è interno
\newcommand{\isEsterno}{1}

% Margini della pagina
\geometry{a4paper, margin=1in}

% Intestazione personalizzata
\pagestyle{fancy}
\fancyhf{}
\fancyhead[L]{Code7Crusaders - Software Development Team}
\fancyhead[R]{\thepage}

% Spaziatura delle righe
\setstretch{1.2}

\begin{document}
%%%%%%%%%%%%%%%%%%%%%%%%%%%%%%%%%%%%%%%%%%%%%%%%%%%%%%%%%%%%%%%%%%%%%%%%%%%%%%%%%%%%%%



%%% Sezione del titolo %%%%%%%%%%%%%%%%%%%%%%%%%%%%%%%%%%%%%%%%%%%%%%%%%%%%%%%%%%%%%%%
\begin{titlepage}

    \AddToHookNext{shipout/background}{
        \begin{tikzpicture}[remember picture,overlay]
        \node at (current page.center) {
            \includegraphics{../../img/background.png}
        };
        \end{tikzpicture}
    }

    \centering
    \vspace*{2cm}
    
    \includegraphics[width=0.3\textwidth]{../../img/logo/7Crusaders_logo.png} % logo
    \vspace{1cm}
    
    {\Huge \textbf{Code7Crusaders}}\\
    \vspace{0.5cm}
    {\Large Software Development Team}\\
    \vspace{2cm}
    
    {\large \textbf{Piano di Progetto}}\\
    \vspace{5cm}
    
    
    \textbf{Membri del Team:}\\
    Enrico Cotti Cottini, Gabriele Di Pietro, Tommaso Diviesti \\
    Francesco Lapenna, Matthew Pan, Eddy Pinarello, Filippo Rizzolo \\
    \vspace{0.5cm}
    
    \vspace{1cm}
\end{titlepage}
%%%%%%%%%%%%%%%%%%%%%%%%%%%%%%%%%%%%%%%%%%%%%%%%%%%%%%%%%%%%%%%%%%%%%%%%%%%%%%%%%%%%%%



% Versioni %%%%%%%%%%%%%%%%%%%%%%%%%%%%%%%%%%%%%%%%%%%%%%%%%%%%%%%%%%%%%%%%%%%%%%%%%%%
\newpage
\begin{table}[h!]
\centering
\textbf{Versioni} \\ % Titolo sopra la tabella
\vspace{2mm} % Spazio tra il titolo e la tabella
\begin{tabular}{|c|c|c|c|c|}
    \hline
    \textbf{Ver.} & \textbf{Data} & \textbf{Autore} & \textbf{Verificatore} & \textbf{Descrizione} \\
    \hline
    0.2 & 3/12/2024 & Diviesti Tommaso & Nome Verificatore & Continuazione e revisione del documento \\
    0.1 & 29/11/2024 & Lapenna Francesco & Nome Verificatore & Prima stesura del documento \\  
    \hline
\end{tabular}
\end{table}
%%%%%%%%%%%%%%%%%%%%%%%%%%%%%%%%%%%%%%%%%%%%%%%%%%%%%%%%%%%%%%%%%%%%%%%%%%%%%%%%%%%%%%



% Indice %%%%%%%%%%%%%%%%%%%%%%%%%%%%%%%%%%%%%%%%%%%%%%%%%%%%%%%%%%%%%%%%%%%%%%%%%%%%%
\newpage
\tableofcontents
\listoftables
\listoffigures
%%%%%%%%%%%%%%%%%%%%%%%%%%%%%%%%%%%%%%%%%%%%%%%%%%%%%%%%%%%%%%%%%%%%%%%%%%%%%%%%%%%%%%



% Sezione Introduzione %%%%%%%%%%%%%%%%%%%%%%%%%%%%%%%%%%%%%%%%%%%%%%%%%%%%%%%%%%%%%%%
\newpage
\section{Introduzione}

    \subsection{Scopo del documento}
    Questo documento ha lo scopo di fornire una guida dettagliata e strutturata su come 
    il progetto verrà eseguito e gestito. In particolare, verrano trattati i seguenti 
    argomenti:
    \begin{itemize}  % TODO: rivedere
        \item analisi del capitolato
        \item analisi delle risorse;
        \item analisi dei rischi;
        \item pianificazione;
        \item stime dei costi;
        \item modello di sviluppo adottato;
        \item log degli sprint;
    \end{itemize}

    \subsection{Scopo del prodotto}
    Il prodotto consiste in una webapp avanzata che integra una chatbot alimentata da intelligenza artificiale, 
    pensata per fornire informazioni precise e approfondite su una vasta selezione di bevande. L’obiettivo principale è 
    offrire alle aziende uno strumento semplice ed efficace per accedere a dettagli fondamentali riguardo le bevande che 
    desiderano acquistare, assicurando maggiore trasparenza e chiarezza in ogni fase del processo di selezione. Grazie alla 
    nostra soluzione, le aziende possono ottenere risposte immediate su una serie di parametri chiave e informazioni su bibite e relativi produttori/venditori.   
    Tutto ciò permette di ridurre incertezze e ambiguità, riducendo i rischi delle aziende legati alla scelta di prodotti non adatti alle proprie esigenze.
    Inoltre, le chat recenti vengono salvate e rese facilmente accessibili agli utenti, permettendo loro di rivedere in qualsiasi 
    momento le informazioni precedentemente richieste. Questa funzionalità risulta particolarmente utile per consultare rapidamente 
    risposte a domande frequenti o per confrontare dettagli su diverse bevande, senza dover rifare ogni ricerca, garantendo così 
    un'esperienza più efficiente e personalizzata.

    \subsection{Glossario}
    Per avere maggiore chiarezza ed evitare ambiguità per quanto riguarda i termini utilizzati all'interno dei vari documenti,
    viene adottato un Glossario\textsuperscript{G} che contiene una serie di termini e relativa definizione.
    Grazie ad esso, sarà possibile cliccare su una determinata porzione di testo, evidenziata grazie ad uno stile specifico, 
    all'interno di un qualsiasi documento e in questo modo si potrà visualizzare la sua definizione all'interno
    del Glossario\textsuperscript{G} stesso. Questa soluzione permetterà agli utenti di avere maggiore chiarezza sugli argomenti da noi 
    trattati nei vari file di documentazione.

    \subsection{Riferimenti}
    % TODO

    \subsection{Preventivo iniziale}
    Il preventivo iniziale è stato presentato durante la fase di Candidatura ed è pari a \textbf{12805}\euro.
    \\ Per ulteriori informazioni è possibile visualizzare il documento di analisi dei costi e assunzione impegni al seguente link:
    \\ \url{https://code7crusaders.github.io/docs/Candidatura/Preventivo_costi.html}

%%%%%%%%%%%%%%%%%%%%%%%%%%%%%%%%%%%%%%%%%%%%%%%%%%%%%%%%%%%%%%%%%%%%%%%%%%%%%%%%%%%%%%



% Analisi del Capitolato %%%%%%%%%%%%%%%%%%%%%%%%%%%%%%%%%%%%%%%%%%%%%%%%%%%%%%%%%%%%%
\newpage
\section{Analisi del Capitolato}
    \subsection{Obbiettivi del progetto}
    \begin{itemize}
        \item Realizzare un Assistente Virtuale che supporti i clienti nella ricerca 
        di informazioni sui prodotti disponibili in catalogo.
        \item Automatizzare le risposte alle domande più frequenti, migliorando 
        l'efficienza del servizio clienti.
        \item Integrare un modello LLM esistente per garantire risposte accurate e 
        un'interfaccia user-friendly.
        \item Memoria a lungo termine/Salvataggio chat recenti (lo Specialist
        potrebbe non ricordarsi tutti i dettagli)
        \item Velocità di risposta e disponibilità 24/7
    \end{itemize}

    \subsection{Ambito del Progetto}

        \subsubsection{Inclusioni}
        \begin{itemize}
            \item Database relazionale per la gestione dei dati sui prodotti.
            \item Integrazione di un modello LLM tramite API.
            \item Interfaccia utente mobile per l'interazione con l'IA.
            \item Funzionalità di configurazione backend per template di domande e risposte.
        \end{itemize}
        \subsubsection{Esclusioni}
        \begin{itemize}
            \item Creazione di un nuovo modello LLM.
            \item Supporto a lingue non previste dal modello LLM scelto.
        \end{itemize}

    \subsection{Funzionamento}
    \begin{itemize}
        \item Da un’interfaccia utente, viene catturata una domanda da parte dell’utente
        \item La domanda viene inoltrata al sistema attraverso delle API REST risiedenti in un Web Server
        \item La query ricevuta viene gestita dall’Embedding Model che trasforma la domanda in rappresentazione vettoriale
        \item La rappresentazione vettoriale viene utilizzata per effettuare una ricerca all’interno del database vettoriale da dove vengono reperiti i vettori più simili
        \item Sia la domanda sia i risultati della ricerca nel database vettoriale, vengono inviati al sistema LLM che costruirà la risposta utilizzando il contesto fornito
        \item Attraverso API REST, il sistema inoltra la riposta al dispositivo dell’utente
    \end{itemize}

    \subsection{Tecnologie e Strumenti Consigliati}
    L’azienda proponente è disponibile a fornire i dati di un caso di studio da utilizzare
    per lo sviluppo del progetto. I dati potranno essere dati in ingresso al sistema così da eseguire la fase di
    training e poi interagire con il sistema per valutarne le prestazioni sfruttando un caso reale.
    Di seguito vengono suggerite alcune tecnologie utilizzabili per il sistema esposto:
    \begin{itemize}
        \item \textbf{Database}: MySQL o PostgreSQL.
        \item \textbf{LLM}: BLOOM o Italia by iGenius, in base alle prestazioni richieste.
        \item \textbf{Backend}: Node.js con Express.js o .NET.
        \item \textbf{Frontend}: .NET MAUI per applicazioni mobile multipiattaforma.
        \item \textbf{API REST}: Per la comunicazione tra LLM e interfaccia utente.
        \item \textbf{Controllo Versione}: Git (GitHub per repository pubblico).
    \end{itemize}

    \subsection{Architettura proposta}
    %immagine architettura

    \subsection{Supporto}
    Per il progetto, l’azienda proponente fornirà ampio supporto da parte del team interno in varie fasi
    di sviluppo. L’interazione potrà avvenire sia nei locali aziendali sia da remoto tramite chat e/o chiamate. 
    Inoltre, mette a disposizione una serie di link e corsi utili che trattano le tecnologie relative ai sistemi
    LLM e allo sviluppo software.

%%%%%%%%%%%%%%%%%%%%%%%%%%%%%%%%%%%%%%%%%%%%%%%%%%%%%%%%%%%%%%%%%%%%%%%%%%%%%%%%%%%%%%



% Pianificazione %%%%%%%%%%%%%%%%%%%%%%%%%%%%%%%%%%%%%%%%%%%%%%%%%%%%%%%%%%%%%%%%%%%%%%%%
\newpage
\section{Pianificazione}
    \subsection{Struttura del Team}
        \subsubsection{Ruoli}
        I ruoli in seguito descritti sono equamente divisi tra i vari componenti del Team. Ogni ruolo possiede diversi incarichi e obbiettivi:
        \begin{itemize}
            \item \textbf{Responsabile}: coordina il gruppo di lavoro, controlla le attività e gestisce le risorse. Si occupa di garantire che il progetto venga portato a termine nei tempi stabiliti e con le risorse disponibili.
            \item \textbf{Amministratore}: si occupa della gestione delle risorse e delle infrastrutture, incluso il setup degli strumenti di supporto alla produzione del software. Garantisce inoltre l’uso corretto delle procedure per assicurare efficienza e produttività.
            \item \textbf{Analista}: gioca un ruolo fondamentale nella fase iniziale del progetto. È responsabile della definizione dei requisiti e dell’analisi delle funzionalità del software, delineando i casi d'uso. Essendo necessario principalmente all'inizio del progetto, il numero di ore assegnato al ruolo è relativamente ridotto.
            \item \textbf{Progettista}: definisce l'architettura del software, descrivendo le componenti e le loro interazioni sulla base dei requisiti stabiliti dall'Analista. Questo ruolo ha un numero di ore significativamente elevato perché è essenziale per garantire una struttura solida, soprattutto considerando l’implementazione di modelli \emph{LLM}, che richiedono un'architettura ben progettata e adattata a tali tecnologie.
            \item \textbf{Programmatore}: si occupa di scrivere il codice del software seguendo le specifiche del progettista. Il numero di ore assegnato è alto, dato che rappresenta il cuore della fase di sviluppo. Tuttavia, il ruolo ha leggermente meno ore rispetto al Verificatore, poiché abbiamo scelto di adottare una metodologia incentrata sui test, che richiede un’accurata verifica del software.
            \item \textbf{Verificatore}: verifica che il software e la documentazione siano conformi alle norme e alle specifiche. Questo ruolo richiede un numero di ore superiore alla media, data la necessità di test approfonditi e continui, in particolare per un progetto basato su \emph{LLM}, dove ogni componente deve essere rigorosamente validato per garantire la precisione e l’affidabilità del sistema.
        \end{itemize}
        \subsubsection{Stakeholder}
        \begin{itemize}
            \item \textbf{Cliente}: Ergon Informatica Srl.
            \item \textbf{Referente interno}: Gianluca Carlesso.
        \end{itemize}

    \subsection{Budget e Risorse}
        \subsubsection{TODO: }
        \begin{itemize}
            \item Allocazione hardware: server per database e API, risorse cloud per il modello LLM.
            \item Licenze software e costi del modello LLM (se applicabile).
        \end{itemize}
        \subsubsection{Distribuzione ore/ruolo}
        Di seguito, si riporta il costo orario in base al ruolo assunto:
        \begin{table}[!h]
            \begin{center}
                \begin{tabular}{ |c|c|c|c| }
                    \hline
                    \textbf{Ruolo}          & \textbf{Costo orario} (\euro) &  \textbf{per ruolo}   & \textbf{Ore per membro} \\
                    \hline          
                    Responsabile   & 30           &     54       &       8        \\
                    Amministratore & 20           &     64       &       9        \\
                    Analista       & 25           &     65       &       9       \\
                    Progettista    & 25           &     105      &       15       \\
                    Programmatore  & 15           &     184      &       26       \\
                    Verificatore   & 15           &     193      &       28       \\
                    \hline
                    \textbf{Totale}         &    12805    &     665       &       95       \\
                    \hline
                \end{tabular}
                \caption{Costo orario e totale}
            \end{center}
        \end{table}

        \subsubsection{Distribuzione ore/membro}
        Tutti i componenti del Team Code7Crusaders si impegnano a dedicare un totale di \textbf{95 ore} di lavoro effettivo partizionate 
        settimanalmente in base al ruolo di riferimento, per lo svolgimento del capitolato \textbf{C7} di \textbf{Ergon Informatica}. Inoltre,
        ciascun membro garantisce la conclusione del progetto entro la data prevista e preventivata nel paragrafo 5 di questo documento.
        \\
        Ripartizione delle ore per membro del team:
        \begin{table}[!h]
            \begin{center}
                \begin{tabular}{ |c|c|c|c|c|c|c|c| }
                    \hline
                    \textbf{Membro}    & \textbf{Re} & \textbf{Am} & \textbf{An} & \textbf{Pj} & \textbf{Pg} & \textbf{Ve} & \textbf{Totale} \\
                    \hline
                    Enrico Cotti Cottini     & 8           & 9           & 9          & 15          & 26          & 28          & 95              \\
                    Gabriele Di Pietro       & 8           & 9           & 9          & 15          & 26          & 28          & 95              \\
                    Tommaso Diviesti         & 8           & 9           & 9          & 15          & 26          & 28          & 95              \\
                    Francesco Lapenna        & 8           & 9           & 9          & 15          & 26          & 28          & 95              \\
                    Matthew Pan              & 8           & 9           & 9          & 15          & 26          & 28          & 95              \\
                    Eddy Pinarello           & 8           & 9           & 9          & 15          & 26          & 28          & 95              \\
                    Filippo Rizzolo          & 8           & 9           & 9          & 15          & 26          & 28          & 95              \\
                    \hline
                \end{tabular}
                \caption{Impegni orari a persona} 
            \end{center}
        \end{table}
        \\
        \textsc{Legenda:} \\
            \textbf{Re} = Responsabile \\
            \textbf{Am} = Amministratore \\
            \textbf{An} = Analista \\
            \textbf{Pj} = Progettista \\
            \textbf{Pg} = Programmatore \\
            \textbf{Ve} = Verificatore \\

    \subsection{Analisi dei rischi}
    In questa sezione vengono elencati i rischi che potrebbero verificarsi durante lo svolgimento del progetto e le relative contromisure. Ad ogni rischio è associato un \textbf{indice di Gravità e Probabilità},
    in modo da poter valutare la criticità di ciascuno di essi.
        \subsubsection{Definizione degli indici}
        I fattori chiave per l’identificazione dei rischi sono:
        \begin{itemize}
            \item l’\textbf{indice}, un valore numerico incrementale che identifica univocamente il rischio
            \item il \textbf{tipo}, che rappresenta l’impatto che un rischio può avere sul progetto, il quale puo essere basso, medio o alto
            \item la \textbf{gravità}, che descrive l'impatto o le conseguenze del rischio che si verificano qualora si presenti
            \item la \textbf{probabilità}, che rappresenta la possibilità che un rischio si verifichi
        \end{itemize}
        I valori dell'\textbf{Indice di Gravità} e dell'\textbf{Indice di Probabilità} sono definiti come segue:
        \begin{table}[h!]
            \centering
            \begin{tabular}{|c|c|p{6cm}|p{6cm}|}
                \hline
                \textbf{Indice} & \textbf{Tipo} & \textbf{Gravità} & \textbf{Probabilità} \\
                \hline
                \textbf{1} & Basso & Ha un impatto minimo o trascurabile sul progetto, come un lieve rallentamento che non incide sui tempi di consegna & Improbabile che si verifichi, ma esistono fattori che potrebbero contribuire alla sua realizzazione \\
                \hline
                \textbf{2} & Medio & Se si concretizza, richiede risorse aggiuntive o modifica parzialmente il piano di progetto, causando impatti gestibili ma che comportano sforzi supplementari & C'è una possibilità realistica che l'evento di rischio si verifichi \\
                \hline
                \textbf{3} & Alto & Causa ritardi significativi, aumento dei costi o degrado della qualità che incide sull'esperienza utente, richiedendo interventi importanti per mantenere il progetto nei tempi e nel budget & Esistono molti fattori o segni che indicano che il rischio potrebbe accadere, e il team considera probabile la sua manifestazione \\
                \hline
            \end{tabular}
            \caption{Definizione degli Indici di Gravità e Probabilità}
            \label{tab:definizione_indici}
        \end{table}
        
        \subsubsection{Rischi}
        \begin{table}[h]
            \centering
            \begin{tabular}{|p{0.5cm}|p{7cm}|p{2cm}|p{2cm}|}
                \hline
                \textbf{ID} & \textbf{Rischio} & \textbf{Gravità} & \textbf{Probabilità} \\
                \hline
                1 & Difficoltà nell'uso di nuove tecnologie & 2 & 3 \\ 
                \hline
                2 & Codice non completato dal delegato & 2 & 2 \\ 
                \hline
                3 & Riduzione del carico e delle ore di lavoro durante le festività & 1 & 2 \\ 
                \hline 
                4 & Scarsa collaborazione da parte di uno o più membri del team & 3 & 1 \\ 
                \hline
                5 & Impegni personali e universitari & 1 & 2 \\ 
                \hline 
                6 & Deviazione dai tempi e costi previsti & 3 & 1 \\
                \hline
            \end{tabular}
            \caption{Analisi dei rischi}
            \label{tab:analisi_rischi}
        \end{table}

        \subsubsection{Contromisure}
        \begin{itemize}
            \item{Contromisura rischio 1:\\}
            Il gruppo si impegnerà a studiare in modo approfondite le tecnologie richieste dal capitolato in particolar modo lo studio dei \emph{LLM}. E verranno organizzati incontri di formazione interna in modo tale da poter condividere le conoscenze acquisite per essere tutti sullo stesso livello.
            \item{Contromisura rischio 2:\\}
            Il gruppo si impegnerà a chiedere supporto all'azienda e si cercherà di massimizzare le risorse nel team nella soluzione di un problema.
            \item{Contromisura rischio 3:\\}
            Il gruppo cercherà di mantenere i ritmi feriali impostando un tempo minimo di lavoro settimanale.
            \item{Contromisura rischio 4:\\}
            Il gruppo si impegna nella comprensione e nel chiarire quali siano i ruoli, inoltre una comunicazione costante e trasparente aiuterà sull'affrontare le diverse difficoltà e nel segnalare tempestivamente eventuali problemi
            \item{Contromisura rischio 5:\\}
            Progettazione di un calendario condiviso dove ogni componente può segnalare i propri impegni personali con anticipo. Di conseguenza pianificare bene le varie attività evitando sovrapposizioni
            \item{Contromisura rischio 6:\\}
            Monitorare il progresso delle attività e svolgere frequenti riunioni per valutare lo stato di avanzamento del progetto. 
        \end{itemize}

    \subsection{Piano di Comunicazione}
    \begin{itemize}
        \item Riunioni settimanali con il team.
        \item Report di avanzamento per il referente aziendale ogni 2 settimane.
        \item Feedback continuo attraverso test intermedi.
    \end{itemize}

    \subsection{Pianificazione delle Attività}
        Il gruppo Code7Crusaders 
        si impegna a consegnare il progetto entro il 14/03/2025. La pianificazione prevede 
        19 settimane di lavoro, suddivise come segue:
        \begin{itemize}
            \item \textbf{Sviluppo PoC (\textit{Proof of Concept}): 6 settimane}
            \item \textbf{Sviluppo MVP (\textit{Minimum Viable Product}): 13 settimane}
        \end{itemize}

        \subsubsection{Fasi principali}  % TODO: rivedere le fasi
        \begin{enumerate}
            \item \textbf{Analisi dei requisiti} (2 settimane):
            \begin{itemize}
                \item Revisione del capitolato.
                \item Identificazione delle tecnologie e dei modelli LLM adatti.
            \end{itemize}
            \item \textbf{Progettazione} (2 settimane):
            \begin{itemize}
                \item Progettazione architetturale.
                \item Definizione dello schema del database.
            \end{itemize}
            \item \textbf{Sviluppo Backend} (4 settimane):
            \begin{itemize}
                \item Configurazione del database.
                \item Implementazione delle API REST.
            \end{itemize}
            \item \textbf{Integrazione LLM} (3 settimane):
            \begin{itemize}
                \item Pre-processing dei dati e integrazione del modello LLM.
            \end{itemize}
            \item \textbf{Sviluppo Frontend} (4 settimane):
            \begin{itemize}
                \item Creazione dell'interfaccia utente mobile.
            \end{itemize}
            \item \textbf{Test e validazione} (2 settimane):
            \begin{itemize}
                \item Test funzionali e di usabilità.
            \end{itemize}
            \item \textbf{Rilascio e documentazione} (2 settimane).
        \end{enumerate}

        \subsubsection{Cronoprogramma}
        % Diagramma di Gantt %%%%%%%%%%%%%%%%%%%%%%%%%%%%%%%%%%%%%%%%%%%%%%%%%%%%%%%%%%%%
        \resizebox{\textwidth}{!}{ % Scales the chart to fit the text width
            \begin{ganttchart}[
                hgrid, vgrid, % Adds grid lines
                x unit=1.1cm, % Adjust the width of each time slot
                title/.append style={draw=black, thick, fill=red!20, line width=1pt, draw opacity=1},
                bar/.append style={fill=red!50}, % Custom bar styling
                milestone/.append style={fill=red!70}, % Milestone style
                ]{1}{19} % Timeline from week 1 to week 19
                \gantttitle{Project Plan (Weeks)}{19} \\ % Title spanning 19 weeks
                \gantttitle{Novembre 2024}{4}
                \gantttitle{Dicembre 2024}{5}
                \gantttitle{Gennaio 2025}{4}
                \gantttitle{Febbraio 2025}{4}
                \gantttitle{Marzo 2025}{2}
                \\
                \gantttitle{4 nov}{1}  % TODO: decidere cosa tenere tra i titoli
                \gantttitle{11 nov}{1}
                \gantttitle{18 nov}{1}
                \gantttitle{25 nov}{1}
                \gantttitle{2 dic}{1}
                \gantttitle{9 dic}{1}
                \gantttitle{16 dic}{1}
                \gantttitle{23 dic}{1}
                \gantttitle{30 dic}{1}
                \gantttitle{6 gen}{1}
                \gantttitle{13 gen}{1}
                \gantttitle{20 gen}{1}
                \gantttitle{27 gen}{1}
                \gantttitle{3 feb}{1}
                \gantttitle{10 feb}{1}
                \gantttitle{17 feb}{1}
                \gantttitle{24 feb}{1}
                \gantttitle{3 mar}{1}
                \gantttitle{10 mar}{1} \\
                \gantttitlelist{1,...,19}{1} \\ % Weekly columns
                
                % Define groups and tasks
                \ganttgroup{Sviluppo PoC}{1}{6} \\
                \ganttmilestone{RTB}{6} \\
                \ganttgroup{Sviluppo MVP}{7}{19} \\
                \ganttmilestone{PB}{19} \\

                \ganttbar{Analisi dei requisiti}{1}{2} \\
                \ganttbar{Progettazione}{3}{4} \\
                \ganttbar{Sviluppo backend}{5}{8} \\
                \ganttbar{Integrazione LLM}{9}{11} \\
                \ganttbar{Sviluppo frontend}{12}{15} \\
                \ganttbar{Test e validazione}{16}{17} \\
                \ganttbar{Rilascio e documentazione}{18}{19} \\

                \ganttlink{elem4}{elem5}  % TODO: decidere se tenere o togliere i link
                \ganttlink{elem5}{elem6}
                \ganttlink{elem6}{elem7}
                \ganttlink{elem7}{elem8}
                \ganttlink{elem8}{elem9}
                \ganttlink{elem9}{elem10}
            
            \end{ganttchart}
        }
        %%%%%%%%%%%%%%%%%%%%%%%%%%%%%%%%%%%%%%%%%%%%%%%%%%%%%%%%%%%%%%%%%%%%%%%%%%%%%%%%%

    
%%%%%%%%%%%%%%%%%%%%%%%%%%%%%%%%%%%%%%%%%%%%%%%%%%%%%%%%%%%%%%%%%%%%%%%%%%%%%%%%%%%%%%



%%%%%%%%%%%%%%%%%%%%%%%%%%%%%%%%%%%%%%%%%%%%%%%%%%%%%%%%%%%%%%%%%%%%%%%%%%%%%%%%%%%%%%
\newpage
\section{Documentazione}  % TODO: vedere se tenere
\begin{itemize}
    \item \textbf{Da consegnare}:
    \begin{itemize}
        \item Schema architetturale.  % TODO: riferimenti ad altri documenti?
        \item Progettazione del database.
        \item Manuale per l'utilizzo della piattaforma.
        \item Codice sorgente e repository Git.
    \end{itemize}
\end{itemize}
%%%%%%%%%%%%%%%%%%%%%%%%%%%%%%%%%%%%%%%%%%%%%%%%%%%%%%%%%%%%%%%%%%%%%%%%%%%%%%%%%%%%%%



%%%%%%%%%%%%%%%%%%%%%%%%%%%%%%%%%%%%%%%%%%%%%%%%%%%%%%%%%%%%%%%%%%%%%%%%%%%%%%%%%%%%%%
\newpage
\section{Modello adottato e motivazioni}
    % TODO
    \subsection{Sprint Log}
        \subsection{RTB}
            % TEMPLATE PER SPRINT %%%%%%%%%%%%%%%%%%%%%%%%%%%%%%%%%%%%%%%%%%%%%%%%%%%%
            % \subsubsection{Sprint Template} % TODO metti numero sprint
            %     \paragraph{Pianificazione}
            %     \begin{itemize}
            %         % TODO segnare inizio e fine
            %         \item inizio:
            %         \item fine:
            %     \end{itemize}
            %     TODO segnare cosa abbiamo fatto in questo sprint \\
            %     TODO segnare i rischi attesi???

            %     \paragraph{Preventivo}\mbox{}\\
            %     \begin{table}[H]
            %         \centering
            %         \begin{tabular}{|c|c|c|c|c|c|c|c|}
            %         \hline
            %          & Re & Am & An & Pj & Pg & Ve & Totale per persona \\ \hline
            %         Cotti Cottini & - & - & - & - & - & - & - \\ \hline
            %         Di Pietro & - & - & - & - & - & - & - \\ \hline
            %         Diviesti & - & - & - & - & - & - & - \\ \hline
            %         Lapenna & - & - & - & - & - & - & - \\ \hline
            %         Pan & - & - & - & - & - & - & - \\ \hline
            %         Pinarello & - & - & - & - & - & - & - \\ \hline
            %         Rizzolo & - & - & - & - & - & - & - \\ \hline
            %         Totale per ruolo & - & - & - & - & - & - & \\ \hline
            %         \end{tabular}
            %         \caption{Preventivo orario per ruolo nello Sprint 1}
            %     \end{table}

            %     \paragraph{Consuntivo}\mbox{}\\
            %     \begin{table}[H]
            %         \centering
            %         \begin{tabular}{|c|c|c|c|c|c|c|c|}
            %         \hline
            %         & Re & Am & An & Pj & Pg & Ve & Totale per persona \\ \hline
            %         Cotti Cottini & - & - & - & - & - & - & - \\ \hline
            %         Di Pietro & - & - & - & - & - & - & - \\ \hline
            %         Diviesti & - & - & - & - & - & - & - \\ \hline
            %         Lapenna & - & - & - & - & - & - & - \\ \hline
            %         Pan & - & - & - & - & - & - & - \\ \hline
            %         Pinarello & - & - & - & - & - & - & - \\ \hline
            %         Rizzolo & - & - & - & - & - & - & - \\ \hline
            %         Totale per ruolo & - & - & - & - & - & - & \\ \hline
            %         \end{tabular}
            %         \caption{Consuntivo orario per ruolo nello Sprint 1}
            %     \end{table}

            %     \paragraph{Prospetto Economico}\mbox{}\\
            %     \begin{table}[H]
            %         \centering
            %         \begin{tabular}{|c|c|c|c|}
            %         \hline
            %         \textbf{Ruolo} & \textbf{Ore} & \textbf{Costo} & \textbf{Ore rimanenti} \\ \hline
            %         Responsabile & ? & ore x €30 & ore rimanenti settimana precedente - ore consuntivo \\ \hline
            %         Amministratore & ? & ore x €20 & ?? \\ \hline
            %         Analista & ? & ore x €25 & ?? \\ \hline
            %         Progettista & ? & ore x €25 & ?? \\ \hline
            %         Programmatore & ? & ore x €15 & ?? \\ \hline
            %         Verificatore & ? & ore x €15 & ?? \\ \hline
            %         \textbf{Totale} & \textbf{??} & \textbf{???} & \textbf{???} \\ \hline
            %         \end{tabular}
            %         \caption{Prospetto economico e ore rimanenti}
            %     \end{table}

            %     \paragraph{Retrospettiva}
            %     TODO Retrospettiva dello sprint \\
            %     TODO segnare i rischi effettivamente occorsi???
            %%%%%%%%%%%%%%%%%%%%%%%%%%%%%%%%%%%%%%%%%%%%%%%%%%%%%%%%%%%%%%%%%%%%%%%%%%
            \subsubsection{Sprint 1}
            \subsubsection{Sprint 2}
            \subsubsection{Sprint ...}
        \subsection{PB}
            \subsubsection{Sprint 1}
            \subsubsection{Sprint 2}
            \subsubsection{Sprint ...}

%%%%%%%%%%%%%%%%%%%%%%%%%%%%%%%%%%%%%%%%%%%%%%%%%%%%%%%%%%%%%%%%%%%%%%%%%%%%%%%%%%%%%%

\end{document} 

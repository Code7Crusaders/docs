%%% Settings %%%%%%%%%%%%%%%%%%%%%%%%%%%%%%%%%%%%%%%%%%%%%%%%%%%%%%%%%%%%%%%%%%%%%%%%%
\documentclass{article}

\usepackage{graphicx}  % serve per inserire immagini
\usepackage{fancyhdr}  % creazione header-footer
\usepackage{tabularx}  % serve per creare tabelle con colonne a larghezza variabile
\usepackage{ifthen}  % serve per mostrare cose diverse in base a condizioni
\usepackage{geometry}
\usepackage{setspace}
\usepackage{tikz}
\usepackage[italian]{babel}
\usepackage[hidelinks]{hyperref}
\usepackage{pgfgantt}  % per i diagrammi di Gantt

% setta a 1 se il verbale è esterno, 0 se è interno
\newcommand{\isEsterno}{1}

% Margini della pagina
\geometry{a4paper, margin=1in}

% Intestazione personalizzata
\pagestyle{fancy}
\fancyhf{}
\fancyhead[L]{Code7Crusaders - Software Development Team}
\fancyhead[R]{\thepage}

% Spaziatura delle righe
\setstretch{1.2}

\begin{document}
%%%%%%%%%%%%%%%%%%%%%%%%%%%%%%%%%%%%%%%%%%%%%%%%%%%%%%%%%%%%%%%%%%%%%%%%%%%%%%%%%%%%%%



%%% Sezione del titolo %%%%%%%%%%%%%%%%%%%%%%%%%%%%%%%%%%%%%%%%%%%%%%%%%%%%%%%%%%%%%%%
\begin{titlepage}

    \AddToHookNext{shipout/background}{
        \begin{tikzpicture}[remember picture,overlay]
        \node at (current page.center) {
            \includegraphics{../../img/background.png}
        };
        \end{tikzpicture}
    }

    \centering
    \vspace*{2cm}
    
    \includegraphics[width=0.3\textwidth]{../../img/logo/7Crusaders_logo.png} % logo
    \vspace{1cm}
    
    {\Huge \textbf{Code7Crusaders}}\\
    \vspace{0.5cm}
    {\Large Software Development Team}\\
    \vspace{2cm}
    
    {\large \textbf{Piano di Progetto}}\\
    \vspace{5cm}
    
    
    \textbf{Membri del Team:}\\
    Enrico Cotti Cottini, Gabriele Di Pietro, Tommaso Diviesti \\
    Francesco Lapenna, Matthew Pan, Eddy Pinarello, Filippo Rizzolo \\
    \vspace{0.5cm}
    
    \vspace{1cm}
\end{titlepage}
%%%%%%%%%%%%%%%%%%%%%%%%%%%%%%%%%%%%%%%%%%%%%%%%%%%%%%%%%%%%%%%%%%%%%%%%%%%%%%%%%%%%%%



% Versioni %%%%%%%%%%%%%%%%%%%%%%%%%%%%%%%%%%%%%%%%%%%%%%%%%%%%%%%%%%%%%%%%%%%%%%%%%%%
\newpage
\begin{table}[h!]
\centering
\textbf{Versioni} \\ % Titolo sopra la tabella
\vspace{2mm} % Spazio tra il titolo e la tabella
\begin{tabular}{|c|c|c|c|c|}
    \hline
    \textbf{Ver.} & \textbf{Data} & \textbf{Autore} & \textbf{Verificatore} & \textbf{Descrizione} \\
    \hline
    0.1 & 29/11/2024 & Lapenna Francesco & Nome Verificatore & Prima stesura del documento \\ 
    \hline
\end{tabular}
\end{table}
%%%%%%%%%%%%%%%%%%%%%%%%%%%%%%%%%%%%%%%%%%%%%%%%%%%%%%%%%%%%%%%%%%%%%%%%%%%%%%%%%%%%%%



% Indice %%%%%%%%%%%%%%%%%%%%%%%%%%%%%%%%%%%%%%%%%%%%%%%%%%%%%%%%%%%%%%%%%%%%%%%%%%%%%
\newpage
\tableofcontents
%%%%%%%%%%%%%%%%%%%%%%%%%%%%%%%%%%%%%%%%%%%%%%%%%%%%%%%%%%%%%%%%%%%%%%%%%%%%%%%%%%%%%%



% Sezione Introduzione %%%%%%%%%%%%%%%%%%%%%%%%%%%%%%%%%%%%%%%%%%%%%%%%%%%%%%%%%%%%%%%
\newpage
\section{Introduzione}

    \subsection{Scopo del documento}
    Questo documento ha lo scopo di fornire una guida dettagliata e strutturata su come 
    il progetto verrò eseguito e gestito. In particolare, verrano trattati i seguenti 
    argomenti:
    \begin{itemize}
        \item analisi dei rischi;
        \item risorse necessarie;
        \item suddivisione dei ruoli;
        \item stime dei costi;
        \item modello di sviluppo adottato;
    \end{itemize}

    \subsection{Glossario}
    % TODO

    \subsection{Riferimenti}
    % TODO

%%%%%%%%%%%%%%%%%%%%%%%%%%%%%%%%%%%%%%%%%%%%%%%%%%%%%%%%%%%%%%%%%%%%%%%%%%%%%%%%%%%%%%



% Analisi del Capitolato %%%%%%%%%%%%%%%%%%%%%%%%%%%%%%%%%%%%%%%%%%%%%%%%%%%%%%%%%%%%%
\section{Analisi del Capitolato}
    \subsection{Obbiettivi del progetto}
    \begin{itemize}
        \item Realizzare un Assistente Virtuale che supporti i clienti nella ricerca 
        di informazioni sui prodotti disponibili in catalogo.
        \item Automatizzare le risposte alle domande più frequenti, migliorando 
        l'efficienza del servizio clienti.
        \item Integrare un modello LLM esistente per garantire risposte accurate e 
        un'interfaccia user-friendly.
    \end{itemize}

    \subsection{Ambito del Progetto}
        \subsubsection{Inclusioni}
        \begin{itemize}
            \item Database relazionale per la gestione dei dati sui prodotti.
            \item Integrazione di un modello LLM tramite API.
            \item Interfaccia utente mobile per l'interazione con l'IA.
            \item Funzionalità di configurazione backend per template di domande e risposte.
        \end{itemize}
        \subsubsection{Esclusioni}
        \begin{itemize}
            \item Creazione di un nuovo modello LLM.
            \item Supporto a lingue non previste dal modello LLM scelto.
        \end{itemize}

    \subsection{Tecnologie e Strumenti Consigliati}
    \begin{itemize}
        \item \textbf{Database}: MySQL o PostgreSQL.
        \item \textbf{LLM}: BLOOM o Italia by iGenius, in base alle prestazioni richieste.
        \item \textbf{Backend}: Node.js con Express.js o .NET.
        \item \textbf{Frontend}: .NET MAUI per applicazioni mobile multipiattaforma.
        \item \textbf{API REST}: Per la comunicazione tra LLM e interfaccia utente.
        \item \textbf{Controllo Versione}: Git (GitHub per repository pubblico).
    \end{itemize}

%%%%%%%%%%%%%%%%%%%%%%%%%%%%%%%%%%%%%%%%%%%%%%%%%%%%%%%%%%%%%%%%%%%%%%%%%%%%%%%%%%%%%%



% Pianificazione %%%%%%%%%%%%%%%%%%%%%%%%%%%%%%%%%%%%%%%%%%%%%%%%%%%%%%%%%%%%%%%%%%%%%
\section{Pianificazione}
    \subsection{Struttura del Team}
        \subsubsection{Ruoli (da sistemare !!!)}
        \begin{itemize}
            \item \textbf{Project Manager}: Coordinazione generale.
            \item \textbf{Data Engineer}: Gestione del database e pre-processing dei dati.
            \item \textbf{Sviluppatore Backend}: Implementazione delle API REST e configurazione della piattaforma.
            \item \textbf{Sviluppatore Frontend}: Creazione dell’interfaccia utente.
            \item \textbf{Esperto LLM}: Selezione e integrazione del modello linguistico.
            \item \textbf{Tester}: Validazione e verifica delle funzionalità.
        \end{itemize}
        \subsubsection{Stakeholder}
        \begin{itemize}
            \item \textbf{Cliente}: Ergon Informatica Srl.
            \item \textbf{Referente interno}: Gianluca Carlesso.
        \end{itemize}

    \subsection{Budget e Risorse}
    \begin{itemize}
        \item Stima delle ore/uomo per ogni fase.
        \item Allocazione hardware: server per database e API, risorse cloud per il modello LLM.
        \item Licenze software e costi del modello LLM (se applicabile).
    \end{itemize}

    \subsection{Rischi e Mitigazioni}
    \begin{itemize}
        \item \textbf{Rischio}: Prestazioni del modello LLM inferiori alle aspettative.
        \begin{itemize}
            \item \textbf{Mitigazione}: Test preliminari con più modelli per selezionare quello ottimale.
        \end{itemize}
        \item \textbf{Rischio}: Complessità nell'integrazione tra LLM e database.
        \begin{itemize}
            \item \textbf{Mitigazione}: Utilizzo di middleware standardizzato.
        \end{itemize}
        \item \textbf{Rischio}: Problemi di usabilità nell'interfaccia utente.
        \begin{itemize}
            \item \textbf{Mitigazione}: Coinvolgimento degli utenti in fase di test.
        \end{itemize}
    \end{itemize}

    \subsection{Piano di Comunicazione}
    \begin{itemize}
        \item Riunioni settimanali con il team.
        \item Report di avanzamento per il referente aziendale ogni 2 settimane.
        \item Feedback continuo attraverso test intermedi.
    \end{itemize}

    \subsection{Pianificazione delle Attività}
        Il gruppo Code7Crusaders 
        si impegna a consegnare il progetto entro il 14/03/2025. La pianificazione prevede 
        19 settimane di lavoro, suddivise come segue:
        \begin{itemize}
            \item \textbf{PoC (\textit{Proof of Concept}): 6 settimane}
            \item \textbf{MVP (\textit{Minimum Viable Product}): 13 settimane}
        \end{itemize}

        \subsubsection{Fasi principali}
        \begin{enumerate}
            \item \textbf{Analisi dei requisiti} (2 settimane):
            \begin{itemize}
                \item Revisione del capitolato.
                \item Identificazione delle tecnologie e dei modelli LLM adatti.
            \end{itemize}
            \item \textbf{Progettazione} (2 settimane):
            \begin{itemize}
                \item Progettazione architetturale.
                \item Definizione dello schema del database.
            \end{itemize}
            \item \textbf{Sviluppo Backend} (4 settimane):
            \begin{itemize}
                \item Configurazione del database.
                \item Implementazione delle API REST.
            \end{itemize}
            \item \textbf{Integrazione LLM} (3 settimane):
            \begin{itemize}
                \item Pre-processing dei dati e integrazione del modello LLM.
            \end{itemize}
            \item \textbf{Sviluppo Frontend} (4 settimane):
            \begin{itemize}
                \item Creazione dell'interfaccia utente mobile.
            \end{itemize}
            \item \textbf{Test e validazione} (2 settimane):
            \begin{itemize}
                \item Test funzionali e di usabilità.
            \end{itemize}
            \item \textbf{Rilascio e documentazione} (2 settimane).
        \end{enumerate}

    \subsection{Cronoprogramma}
    % Diagramma di Gantt %%%%%%%%%%%%%%%%%%%%%%%%%%%%%%%%%%%%%%%%%%%%%%%%%%%%%%%%%%%%%
    \resizebox{\textwidth}{!}{ % Scales the chart to fit the text width
        \begin{ganttchart}[
            hgrid, vgrid, % Adds grid lines
            x unit=1.1cm, % Adjust the width of each time slot
            title/.append style={draw=black, thick, fill=red!20, line width=1pt, draw opacity=1},
            bar/.append style={fill=red!50}, % Custom bar styling
            milestone/.append style={fill=red!70}, % Milestone style
            ]{1}{19} % Timeline from week 1 to week 19
            \gantttitle{Project Plan (Weeks)}{19} \\ % Title spanning 19 weeks
            \gantttitle{Novembre 2024}{4}
            \gantttitle{Dicembre 2024}{5}
            \gantttitle{Gennaio 2025}{4}
            \gantttitle{Febbraio 2025}{4}
            \gantttitle{Marzo 2025}{2}
            \\
            \gantttitle{4 nov}{1}  % TODO: decidere cosa tenere tra i titoli
            \gantttitle{11 nov}{1}
            \gantttitle{18 nov}{1}
            \gantttitle{25 nov}{1}
            \gantttitle{2 dic}{1}
            \gantttitle{9 dic}{1}
            \gantttitle{16 dic}{1}
            \gantttitle{23 dic}{1}
            \gantttitle{30 dic}{1}
            \gantttitle{6 gen}{1}
            \gantttitle{13 gen}{1}
            \gantttitle{20 gen}{1}
            \gantttitle{27 gen}{1}
            \gantttitle{3 feb}{1}
            \gantttitle{10 feb}{1}
            \gantttitle{17 feb}{1}
            \gantttitle{24 feb}{1}
            \gantttitle{3 mar}{1}
            \gantttitle{10 mar}{1} \\
            \gantttitlelist{1,...,19}{1} \\ % Weekly columns
            
            % Define groups and tasks
            \ganttgroup{Sviluppo PoC}{1}{6} \\
            \ganttmilestone{RTB}{6} \\
            \ganttgroup{Sviluppo MVP}{7}{19} \\
            \ganttmilestone{MVP}{19} \\

            \ganttbar{Analisi dei requisiti}{1}{2} \\
            \ganttbar{Progettazione}{3}{4} \\
            \ganttbar{Sviluppo backend}{5}{8} \\
            \ganttbar{Integrazione LLM}{9}{11} \\
            \ganttbar{Sviluppo frontend}{12}{15} \\
            \ganttbar{Test e validazione}{16}{17} \\
            \ganttbar{Rilascio e documentazione}{18}{19} \\

            \ganttlink{elem4}{elem5}  % TODO: decidere se tenere o togliere i link
            \ganttlink{elem5}{elem6}
            \ganttlink{elem6}{elem7}
            \ganttlink{elem7}{elem8}
            \ganttlink{elem8}{elem9}
            \ganttlink{elem9}{elem10}
        
        \end{ganttchart}
    }
    %%%%%%%%%%%%%%%%%%%%%%%%%%%%%%%%%%%%%%%%%%%%%%%%%%%%%%%%%%%%%%%%%%%%%%%%%%%%%%%%%%

    \subsection{Preventivo}
    Visti e considerati i precedenti punti, il preventivo del progetto viene calcolato 
    pari a €12.805,00.
%%%%%%%%%%%%%%%%%%%%%%%%%%%%%%%%%%%%%%%%%%%%%%%%%%%%%%%%%%%%%%%%%%%%%%%%%%%%%%%%%%%%%%



%%%%%%%%%%%%%%%%%%%%%%%%%%%%%%%%%%%%%%%%%%%%%%%%%%%%%%%%%%%%%%%%%%%%%%%%%%%%%%%%%%%%%%
\section{Documentazione}  % TODO: vedere se tenere
\begin{itemize}
    \item \textbf{Da consegnare}:
    \begin{itemize}
        \item Schema architetturale.  % TODO: riferimenti ad altri documenti?
        \item Progettazione del database.
        \item Manuale per l'utilizzo della piattaforma.
        \item Codice sorgente e repository Git.
    \end{itemize}
\end{itemize}
%%%%%%%%%%%%%%%%%%%%%%%%%%%%%%%%%%%%%%%%%%%%%%%%%%%%%%%%%%%%%%%%%%%%%%%%%%%%%%%%%%%%%%



%%%%%%%%%%%%%%%%%%%%%%%%%%%%%%%%%%%%%%%%%%%%%%%%%%%%%%%%%%%%%%%%%%%%%%%%%%%%%%%%%%%%%%
\section{Gestione del modello e sprint}
    \subsection{Modello adottato e motivazioni}
%%%%%%%%%%%%%%%%%%%%%%%%%%%%%%%%%%%%%%%%%%%%%%%%%%%%%%%%%%%%%%%%%%%%%%%%%%%%%%%%%%%%%%

\end{document} 
